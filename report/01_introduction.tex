\section{Introduction}

There is little doubt that algorithms will play an increasingly important role in everyday life. Automated and dynamic pricing are frequently used in online retail \parencite{chen_empirical_2016}, but are also applied in the tourist industry \parencite[p.4]{den_boer_dynamic_2015} and at petrol stations \parencite[pp.7-9]{assad_algorithmic_2020}. 

\textbf{where is it prevalent}


As with many other technological advances, the economic advantages are conspicuous. Not only does automating pricing decisions cut costs and free up resources, algorithms may also be better at predicting demand and react faster to changing market conditions \parencite[p. 15]{oecd_algorithms_2017}. 
This non-exhaustive list of examples leaves little doubt that pricing algorithms may be used as tool by companies to gain competitive advantages and it is worth pointing out that consumers also benefit from intensified competition.\footnote{Moreover, there exist other types of algorithms that benefit consumers. Price comparison tools have been around for a while but the applications extend beyond that. \textcite{gal_algorithmic_2017} champion \emph{algorithmic consumers}, electronic assistants who make sophisticated product comparisons at low transaction costs enabling humans to completely outsource their purchase decisions. Moreover, algorithmic consumers may challenge market power of suppliers by bundling consumer interests.}

Nevertheless, concerns have been raised that ceding pricing authority to algorithms has the potential to create new forms of collusion that contemporaneous competition policy is not well equipped to deal with. The main issue is that the traditional dichotomy between \emph{explicit} and \emph{tacit} collusion is potentially unsuitable in the case of pricing software. Traditionally, competition authorities only prohibit and punish explicit pricing agreements. On the contrary, tacit collusion (e.g.\ \emph{intelligent market adaption}) is tolerated \textbf{citation needed?} despite the economic effect on consumers being equally detrimental \parencite[p. 141]{motta_competition_2004}.\footnote{A different issue is that pricing algorithms with information on consumer characteristics may be able to augment the scope of \emph{perfect price discrimination}, i.e.\ companies extracting rent by charging to every consumer the highest price he is willing to pay. Under which circumstances competition authorities should be concerned with this possibility is outlined in \textcite{oecd_price_2016}. \textcite{ezrachi_algorithmic_2017} develop a scenario where discriminatory pricing and tacit collusion occur simultaneously. Both issues remain outside the scope of this study.}


\textbf{The distinction was elusive before the advent of pricing algorithms. But it is even less clear now and might not capture the danger of algorithmic collusion}

\textbf{painpoint of of algorithms --> why do we care?} 

Unfortunately, there is a lack of empirical studies assessing the effects of autonomous pricing software in the real world. A notable study of the German retail gasoline market by \textcite{assad_algorithmic_2020} documents that margins in duopoly markets increased substantially after both gasoline stations switched from manual pricing to algorithmic-pricing software. Further field studies could prove instrumental to confirm and refine these findings. On the other hand, there is a growing number of simulation studies that show the capacity of \emph{reinforcement learning} algorithms to create and sustain collusive equilibria in repeated pricing competition games (see \autoref{simulation_studies}). However, the direct transferability of these findings to real markets is questionable. Most studies use tabular learning methods, mainly \emph{Q-Learning}, which requires discretizing prices and does not scale well if the complexity of the environment increases.

This study attempts to get rid of the problem by employing \emph{linear function approximation} to estimate the value of actions. More specifically, I develop three methods of function approximation and then run a series of experiments to assess how they compare to tabular learning. Moreover, I utilize \emph{eligibility traces} as an efficient way to increase the memory of agents interacting in the environment.\footnote{Neither linear function approximation nor eligibility traces are new concepts in reinforcement learning. However, to my knowledge, this is the first study to apply them to a repeated pricing game.}

To foreshadow the results, the simulations show that the developed function approximation methods, like tabular learning, result in supra-competitive prices upon convergence. However, \emph{unlike} tabular learning, the learned strategies are easy to exploit. By forcing one of the agents to diverge from the convergence equilibrium, I show that the cheated agent fails to punish that deviation. This indicates that the learned equilibrium strategies are unstable vis-à-vis rational agents with full information. This observation is robust to a number of variations and extensions. Also, the impact of eligibility traces in this study is small.

The remainder of this paper is organized as follows. The next section briefly reviews literature on algorithmic competition as well as contemporaneous regulation and presents results from previous simulations similar to this study. Section \ref{enironment} introduces the repeated pricing environment I let the competitors interact with. Section \ref{algorithm} present in detail the deployed learning algorithm with its parametrization and \autoref{feature_extraction} discusses the developed methods to estimate action values with function approximation. I present the results in \autoref{results} and consider variations and extensions in \autoref{robustness}. Finally, \autoref{conclusions} concludes.