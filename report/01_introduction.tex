\section{Introduction}

There is little doubt that algorithms will play an increasingly important role in economic life. Dynamic pricing software is frequently used in online retail markets \parencite{chen_empirical_2016}, the tourist industry \parencite[p.4]{den_boer_dynamic_2015} and at petrol stations \parencite[pp.7-9]{assad_algorithmic_2020}. As with many other technological advances, the economic advantages are conspicuous. Not only does automating pricing decisions cut costs and free up resources, algorithms may also be better at predicting demand and react faster to changing market conditions \parencite[p. 15]{oecd_algorithms_2017}. Overall, there is little doubt that pricing algorithms may be used as a tool by companies to gain competitive advantages. It is worth pointing out that thereby intensified competition also benefits consumers.\footnote{Of course other types of algorithms that benefit consumers exist. Price comparison tools have been around for a while but applications extend beyond mere reduction of search costs. \textcite{gal_algorithmic_2017} champion \emph{algorithmic consumers}, electronic assistants that compare product characteristics at low transaction costs enabling humans to completely outsource their purchase decisions. Moreover, algorithmic consumers may challenge market power of suppliers by bundling consumer interests.}

Nevertheless, concerns have been raised that ceding pricing authority to algorithms has the potential to create new forms of collusion that contemporaneous competition policy is not well equipped to deal with. The main issue is that the traditional dichotomy between \emph{explicit} and \emph{tacit} collusion is potentially unsuitable in the case of pricing software. Traditionally, competition authorities only prohibit and punish explicit pricing agreements. On the contrary, tacit collusion (e.g.\ \emph{intelligent market adaption}) is typically tolerated despite the economic effect on consumers being equally detrimental \parencite[p. 141]{motta_competition_2004}. In practice, the distinction is sometimes vague and the advent of pricing algorithms is believed to blur the line. It is still unclear when \emph{algorithmic collusion} could elude competition enforcement.\footnote{A different issue is that pricing algorithms with information on consumer characteristics may be able to augment the scope of \emph{price discrimination}, i.e.\ companies extracting rent by charging to every consumer the highest price he is willing to pay. Under which circumstances competition authorities should be concerned with this possibility is outlined in \textcite{oecd_price_2016}. \textcite{ezrachi_algorithmic_2017} develop a scenario where discriminatory pricing and tacit collusion occur simultaneously. Both issues remain outside the scope of this study.}

Unfortunately, there is a lack of empirical studies assessing the effects of autonomous pricing software in the real world. A notable study of the German retail gasoline market by \textcite{assad_algorithmic_2020} documents that margins in duopoly markets increased substantially after both duopolists switched from manual pricing to algorithmic-pricing software. Further field studies could prove instrumental to confirm and refine these findings. As a substitute, there is a growing number of simulation studies that show the capacity of \emph{reinforcement learning} algorithms to create and sustain collusive equilibria in repeated games of competition (see \autoref{literature review}). However, the direct transferability of these findings to real markets is questionable. Most studies use a simple tabular learning method, called \emph{Q-Learning}, that requires discretizing prices and does not scale well if the complexity of the environment increases.

This study attempts to mitigate these problems by employing \emph{linear function approximation} to estimate the value of actions. More specifically, I develop three methods of function approximation and run a series of experiments to assess how they compare to tabular learning. Moreover, I utilize \emph{eligibility traces} as an efficient way to increase the memory of agents interacting in the environment.\footnote{Neither linear function approximation nor eligibility traces are new concepts in reinforcement learning. However, to my knowledge, this is the first study to apply them to a repeated pricing game.}

To foreshadow the results, the simulations show that the developed function approximation methods, like tabular learning, result in supra-competitive prices upon convergence. However, \emph{unlike} tabular learning, the learned strategies are easy to exploit. By forcing one of the agents to diverge from the convergence equilibrium, I show that the cheated agent fails to punish that deviation. This indicates that the learned equilibrium strategies are unstable vis-à-vis rational agents with full information. This observation is robust to a number of variations and extensions. Also, with respect to eligibility traces, excessively increasing memory tends to destabilize the learning process, but the overall impact for reasonable parametrization appears small.

The remainder of this paper is organized as follows. The next section briefly surveys simulation studies similar to this one. I also review the scholarly literature on algorithmic competition and contemporaneous regulation. Section \ref{enironment} introduces the repeated pricing environment in which the artificial competitors interact. Section \ref{algorithm} presents in detail the deployed learning algorithm with its parametrization and \autoref{feature_extraction} discusses the developed methods to estimate action values with function approximation. I present the results in \autoref{results} and consider variations and extensions in \autoref{robustness}. Section \ref{conclusions} concludes.