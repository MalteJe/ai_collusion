\section{Introduction}

Collusion is traditionally distinguished between \emph{explicit} and \emph{tacit} collusion. While the detrimental effect on consumers is (qualitatively) the same, authorities currently prohibit and punish only explicit pricing agreements but condone tacit collusion as a consequence of e.g. \emph{intelligent market adaption} \textbf{citation needed?}.


\footnote{A different issue is that pricing algorithms with information on consumer characteristics may be able to augment the scope of \emph{perfect price discrimination}, i.e.\ companies extracting consumer rent by charging to every individual the highest price he is willing to pay. Under which circumstances competition competition authorities should be concerned with this possibility is outlined in \textcite{oecd_price_2016}. \textcite{ezrachi_algorithmic_2017} develop a scenario where discriminatory pricing and tacit collusion occur simultaneously. Though both issues remain outside the scope of this study.}