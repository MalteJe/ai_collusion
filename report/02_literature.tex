\section{Literature Review}
This study is related to three literature streams: (i) the scholarly debate on how competition law is supposed to manage autonomous pricing software, (ii)  repeated games in the realm of algorithms, and (iii) an increasing number of simulation studies that empirically examine the behavior of algorithms in simplified economic environments. I will provide a brief summary of the recent developments in each of these fields.

\subsection{Competition Policy and pricing algorithms}
As algorithms increasingly take over pricing authority from humans in a number of industries, some scholars have voiced concerns about the adequacy of current competition laws and practices. \textcite{mehra_antitrust_2015} points out that the traditional distinction between explicit and tacit collusion emerged with \emph{human sellers} in mind who differ from \emph{robo-sellers}. He argues that the latter is more likely to achieve cartel solutions in an oligopolistic setting due to superior speed, accuracy and even rationality when analyzing and adjusting prices. He concludes that the increasing prevalence of automated pricing software warrants a reassessment of current competition law and enforcement. 

* no human intent to achieve supracompetitive prices
	* but employed algorithms achieved that anyway. 
	* outcome eludes explicit agreement, still detrimental
 light of recent development 

Whereas \emph{human pricing} 



\textcite{calvano_artificial_2019} simulate two Q-Learning agents that not only achieve collusive outcomes, but also sustain their tacit agreement through a \emph{reward-punishment} scheme. \textcite{klein_autonomous_2019} shows that Q-Learning algorithms converge faster in a sequential price setting environment.