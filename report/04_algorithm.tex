\section{Reinforcement Learning with Function Approximation}

if the system were stationary --> convergence guarantee (e.g. Jaakoola et al. 1994). However, non-stationarity induced by multi-agent learning breaks that guarantee. To my knowledge, no guarantees, but empirically strong results

Though both agents repeatedly face the environment as presented in \autoref{enironment}, I will present this section from the vantage point of a single agent. Accordingly, the subscript {i} is dropped when appropriate.

\subsection{value approximation}\label{value_approximation}

The agent learns to approximate the value of an action given the available information. The potential actions reflect the available prices in the current period. It is useful to discretize the action space.\footnote{This discretization usually implies that agents will not charge exactly $p_n$ or $p_m$.} Compared to the baseline specification in \textcite{calvano_artificial_2019}, I consider a wider price range confined by a lower bound $A^L$ and an upper bound $A^U$:

\begin{gather}
A^{L} = c
\end{gather}

\begin{gather}
A^{U} = p_m + \zeta (p_n - c)
\end{gather}

The lower bound ensures positive margins. It is conceivable that a human manager could implement a sanity restriction like that before conceding pricing authority to an algorithm. The parameter $\zeta$ controls the extent to which the upper bound $A^U$ exceeds the monopoly price. With $\zeta = 1$, the difference between $A^{L}$ and $p_n$ is equal to the difference between $A^{U}$ and $p_m$. The available set of prices $\mathcal{A}$ is then evenly spaced out in the interval $[A^L, A^U]$:

\begin{gather}
	\mathcal{A} = (A^L, A^L + \frac{1(A^U - A^L)}{m-1}, A^L + \frac{2(A^U - A^L)}{m-1}~ , ... , ~ A^L + \frac{(m-2)(A^U - A^L)}{m-1}, A^U)
\end{gather}

$m$ determines the number of feasible prices. Following \textcite{sutton_reinforcement_2018}, I denote any possible action as $a \in \mathcal{A}$ and the actual realization at time $t$ as $A_t$.

In this simulation, the state set $S_t$ comprises merely the prices of the previous period $t-1$:

\begin{gather}
S_t = \{ p_{i, t-1}, p_{j, t-1} \}
\end{gather}


Accordingly, for every state variable, the set of possible states is identical to the feasible actions, i.e. $\mathcal{A}$. However, this is not required with function approximation methods. Theoretically, any state variable could be continuous and unbounded. Similarly to actions, $s \in \mathcal{S}$ denotes any possible state set and $S_t$ refers to the actual states at $t$.






Lastly, a set of parameters $\boldsymbol{w} = \{w_1, w_2, ..., w_D\}$, where $d \in \{1, 2, ..., D\}$, maps any combination of $S_t$ and $A_t$ to a value estimate $\hat{q-}_t$.\footnote{In the computer science literature, $\boldsymbol{w}$ is typically referred to as \emph{weights}. I will stick to the economic vocabulary and declare $\boldsymbol{w}$ parameters.} Hence:

\begin{gather}\label{q_estimation}
	\hat{q-}_t = \hat{q}(S_t,A_t,\boldsymbol{w}_t) = \hat{q}(p_{i, t-1}, p_{j, t-1}, p_{i, t}, \boldsymbol{w}_t)
\end{gather}

More specifically, any state-action combination is represented by a \emph{feature vector} $\boldsymbol{x}_t = \boldsymbol{x}(S_t, A_t) = \{x_1(S_t, A_t), x_2(S_t, A_t), ..., x_D(S_t, A_t)\}$ and every \emph{feature} $x_d = x_d(S_t, A_t)$, where every element is derived from a state, an action or a combination thereof. Moreover each $x_d$ is associated with a counterpart $w_d$. In section \ref{feature_extraction} the mechanisms to extract features are outlined in more detail. For now, note that I will only consider linear functions of $\hat{q}$. In this case, \autoref{q_estimation} can be written as the inner product of the feature vector and the set of parameters, i.e.\ $\boldsymbol{x}_t \top \boldsymbol{w} = \sum_{d=1}^{D} x_d * w_d$

Two elements are required for the algorithms to work successfully. First, the agents must mix between \emph{exploration and exploitation}. Second, the set of parameters $\boldsymbol{w}$ must be continuously optimized.

\paragraph{Exploration and Exploitation} 
In every period, the agent chooses either to \emph{exploit} its current knowledge and pick the supposedly optimal action or to \emph{explore} in order to test the merit of alternative choices that are perceived sub-optimal but may turn out to be superior. As is common, I use a simple $\epsilon$-greedy policy to steer this tradeoff:

\begin{gather}\label{action_selection}
 A_t = \begin{cases} arg ~\underset{a}{max} ~ \hat{q}(S_t,a,\boldsymbol{w}_t) & \quad \text{with probability } 1 - \epsilon_t\\
\text{randomize over } \mathcal{A} & \quad \text{with probability } \epsilon_t\\ \end{cases} 
\end{gather}

In words, the agent chooses to play the action that is regarded optimal with probability $1-\epsilon_t$ and randomizes over all prices with probability $\epsilon_t$.\footnote{If more than one $a$ maximizes $\hat{q}$, ties are broken randomly.} The subscript suggests that exploration varies over time. The explicit definition is given by:

\begin{gather}
	\epsilon_t = \psi e^{-\beta t}~ \text{, where}
\end{gather}

$\psi \in [0, 1]$ defines the initial exploration rate at $t = 0$ and $\beta$ controls the speed of its decay. This \emph{time-declining} exploration rate ensures that the agent randomizes actions frequently at the beginning of the simulation and stabilizes its behavior over time. 

After both agents selected an action, the quantities and profits are realized in accordance with equations \ref{quantity} \& \ref{profit}. The agents' actions in period $t$ become the state set in $t+1$ and new actions are chosen as dictated by equations \ref{q_estimation} \& \ref{action_selection}.

Irrespective of the \emph{exploit vs explore} decision, the agent proceeds to leverage the observed outcomes to refine $\boldsymbol{w}$.



\paragraph{Update}

After observing the opponent's price and the own profits, the agent exploits this new information to improve $\boldsymbol{w}$. A good starting point to introduce the utilized update rules is the so called \emph{TD error}, denoted $\delta_t$ (\textbf{finalize footnote}).\footnote{Without function approximation, versions of the \emph{TD error} usually encompass a discount factor $gamma$, such as:
	\begin{center}
		$\delta_t^{SARSA} = \pi_t + \gamma \hat{q}(S_{t+1}, A_{t+1}, \boldsymbol{w}) - \hat{q}(S_t, A_t, \boldsymbol{w})$
		
		$\delta_t^{Q-Learning} = \pi_t + \gamma ~ \underset{a}{max} ~ \hat{q}(S_{t+1}, a, \boldsymbol{w}) - \hat{q}(S_t, A_t, \boldsymbol{w})$
\end{center}
While they come with a meaningful economic interpretation, \textcite{sutton_reinforcement_2018} and \textcite{naik_discounted_2019} show that their use is inappropriate in infinite sequences with function approximation settings. Moreover, a policy maximizing average rewards is equivalent to a policy maximizing the average of discounted future values - irrespective of the particular discount factor.}

\textbf{average setting update}

\begin{gather}
	\delta_t = r_t - \widetilde{R}_{t-1} + \hat{q}(S_{t+1}, A_{t+1}, \boldsymbol{w}_t) - \hat{q}(S_t, A_t, \boldsymbol{w}_t) ~~   \text{,}
\end{gather}

where the reward $r_t = \pi_t - p_n$ reflects the profits relative to the Nash solution and $\widetilde{R}_{t-1}$ is a (weighted) average reward.\footnote{Please note that I explicitly distinguish profits and rewards. Profits, $\pi$, represent the monetary remuneration from operating in the environment and can be interpreted economically. However, profits do not enter the learning algorithm directly. Instead, rewards, $r$, immediate successors of profits, constitute the signal that is utilized as feedback by the agents to refine their algorithms.} $\delta_t$ measures the difference between the \emph{ex ante} ascribed value to the selected state-action combination in $t$ and the \emph{ex post} \emph{differential} profit $\pi_t - \widetilde{R}_{t-1}$ in conjunction with the estimated value of the newly arising state-action combination in $t+1$. A positive $\delta_t$ indicates that the actual realization turned out to exceed the original expectation. Likewise, a negative $\delta_t$ suggests that the realization failed short of the expected reward of playing the particular state-action combination. In both instances, $\boldsymbol{w}$ will be adjusted accordingly, such that the state-action combination is valued respectively higher or lower next time. Note that $\delta_t$ can only be calculated after the action in the next period has been taken.\footnote{This is often referred to as \emph{SARSA}, abbreviating a state-action-reward-state-action sequence.}


Surprisingly, this system 
* discount factor can be = 1






\emph{Semi-gradient} methods constitute a basic procedure for such continuous optimization. They serve as a good benchmark before developing more complex algorithms. Essentially, the direction and magnitude of updating parameters is driven by the \emph{TD error} $\delta_t$ and the gradient of $\hat{q}_t(S_t, A_t, \boldsymbol{w})$ with respect to $\boldsymbol{w}$:
$\frac{\Delta \hat{q}}{\Delta \boldsymbol{w}} =
\{ \frac{\Delta \hat{q}}{\Delta w_1},
\frac{\Delta \hat{q}}{\Delta w_2},
...,
\frac{\Delta \hat{q}}{\Delta w_d}  \}$. The update rule is:

\begin{gather}
 \boldsymbol{w}_{t+1} \leftarrow \boldsymbol{w}_t +
 	\alpha \delta_t
 	\frac{\Delta \hat{q}}{\Delta \boldsymbol{w}} ~~ \text{,}
\end{gather}

where $\alpha$ steers the speed of learning. As indicated earlier, I will only consider approximations that are linear in parameters. Thus, $\frac{\Delta \hat{q}}{\Delta \boldsymbol{w}}$ simplifies to the \emph{feature vector} $\boldsymbol{x}_t = \{x_1, x_2, ..., x_d\}$.

\textcite{seijen_true_2014} showed that the performance of that algorithm can be improved by keeping track of an eligibility vector $\boldsymbol{z} = \{z_1, z_2, ..., z_d\}$, called the \emph{Dutch Trace}. Like $\boldsymbol{w}$, this trace vector is updated at every time step. The trick is that the eligibility trace controls the magnitude by which individual parameters are updated, prioritizing those that contributed to producing an estimate of $hat{q}$. The update rule for the eligibility trace is:

\begin{gather}
\boldsymbol{z}_{t} \leftarrow \boldsymbol{z}_{t-1} \gamma \lambda + \boldsymbol{x_t} (1 - \alpha \gamma \lambda  {  \boldsymbol{x_t} \intercal \boldsymbol{z}_{t-1} }) ~~ \text{,}
\end{gather}

where $ \boldsymbol{x}_t \top \boldsymbol{z}_t $ denotes the inner product of $\boldsymbol{x}_t$ and $\boldsymbol{z}_t$, i.e. $ \boldsymbol{x}_t \top \boldsymbol{z}_t  = \sum_{i}^{d} x_i z_i$. $\boldsymbol{z}$ is then used in the refined parameter update:

\begin{gather}
	\boldsymbol{w}_{t+1} \leftarrow
		\boldsymbol{w}_{t} +
		\alpha \delta \boldsymbol{z}_t +
		\alpha ( \boldsymbol{w}_t \intercal \boldsymbol{x}_t  -
				 \boldsymbol{w}_{t-1} \intercal \boldsymbol{x}_t)
				(\boldsymbol{z}_t - \boldsymbol{x}_t)
\end{gather}





\subsection{Feature Extraction}\label{feature_extraction}

\textbf{TBD: introduction}
As outlined in \autoref{value_approximation}, the state-action space contains just 3 variables. Assigning a single coefficient to each variable certainly fails to do justice to the complexity of the optimization problem. In particular, a \emph{reward-punishment} theme requires that actions are chosen conditional on past prices (i.e.\ the state space). Hence, it is imperative to consider interactions and non-linearities. Therefore, I utilize various methods to extract features form the state-action space.

In reinforcement learning, a common approach is to store a distinct set of coefficients for every feasible action.\footnote{In this case, the vector of coefficients contains \emph{m} times features components} This is a sensible approach with qualitative action spaces. However, very much like tabular learning, a separate set of coefficients neglects the (quasi-) continuous nature of prices. Therefore two issues arise. First, discretizing the action space doesn't scale well if the number of feasible prices increases. Consequently, learning requires relatively many periods with large $m$. Second, observing a particular reward may not only constitute an informative feedback for the particular action undertaken, but also for 'similar' prices. Using and updating coefficients valid for (a subset of) all feasible prices exploits this.

For this simulation, I use \emph{polynomials}, \emph{polynomial splines} and \emph{tile coding} to extract features from the state-action space.

\subsubsection{Polynomials}

\emph{Polynomial approximation} of order $k$ maps states and action to a set of features, where a single feature corresponds to:



\begin{gather}
x_i^{Poly} = p_{1, t-1}^{\kappa_1} ~ p_{2, t-1}^{\kappa_2} ~ p_{1, t}^{\kappa_3}
\end{gather}


Every combination of exponents that adheres to the restrictions

\begin{itemize}
	\item $0 < \kappa_1 + \kappa_2 + \kappa_3 \leq k$ and
	\item $\kappa_1, \kappa_2, \kappa_3 \in \{0, 1, ..., k\}$
\end{itemize}

constitutes one feature. Using polynomial approximation, the feature vector $\boldsymbol{x}$ contains ${k + 3\choose3}  - 1$ elements.

\subsubsection{Normalized Polynomials}

\textbf{TBD}

\subsubsection{Polynomial Splines}

\textbf{TBD}

\subsubsection{Tile Coding}

In reinforcement learning, \emph{Tile Coding} is a common way to extract linear, in fact binary, features from a state-action space.\footnote{for an extensive introduction with instructive illustrations refer to \textcite{sutton_reinforcement_2018}} The idea is that several \emph{tilings} superimpose the state-action space. The $\mathcalligra{T}$ \ tilings are offset but each tiling covers the entire state-action space:

\begin{gather}
	 \mathcal{T}^L \leq A^L  ~ \& ~ \mathcal{T}^U \geq A^U    \text{for } \mathcal{T} \in \{1, 2, ..., \mathcalligra{T} ~ \}
\end{gather}

Each tiling is itself composed of uniformly spaced out \emph{tiles}.\footnote{With 2 dimensions, a tiling simply corresponds to a grid. In our case, the state-action space is 3-dimensional, so it may prove more intuitive to think of cubes instead of tilings and tiles.} Every tile is uniquely demarcated by a lower and an upper threshold for every dimension. Consequently, the number of tiles per tiling is controlled by the number of thresholds. For this simulation, it suffices to define a single set of thresholds per tiling that applies to all 3 dimensions. More specifically, the thresholds are spaced out evenly in the tiling-specific interval $[\mathcal{T}^L, \mathcal{T}^U]$:

\begin{gather}
\mathcal{T} = (
\mathcal{T}^L,
\mathcal{T}^L + \frac{1(\mathcal{T}^U - \mathcal{T}^L)}{\tau},
\mathcal{T}^L + \frac{2(\mathcal{T}^U - \mathcal{T}^L)}{\tau}~ , ... , ~
\mathcal{T}^L + \frac{(\tau-1)(\mathcal{T}^U - \mathcal{T}^L)}{\tau},
\mathcal{T}^U)
\end{gather}

This gives rise to $\tau^3$ tiles per tiling. Tiles are binary, i.e.\ if a state-action observation falls into a particular demarcation, the corresponding tile is \emph{activated}:

\begin{gather}\label{tile_activation}
x_i^{Tiling} = \begin{cases}
1 & \quad \text{if } \{p_{1, t-1}, p_{2, t-1}, p_{1, t}\} \text{~in tile demaraction}_i  \\
0 & \quad \text{if } \{p_{1, t-1}, p_{2, t-1}, p_{1, t}\} \text{~not in tile demarcation}_i \\ \end{cases} 
\end{gather}

Since tiles within a tiling are non-overlapping, any state-action combination activates exactly $\mathcal{T}$ tiles, one per tiling. The total number of features is simply $\mathcalligra{T}~\tau^3$. Note that the tabular case can be recovered as a special case by setting $\mathcalligra{T}~ = 1$ and $\tau \leq m$. In this case, every tile is activated by at most one feasible state-action combination which is equivalent to storing a dedicated coefficient for every state-action combination.\footnote{If $\tau > m$, some tiles would never be activated. But again, every table entry would correspond to a unique tile.}

\textbf{TBD: This feature exhibits the advantage of function approximation in large state spaces.  dimensionality problem when increasing the exponent, or $\tau$ not so much when increasing the number of tilings}
