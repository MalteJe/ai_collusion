
\subsection{Feature Extraction}\label{feature_extraction}

\textbf{TBD: introduction}
As outlined in \autoref{value_approximation}, the state-action space contains just 3 variables. Assigning a single coefficient to each variable certainly fails to do justice to the complexity of the optimization problem. In particular, a \emph{reward-punishment} theme requires that actions are chosen conditional on past prices (i.e.\ the state space). Hence, it is imperative to consider interactions and non-linearities. Therefore, I utilize various methods to extract features form the state-action space.

In reinforcement learning, a common approach is to store a distinct set of coefficients for every feasible action.\footnote{In this case, the vector of coefficients contains \emph{m} times features components} This is a sensible approach with qualitative action spaces. However, very much like tabular learning, a separate set of coefficients neglects the (quasi-) continuous nature of prices. Therefore two issues arise. First, discretizing the action space doesn't scale well if the number of feasible prices increases. Consequently, learning requires relatively many periods with large $m$. Second, observing a particular reward may not only constitute an informative feedback for the particular action undertaken, but also for 'similar' prices. Using and updating coefficients valid for (a subset of) all feasible prices exploits this.

For this simulation, I use \emph{polynomials}, \emph{polynomial splines} and \emph{tile coding} to extract features from the state-action space.

\subsubsection{Polynomials}

\emph{Polynomial approximation} of order $k$ maps states and action to a set of features, where a single feature corresponds to:



\begin{gather}
x_i^{Poly} = p_{1, t-1}^{\kappa_1} ~ p_{2, t-1}^{\kappa_2} ~ p_{1, t}^{\kappa_3}
\end{gather}


Every combination of exponents that adheres to the restrictions

\begin{itemize}
	\item $0 < \kappa_1 + \kappa_2 + \kappa_3 \leq k$ and
	\item $\kappa_1, \kappa_2, \kappa_3 \in \{0, 1, ..., k\}$
\end{itemize}

constitutes one feature. Using polynomial approximation, the feature vector $\boldsymbol{x}$ contains ${k + 3\choose3}  - 1$ elements.

\subsubsection{Normalized Polynomials}

\textbf{TBD}

\subsubsection{Polynomial Splines}

\textbf{TBD}

\subsubsection{Tile Coding}

In reinforcement learning, \emph{Tile Coding} is a common way to extract linear, in fact binary, features from a state-action space.\footnote{for an extensive introduction with instructive illustrations refer to \textcite{sutton_reinforcement_2018}} The idea is that several \emph{tilings} superimpose the state-action space. The $\mathcalligra{T}$ \ tilings are offset but each tiling covers the entire state-action space:

\begin{gather}
	 \mathcal{T}^L \leq A^L  ~ \& ~ \mathcal{T}^U \geq A^U    \text{for } \mathcal{T} \in \{1, 2, ..., \mathcalligra{T} ~ \}
\end{gather}

Each tiling is itself composed of uniformly spaced out \emph{tiles}.\footnote{With 2 dimensions, a tiling simply corresponds to a grid. In our case, the state-action space is 3-dimensional, so it may prove more intuitive to think of cubes instead of tilings and tiles.} Every tile is uniquely demarcated by a lower and an upper threshold for every dimension. Consequently, the number of tiles per tiling is controlled by the number of thresholds. For this simulation, it suffices to define a single set of thresholds per tiling that applies to all 3 dimensions. More specifically, the thresholds are spaced out evenly in the tiling-specific interval $[\mathcal{T}^L, \mathcal{T}^U]$:

\begin{gather}
\mathcal{T} = (
\mathcal{T}^L,
\mathcal{T}^L + \frac{1(\mathcal{T}^U - \mathcal{T}^L)}{\tau},
\mathcal{T}^L + \frac{2(\mathcal{T}^U - \mathcal{T}^L)}{\tau}~ , ... , ~
\mathcal{T}^L + \frac{(\tau-1)(\mathcal{T}^U - \mathcal{T}^L)}{\tau},
\mathcal{T}^U)
\end{gather}

This gives rise to $\tau^3$ tiles per tiling. Tiles are binary, i.e.\ if a state-action observation falls into a particular demarcation, the corresponding tile is \emph{activated}:

\begin{gather}\label{tile_activation}
x_i^{Tiling} = \begin{cases}
1 & \quad \text{if } \{p_{1, t-1}, p_{2, t-1}, p_{1, t}\} \text{~in tile demaraction}_i  \\
0 & \quad \text{if } \{p_{1, t-1}, p_{2, t-1}, p_{1, t}\} \text{~not in tile demarcation}_i \\ \end{cases} 
\end{gather}

Since tiles within a tiling are non-overlapping, any state-action combination activates exactly $\mathcal{T}$ tiles, one per tiling. The total number of features is simply $\mathcalligra{T}~\tau^3$. Note that the tabular case can be recovered as a special case by setting $\mathcalligra{T}~ = 1$ and $\tau \leq m$. In this case, every tile is activated by at most one feasible state-action combination which is equivalent to storing a dedicated coefficient for every state-action combination.\footnote{If $\tau > m$, some tiles would never be activated. But again, every table entry would correspond to a unique tile.}

\textbf{TBD: This feature exhibits the advantage of function approximation in large state spaces.  dimensionality problem when increasing the exponent, or $\tau$ not so much when increasing the number of tilings}
