

\pagebreak
\begin{algorithm}
	\caption{Gradient Descend Expected SARSA}
	\begin{algorithmic}[]
		\small
		\STATE Initialize state S
		\WHILE{convergence is not achieved,}
		\STATE Choose action A \~{} $\pi(.|S)$
		\STATE observe profit $\pi$, adjust to reward $R$
		\STATE observe next state: $S_{t+1} = A_t$
		\STATE calculate TD-error: $\delta \leftarrow R +  \gamma \bar{V}(S_{t+1}) - q(S_t)$
		\STATE update eligibility trace: $\boldsymbol{z} \leftarrow \gamma \lambda \rho \boldsymbol{z} + \boldsymbol{x} $
		\STATE update parameter vector: $\boldsymbol{w} \leftarrow \boldsymbol{w} + \alpha  \delta  \boldsymbol{z}$
		\STATE $S \leftarrow S_{t+1}$
		\STATE $t \leftarrow t+1$
		\ENDWHILE
	\end{algorithmic}
\end{algorithm}
















\section{Results}
Starting, with the baseline specification, this section reports on the simulation results. \textbf{TBD}. To foreshadow the results, profits mostly exceed Nash-predictions, but remain below monopoly profits. While agents learn to charge supra-competitive prices, they fail to incorporate \emph{reward-punishment} schemes consistently. Overall, the results crucially hinge on the combination of feature extraction method and selected parameters. Only tabular learning exhibits a clear tendency to punish deviations with lower prices in subsequent periods.

I report results for various specifications and will refer to every unique combination of feature extraction method and parameters as an \emph{experiment}. Every experiment consists of 48 \emph{runs}, i.e. repeated simulations with the exact same set of starting conditions. Lastly, within the scope of a particular \emph{run}, time steps are called \emph{periods}.

\subsection{Convergence}

\textbf{TBD: As indicated,} convergence is not guaranteed in a non-stationary environment, much less so with function approximation. Notwithstanding the lack of a theoretical convergence guarantee, prior experiments have shown that simulation runs tend to approach a stable equilibrium in practice (\cite{calvano_artificial_2019} \textbf{and others}). Note that \emph{stability} simply refers to the observation that the same set of prices continuously recur over a longer time interval. The strategies upon convergence need not coincide with economic theory. In fact, at times the observed outcomes contradict predictions from game theory. For instance, despite symmetric profit functions, the converged outcomes may display asymmetric prices. Moreover, price cycles, i.e.\ a recurring sequence of price combinations, occur frequently.\footnote{The model from \autoref{quantity} predicts symmetric outcomes without cycles. This is typical for simultaneous pricing games, but not universal across economic models. For instance, collusive outcomes in quantity competition (i.e.\ Cournot) may exhibit price asymmetries. The relevance of that prediction has been fortified in experimental settings, e.g.\ in \textcite{fischer_collusion_2019}. \textcite{maskine-tirole} pioneer a sequential pricing game that predicts \emph{Edgeworth price cycles} where agents successively undercut each other until one firm prefers to reset the cycle and increases its price. Based on their model, \textcite{klein_autonomous_2019} shows that \emph{Q-Learning} agents are indeed capable of learning those dynamic strategies.}


The following, arbitrary but practical, convergence rule was employed. If a price cycle recurred for the last 10,000 episodes, the algorithm is considered \emph{converged}. For efficiency reasons, a check for convergence is undertaken only every 2,000 episodes. Price cycles up to a length of 10 are considered. If no convergence is achieved until 500,000 episodes, the simulation stops and the run is deemed \emph{not converged}. There are a number of runs that \emph{failed to complete} as a consequence of the program running into an error. Unfortunately, the program code does not allow to examine the exact cause of such occurrences in retrospect. However, by and large, the failed runs occurred with unsuitable specifications (see below for a detailed discussion).

In accordance with the outlined convergence criteria above, \autoref{converged} displays the share of runs that, respectively, converged successfully, did not converge until the end of the simulation or failed to complete. Two main conclusions emerge. First, failed runs are mainly prevalent in specifications with a high value of $\alpha$ in conjunction with a polynomial feature method. Second, the tiling methods are more likely to converge.

\begin{figure}
	\includegraphics[width=\linewidth]{plots/converged.png}
	\caption{Number of runs that achieved convergence per experiment.}
	\label{converged}
\end{figure}

Regarding the failed runs, \textbf{recall} from \autoref{feature_extraction} that features of polynomial extraction are not binary and warrant cautious adjustments of the coefficient vector. I suspect that with unreasonably large values of $\alpha$, the estimates of $\boldsymbol{\theta}$ overshoot early in the simulation, don't recover and at some point exceed the software's numerical limits.\footnote{Controlled runs where I could carefully monitor the development of the coefficient vector $\boldsymbol{w}$ seem to confirm the hypothesis. However, isolated errors \emph{with} reasonable parameter settings remain unexplained.} While imoportant to acknowledge, the failed runs are largely an artifact of unreasonable specifications and I will \textbf{disregard them for the remainder of this chapter}. For instance, the percentages in the subsequent paragraph don't account for the failed runs.

Focusing on the completed runs, 95.4\% of runs did converge. But there are subtle differences between feature extraction methods. With only one exception, both tiling methods converged consistently for various $\alpha$. With 85.4\% of converged runs, separate polynomials constitute the other extreme. The figure indicates that convergence becomes less likely for low values of $\alpha$. With tabular learning, 92.9\% of runs converged without clear relation to different values of $\alpha$.

\autoref{convergence_at} displays a frequency polygon of the runs that achieved convergence within 500,000 episodes. Clearly, the distribution is fairly uniform across feature extraction methods. Most runs converged between 200,000 and 300,000 runs. This is an artifact of the decay in exploration as dictated by $\beta$. Before the focal point of 200,000 is reached, agents probabilistically experiment too frequently to observe 10,000 consecutive episodes without any deviation from the learned strategies. Thereafter, it becomes increasingly likely that both agents keep \emph{exploiting} their current knowledge and continuously play the same strategy for a sufficiently long time to trigger the convergence criteria. Note that the low quantity of runs converging between 300,000 and 500,000 suggests that increasing the maximum of allowed episodes would not necessarily entail a significantly higher portion of converged runs.

\begin{figure}
	\includegraphics[width=\linewidth]{plots/convergence_at.png}
	\caption{timing of convergence, runs that did not converge or failed to complete are excluded. Width of bins: 8,000}
	\label{convergence_at}
\end{figure}

\autoref{cycle_length} visualizes the distribution of cycle length and offers some interesting insights. Unsurprisingly, a first glance suggests that the frequency of runs decreases with cycle length. Not accounting for differences between selection methods, the bars appear similar to a geometric distribution with the largest bar corresponding to a 'cycle length of 1' (i.e.\ no cycle at all). Moving towards the right, the frequency of observed runs decreases with cycle length, though at a decreasing pace. In fact, there are even 7 runs with the maximum cycle length of 10. Interestingly, there are substantial differences between the different feature extraction methods. Polynomial tiling follows the described decaying pattern. Similarly, simple tile coding rarely converges in long cycles, though its spike of 194 runs corresponds to a cycle length of 2. Contrary, almost all runs of the separated polynomials converged without cycles.\footnote{Though barely visible with the naked eye, there are 2 runs with a cycle length of 2.}. Lastly, the frequency of cycle length of converged tabular runs is distributed almost uniformly. This observation also suggests that the employed convergence rule may well have misclassified some of the runs in the top left panel of \autoref{converged} as \emph{not converged} where in reality the convergence cycle length simply exceeded the arbitrary threshold of 10. 

\begin{figure}
	\includegraphics[width=\linewidth]{plots/cycle_length.png}
	\caption{Number of converged runs with particular cycle length.}
	\label{cycle_length}
\end{figure}

\textbf{price range of cycles}
* price range 
* price changes within a cycle?
* profits as a function of cycle length

\subsection{Profits}

In order to assess the simulation results in more detail, I normalize profits similar to \textcite{calvano_algorithmic_2018}:

\begin{gather}
\Delta = \frac{\bar{\pi} - p_n}{p_m - p_n}.
\end{gather}

$\bar{\pi}$ represents profits averaged over the last 100 time steps upon convergence and over both firms in a single run\footnote{Instead of looking just at the convergence profits, I average over the last 100 time steps to account for price cycles}. The normalization implies that $\Delta = 0$ and $\Delta = 1$ respectively reference the Nash and monopoly solution. Note that it is possible to obtain a $\Delta$ below $0$ (e.g. if both agents charge prices equal to marginal costs), but not above $1$. \autoref{alpha} displays the convergence profits as a function of the feature extraction method and $\alpha$. Every data point represents one experiment, more specifically the mean of $\Delta$ across all runs.

\begin{figure}
	\includegraphics[width=\linewidth]{plots/alpha.png}
	\caption{average $\Delta$ for various combinations of the feature selection method and $\alpha$. Beware the logarithmic x-scale.}
	\label{alpha}
\end{figure}

It is noteworthy that average profits consistently remain between both benchmarks $p_m$ and $p_n$ across specifications.


As choosing a sensible $alpha$ clearly depends on the feature extraction method, I will choose a different \emph{optimized} $\alpha$ to present the further results. Specifically, the considered experiments are:

\begin{enumerate}
	\item tabular: $\alpha = 0.1$
	\item tiling: $\alpha = 0.001$
	\item poly-separated: $\alpha = 10^{-6}$
	\item poly-tiling: $\alpha = 10^{-8}$
\end{enumerate}

\textbf{TBD: explanation why exactly these values?}

\autoref{all_runs} displays the development of profits and prices of all runs for the 'optimized' $\alpha$'s. Both metrics are averaged over 50,000 episodes apiece and over both players. Again, note that, by and large, prices and profits remain within the benchmarks of Nash competition and the cartel case.

\begin{figure}
	\includegraphics[width=\linewidth]{plots/all_runs.png}
	\caption{all runs for manually optimized $\alpha$}
	\label{all_runs}
\end{figure}



\clearpage
\subsection{Deviations}

This section examines whether the learned strategies are stable in the face of deviations. As outlined before, collusion requires a \emph{reward punishment scheme} and it seems instructive to assess whether the agents learned to punish deviations. In order to scrutinize that, I had one agent deviate from the stable price cycle by playing the short-term best response to maximize own profits \emph{after} convergence was detected. Subsequently, both agents played the learned strategies again for 10 episodes. For the period of that intervention, learning and exploration was turned off.

\autoref{average_intervention} displays the average price trajectory around the manually imposed deviation. It exhibits clear differences between the considered feature extraction methods. 

\begin{figure}
	\includegraphics[width=\linewidth]{plots/average_intervention.png}
	\caption{average price trajectory around deviation}
	\label{average_intervention}
\end{figure}

As the average price trajectory might hide subtle differences between runs even within the same experiment, \autoref{intervention_violin} displays the distribution of at and after the deviation.


\begin{figure}
	\includegraphics[width=\linewidth]{plots/intervention_violin.png}
	\caption{distribution of prices at and after deviation relative to prior equilibrium. Every violin has the same maximum width, i.e.\ width between violins are not comparable. Tails are trimmed}
	\label{intervention_violin}
\end{figure}

\pagebreak
\subsection{responses off equilibrium}

\textbf{TBD}



