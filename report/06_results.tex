
\section{Results}
This section reports on the simulation outcomes of the baseline specification. To foreshadow the results, profits mostly exceed Nash-predictions, but remain below monopoly profits. While agents learn to charge supra-competitive prices, they fail to incorporate \emph{reward-punishment} schemes consistently. Overall, the results crucially hinge on the combination of feature extraction method and selected parameters. Only tabular learning exhibits a clear tendency to punish deviations with lower prices in subsequent periods.

I report results for various specifications and will refer to every unique combination of feature extraction method and parameters as an \emph{experiment}. Every experiment consists of 48 \emph{runs}, i.e. repeated simulations with the exact same set of starting conditions. Lastly, within the scope of a particular \emph{run}, time steps are called \emph{periods}.

\subsection{Convergence}\label{convergence}

\textbf{TBD: As indicated,} convergence is not guaranteed in a non-stationary environment, much less so with function approximation. Notwithstanding the lack of a theoretical convergence guarantee, prior experiments have shown that simulation runs tend to approach a stable equilibrium in practice (\cite{calvano_artificial_2019} \textbf{and others}). Note that \emph{stability} simply refers to the observation that the same set of prices continuously recur over a longer time interval. The strategies upon convergence need not coincide with economic theory. In fact, at times the observed outcomes in this study contradict predictions from game theory. For instance, despite symmetric profit functions, the converged outcomes may display asymmetric prices. Moreover, price cycles, i.e.\ a recurring sequence of price combinations, occur frequently.\footnote{The model from \autoref{quantity} predicts symmetric outcomes without cycles. This is typical for simultaneous pricing games, but not universal across economic models. For instance, collusive outcomes in quantity competition (i.e.\ Cournot) may exhibit price asymmetries. The relevance of that prediction has been fortified in experimental settings, e.g.\ in \textcite{fischer_collusion_2019}. \textcite{maskine-tirole} pioneer a sequential pricing game that predicts \emph{Edgeworth price cycles} where agents successively undercut each other until one firm prefers to reset the cycle and increases its price. Based on their model, \textcite{klein_autonomous_2019} shows that \emph{Q-Learning} agents are indeed capable of learning those dynamic strategies.}


The following, arbitrary but practical, convergence rule was employed. If a price cycle recurred for 10,000 consecutive episodes, the algorithm is considered \emph{converged} and the simulation concludes. A price cycle requires both agents' adherence.\footnote{Of course it is possible that the cycle length differs between agents. For instance, one agent may continuously play the same price while the opponent keeps alternating between two prices. In this case, the cycle length is $1*2=2$.}

For efficiency reasons, price cycles up to a length of 10 are considered and a check for convergence is undertaken only every 2,000 episodes. If no convergence is achieved until 500,000 episodes, the simulation stops and the run is deemed \emph{not converged}. Furthermore, there are a number of runs that \emph{failed to complete} as a consequence of the program running into an error. Unfortunately, the program code does not allow to examine the exact cause of such occurrences in retrospect. However, by and large, the failed runs occurred with unsuitable specifications (see below for a detailed discussion).

\begin{figure}
	\includegraphics[width=\linewidth]{plots/converged.png}
	\caption{Number of runs per experiments that achieved convergence as a function of $\alpha$.}
	\label{converged}
\end{figure}

In accordance with the outlined convergence criteria above, \autoref{converged} displays the share of runs that, respectively, converged successfully, did not converge until the end of the simulation or failed to complete. Two main conclusions emerge. First, failed runs are mainly prevalent in specifications with a high value of $\alpha$ in conjunction with a polynomial feature method. Second, the tiling methods are more likely to converge. Both points deserve some further exploration.

Regarding the failed runs, \textbf{recall} from \autoref{feature_extraction} that features of polynomial extraction are not binary and warrant cautious adjustments of the coefficient vector. I suspect that with unreasonably large values of $\alpha$, the estimates of $\boldsymbol{w}$ overshoot early in the simulation, don't recover and at some point exceed the software's numerical limits.\footnote{Controlled runs where I could carefully monitor the development of the coefficient vector $\boldsymbol{w}$ seem to confirm the hypothesis. However, isolated errors \emph{with} reasonable parameter settings remain unexplained, see the top right panel in \autoref{converged}.} While imoportant to acknowledge, the failed runs are largely an artifact of unreasonable specifications and I will \textbf{disregard them for the remainder of this chapter}. For instance, the percentages in the subsequent paragraph don't account for the failed runs.

Out of the completed runs without program failure, 95.4\% did converge. Though there are subtle differences between feature extraction methods. With only one exception, both tiling methods converged consistently for various $\alpha$. With only 85.4\% of runs converging, separate polynomials constitute the other extreme. The figure also indicates that convergence becomes less likely for low values of $\alpha$. With tabular learning, 92.9\% of runs converged without clear relation to different values of $\alpha$.

\autoref{convergence_at} displays a frequency polygon of the runs that achieved convergence within 500,000 episodes. Clearly, the distribution is fairly uniform across feature extraction methods. Most runs converged between 200,000 and 300,000 runs. This is an artifact of the decay in exploration as dictated by $\beta$. Before the focal point of 200,000 is reached, agents probabilistically experiment too frequently to observe 10,000 consecutive episodes without any deviation from the learned strategies. Thereafter, it becomes increasingly likely that both agents keep \emph{exploiting} their current knowledge and continuously play the same strategy for a sufficiently long time to trigger the convergence criteria. Note that the low quantity of runs converging between 300,000 and 500,000 suggests that increasing the maximum of allowed episodes would not necessarily entail a significantly higher portion of converged runs.

\begin{figure}
	\includegraphics[width=\linewidth]{plots/convergence_at.png}
	\caption{timing of convergence, runs that did not converge or failed to complete are excluded. Width of bins: 8,000}
	\label{convergence_at}
\end{figure}

\autoref{cycle_length} visualizes the distribution of cycle length and offers some interesting insights. Unsurprisingly, a first glance suggests that the frequency of runs decreases with cycle length. Not accounting for differences between selection methods, the bars appear similar to a geometric distribution with the largest bar corresponding to a 'cycle length of 1' (i.e.\ no cycle at all). Moving towards the right, the frequency of observed runs decreases with cycle length, though at a decreasing pace. In fact, there are even 7 runs with the largest considered cycle length of 10.

Again, there are substantial differences between the different feature extraction methods. Polynomial tiling largely follows the described decaying pattern. Similarly, simple tile coding rarely converges in long cycles, though its spike of 194 runs corresponds to a cycle length of 2. Contrary, almost all runs of the separated polynomials converged without cycles.\footnote{Though barely visible in \autoref{convergence_at}, there are 2 runs with a cycle length of 2.}. Lastly, the frequency of cycle length of converged tabular runs is distributed almost uniformly. This observation also suggests that the employed convergence rule may well have misclassified some of the runs in the top left panel of \autoref{converged} as \emph{not converged} where in reality the convergence cycle length simply exceeded the threshold arbitrarily set at 10. 

\begin{figure}
	\includegraphics[width=\linewidth]{plots/cycle_length.png}
	\caption{Number of converged runs with particular cycle length.}
	\label{cycle_length}
\end{figure}

\autoref{prices} unveils the ranges of prices within a cycle. For now, I proceed by examining profits upon convergence.

\subsection{Profits}

In order to benchmark the simulation profits, I normalize profits similar to \textcite{calvano_algorithmic_2018}:

\begin{gather}
\Delta = \frac{\bar{\pi} - p_n}{p_m - p_n} ~~ \text{,}
\end{gather}

where $\bar{\pi}$ represents profits averaged over the last 100 time steps upon convergence and over both agents in a single run. The normalization implies that $\Delta = 0$ and $\Delta = 1$ respectively reference the Nash and monopoly solution. Note that it is possible to obtain a $\Delta$ below $0$ (e.g. if both agents charge prices equal to marginal costs), but not above $1$.\footnote{Strictly speaking, exactly 1 is not attainable either. Recall that $m$ was chosen to allow for prices very close, but not equal to both benchmark prices. With $m = 19$, the highest feasible $\Delta$ is 0.9997.} \autoref{alpha} displays the convergence profits as a function of the feature extraction method and $\alpha$. Every data point represents one experiment, more specifically the mean of $\Delta$ across all runs making up the experiment.

\begin{figure}
	\includegraphics[width=\linewidth]{plots/alpha.png}
	\caption{average $\Delta$ for various experiments. Includes converged and non-converged runs. One data point (poly tiling, $\alpha = 0.0004$) is excluded for better presentability. Beware the logarithmic x-scale.}
	\label{alpha}
\end{figure}

First of all, note that average profits consistently remain between both benchmarks $p_m$ and $p_n$ across specifications.\footnote{There is one exception. One data point is hidden in the plot to preserve reasonable y axis limits. More specifically, for the polynomial tiles and $\alpha = 0.0001$, the average $\Delta$ is -1.73. This extends the observation in \autoref{convergence}. It appears that this particular $\alpha$ constitutes a critical point. While the program does not crash, agents only learn strategies void of any reasonableness. As \autoref{alpha} displays, outcomes within the benchmarks are obtained by further decreasing $\alpha$.} As with prior results, the plot unveils salient differences between feature extraction methods.  On average, polynomial tiling runs yield the highest profits. The average $\Delta$ peaks at 0.85 for $\alpha = 10^{-8}$. Higher values of $\alpha$ tend to progressively decrease profits. Moving downwards on the y-axis, both the tabular method and tile coding yield similar average values of $\Delta$. Furthermore, the level of $\alpha$ does not seem to impact $\Delta$ much. For both methods $\alpha = 10^{-4}$ induces the highest average $\Delta$ at 0.487 and 0.478 respectively. Similarly for separated polynomials, $\Delta$ does not seem to respond to variations in $\alpha$. The maximum $\Delta$ is 0.35.

Naturally, averaging $\Delta$ over runs as in \autoref{alpha} potentially hides subtleties in the distribution of $\Delta$. Therefore, \autoref{alpha_violin} displays a violin plot that shows the distribution of $\Delta$ per experiment. The distribution largely confirms the conclusion that most runs converge between $\Delta_m$ and $\Delta_n$. The only method with a significant quantity of runs with profits below the Nash benchmark are the separated polynomials. Overall, 19.8\% of runs converged with profits below the Nash equilibrium, though most of them ended up reasonably close. The percentage is largest for the experiment with $\alpha = 10^{-8}$: 27.1\%. While the other methods tend to elicit runs within the set up benchmarks, the variability remains quite high. This indicates a degree of path dependence and suggests that the algorithms are prone to stick to early explored strategies that are \emph{above-average}, but \emph{sub-optimal}. Polynomial tiles exhibit the narrowest $\Delta$ range, in particular for low $\alpha$.

\begin{figure}
	\includegraphics[width=\linewidth]{plots/alpha_violin.png}
	\caption{distribution of $\Delta$ for various experiments. Includes converged and non-converged runs. Violin widths are scaled to maximize width of single violins, comparisons of widths between violins are not meaningful. Violins are trimmed at smallest and largest observation respectively. One violin (poly tiling, $\alpha = 0.0004$) is excluded for better presentability. Horizontal lines represent the median. Beware the logarithmic x-scale.}
	\label{alpha_violin}
\end{figure}


\autoref{convergence} and \autoref{alpha} established that, what constitutes a sensible value of $\alpha$ clearly depends on the feature extraction method. Hence, for the remainder of this chapter, I will select an 'optimal' $\alpha$ for every feature extraction method and present further results only for these combinations. In determining \emph{optimality} of $\alpha$, I don't rely on a single hard criteria, rather I consider a number of factors including the percentage of converged runs, comparability with previous studies and prefer to select experiments with high average $\Delta$ as they are most central to the purpose of this study. \autoref{justifications} provides a justification for every experiments setting deemed \emph{optimal}. To get a sense of the variability of runs within the experiments and the price trajectory over time, \autoref{trajectory_Delta} displays the development of profits of all runs for the optimal values of $\alpha$. Moreover, \autoref{appendix} contains further trajectory visualizations of prices and profits.

\begin{center}
	\begin{table}
		
		\begin{tabular}{|l|c|l|}
			\hline
			\textbf{Feature Extraction Method}&$\boldsymbol{\alpha}$&\textbf{justification} \\
			\hline
			Tabular&0.1&- comparability with previous simulation studies \\
			&&- most pronounced response to price deviations \\
			&& \ \ (see \autoref{deviations}) \\
			\hline
			Tile Coding&0.001&- high $\Delta$ \\
			&&- most pronounced response to price deviations \\
			&&\ \ (see \autoref{deviations}) \\
			\hline
			Separated Polynomials&$10^{-6}$&- high percentage of converged runs \\
			\hline
			Polynomial Tiles&$10^{-8}$&- high $\Delta$ \\
			\hline
		\end{tabular}
		\caption{\emph{Optimized} values of $\alpha$ by feature extraction method}
		\label{justifications}
	\end{table}
\end{center}


\begin{figure}
	\includegraphics[width=\linewidth]{plots/trajectory_Delta.png}
	\caption{distribution of $\Delta$ over time in 'optimized' experiments. For individual runs, $\Delta$ is averaged over 50,000 periods apiece and both players. Plot includes converged and non-converged runs. Violin widths represent quantity of active runs at $t$ which enables comparisons between violins. As most runs converge after 200,000 to 300,000 episodes, violin widths decrease thereafter. Violins are trimmed at smallest and largest observation respectively. Horizontal lines represent the median.}
	\label{trajectory_Delta}
\end{figure}

\textbf{paragraph on differences between player profits?}

\subsection{Price Ranges}\label{prices}

As established in \autoref{convergence}, many simulations converge in price cycles of various lengths. \autoref{price_range} plots the range between the lowest and highest price a single agent charges in a cycle upon convergence. Naturally, the price range is null if no cycle is present. Perhaps unsurprisingly, the price range then tends to increase with cycle length, at times becoming remarkably high. The range of prices due to tabular learning frequently exceeds the range between collusive and Nash prices. This is a clear indication that price setting is occasionally irrational. Irrespective of agents competing or colluding, prices outside this range are not economically optimal. Recall from \autoref{convergence} that other feature extraction methods tend to yield lower cycle lengths. \autoref{price_range} extends that observation with the insight that those methods also produce lower price ranges. However, the inversion of the previous argument is dangerous. One should not deduct that behavior is \emph{closer to optimal} from the mere fact that prices appear more stable. In fact, the next section shows that the strategies learned with function approximation are often far from optimal and easy to exploit. Potential consequences of frequent price changes of significant magnitude are discussed \textbf{in section X}.

\begin{figure}
	\includegraphics[width=\linewidth]{plots/price_range.png}
	\caption{Price range of individual runs. Every point represents a particular run. Within groups, points are spaced out horizontally. Price range is defined as the difference between the highest and lowest price an agent charges within a cycle. Relationships to the opponent's prices are not examined. Only converged runs are considered (as cycle length is unavailable for other runs). Dashed line represents the difference between collusive and Nash outcome (i.e.\ $p_m - p_n$).}
	\label{price_range}
\end{figure}


\subsection{Deviations}\label{deviations}

This section examines whether the learned strategies are stable in the face of deviations from the learned behavior. There are at least two explanations for the existence of supra-competitive outcomes. First, agents simply fail to learn how to compete effectively and miss out on opportunities to undercut their opponent. Second, agents avoid deviating from the stable strategy because they fear retaliation and lower (discounted) profits in the long run. \textbf{As outlined before}, only the latter cause, some form of a \emph{reward punishment scheme}, allows to label the supra-competitive outcomes as \emph{collusive} and warrants attention from competition policy \parencite{assad_algorithmic_2020}. Therefore, the following \emph{deviation experiment} was conducted to scrutinize whether agents learn to actually punish deviations. Denote the period in which convergence was detected as $\tau = 0$. At this point, both agents played for 10,000 episodes an equilibrium strategy they mutually regard as optimal (on path). At $\tau = 1$, I force one agent to deviate from her learned strategy and play instead the short-term best response that mathematically maximizes profits. Subsequently, she reverts again to the learned strategy. In order to verify whether the non deviating agent proceeds to punish the cheater, he sticks to his learned behavior throughout the deviation experiment. The \emph{deviation experiment} lasts 10 episodes in total. Learning and exploration are disabled (i.e.\ $\alpha = \epsilon = 0$).\footnote{See \textbf{section TBD} for prolonged deviations and continued learning \emph{after} detected convergence.} In order to evaluate the deviation, it appears useful to define a \emph{counterfactual} situation where both agents stick to their learned strategies. Comparing (discounted) profits between the experiment and the counterfactual allows to assess the profitability of the deviation.

\textbf{words on punishment strategies: grim trigger vs. slow reversal }

\begin{figure}
	\includegraphics[width=\linewidth]{plots/average_intervention.png}
	\caption{Average price trajectory around deviation. }
	\label{average_intervention}
\end{figure}

As the responses to one agent's deviation vastly differ across feature extraction methods, it is natural to discuss them separately at first and contrast differences only thereafter.  It is difficult to summarize all information in a single graph or table, so I will consult \autoref{average_intervention}, \autoref{intervention_boxplot} and \autoref{share_deviation_profitability} simultaneously to describe the deviation and response patterns. Before that, a brief description of the plots is in line. \autoref{average_intervention} displays the price trajectory around the forced deviations averaged over all runs of an experiment.\footnote{To reiterate the result from the previous section, the plot reinforces that the price variation between periods is non-negligible even \emph{before} the deviation takes place - despite averaging over all runs of the optimal experiments.}  Since the average price trajectory might veil important differences between runs, \autoref{intervention_boxplot} illustrates the range of deviation and punishment prices compared to the counterfactual price that would have materialized if no deviation had taken place and agents kept following their learned strategies.\footnote{\autoref{intervention_profit_boxplot} in \autoref{appendix} shows a similar plot for profits.} Note that in the presence of price cycles, part of the variation can be explained by \emph{cycle shifting}, a phenomenon where the agents return to the learned cycle but the intervals are not aligned with the counterfactual path. These differences should even out over runs and thus, not systematically bias the boxes in either direction. Similarly, the average price response in \autoref{average_intervention} is largely unaffected by this phenomenon. Finally, \autoref{share_deviation_profitability} reports the share of deviations that turned out to be profitable compared to the counterfactual.\footnote{\autoref{intervention_profitability_polygon} in \autoref{appendix} displays a more detailed frequency polygon to gauge how much more or less profitable the deviation is compared to the counterfactual of sticking to the learned strategy.}

\begin{figure}
	\includegraphics[width=\linewidth]{plots/intervention_boxplot.png}
	\caption{distribution of prices at and after deviation relative to alternative path \emph{without} forced deviation, i.e.\ the difference to the price at the same $\tau$ had no deviation taken place. Only includes converged runs because a clear counterfactual exists. Boxes demarcate 15th and 85th percentiles and are extended by whiskers that mark the entire range of price differences. Horizontal lines represent the group median.}
	\label{intervention_boxplot}
\end{figure}

Most importantly, only tabular learning evokes a conspicuous punishment from the non deviating agent. At $\tau = 2$, the non deviating agent tends to match, arguably even undercut, the deviation price whereas the deviating agent already begins reverting to pre-deviation prices. Though this result's general validity is qualified. \autoref{intervention_boxplot} unveils that the non deviating agent does not always reduce prices compared to the counterfactual. Despite the existence of punishment prices in some runs, agents are fairly quick to return to the price levels observed before the deviation was forced upon them. As early as $\tau = 4$ there is no visible difference between average pre- and post-deviation price levels.\footnote{Previous studies showcase a strong deviation is usually followed by a more gradual reversion to pre-deviation behavior (around 5-10 episodes), see in particular Figure 4 in \textcite{calvano_algorithmic_2018} and Figure 3 in \textcite{klein_autonomous_2019}.} This might partly follow from prices being relatively close to the Nash equilibrium in the first place. The punishments ensure that deviating is (strictly) profitable in only 18\% of runs. This suggests that, upon convergence, agents stick to a stable equilibrium, from which deviations tend to be unprofitable due to the cheated agent retaliating.

\begin{center}
	\begin{table}
		% latex table generated in R 3.6.1 by xtable 1.8-4 package
% Thu May 27 09:48:14 2021
\begin{tabular}{llrr}
  \hline
feature\_method & agent & share profitable & share unprofitable \\ 
  \hline
tabular & deviating & 0.18 & 0.56 \\ 
  tabular & non deviating & 0.04 & 0.69 \\ 
  tiling & deviating & 0.56 & 0.33 \\ 
  tiling & non deviating & 0.06 & 0.83 \\ 
  poly-separated & deviating & 0.72 & 0.00 \\ 
  poly-separated & non deviating & 0.02 & 0.70 \\ 
  poly-tiling & deviating & 0.96 & 0.02 \\ 
  poly-tiling & non deviating & 0.00 & 0.98 \\ 
   \hline
\end{tabular}

		\caption{Share of profitable and non-profitable deviations by agent and feature extraction method. Deviations are deemed \emph{profitable} if the discounted profits until $\tau = 10$ due to the deviation exceed cash flows from a counterfactual without deviation. Only includes converged runs because a clear counterfactual exists. Discounting is equivalent paramount to $\gamma$ in \autoref{td_error_expected}, i.e.\ 0.95. A significant number of 'deviations' are neither profitable nor unprofitable. In those runs, the learned strategy of the deviating agent is actually the best response at $\tau = 1$ and both agents keep following their respective price cycle.}
		\label{share_deviation_profitability}
	\end{table}
\end{center}

When examining the outcomes of the other feature extraction methods, different conclusions emerge. Recall from \autoref{alpha} that tile coding yielded convergence profits very similar to tabular learning. Yet, \autoref{average_intervention} and \autoref{intervention_boxplot} only hint at slight punishments in some runs. In fact, the median of the cheated agent's price at $\tau = 2$ is exactly 0, which amounts to a complete absence of a response. This lack of punishment renders 56\% of the cheater's deviations profitable. In light of that, it is surprising the cheating agent tends to return to pre-intervention price levels instead of continuing to exploit her opponent's failure to punish deviations.

This is even more true for the feature extraction method utilizing separate polynomials. \autoref{convergence} outlined that this method typically did not converge in price cycles but a single, continuously played price instead. (In that sense, the opponent's behavior is more predictable and deviations easier to undertake.) The deviation impulse and responses are easy to summarize. In all runs, after the forced intervention at $\tau = 1$, both agents immediately return to the pre-deviation equilibrium. This is remarkable for two reasons. First, the non deviating agent completely fails to punish the cheater's behavior and does not respond to the price cut whatsoever. Consequently, 72\% of the deviations are profitable.\footnote{The remaining 31\% comprise runs where the deviating agent was already playing the short-term best response. Remember that the separated polynomial method tends to result in prices at or close to the Nash equilibrium.} This leads to the second point. Despite the obvious advantage of cheating, the deviating agent returns to the pre-deviation price without exception, thus failing to exploit her opponent's weakness. To be fair, the initial price levels are fairly close to the Nash equilibrium and the deviation's profitability is relatively small compared to the potential gains realizable in other experiments (see also \autoref{intervention_profitability_polygon} in \autoref{appendix}). Still, it is puzzling that such a simple strategy improvement remains consistently untapped. In conclusion, the agents' failure to play economically sound strategies casts doubts on the viability of the feature extraction method in reality.

Finally, turn your attention to the experiment with polynomial tiling. Recall that this experiment generated outcomes closest to perfect collusion. By and large, the deviation experiment for polynomial tiling is similar to the one with separated polynomials, but there are some variations between runs that warrant detailed examination. Consider first the non deviating agent. Again, the majority of runs exhibits a failure to respond to the price cut. However, selected runs show a \emph{matching} strategy where the cheated agent meets the price cut with a similar price. Notably, in those circumstances, agents \emph{do not return} to the previously learned path but quickly establish a new equilibrium. Moreover, note that \autoref{intervention_boxplot} displays a slight bias downwards over all periods. This is indicative of \emph{continued cheating} of the deviating agent. After being forced to undercut the price, she proceeds to set prices below pre-deviation levels without getting punished. This, too, results in a new equilibrium. \autoref{intervention_poly_tiling} in \autoref{appendix} illustrates both phenomena (price matching and continued cheating) through the exact price sequence of exemplary runs. In light of high pre-deviation prices and the lack of retaliatory prices, it is unsurprising that 96\% of deviations are profitable. The conclusions for polynomial tiling are similar to the separated polynomials. Baring a few exceptions, the non deviating agent fails to respond to a price cut and is easy to exploit. On the other hand, the deviating agent tends to leave that weakness unexploited.
Overall the deviation exercise suggests that while algorithmic agents manage to sustain high prices when playing each other, their strategies are incomplete and easy to exploit.

\textbf{summary of all methods}

Evidently, under the regime of this study's simulations, tabular learning is better in producing stable supra-competitive outcomes than the functional approximation methods. To illustrate the notion of \emph{stability} in this context, consider the following thought experiment of a \emph{superagent}. Upon convergence, a rational player with perfect information on the economic environment and the learned best responses of both agents enters the game and takes over pricing authority from one of them.\footnote{For the sake of the argument it is irrelevant whether this superagent is human or not.} Importantly, the superagent could anticipate the opponent's price in the next period and calculate the short-term maximizing response as well as the opponent's reaction to the deviation and so on. When playing against a tabular learning agent, the superagent would deliberately stick to the convergence pricing scheme as a cheating surely evokes a retaliation rendering a deviation unprofitable (see \autoref{share_deviation_profitability}). Contrary, when facing an opponent who learned utilizing a functional approximation method, the superagent could easily cheat on the opponent to increase short-term profits without being punished in subsequent periods.

The evidence also suggests that functional approximation methods create hesitancy in the agents to change best responses. Probabilistically, \emph{exploration} ensures that both agents will undercut the price of their opponent and realize excess profits similar to those in the forced deviation experiment. However, it appears that agents fail to learn (enough) from such \emph{explored cheating}. As evidenced by the undertaken deviation experiment, they typically return to the prior price (cycle) immediately. This rigidity in adjusting strategies potentially points to a problem with the specific algorithm or the tuning of its parameters. For instance, a higher $\alpha$ could enable the cheater to learn faster that an unpunished deviation is more profitable than adhering to the learned strategy. 

Recall that learning and exploration were turned off for the deviation experiment. This enables an objection to the presented results. The non deviating agent, stripped of his ability to adjust his strategy, might only be exploitable for a finite number of episodes until he adjusts his strategy. In fact, a generosity to condone isolated price cuts might be conducive to establishing high price levels early in the simulation runs. However, \autoref{prolonged deviation} demonstrates that the lack of punishment in response to a deviation remains ubiquitous in prolonged deviation experiments with enabled learning ($\alpha > 0$).

\subsection{responses off equilibrium}

\textbf{TBD}



