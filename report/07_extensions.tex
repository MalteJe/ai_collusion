\section{Robustness and variations}\label{robustness}

\begin{enumerate}
	\item vary $\beta$
	\item vary discount factor
	\item vary $lambda$
	\item vary price range
	\item vary number of prices ($m$)
	\item other algorithms
	\item average reward setting
\end{enumerate}

\subsection{Algorithm Variations}\label{vary_algorithm}

\subsubsection{On Policy}

Note that $\delta_t$ can only be calculated after the action in the next period has been taken.\footnote{}

\begin{gather}
	\delta_t^{SARSA} = \pi_t + \gamma \hat{q}(S_{t+1}, A_{t+1}, \boldsymbol{w}) - \hat{q}(S_t, A_t, \boldsymbol{w})
\end{gather}
		

\subsection{Differential Reward Setting}

\begin{gather}
\delta_t = r_t - \widetilde{R}_{t-1} + \hat{q}(S_{t+1}, A_{t+1}, \boldsymbol{w}_t) - \hat{q}(S_t, A_t, \boldsymbol{w}_t) ~~   \text{,}
\end{gather}

$\widetilde{R}_{t-1}$ is a (weighted) average reward

\emph{ex post} \emph{differential} profit $\pi_t - \widetilde{R}_{t-1}$ in conjunction with the estimated value of the newly arising state-action combination in $t+1$

While they come with a meaningful economic interpretation, \textcite{sutton_reinforcement_2018} and \textcite{naik_discounted_2019} show that their use is inappropriate in infinite sequences with function approximation settings. Moreover, a policy maximizing average rewards is equivalent to a policy maximizing the average of discounted future values - irrespective of the particular discount factor.

Surprisingly, this system 
* discount factor can be = 1




\pagebreak