\section{Conclusions}\label{conclusions}

I argued that despite the success of Q-Learning in achieving collusion in repeated games of price competition, tabular learning methods might face practical challenges when applied to real markets. Therefore, I developed three linear function approximation methods that scale better with the learning task's complexity and combined them with suitable reinforcement learning algorithms. To assess their merit, I deployed them to a simultaneous pricing environment and compared their performance to tabular learning.

The simulations have shown that all \gls{fem}s tend to converge in supra-competitive profits as long as parameter specifications remain reasonable.  As is shown in other studies, tabular learning agents acquire truly collusive strategies and show a stubborn resilience to return to the convergence equilibrium after episodes of forced deviation. On the contrary, the convergence equilibra arising in simulations with function approximation \gls{fem}s are not supported by a reward-punishment scheme. I show in various deviation exercises that agents have not learned to systematically punish price cuts. Thus, the simulation \emph{failed to provide evidence of collusion with function approximation \gls{fem}s}. Furthermore, despite the obvious lack of a credible deterrent, deviating agents are unable to exploit that weakness and return to pre-deviation prices. This is clear evidence of irrational behavior on the side of the agents. These findings also apply when using different algorithms and are robust to variations in learning and environment parameters. In particular, the introduction of eligibility traces does not qualitatively change the conclusions.

I stress that the mere existence of supra-competitive prices in the simulations does not make the \gls{fem}s viable. In fact, the only reason supra-competitive prices arise in settings with function approximation is that \emph{both} agents fail to compete efficiently. Indeed, their success hinges on the other agent also playing an inferior strategy. It is easy to see that playing such exploitable strategies are unlikely to succeed in real market settings. A potential exception is the \emph{hub and spoke} scenario envisioned by \textcite{ezrachi_algorithmic_2017}. The exploitable algorithms could prove successful if a vendor was able to supply it to \emph{all} competitors in an industry.

It is hard to pinpoint the exact causes of these failures. An obvious lever for improvement is the parametrization. The default specification was largely arbitrary and I did not systematically optimize parameters for computational reasons. The most obvious candidate for improving strategies is the exploration rate. But since the results are similar for a number of specifications, I am doubtful that optimizing parameters would make much of a difference. Alternatively, one could trial other, more sophisticated \gls{fem}s. However, linear function approximation might be generally inadequate to learn collusive strategies precisely because stable strategies require non-linear responses. I suspect that \emph{linear} function approximation could be a dead end in the realm of multi-agent reinforcement learning in economic environments. Nevertheless, absence of evidence does not equate evidence of absence. Indeed, \textcite{hettich_algorithmic_2021} proves that agents learning with \emph{non-linear} function approximation can be very successful in forging collusion.

I leave open two avenues for future research. First, further simulation studies could prove instrumental to understand under which conditions algorithmic collusion is likely. Most considered environments (including the one in this study) are rather simple and prefabricated. It would be interesting to see how algorithms behave in more challenging conditions (e.g.\ many players, dynamic demand, multi-sided markets). Possible extensions include \emph{actor-critic models} that allow to incorporate continuous action spaces.\footnote{\textcite[p.16]{hettich_algorithmic_2021} makes the same suggestion.} Second, empirical studies on real markets are imperative to get a refined understanding of how real the threat of algorithmic collusion is. \textcite{assad_algorithmic_2020} show that increased price margins in the wake of independently acquired algorithms are \emph{possible}. Whether this results holds for other industries and over time remains to be seen.