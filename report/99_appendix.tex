\autoref{trajectory_price} displays the distribution of average prices in various runs of the optimized experiments. \autoref{all_runs} displays the price and profit trajectory of single runs over time. Only runs of the \emph{optimal} runs are printed, as explained in \autoref{justifications}. Both metrics are averaged over 50,000 episodes apiece and over both players. Again, note that, by and large, prices and profits remain within the benchmarks of Nash competition and the cartel case.


\begin{figure}
	\includegraphics[width=\linewidth]{plots/trajectory_price.png}
	\caption{distribution of $p$ over time of 'optimized' experiments. For individual runs, $p$ is averaged over 50,000 periods apiece and both players. Plot includes converged and non-converged runs. Violin widths represent quantity of active runs at $t$ which enables comparisons between violins. As most runs converge after 200,000 to 300,000 episodes, violin widths decrease thereafter. Violins are trimmed at smallest and largest observation respectively. Horizontal lines represent the median.}
	\label{trajectory_price}
\end{figure}

\begin{figure}
	\includegraphics[width=\linewidth]{plots/all_runs.png}
	\caption{all runs for manually optimized $\alpha$}
	\label{all_runs}
\end{figure}


\autoref{intervention_profit_boxplot} displays the difference between profits agents receive after the forced deviation takes place and profits of the alternative path with no deviation.

\begin{figure}
	\includegraphics[width=\linewidth]{plots/intervention_profit_boxplot.png}
	\caption{distribution of profits at and after deviation relative to alternative path \emph{without} forced deviation, i.e.\ the difference to the profits at the same $\tau$ had no deviation taken place. Boxes demarcate 15th and 85th percentiles and are extended by whiskers that mark the entire range of price differences. Horizontal lines represent the group median.}
	\label{intervention_profit_boxplot}
\end{figure}


\begin{figure}
	\includegraphics[width=\linewidth]{plots/intervention_profitabiliy_polygon.png}
	\caption{Distribution of additional profitability due to deviation by agent and feature extraction method. Width of bins: 0.02. 6 extreme data points ($<-0.5$ or $>0.5$) are excluded for better presentability. Deviations are deemed \emph{profitable} if the discounted profits until $\tau = 10$ due to the deviation exceed cash flows from a counterfactual without deviation. Discounting is paramount to $\Delta$ in \textbf{equation X}, i.e.\ 0.95. The spike at zero is due to a significant number of 'deviations' that are neither profitable nor unprofitable. In those runs, the learned strategy of the deviating agent at $\tau = 1$ amounts to the best response and both agents keep following their respective price cycle.}
	\label{intervention_profitability_polygon}
\end{figure}


\begin{figure}
	\includegraphics[width=\linewidth]{plots/intervention_poly_tiling.png}
	\caption{Prices and profits of 3 individual runs belonging to the \emph{optimal } polynomial tiling experiment. Run \emph{01} shows the common pattern of immediate reversal to pre-deviation prices and profits. Run \emph{06} has the deviating agent continue to cheat and increasing her long-term profits. Run \emph{12} shows the non deviating agent meeting the price cut. This also yields a new equilibrium with lower prices. Since the pre-deviation prices were slightly above the collusive benchmark, the new equilibrium actually improves both agent's profits in this particular example.}
	\label{intervention_poly_tiling}
\end{figure}