\documentclass[a4paper]{scrartcl}
%alternatives to classic article are: book, report, scrartcl (recommended), scrreprt, scrbook
%more options Optionen: 11pt, 12pt, twoside, twocolumn

\usepackage[english]{babel}  %for German texts: ngerman
%\usepackage[utf8]{inputenc}   %encode in UTF-8 (I think TexStudio does this by default)

%\usepackage{natbib}  %used for bibliography (exact purpose unknown)
%\bibliographystyle{natdin} %bibliography style

\usepackage[backend = biber, citestyle=authoryear, bibstyle = authoryear]{biblatex} %more options, see e.g. https://www.overleaf.com/learn/latex/Biblatex_citation_styles
\addbibresource{zotero_refs.bib} %pass the name of the bib file

%Formatting and Layout
%\usepackage[left = 1.5cm, right = 1.5cm]{geometry} %This package allows for granular specified layouts. Optional arguments are, among others: left, right, top, bottom, width, height, textwidth, textheight, includeheadfoot
%\usepackage[onehalfspacing]{setspace} %controls space between lines (line spacing)
%\renewcommand{\baselinestretch}{1.5} %controls space between lines (line spacing)
\usepackage{blindtext} % to test layout via \blindtext


\usepackage{calligra}
\DeclareMathAlphabet{\mathcalligra}{T1}{calligra}{m}{n}


\pagestyle{headings}   %optional: section name is printed on the head of each page


\usepackage{xcolor}    %Colour package
\usepackage{graphicx}  %for graphics
\usepackage{amsmath,amsfonts,amssymb,amsthm,mathtools}  %math packages

%References
\usepackage[hidelinks]{hyperref} % should be called as last package
\usepackage{cleveref}


\begin{document}
	
	\def\sectionautorefname{section}
	\def\equationautorefname{equation}
	
	
	
	
	\title{\LaTeX Template}
	\author{Malte Jeschonneck}
	\date{\today}
	\maketitle
	
	\newpage
	
	\begin{titlepage}
		\centering
		\vspace{1cm}
		{\Large\bfseries \LaTeX Template \par}
		\vspace{4cm}
		{\large\itshape Heinrich-Heine-University Düsseldorf\par}
		\vspace{0.5cm}
		{\large\itshape Faculty of Business Administration and Economics\par}
		\vspace{0.5cm}
		{\large\itshape MW70 Competition Law and Policy\par}
		\vspace{0.5cm}
		{\large\itshape Summer Term 2018\par}
		{\large\itshape \par}
		\vfill
		by\par
		\vspace{0.5cm}
		Malte Jeschonneck\\
		Malte.Jechonneck@uni-duesseldorf.de\\
		Matriculation Number: 2307497\\
		Mörsenbroicher Weg 179, 40470 Düsseldorf\\
		Program: Economics, M. Sc.\\
		First Term
		
		
		\vfill
		
		% Bottom of the page
		{\large \today\par}
	\end{titlepage}

\newpage
	
	
	\tableofcontents
	\newpage
	\listoffigures
	\newpage
	\listoftables
	\newpage
	
	\section{Introduction}

There is little doubt that algorithms will play an increasingly important role in economic life. Dynamic pricing software is frequently used in online retail markets \parencite{chen_empirical_2016}, the tourist industry \parencite[p.4]{den_boer_dynamic_2015} and at petrol stations \parencite[pp.7-9]{assad_algorithmic_2020}. As with many other technological advances, the economic advantages are conspicuous. Not only does automating pricing decisions cut costs and free up resources, algorithms may also be better at predicting demand and react faster to changing market conditions \parencite[p. 15]{oecd_algorithms_2017}. Overall, there is little doubt that pricing algorithms may be used as a tool by companies to gain competitive advantages. It is worth pointing out that thereby intensified competition also benefits consumers.\footnote{Of course other types of algorithms that benefit consumers exist. Price comparison tools have been around for a while but applications extend beyond mere reduction of search costs. \textcite{gal_algorithmic_2017} champion \emph{algorithmic consumers}, electronic assistants that compare product characteristics at low transaction costs enabling humans to completely outsource their purchase decisions. Moreover, algorithmic consumers may challenge market power of suppliers by bundling consumer interests.}

Nevertheless, concerns have been raised that ceding pricing authority to algorithms has the potential to create new forms of collusion that contemporaneous competition policy is not well equipped to deal with. The main issue is that the traditional dichotomy between \emph{explicit} and \emph{tacit} collusion is potentially unsuitable in the case of pricing software. Traditionally, competition authorities only prohibit and punish explicit pricing agreements. On the contrary, tacit collusion (e.g.\ \emph{intelligent market adaption}) is typically tolerated despite the economic effect on consumers being equally detrimental \parencite[p. 141]{motta_competition_2004}. In practice, the distinction is sometimes vague and the advent of pricing algorithms is believed to blur the line. It is still unclear when \emph{algorithmic collusion} could elude competition enforcement.\footnote{A different issue is that pricing algorithms with information on consumer characteristics may be able to augment the scope of \emph{price discrimination}, i.e.\ companies extracting rent by charging to every consumer the highest price he is willing to pay. Under which circumstances competition authorities should be concerned with this possibility is outlined in \textcite{oecd_price_2016}. \textcite{ezrachi_algorithmic_2017} develop a scenario where discriminatory pricing and tacit collusion occur simultaneously. Both issues remain outside the scope of this study.}

Unfortunately, there is a lack of empirical studies assessing the effects of autonomous pricing software in the real world. A notable study of the German retail gasoline market by \textcite{assad_algorithmic_2020} documents that margins in duopoly markets increased substantially after both duopolists switched from manual pricing to algorithmic-pricing software. Further field studies could prove instrumental to confirm and refine these findings. As a substitute, there is a growing number of simulation studies that show the capacity of \emph{reinforcement learning} algorithms to create and sustain collusive equilibria in repeated games of competition (see \autoref{literature review}). However, the direct transferability of these findings to real markets is questionable. Most studies use a simple tabular learning method, called \emph{Q-Learning}, that requires discretizing prices and does not scale well if the complexity of the environment increases.

This study attempts to mitigate these problems by employing \emph{linear function approximation} to estimate the value of actions. More specifically, I develop three methods of function approximation and run a series of experiments to assess how they compare to tabular learning. Moreover, I utilize \emph{eligibility traces} as an efficient way to increase the memory of agents interacting in the environment.\footnote{Neither linear function approximation nor eligibility traces are new concepts in reinforcement learning. However, to my knowledge, this is the first study to apply them to a repeated pricing game.}

To foreshadow the results, the simulations show that the developed function approximation methods, like tabular learning, result in supra-competitive prices upon convergence. However, \emph{unlike} tabular learning, the learned strategies are easy to exploit. By forcing one of the agents to diverge from the convergence equilibrium, I show that the cheated agent fails to punish that deviation. This indicates that the learned equilibrium strategies are unstable vis-à-vis rational agents with full information. This observation is robust to a number of variations and extensions. Also, with respect to eligibility traces, excessively increasing memory tends to destabilize the learning process, but the overall impact for reasonable parametrization appears small.

The remainder of this paper is organized as follows. The next section briefly surveys simulation studies similar to this one. I also review the scholarly literature on algorithmic competition and contemporaneous regulation. Section \ref{enironment} introduces the repeated pricing environment in which the artificial competitors interact. Section \ref{algorithm} presents in detail the deployed learning algorithm with its parametrization and \autoref{feature_extraction} discusses the developed methods to estimate action values with function approximation. I present the results in \autoref{results} and consider variations and extensions in \autoref{robustness}. Section \ref{conclusions} concludes.
	
	\section{Literature review}\label{literature review}

This study concerns itself with the ability of algorithms to forge collusion without being explicitly instructed. Situations in which  humans would be unable to achieve such schemes are of special interest. This section provides an overview of academic and institutional assessments of the controversial topic.

First, it is helpful to define collusion. In this text, I will follow \textcite[pp.334-336]{harrington_developing_2018} who recognizes that \emph{supra-competitive} prices, i.e.\ prices above a competitive benchmark, must be underpinned by a \emph{reward-punishment scheme} to be labeled as \emph{collusive}. The scheme is maintained by a mutual understanding that a firm's current behavior affects its competitors' future conduct. Specifically, a participant's adherence to high prices is rewarded with high future prices of competitors ensuring high industry margins. Conversely, deviations are punished with price cuts.\footnote{Of course, competitors can collude not only on prices, but in a variety of ways. The concept easily extends to other dimensions, e.g.\ investment or product quality.} Naturally, it is yet to be seen how likely the scenario of algorithms achieving collusion in real markets is. It is also unclear whether the scenario constitutes illicit behavior and warrants intervention from competition authorities. This question is not a primary concern of this paper. Nevertheless, I will sketch some of the main positions before moving on.

The central issue is that contemporaneous competition policy does not consider collusion itself as illicit behavior. Rather, it is the process, by which it is achieved, that determines legality \parencite[p.339-341]{harrington_developing_2018}. Explicit communication among competitors who consciously agree on price levels is clearly illegal. Smart adaption to market conditions by individual agents is not. These distinct cases are often referred to as \emph{explicit} and \emph{tacit} collusion. Whether to put cooperating algorithms in the former or the latter category is subject to ongoing controversy. A binary answer probably does not do justice to the problem's complexity.  

At the very least, academics consent that algorithms could be utilized to facilitate \emph{existing} collusive agreements \parencite[p.219]{ezrachi_sustainable_2018}. For instance, cartel members could automate detection and punishment of deviations from an agreement through an algorithm. Other conceivable schemes include facilitated market segmentation and price \emph{signalling} \parencite[p.29]{oecd_price_2016}. While these scenarios may alter the operational scope of market investigations to account for the role of deployed algorithms, they are well covered by contemporary competition practices.\footnote{See e.g.\ a joint statement by the German and French federal cartel authorities \parencite{bundeskartellamt_working_nodate} and \textcite{cma_case_2016}, \textcite{oefgen_decision_2019} for two exemplary cases with algorithms \emph{facilitating} collusive agreements.} In fact, the specifics of \emph{facilitating} algorithms might not be highly important because the mere \emph{intention} to collude suffices to invoke competition laws \parencite[p.29]{bundeskartellamt_working_nodate}.

The most interesting scenario concerns independently developed or acquired algorithms that align pricing behavior. The US Department of Justice states that competitors are unlikely to be held liable if they independently adopt similar pricing software \parencite[p.6]{doj_algorithms_2017}. A note from the European Union indicates that companies can not expect to completely avoid liability by referring to their pricing algorithms. Moreover, they do not rule out the option that algorithms \emph{decoding} each other may be within the scope of explicit communication \parencite[p.8-9]{eu_algorithms_2017}. 

\textcite[pp.105-106]{gal_algorithms_2019} argues that in the special case of \emph{rule based} algorithms, a programmer's intent to create coordination could in principal be derived from her developed code. For instance, the conscious decision to include punishment mechanisms if a competitor's price falls below a certain threshold seems incriminating. However, many algorithms do not follow a rigid set of rules. Rather, its programmer defines higher-level objectives such as \emph{profit maximization} and the algorithm itself figures out how to act in a specific situation based on data and experience.\footnote{Every form of reinforcement learning falls into this second category.} Obviously, it is hard to infer intent from these types of algorithms. On a similar note, \textcite[p.350-351]{harrington_developing_2018} emphasizes that determining intentions from program code is conceptually appealing, but costly to implement.

\textcite{mehra_antitrust_2015} takes a more controversial view. He points out that the traditional distinction between explicit and tacit collusion emerged with human agents in mind and competition laws did not foresee pricing algorithms that are more likely to achieve cartel solutions in oligopolistic settings due to superior speed, accuracy and even rationality when analyzing and adjusting prices. Consequently, he argues that the increasing prevalence of automated pricing software warrants a reassessment of current competition law and enforcement. Moving forward, I will discuss the likelihood of collusion among algorithms arising in real markets. Due to a lack of empirical evidence, I will focus on theoretical considerations and simulation studies.

As pointed out earlier, the scarcity of field studies on pricing algorithms in real markets prohibits generalized conclusions. However, theoretical considerations might shed light on when algorithmic coordination is a valid concern. Any form of collusion requires timely detection of deviations and a credible threat of punishment \parencite[pp.48-56]{stigler_theory_1964}.\footnote{Naturally, industry characteristics play an important role (e.g.\ number of firms, market entry barriers or product homogeneity). See \textcite[p.142-149]{motta_competition_2004} for an extensive list of structural factors and their impact on the likelihood of collusion arising.}  Surely, algorithms are able to fulfill these conditions but so do humans. \textcite{schwalbe_algorithms_2018} argues that the advent of algorithms does not raise \emph{novel} competition problems. Indeed, humans might be replaced by pricing software but this does not inevitably make collusion more likely. The argument is strengthened by a list of experimental settings where algorithms fail to cooperate. Schwalbe stresses that the ability to communicate is vital to achieve collusive outcomes in markets with more than two participants and raises the question whether algorithms are better at communicating than humans.

\textcite[p.10-13]{ittoo_algorithmic_2017} emphasize the challenges associated with applying \emph{reinforcement learning} algorithms to real markets. First, there are practical implementation issues and mapping real market conditions to a reinforcement learning data problem is not always natural. Second, convergence guarantees break down as soon as the market is subject to changing conditions.\footnote{Technically, guarantees of convergence in reinforcement learning tasks are only valid if the environment is stationary, an assumption that is violated if demand conditions change or competitors price dynamically (see \autoref{convergence_considerations}). However, absence of convergence guarantees does not render convergence impossible.} Third, tabular learning methods do not scale well with the complexity of learning tasks. Consequently, mastering collusion might take a long time.\footnote{I will revisit this point in \autoref{tabular}.} They conclude that the deployment of pricing algorithms possibly, but not inevitably, leads to collusive outcomes.

\textcite[pp.6-17]{ezrachi_algorithmic_2017} argue that algorithms have the potential to establish tacit collusion in markets where conscious parallelism among humans is unrealistic. For instance, they develop a \emph{hub and spoke} scenario in which a third party software vendor provides the same or a similar pricing algorithm to competing sellers. The single algorithm could then align the pricing behavior of competitors resulting in conditions conducive to collusion. The authors suggest counter measures such as imposing restrictions on the allowed frequency of price changes or artificially reducing price transparency.


While there are numerous studies on the behavior of learning algorithms in cooperative and competitive multi-agent games\footnote{See e.g.\ \textcite{leibo_multi-agent_2017} and \textcite{crandall_cooperating_2018} for recent large-scale experimental studies.}, their application in oligopolistic environments has been rare and the trialed algorithms have been relatively simple. A seminal study by \textcite{waltman_q-learning_2008} examines two \emph{Q-Learning} agents in a \emph{Cournot} environment. Their simulations result in supra-competitive outcomes. However, even \emph{memoryless} agents without knowledge of past outcomes manage to attain quantities below the one-shot Nash equilibrium. This casts doubt on the viability of the learned strategies vis-à-vis rational agents. Truly memoryless agents can not pursue \emph{punishment strategies} because they are unable to even detect them. Thus, constantly playing the one-shot solution \emph{should} be the only rational strategy. It appears, the agents do not \emph{learn how to collude}, but rather \emph{fail to learn how to compete}.

Two further studies that model agents in games of infinitely repeated quantity competition should be mentioned. Inspired by the management literature, \textcite{kimbrough_learning_2009} trial a \emph{probe and adjust} algorithm.\footnote{To my knowledge, \emph{probe and adjust} is the only algorithm that explored continuous price setting in repeated games of competition to date.} Their agents repeatedly draw prices from a continuous price range. After some time, they assess whether low or high prices yielded better rewards and adjust the range of prices accordingly. They find that agents end up playing one-shot Nash prices unless industry profits enter the reward function in some way. \textcite{siallagan_aspiration-based_2013} propose \emph{aspiration based} learning where agents are allowed to communicate expectations to each other. They find that supra-competitive prices are attainable if the number of available options does not exceed 3.

Recent studies have focused on price instead of quantity competition. \textcite{klein_autonomous_2019} shows that \emph{Q-Learning} algorithms in a sequential price setting environment maintain a supra-competitive price level. He reports two types of equilibria: constant market prices and \emph{Edgeworth price cycles} where competitors sequentially undercut each other until profits become low and one firm resets the cycle by increasing its price significantly.\footnote{\textcite{noel_edgeworth_2008} considers a similar environment. However, he uses \emph{dynamic programming} for learning. His deployed agents \emph{know} their environment in detail, an assumption unlikely to hold in real markets. With \emph{Q-Learning}, agents estimate the action values based on past experiences.} Importantly, the high price levels are underpinned by a \emph{reward-punishment scheme}, i.e.\ a price cut of one agent evokes punishment prices by the opponent. Interestingly, the agents return to pre-deviaton levels within a couple of periods.

This study is closest to \textcite{calvano_artificial_2020} and \textcite{hettich_algorithmic_2021}. The former authors show that Q-Learning agents learn to sustain collusion through a \emph{reward-punishment} scheme in a simultaneous pricing environment. These findings are remarkably robust to variations and extensions. Furthermore, they find that agents learn to price competitively if they are memoryless (i.e.\ can not remember past prices) or short-sighted (i.e.\ do not value future profits). This coincides with predictions from economic theory. An important extension comes from \textcite{hettich_algorithmic_2021}. As in the present study, he utilizes function approximation, specifically a \emph{deep Q-Network algorithm} originally due to \textcite{mnih_human-level_2015}. He shows that the method converges much faster than \emph{Q-Learning}. The importance of that finding is augmented by the fact that the algorithm is much easier to scale to real applications.\footnote{\textcite{johnson_platform_2020} provide another extension. Introducing a \emph{multi-agent reinforcement learning} approach, they show that collusion arises even when the number of agents, often regarded a main inhibitor of cooperative behavior, is significantly increased. Moreover, they show that market design can significantly disturb collusion.}

To summarize, recent simulation studies show that reinforcement learning algorithms are capable of colluding in prefabricated environments. This paper tries to extend those findings by trialing \emph{linear} function approximation and eligibility traces.






	
	\section{Environment}\label{enironment}

This section presents the simulated economic environment that the autonomous pricing agents interact with. I consider an infinitely repeated pricing game with a logit demand as in \textcite{calvano_artificial_2019}. Restricting the analysis to the oligopoly case with $n=2$ agents (where $i = 1,2$), the market comprises \emph{2} differentiated products and an outside option. In every period $t$, both agents simultaneously pick a price $p_i$. Demand for agent $i$ is then determined\footnote{Generalization to a model with \emph{n} agents is straightforward. In fact, the demand formula remains the same. The limitation to 2 agents is merely chosen for computational efficiency and the (intuitive) conjecture that the simulation results generalize to more players provided learning time is sufficiently high.}:

\begin{gather}\label{quantity}
q_{i,t}=\frac{e^{\frac{a_i - p_{i,t}}{\mu}}}{\sum_{j=1}^{n}~ e^{\frac{a_j-p_{j,t}}{\mu}}+e^{\frac{a_0}{\mu}}}
\end{gather}

\textbf{Citations needed}
$\mu$ controls the degree of horizontal differentiation, where $\mu \rightarrow 0$ approximates perfect substitutability. Vertical differentiation is incorporated through the quality parameters $a_i$. $a_0$ reflects the appeal of the outside good. It diminishes as $a_0 \rightarrow -\infty$ \parencite{anderson_logit_1992}. 

Profits of both agents $\pi_i$ are simply calculated as

\begin{gather}\label{profit}
\pi_{i,t} = (p_{i,t} * q_{i,t}) - c_i,
\end{gather}

where $c_i$ is a firm-specific marginal cost. Market entry and exit are not considered. The baseline parametrization emulates \textcite{calvano_artificial_2019}:
$c_i = 1$,
$a_i = 2$,
$a_0 = 0$ and
$\mu = \frac{1}{4}$. These parameters give rise to a static Nash equilibrium with $p_n \approx 1.47$ and $\pi_n \approx 0.23$ per agent. The monopolist solution entails $p_m \approx 1.92$ with $\pi_m \approx 0.34$ for each product. Nevertheless, the following section covers the applied reinforcement learning methods for general parameters.

	
	\section{Reinforcement Learning with Function Approximation}

if the system were stationary --> convergence guarantee (e.g. Jaakoola et al. 1994). However, non-stationarity induced by multi-agent learning breaks that guarantee. To my knowledge, no guarantees, but empirically strong results

Though both agents repeatedly face the environment as presented in \autoref{enironment}, I will present this section from the vantage point of a single agent. Accordingly, the subscript {i} is dropped when appropriate.

\subsection{value approximation}\label{value_approximation}

The agent learns to approximate the value of an action given the available information. The potential actions reflect the available prices in the current period. It is useful to discretize the action space.\footnote{This discretization usually implies that agents will not charge exactly $p_n$ or $p_m$.} Compared to the baseline specification in \textcite{calvano_artificial_2019}, I consider a wider price range confined by a lower bound $A^L$ and an upper bound $A^U$:

\begin{gather}
A^{L} = c
\end{gather}

\begin{gather}
A^{U} = p_m + \zeta (p_n - c)
\end{gather}

The lower bound ensures positive margins. It is conceivable that a human manager could implement a sanity restriction like that before ceding pricing authority to an algorithm. The parameter $\zeta$ controls the extent to which the upper bound $A^U$ exceeds the monopoly price. With $\zeta = 1$, the difference between $A^{L}$ and $p_n$ is equal to the difference between $A^{U}$ and $p_m$. The available set of prices $\mathcal{A}$ is then evenly spaced out in the interval $[A^L, A^U]$:

\begin{gather}
	\mathcal{A} = (A^L, A^L + \frac{1(A^U - A^L)}{m-1}, A^L + \frac{2(A^U - A^L)}{m-1}~ , ... , ~ A^L + \frac{(m-2)(A^U - A^L)}{m-1}, A^U)
\end{gather}

$m$ determines the number of feasible prices. Following \textcite{sutton_reinforcement_2018}, I denote any possible action as $a \in \mathcal{A}$ and the actual realization at time $t$ as $A_t$.

In this simulation, the state set $S_t$ comprises merely the prices of the previous period $t-1$:

\begin{gather}
S_t = \{ p_{i, t-1}, p_{j, t-1} \}
\end{gather}


Accordingly, for every state variable, the set of possible states is identical to the feasible actions, i.e. $\mathcal{A}$. However, this is not required with function approximation methods. Theoretically, any state variable could be continuous and unbounded. Similarly to actions, $s \in \mathcal{S}$ denotes any possible state set and $S_t$ refers to the actual states at $t$.






Lastly, a set of parameters $\boldsymbol{w} = \{w_1, w_2, ..., w_D\}$, where $d \in \{1, 2, ..., D\}$, maps any combination of $S_t$ and $A_t$ to a value estimate $\hat{q-}_t$.\footnote{In the computer science literature, $\boldsymbol{w}$ is typically referred to as \emph{weights}. I will stick to the economic vocabulary and declare $\boldsymbol{w}$ parameters.} Hence:

\begin{gather}\label{q_estimation}
	\hat{q-}_t = \hat{q}(S_t,A_t,\boldsymbol{w}_t) = \hat{q}(p_{i, t-1}, p_{j, t-1}, p_{i, t}, \boldsymbol{w}_t)
\end{gather}

More specifically, any state-action combination is represented by a \emph{feature vector} $\boldsymbol{x}_t = \boldsymbol{x}(S_t, A_t) = \{x_1(S_t, A_t), x_2(S_t, A_t), ..., x_D(S_t, A_t)\}$ and every \emph{feature} $x_d = x_d(S_t, A_t)$, where every element is derived from a state, an action or a combination thereof. Moreover each $x_d$ is associated with a counterpart $w_d$. In section \ref{feature_extraction} the mechanisms to extract features are outlined in more detail. For now, note that I will only consider linear functions of $\hat{q}$. In this case, \autoref{q_estimation} can be written as the inner product of the feature vector and the set of parameters, i.e.\ $\boldsymbol{x}_t \top \boldsymbol{w} = \sum_{d=1}^{D} x_d * w_d$

Two elements are required for the algorithms to work successfully. First, the agents must mix between \emph{exploration and exploitation}. Second, the set of parameters $\boldsymbol{w}$ must be continuously optimized.

\paragraph{Exploration and Exploitation} 
In every period, the agent chooses either to \emph{exploit} its current knowledge and pick the supposedly optimal action or to \emph{explore} in order to test the merit of alternative choices that are perceived sub-optimal but may turn out to be superior. As is common, I use a simple $\epsilon$-greedy policy to steer this tradeoff:

\begin{gather}\label{action_selection}
 A_t = \begin{cases} arg ~\underset{a}{max} ~ \hat{q}(S_t,a,\boldsymbol{w}_t) & \quad \text{with probability } 1 - \epsilon_t\\
\text{randomize over } \mathcal{A} & \quad \text{with probability } \epsilon_t\\ \end{cases} 
\end{gather}

In words, the agent chooses to play the action that is regarded optimal with probability $1-\epsilon_t$ and randomizes over all prices with probability $\epsilon_t$.\footnote{If more than one $a$ maximizes $\hat{q}$, ties are broken randomly.} The subscript suggests that exploration varies over time. The explicit definition is given by:

\begin{gather}
	\epsilon_t = \psi e^{-\beta t}~ \text{, where}
\end{gather}

$\psi \in [0, 1]$ defines the initial exploration rate at $t = 0$ and $\beta$ controls the speed of its decay. This \emph{time-declining} exploration rate ensures that the agent randomizes actions frequently at the beginning of the simulation and stabilizes its behavior over time. 

After both agents selected an action, the quantities and profits are realized in accordance with equations \ref{quantity} \& \ref{profit}. The agents' actions in period $t$ become the state set in $t+1$ and new actions are chosen as dictated by equations \ref{q_estimation} \& \ref{action_selection}.

Irrespective of the \emph{exploit vs explore} decision, the agent proceeds to leverage the observed outcomes to refine $\boldsymbol{w}$.



\paragraph{Update}

After observing the opponent's price and the own profits, the agent exploits this new information to improve $\boldsymbol{w}$. A good starting point to introduce the utilized update rules is the so called \emph{TD error}, denoted $\delta_t$ (\textbf{finalize footnote}).\footnote{Without function approximation, versions of the \emph{TD error} usually encompass a discount factor $gamma$, such as:
	\begin{center}
		$\delta_t^{SARSA} = \pi_t + \gamma \hat{q}(S_{t+1}, A_{t+1}, \boldsymbol{w}) - \hat{q}(S_t, A_t, \boldsymbol{w})$
		
		$\delta_t^{Q-Learning} = \pi_t + \gamma ~ \underset{a}{max} ~ \hat{q}(S_{t+1}, a, \boldsymbol{w}) - \hat{q}(S_t, A_t, \boldsymbol{w})$
\end{center}
While they come with a meaningful economic interpretation, \textcite{sutton_reinforcement_2018} and \textcite{naik_discounted_2019} show that their use is inappropriate in infinite sequences with function approximation settings. Moreover, a policy maximizing average rewards is equivalent to a policy maximizing the average of discounted future values - irrespective of the particular discount factor.}

\textbf{average setting update}

\begin{gather}
	\delta_t = r_t - \widetilde{R}_{t-1} + \hat{q}(S_{t+1}, A_{t+1}, \boldsymbol{w}_t) - \hat{q}(S_t, A_t, \boldsymbol{w}_t) ~~   \text{,}
\end{gather}

where the reward $r_t = \pi_t - p_n$ reflects the profits relative to the Nash solution and $\widetilde{R}_{t-1}$ is a (weighted) average reward.\footnote{Please note that I explicitly distinguish profits and rewards. Profits, $\pi$, represent the monetary remuneration from operating in the environment and can be interpreted economically. However, profits do not enter the learning algorithm directly. Instead, rewards, $r$, immediate successors of profits, constitute the signal that is utilized as feedback by the agents to refine their algorithms.} $\delta_t$ measures the difference between the \emph{ex ante} ascribed value to the selected state-action combination in $t$ and the \emph{ex post} \emph{differential} profit $\pi_t - \widetilde{R}_{t-1}$ in conjunction with the estimated value of the newly arising state-action combination in $t+1$. A positive $\delta_t$ indicates that the actual realization turned out to exceed the original expectation. Likewise, a negative $\delta_t$ suggests that the realization failed short of the expected reward of playing the particular state-action combination. In both instances, $\boldsymbol{w}$ will be adjusted accordingly, such that the state-action combination is valued respectively higher or lower next time. Note that $\delta_t$ can only be calculated after the action in the next period has been taken.\footnote{This is often referred to as \emph{SARSA}, abbreviating a state-action-reward-state-action sequence.}


Surprisingly, this system 
* discount factor can be = 1






\emph{Semi-gradient} methods constitute a basic procedure for such continuous optimization. They serve as a good benchmark before developing more complex algorithms. Essentially, the direction and magnitude of updating parameters is driven by the \emph{TD error} $\delta_t$ and the gradient of $\hat{q}_t(S_t, A_t, \boldsymbol{w})$ with respect to $\boldsymbol{w}$:
$\frac{\Delta \hat{q}}{\Delta \boldsymbol{w}} =
\{ \frac{\Delta \hat{q}}{\Delta w_1},
\frac{\Delta \hat{q}}{\Delta w_2},
...,
\frac{\Delta \hat{q}}{\Delta w_d}  \}$. The update rule is:

\begin{gather}
 \boldsymbol{w}_{t+1} \leftarrow \boldsymbol{w}_t +
 	\alpha \delta_t
 	\frac{\Delta \hat{q}}{\Delta \boldsymbol{w}} ~~ \text{,}
\end{gather}

where $\alpha$ steers the speed of learning. As indicated earlier, I will only consider approximations that are linear in parameters. Thus, $\frac{\Delta \hat{q}}{\Delta \boldsymbol{w}}$ simplifies to the \emph{feature vector} $\boldsymbol{x}_t = \{x_1, x_2, ..., x_d\}$.

\textcite{seijen_true_2014} showed that the performance of that algorithm can be improved by keeping track of an eligibility vector $\boldsymbol{z} = \{z_1, z_2, ..., z_d\}$, called the \emph{Dutch Trace}. Like $\boldsymbol{w}$, this trace vector is updated at every time step. The trick is that the eligibility trace controls the magnitude by which individual parameters are updated, prioritizing those that contributed to producing an estimate of $hat{q}$. The update rule for the eligibility trace is:

\begin{gather}
\boldsymbol{z}_{t} \leftarrow \boldsymbol{z}_{t-1} \gamma \lambda + \boldsymbol{x_t} (1 - \alpha \gamma \lambda  {  \boldsymbol{x_t} \intercal \boldsymbol{z}_{t-1} }) ~~ \text{,}
\end{gather}

where $ \boldsymbol{x}_t \top \boldsymbol{z}_t $ denotes the inner product of $\boldsymbol{x}_t$ and $\boldsymbol{z}_t$, i.e. $ \boldsymbol{x}_t \top \boldsymbol{z}_t  = \sum_{i}^{d} x_i z_i$. $\boldsymbol{z}$ is then used in the refined parameter update:

\begin{gather}
	\boldsymbol{w}_{t+1} \leftarrow
		\boldsymbol{w}_{t} +
		\alpha \delta \boldsymbol{z}_t +
		\alpha ( \boldsymbol{w}_t \intercal \boldsymbol{x}_t  -
				 \boldsymbol{w}_{t-1} \intercal \boldsymbol{x}_t)
				(\boldsymbol{z}_t - \boldsymbol{x}_t)
\end{gather}





\subsection{Feature Extraction}\label{feature_extraction}

\textbf{TBD: introduction}
As outlined in \autoref{value_approximation}, the state-action space contains just 3 variables. Assigning a single coefficient to each variable certainly fails to do justice to the complexity of the optimization problem. In particular, a \emph{reward-punishment} theme requires that actions are chosen conditional on past prices (i.e.\ the state space). Hence, it is imperative to consider interactions and non-linearities. Therefore, I utilize various methods to extract features form the state-action space.

In reinforcement learning, a common approach is to store a distinct set of coefficients for every feasible action.\footnote{In this case, the vector of coefficients contains \emph{m} times features components} This is a sensible approach with qualitative action spaces. However, very much like tabular learning, a separate set of coefficients neglects the (quasi-) continuous nature of prices. Therefore two issues arise. First, discretizing the action space doesn't scale well if the number of feasible prices increases. Consequently, learning requires relatively many periods with large $m$. Second, observing a particular reward may not only constitute an informative feedback for the particular action undertaken, but also for 'similar' prices. Using and updating coefficients valid for (a subset of) all feasible prices exploits this.

For this simulation, I use \emph{polynomials}, \emph{polynomial splines} and \emph{tile coding} to extract features from the state-action space.

\subsubsection{Polynomials}

\emph{Polynomial approximation} of order $k$ maps states and action to a set of features, where a single feature corresponds to:



\begin{gather}
x_i^{Poly} = p_{1, t-1}^{\kappa_1} ~ p_{2, t-1}^{\kappa_2} ~ p_{1, t}^{\kappa_3}
\end{gather}


Every combination of exponents that adheres to the restrictions

\begin{itemize}
	\item $0 < \kappa_1 + \kappa_2 + \kappa_3 \leq k$ and
	\item $\kappa_1, \kappa_2, \kappa_3 \in \{0, 1, ..., k\}$
\end{itemize}

constitutes one feature. Using polynomial approximation, the feature vector $\boldsymbol{x}$ contains ${k + 3\choose3}  - 1$ elements.

\subsubsection{Normalized Polynomials}

\textbf{TBD}

\subsubsection{Polynomial Splines}

\textbf{TBD}

\subsubsection{Tile Coding}

In reinforcement learning, \emph{Tile Coding} is a common way to extract linear, in fact binary, features from a state-action space.\footnote{for an extensive introduction with instructive illustrations refer to \textcite{sutton_reinforcement_2018}} The idea is that several \emph{tilings} superimpose the state-action space. The $\mathcalligra{T}$ \ tilings are offset but each tiling covers the entire state-action space:

\begin{gather}
	 \mathcal{T}^L \leq A^L  ~ \& ~ \mathcal{T}^U \geq A^U    \text{for } \mathcal{T} \in \{1, 2, ..., \mathcalligra{T} ~ \}
\end{gather}

Each tiling is itself composed of uniformly spaced out \emph{tiles}.\footnote{With 2 dimensions, a tiling simply corresponds to a grid. In our case, the state-action space is 3-dimensional, so it may prove more intuitive to think of cubes instead of tilings and tiles.} Every tile is uniquely demarcated by a lower and an upper threshold for every dimension. Consequently, the number of tiles per tiling is controlled by the number of thresholds. For this simulation, it suffices to define a single set of thresholds per tiling that applies to all 3 dimensions. More specifically, the thresholds are spaced out evenly in the tiling-specific interval $[\mathcal{T}^L, \mathcal{T}^U]$:

\begin{gather}
\mathcal{T} = (
\mathcal{T}^L,
\mathcal{T}^L + \frac{1(\mathcal{T}^U - \mathcal{T}^L)}{\tau},
\mathcal{T}^L + \frac{2(\mathcal{T}^U - \mathcal{T}^L)}{\tau}~ , ... , ~
\mathcal{T}^L + \frac{(\tau-1)(\mathcal{T}^U - \mathcal{T}^L)}{\tau},
\mathcal{T}^U)
\end{gather}

This gives rise to $\tau^3$ tiles per tiling. Tiles are binary, i.e.\ if a state-action observation falls into a particular demarcation, the corresponding tile is \emph{activated}:

\begin{gather}\label{tile_activation}
x_i^{Tiling} = \begin{cases}
1 & \quad \text{if } \{p_{1, t-1}, p_{2, t-1}, p_{1, t}\} \text{~in tile demaraction}_i  \\
0 & \quad \text{if } \{p_{1, t-1}, p_{2, t-1}, p_{1, t}\} \text{~not in tile demarcation}_i \\ \end{cases} 
\end{gather}

Since tiles within a tiling are non-overlapping, any state-action combination activates exactly $\mathcal{T}$ tiles, one per tiling. The total number of features is simply $\mathcalligra{T}~\tau^3$. Note that the tabular case can be recovered as a special case by setting $\mathcalligra{T}~ = 1$ and $\tau \leq m$. In this case, every tile is activated by at most one feasible state-action combination which is equivalent to storing a dedicated coefficient for every state-action combination.\footnote{If $\tau > m$, some tiles would never be activated. But again, every table entry would correspond to a unique tile.}

\textbf{TBD: This feature exhibits the advantage of function approximation in large state spaces.  dimensionality problem when increasing the exponent, or $\tau$ not so much when increasing the number of tilings}

	
	\section{Results}
	
	\section{Robustness \& Extensions}
	
	\section{Conclusions}
	
	\section{To Do}
	List of useful \emph{quick \& dirty commands}:
	
	\begin{itemize}
		\item additional \space \space spaces \space \space \space with \texttt{\textbackslash space}
		\item \textbf{\textbackslash} \space via \texttt{\textbackslash textbackslash}
		\item \textbf{\textasciitilde} \space via \texttt{\textbackslash textasciitilde}
		\item \textbf{\ldots} \space via \texttt{\textbackslash ldots}
		
	\end{itemize}
	{\Huge Bibliography}
	also look at further hints in Microsoft/LinkedIn assignment file.
	
	\section{Hierarchy and References}\label{HaR}
	
	\subsection{Hierarchy}\label{Hie}
	
	The hierarchy is rather simple. There are \texttt{sections}, \texttt{subsections}, \texttt{subsubsections}, \texttt{paragraphs} \& \texttt{subparagraphs}. For instance, this text is located within a subsection. Luckily for us, \LaTeX \space takes care of incrementation.
	
	\subsubsection{Block Numbering}\label{sssbn}
	If we want don't want \LaTeX to assign numbers to a particular hierarchical element, one assign a \texttt{*} right after the command, e.g. \texttt{\textbackslash subsubsection*}
	
	\subsubsection*{Section without number}
	Hierarchically, this subsubsection is the successor of the previous subsubsection \ref{sssbn}.
	
	\subsubsection{Further hierarchy commands}
	The following hierarchical elements are rather uncommon and their functionality won't be described in detail. So, the following list serves merely as a collection:
	
	\begin{itemize}
		\item Naturally, \texttt{\textbackslash appendix} is used in the end of the text
		\item \texttt{\textbackslash part} \& \texttt{\textbackslash chapter} can only be used in the document class \emph{book}
		\item \texttt{\textbackslash frontmatter} is printed before the main text (if it is used, the main text should be started with \texttt{\textbackslash mainmatter})
		\item Similarly, \texttt{backmatter} is appended at the end
	\end{itemize}

\subsection{Calling other text files}
	In order to keep the size of a file manageable, a script can include references to other text files. The compiler will then include the text from that text file in the generated file. The required command is \texttt{\textbackslash include}. 

\subsection{References}
	\LaTeX \space can reference pretty much anything. The requirement is that the referenced object is labeled by \texttt{\textbackslash label}.
	
	\subsubsection{Default References}\label{dr}
	The default command to reference something is \texttt{\textbackslash ref}. For instance, we can refer to section \ref{HaR}, subsection \ref{Hie}, subsubsection \ref{sssbn}, table \ref{diamondsreg} \& figure \ref{tokyo}. On top of that one can refer to pages, e.g. the start of this paragraph is on page \pageref{dr}.
	
	\subsubsection{Hyperref}
	The package \emph{hyperref} steps up the default referencing. It introduces the command \texttt{\textbackslash autoref} which automatically returns the type of reference as well (e.g. you don't need to type `section' or `subsection' by yourself): As an example, look at the referencing towards \autoref{HaR}, \autoref{Hie}, \autoref{diamondsreg} \& \autoref{tokyo}.
	Hyperref does two more things:
	
	\begin{enumerate}
		\item it creates links between references and the Table of Contents. This allows a reader of the final \emph{pdf} to comfortably jump between sections.
		\item by default, it marks all references with a red frame. Personally, I like this while working in TexStudio but not really in the final pdf. It can be turned off by the option \texttt{hidelinks} when calling the package.
	\end{enumerate}



	\subsubsection{URLs and local file paths}
	
	Referencing URLs also works with \emph{hyperref}. Two options are available:
	\begin{itemize}
		\item using the \texttt{\textbackslash url} command: \url{https://r4ds.had.co.nz/index.html} 
		\item using the \texttt{\textbackslash href} command:
		\href{https://r4ds.had.co.nz/index.html}{R for Data Science}
	\end{itemize}
Both examples link the same page. The first displays the full link while the second uses a description of the page. Local file paths can be referenced in the same way. Here are links to the book
\href{run:C:/Users/psymo/OneDrive/Studium/Statistik/Dokumente/ISL/An Introduction to Statistical Learning.pdf}{\emph{Introduction to Statistical Learning}} and a picture of \href{run:C:/Users/psymo/Pictures/Korea/Korea/20190630_154735_HDR.jpg}{\emph{Ha Long Bay}}. Note that this linkage doesn't work in the previewer of TexStudio. You'll have to compile and open the real pdf for the links to work.

\subsection{Footnotes}
Footnotes are implemented with \texttt{\textbackslash footnote}. This is a sentence with an exemplary footnote\footnote{This is a footnote}.

	\section{Various}
	Due to the UTF-8 encoding, all German peculiarities are processed without problems: "ÄäÖöÜüß". Also note, that the language checking works as specified in the \emph{babel} package.
	
	That is great. At this point I want to check if \LaTeX knows about hyphenation and in order to do so I'll have to think of some sentences with some especially long word creations. The previous sentence convinced me that \LaTeX is capable of proper hyphenation.
	
	\subsection{Font Types}
	Single words can be printed in a different font. There are many types of fonts:
	\begin{itemize}
		\item \textmd{standard fonts via} \texttt{\textbackslash textmd}
		\item \textit{Italic} 
		\item \textbf{fat}
		\item \textsf{without serifs}
		\item \texttt{designed to print code where every letter has the same width}
		\item \textsc{capped names}
		\item \textsl{aslope}
		\item \underline{Underlined}
		\item \textcolor{blue}{Colored text} 
		\item \colorbox{green}{Marked text}
	\end{itemize}

\texttt{\textbackslash emph} ensures that an appropriate \emph{accentuation} for the current environment is chosen at all times.

	\subsection{Font Size}
One can change the font size, though in most circumstances this is not recommended, especially in a scientific text. The levels are:

\begin{itemize}
	\item {\tiny tiny}
	\item {\scriptsize scriptsize}
	\item {\footnotesize footnotesize}
	\item {\small small}
	\item {\normalsize normalsize (the default)}
	\item {\large large}
	\item {\Large Large}
	\item {\LARGE LARGE}
	\item {\huge huge}
	\item {\Huge Huge}
\end{itemize}


	\begin{scriptsize}
		Alternatively an environment can be called to change the text size for a longer text passage.
	\end{scriptsize}


\subsection{Quote and dash variations}
	There are many ways to quote. Which way is correct depends on the type of text and personal preferences. The most common ways are:
	
	\begin{itemize}
		\item `British single quotes' (I think this is my favorite)
		\item ``American double quotes'' (I think more common)
		\item \glqq German Quotation.\grqq.Test
		\item ``double quotes allows for `quotes inside quotes' ''
	\end{itemize}

The typical dashes that are available include:

	\begin{itemize}
		\item default - dash
		\item some -- dash
		\item large --- dash
		\item math $-$ dash
	\end{itemize}





	\subsection{Formatting}
	
\texttt{\textbackslash blindtext} is a nice way to test the current formatting options that are applied at a particular location of the text (in this example within the \texttt{quote} environment):
\begin{quote}
	\blindtext
\end{quote}


\subsubsection{line breaks}
Line breaks can be forced with \texttt{\textbackslash \textbackslash}. In order to increase the tweak the line spacing an optional argument can be used in square brackets.\\

{\setlength\parindent{0pt}
	I'm lost for words\\
	The truth hurts\\
	Behind walls of silence\\
	where I am caught \\[0.5cm]
	I'm lost for words\\
	The truth hurts\\
	The times I have the most to say\\
	are the times I can't talk
\\

Words or names that belong together can be glued together. A counterexample is \emph{Rio de Janeiro}. Spelling out the city name over two lines looks odd. Instead, use \textasciitilde \space to connect the single components as in \emph{Rio~de~Janeiro}.
}
\subsubsection{Indention}
By default, \LaTeX indents the first word of a a paragraph after another one has ended. Usually, this is useful and looks professional. However, sometimes this might be unwanted.

\begin{itemize}
	\item For instance, after an itemization, I think indention isn't particularly useful.
\end{itemize}

{\setlength\parindent{0pt}  \texttt{\textbackslash setlength\textbackslash parindent} in curly parentheses controls the amount of indention. Setting the mandatory argument to \emph{0pt} avoids indention altogether.}

Moreover, the indention of whole passage works with \texttt{\textbackslash quote}:
\begin{quote}
	``And I will strike down upon thee, those who attempt to poison and destroy my brothers, and you will know my name is the lord when I lay my vengeance upon thee.''
\end{quote}
		
	\subsubsection{Hyphenation}
	Given the right language package is used, \LaTeX \space usually hyphenates words automatically when appropriate. Occasionally, it may struggle with foreign words such as archae\-opterix. In those cases, use \textbackslash - within the word to mark the appropriate places to hyphenate.
	
	\subsection{Centering}
	Single text elements can be centered by creating a new \texttt{center} environment:
	
	\begin{center}
		This text is centered.\\
		$ a^2 + b^2 = c^2$
	\end{center}

A \emph{quick and dirty} alternative is \texttt{\textbackslash centering} but I don't like it that much.
	
	\subsection{Itemization \& Enumeration}
	Listings work via the \texttt{\textbackslash itemize} environment. Enumerations are built with \texttt{\textbackslash enumerate}. Partitioning over more than one layer works by using more than one environment:
	
	\begin{itemize}
		\item first bulletpoint
		\item second bulletpoint with subsections
		\begin{enumerate}
			\item first option
			\item second option
		\end{enumerate}
		\item third bulletpoint
			\begin{itemize}
				\item [a)]  The style of the bullet points can also be set manually
				\item [b)] However, \LaTeX \space doesn't really like that. It might be a good idea to stick to the defaults.
			\end{itemize}
	\end{itemize}

	
	\newpage
	
	\section{Tables and Figures}
	
	\subsection{Tables}
		Creating tables in \LaTeX is rather inefficient and involved. Below is an example of a manually created table:\\
		
	\begin{table}
		\caption{Excellent Songs}
	\begin{tabular}{|c|c|r|l|}
		\hline
		\multicolumn{4}{|c|}{\textbf{Excellent Songs}} \\
		\hline \hline
		\textbf{song}&\textbf{artist}&\textbf{price}&\textbf{comment} \\
		\hline
		Infinite&Eminem&0,99&Very Good \\
		\hline
		Lady (Hear me Tonight)&Modjo&0,69&Excellent track during summer \\
		\hline
		Still D.R.E.&Dr. D.R.E.&1,29&this is an excellent song \\
		&Snoop Dog&&of two gangsta exhibits who\\
		&&&swear that the game hasn't changed and\\
		&&& both of them remain rightful kings\\
		\hline
	\end{tabular}
	\end{table}
	
	There are many tools which ease the pain of \LaTeX tables (e.g. \emph{Stargazer} to print Regression tables from `R'):
	
	\begin{table}[!htbp] \centering 
		\caption{Diamonds Linear Regression} 
		\label{diamondsreg} 
		\begin{tabular}{@{\extracolsep{5pt}}lc} 
			\\[-1.8ex]\hline 
			\hline \\[-1.8ex] 
			& \multicolumn{1}{c}{\textit{Dependent variable:}} \\ 
			\cline{2-2} 
			\\[-1.8ex] & carat \\ 
			\hline \\[-1.8ex] 
			price & 0.00003$^{***}$ \\ 
			& (0.00000) \\ 
			& \\ 
			x & 0.312$^{***}$ \\ 
			& (0.001) \\ 
			& \\ 
			depth & 0.016$^{***}$ \\ 
			& (0.0002) \\ 
			& \\ 
			Constant & $-$2.133$^{***}$ \\ 
			& (0.016) \\ 
			& \\ 
			\hline \\[-1.8ex] 
			Observations & 53,940 \\ 
			R$^{2}$ & 0.969 \\ 
			Adjusted R$^{2}$ & 0.969 \\ 
			Residual Std. Error & 0.083 (df = 53936) \\ 
			F Statistic & 568,906.900$^{***}$ (df = 3; 53936) \\ 
			\hline 
			\hline \\[-1.8ex] 
			\textit{Note:}  & \multicolumn{1}{r}{$^{*}$p$<$0.1; $^{**}$p$<$0.05; $^{***}$p$<$0.01} \\ 
		\end{tabular} 
	\end{table}
	
	\subsection{Figures}
\LaTeX can import many pictures stored in various file formats and include them in the resulting pdf. The picture file must be located in the same folder as the \emph{.Tex} file.

		\newpage
	\section{Math} \label{form}
	This section is old but might help looking up some commands and templates.
	
	Formeln lassen sich ähnlich Microsofts Formel-Editor einfügen. Der Vorteil ist, dass man aufgrund der befehlbasierten Logik potenzielle Tippfehler besser korrigieren kann: $ x^2 =6 $. Zentrierte Formeln sind auch möglich:
	\begin{center}
		$ y_i =x_1*\beta_1 $
	\end{center}
	
	Hier sind die bekannten Regeln anzuwenden. Man kann tolle Sachen zaubern!
	\subsection{Matrizen}
	Matrizen lassen sich über '\&' sowie '\verb=\=' verarbeiten:
	
	\begin{center}
		$
		\begin{Bmatrix}
		y_{1,1} & y_{1,2} & \dots & y_{1,n} \\
		y_{2,1} & y_{2,2} & \dots & y_{2,n} \\
		\vdots & \vdots & \ddots & \vdots \\
		y_{m,1} & y_{m,2} & \vdots & y_{m,n}
		\end{Bmatrix}
		$
	\end{center}
	
	\subsection{Summen}
	\label{sums}
	Summen sind toll. Hier werden sie über die Alternative zur Einleitung von Formeln eingegeben. Dadurch wird immer ein neuer Absatz erstellt und zentriert.
	\[  \sum_{i=1}^n \]
	
	\subsection{Brüche und Wurzeln}
	Für Brüche wird der \textbf{frac}-Befehl genutzt.
	\[ \frac{a^2+b^2}{c^2+\frac{d}{n}} \]
	
	Wurzeln sind einfach ein Klassiker:
	\[ \sqrt[3]{Inhalt} \]
	
	
	\newpage
	
	\newpage
	\section{Crafting and automating the bibliography}
	Once set up there, \LaTeX in conjunction with JabRef and a bibliography tool takes care of citations. Take this as an example for an imaginary article in a journal citation\footnote{\cite{demo_art}}.
	Here comes a citation of an imaginary book.\footnote{\cite{demo_book}} Moreover, here is an American style reference to an article that was inserted into the pseudo database by Google Scholar's BibTeX function.(\cite{demo_source})
	
	\subsection{Workflow}
	
	Before passing different sources, a few global parameters must be set. These steps must be undertaken once:
	
	\begin{enumerate}
		\item Set up a .bib file and save it in the same directory as the .tex file. JabRef is a nice tool for that. The user interface suggests that a pseudo database is created which makes working with the files easier.
		\item In the file's header load the package \texttt{biblatex} with the appropriate options (see \autoref{config})
		\item Still in the file's header, specify the name of the .bib file with \texttt{\textbackslash addbibresource}
		\item Type \texttt{\textbackslash printbibliography} before the \emph{document} environment is closed (i.e. after the main text but before \texttt{\textbackslash end\{document\} }
	\end{enumerate}
	
	Once that is set up, the workflow for single quotes is like this.
	\begin{enumerate}
		\item include a source in the JabRef pseudo database. There are two options to do this.
			\begin{enumerate}
				\item manually enter all required information including a unique bibtexkey to refer to the source
				\item look online for the BibTex source code. There are some sites where those can be obtained (eg. Google Scholar). A bibtexkey is set by default but can be altered  to fit individual preferences
			\end{enumerate}
		\item use \texttt{\textbackslash cite} and refer to the Bibtexkey in curly brackets.
		\item optionally set up additional parameters in squared brackets to appear before or after the source\footnote{\cite[see][p. 12]{grapov2018rise}}.
	\end{enumerate}

\subsection{Configurations}\label{config}
	There are many options to control the citation style. For instance, the argument \texttt{citestyle} determines how citations are implemented in the text and \texttt{bibstyle} affects how the sources are implemented in the list of references at the end. Some information on style options can be obtained \href{https://www.overleaf.com/learn/latex/Biblatex_citation_styles}{\textbf{\color{blue}here}}.
	
	\subsection{Troubleshooting}
	
	If the workflow doesn't work, the first attempt to solve this should be to re-run the whole thing. If this doesn't work, check the following options:
	\begin{itemize}
		\item \textbf{TexStudio:} Bibliography $\rightarrow$ Type: \emph{BibLaTeX}
		\item \textbf{TexStudio:} Options $\rightarrow$ Configure TexStudio $\rightarrow$ Build\\ $\rightarrow$ Default Bibliography Tool = \emph{Biber}
		\item \textbf{JabRef:} Options $\rightarrow$ General $\rightarrow$ Default Bibliography mode = \emph{biblatex}
	\end{itemize}

The next step is to update or add one entry in JabRef and saving the changes in the .bib file. Following that, delete the existing .aux, .bbl, .bcf files in the directory of the .tex file and re-run three times. If this doesn't work use google for help. Those two sites are pretty extensive for troubleshooting:

\begin{itemize}
	\item \url{https://ipfs-sec.stackexchange.cloudflare-ipfs.com/tex/A/question/63852.html}
	\item 	\url{https://ipfs-sec.stackexchange.cloudflare-ipfs.com/tex/A/question/286706.html}
\end{itemize}

	\printbibliography
	
	\appendix
	\section{Glossary}
	\section{second appendix}
\end{document}
