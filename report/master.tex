\documentclass[a4paper]{scrartcl}
%alternatives to classic article are: book, report, scrartcl (recommended), scrreprt, scrbook
%more options Optionen: 11pt, 12pt, twoside, twocolumn

\usepackage[english]{babel}  %for German texts: ngerman
%\usepackage[utf8]{inputenc}   %encode in UTF-8 (I think TexStudio does this by default)


\usepackage[backend = biber, citestyle=authoryear, bibstyle = authoryear, doi=false,isbn=false,url=false,eprint=false]{biblatex} %more options, see e.g. https://www.overleaf.com/learn/latex/Biblatex_citation_styles

\AtEveryBibitem{
	\clearlist{language} % clears language
	
} 

\renewbibmacro{in:}{}

\usepackage{xpatch}

% No dot before number of articles
\xpatchbibmacro{volume+number+eid}{%
	\setunit*{\adddot}%
}{%
}{}{}

% Number of articles in parentheses
\DeclareFieldFormat[article]{number}{\mkbibparens{#1}}


\addbibresource{zotero_refs.bib} %pass the name of the bib file


% define glossary
\usepackage[nopostdot,nogroupskip,style=super,nonumberlist,automake]{glossaries}

\glstocfalse
\makeglossaries

\newglossaryentry{mdp}{name={MDP},description={Markov Decision Process}}
\newglossaryentry{fem}{name={FEM},description={Feature Extraction Method}}
\newglossaryentry{td_error}{name={TD error},description={temporal-difference error}}


%Formatting and Layout
\usepackage[left = 4cm, right = 2cm, top = 2.5cm, bottom = 2cm]{geometry}

% line spacing: 1.5
\usepackage[onehalfspacing]{setspace}
\renewcommand{\baselinestretch}{1.5}

%no indentation in captions of figures and tables
\setcapindent{0pt} 


% Package for algorithm boxes
\usepackage{algorithm,algorithmic}

% section name is printed on the head of each page
\pagestyle{headings}   


\usepackage{graphicx}  %for graphics
\usepackage{amsmath,amsfonts,amssymb,amsthm,mathtools}  %math packages



% abstract and appendices
\usepackage{abstract}
\usepackage[title]{appendix}
\newcommand*{\Appendixautorefname}{appendix}

%References
\usepackage[hidelinks]{hyperref}
\usepackage{cleveref}


\begin{document}
	
	% overwrite autoreference labels
	\def\sectionautorefname{section}
	\def\subsectionautorefname{section}
	\def\subsubsectionautorefname{section}
	\def\equationautorefname{equation}
	\def\algorithmautorefname{Algorithm}
	

	
	% start roman page numbers
	\pagenumbering{Roman}
	
	% center front page, no page number
	\newgeometry{left = 2cm, right = 2cm}
	\thispagestyle{empty}
	
	\begin{titlepage}
		\centering
		\vspace{1cm}
		{\Large\bfseries \LaTeX Template \par}
		\vspace{4cm}
		{\large\itshape Heinrich-Heine-University Düsseldorf\par}
		\vspace{0.5cm}
		{\large\itshape Faculty of Business Administration and Economics\par}
		\vspace{0.5cm}
		{\large\itshape MW70 Competition Law and Policy\par}
		\vspace{0.5cm}
		{\large\itshape Summer Term 2018\par}
		{\large\itshape \par}
		\vfill
		by\par
		\vspace{0.5cm}
		Malte Jeschonneck\\
		Malte.Jechonneck@uni-duesseldorf.de\\
		Matriculation Number: 2307497\\
		Mörsenbroicher Weg 179, 40470 Düsseldorf\\
		Program: Economics, M. Sc.\\
		First Term
		
		
		\vfill
		
		% Bottom of the page
		{\large \today\par}
	\end{titlepage}

\restoregeometry % go back to default layout

\newpage
	\begin{abstract}
		 The increased prevalence of pricing algorithms incited an ongoing debate about new forms of collusion. The concern is that intelligent algorithms may be able to forge collusive schemes without being explicitly instructed to do so. I attempt to examine the ability of competing \emph{reinforcement learning} algorithms to maintain collusive prices in a simulated oligopoly of price competition. To my knowledge, this study is the first to use a reinforcement learning system with linear function approximation and eligibility traces in an economic environment. I show that the deployed agents sustain supra-competitive prices, but tend to be exploitable in the short-term. The degree of collusion crucially hinges on the utilized method to estimate the qualities of actions. This finding is robust to variations of parameters that control the learning process.
	\end{abstract}
	
\newpage	
	
	% TOC, LOF, LOT, Abbreviations
	\tableofcontents
	\newpage
	\listoffigures
	\newpage
	\listoftables
	\newpage
	\printglossary[title={List of Abbreviations}] %Generate List of Abbreviations
	\newpage
	\pagenumbering{arabic}
	
	\section{Introduction}

There is little doubt that algorithms will play an increasingly important role in everyday life. Automated and dynamic pricing are frequently used in online retail \parencite{chen_empirical_2016}, but are also applied in the tourist industry \parencite[p.4]{den_boer_dynamic_2015} and at petrol stations \parencite[pp.7-9]{assad_algorithmic_2020}. 

\textbf{where is it prevalent}


As with many other technological advances, the economic advantages are conspicuous. Not only does automating pricing decisions cut costs and free up resources, algorithms may also be better at predicting demand and react faster to changing market conditions \parencite[p. 15]{oecd_algorithms_2017}. 
This non-exhaustive list of examples leaves little doubt that pricing algorithms may be used as tool by companies to gain competitive advantages and it is worth pointing out that consumers also benefit from intensified competition.\footnote{Moreover, there exist other types of algorithms that benefit consumers. Price comparison tools have been around for a while but the applications extend beyond that. \textcite{gal_algorithmic_2017} champion \emph{algorithmic consumers}, electronic assistants who make sophisticated product comparisons at low transaction costs enabling humans to completely outsource their purchase decisions. Moreover, algorithmic consumers may challenge market power of suppliers by bundling consumer interests.}

Nevertheless, concerns have been raised that ceding pricing authority to algorithms has the potential to create new forms of collusion that contemporaneous competition policy is not well equipped to deal with. The main issue is that the traditional dichotomy between \emph{explicit} and \emph{tacit} collusion is potentially unsuitable in the case of pricing software. Traditionally, competition authorities only prohibit and punish explicit pricing agreements. On the contrary, tacit collusion (e.g.\ \emph{intelligent market adaption}) is tolerated \textbf{citation needed?} despite the economic effect on consumers being equally detrimental \parencite[p. 141]{motta_competition_2004}.\footnote{A different issue is that pricing algorithms with information on consumer characteristics may be able to augment the scope of \emph{perfect price discrimination}, i.e.\ companies extracting rent by charging to every consumer the highest price he is willing to pay. Under which circumstances competition authorities should be concerned with this possibility is outlined in \textcite{oecd_price_2016}. \textcite{ezrachi_algorithmic_2017} develop a scenario where discriminatory pricing and tacit collusion occur simultaneously. Both issues remain outside the scope of this study.}


\textbf{The distinction was elusive before the advent of pricing algorithms. But it is even less clear now and might not capture the danger of algorithmic collusion}

\textbf{painpoint of of algorithms --> why do we care?} 

Unfortunately, there is a lack of empirical studies assessing the effects of autonomous pricing software in the real world. A notable study of the German retail gasoline market by \textcite{assad_algorithmic_2020} documents that margins in duopoly markets increased substantially after both gasoline stations switched from manual pricing to algorithmic-pricing software. Further field studies could prove instrumental to confirm and refine these findings. On the other hand, there is a growing number of simulation studies that show the capacity of \emph{reinforcement learning} algorithms to create and sustain collusive equilibria in repeated pricing competition games (see \autoref{simulation_studies}). However, the direct transferability of these findings to real markets is questionable. Most studies use tabular learning methods, mainly \emph{Q-Learning}, which requires discretizing prices and does not scale well if the complexity of the environment increases.

This study attempts to get rid of the problem by employing \emph{linear function approximation} to estimate the value of actions. More specifically, I develop three methods of function approximation and then run a series of experiments to assess how they compare to tabular learning. Moreover, I utilize \emph{eligibility traces} as an efficient way to increase the memory of agents interacting in the environment.\footnote{Neither linear function approximation nor eligibility traces are new concepts in reinforcement learning. However, to my knowledge, this is the first study to apply them to a repeated pricing game.}

To foreshadow the results, the simulations show that the developed function approximation methods, like tabular learning, result in supra-competitive prices upon convergence. However, \emph{unlike} tabular learning, the learned strategies are easy to exploit. By forcing one of the agents to diverge from the convergence equilibrium, I show that the cheated agent fails to punish that deviation. This indicates that the learned equilibrium strategies are unstable vis-à-vis rational agents with full information. This observation is robust to a number of variations and extensions. Also, the impact of eligibility traces in this study is small.

The remainder of this paper is organized as follows. The next section briefly reviews literature on algorithmic competition as well as contemporaneous regulation and presents results from previous simulations similar to this study. Section \ref{enironment} introduces the repeated pricing environment I let the competitors interact with. Section \ref{algorithm} present in detail the deployed learning algorithm with its parametrization and \autoref{feature_extraction} discusses the developed methods to estimate action values with function approximation. I present the results in \autoref{results} and consider variations and extensions in \autoref{robustness}. Finally, \autoref{conclusions} concludes.
	
	\section{Literature Review}
This study is related to three literature streams: (i) the scholarly debate on how competition law is supposed to manage autonomous pricing software, (ii)  repeated games in the realm of algorithms, and (iii) an increasing number of simulation studies that empirically examine the behavior of algorithms in simplified economic environments. I will provide a brief summary of the recent developments in each of these fields.

\subsection{Competition Policy and pricing algorithms}
As algorithms increasingly take over pricing authority from humans in a number of industries, some scholars have voiced concerns about the adequacy of current competition laws and practices. \textcite{mehra_antitrust_2015} points out that the traditional distinction between explicit and tacit collusion emerged with \emph{human sellers} in mind who differ from \emph{robo-sellers}. He argues that the latter are more likely to achieve cartel solutions in an oligopolistic setting due to superior speed, accuracy and even rationality when analyzing and adjusting prices. He concludes that the increasing prevalence of automated pricing software warrants a reassessment of current competition law and enforcement. 

* no human intent to achieve supracompetitive prices
	* but employed algorithms achieved that anyway. 
	* outcome eludes explicit agreement, still detrimental

\textcite{ezrachi_sustainable_2018} claim academic consensus that algorithms could at the very least be utilized to facilitate \emph{existing} collusive agreements. For instance, cartel members could automate the detection and even punishment of deviations from an agreement through an automated algorithm. Other conceivable schemes include facilitated market segmentation \parencite{oefgen_decision_2019} and price \emph{signaling} \parencite{oecd_price_2016}. While these scenarios may alter the operational scope of market investigations to account for the role of deployed algorithms, they are well covered by contemporary competition laws.  \footnote{see e.g.\ statements by \cite{bundeskartellamt_bundeskartellamt_nodate}. See \textcite{cma_case_2016} \textcite{oefgen_decision_2019}for two exemplary cases}. 




\textcite{noa}


\textcite{bundeskartellamt_bundeskartellamt_nodate} also note that the specifics of the algorithms are not highly important because the mere \emph{intention} to collude suffices to invoke competition laws.


\subsection{Algorithms in Game Theory}

something moer

\subsection{Simulation Studies}

While there are numerous studies on the behavior of algorithms in cooperative and competitive games, their application in industrial economics has been a little scarcer. A seminal study by \textcite{waltman_q-learning_2008} examines two \emph{Q-Learning} pricing agents in a \emph{Cournot} environment.\footnote{i.e.\ firms compete in quantities.} Their simulations result in supra-competitive equilibria, but even \emph{memoryless} agents without knowledge of past outcomes manage to attain quantities below the one-shot Nash equilibrium.  This casts doubt on the viability of the learned strategies vis-à-vis rational agents. Truly memoryless agents can't pursue \emph{trigger strategies} in the sense that they are unable to punish deviations as they fail to even detect them. Thus, constantly playing the one-shot solution \emph{should} be the only rational strategy. To that end, the agents seem to \emph{fail to learn how to compete} rather than to \emph{learn how to collude} \parencite{cooper_learning_2015}. 






More recently, \textcite{calvano_artificial_2019} simulate two Q-Learning agents that not only achieve collusive outcomes, but also sustain their tacit agreement through a \emph{reward-punishment} scheme. \textcite{klein_autonomous_2019} shows that Q-Learning algorithms converge faster in a sequential price setting environment.







Unfortunately, there is a lack of empirical studies assessing the workings and effects of autonomous pricing software in the real world. To my knowledge, a notable study of the German retail gasoline market by \textcite{assad_algorithmic_2020} remains the only exception. They document that margins in duopoly markets increased substantially if both actors switched from manual pricing to algorithmic-pricing software. Further field studies could prove instrumental to confirm and refine these results.
	
	\section{Environment}\label{enironment}

This section presents the simulated economic environment that the autonomous pricing agents interact with. I consider an infinitely repeated pricing game with a multinominal logit demand as in \textcite{calvano_artificial_2019}. Restricting the analysis to a symmetric oligopoly case with $n=2$ agents (where $i = 1,2$), the market comprises \emph{2} differentiated products and an outside option. In every period $t$, both agents simultaneously pick a price $p_i$. Demand for agent $i$ is then determined \parencite{anderson_logit_1992}:\footnote{Generalization to a model with \emph{n} agents is straightforward. In fact, the demand formula remains the same. The limitation to 2 agents is merely chosen for computational efficiency and the (intuitive) conjecture that the simulation results generalize to more players provided learning time is sufficiently high.}

\begin{gather}\label{quantity}
q_{i,t}=\frac{e^{\frac{a - p_{i,t}}{\mu}}}{\sum_{j=1}^{n}~ e^{\frac{a-p_{j,t}}{\mu}}+e^{\frac{a_0}{\mu}}}
\end{gather}

\textbf{Citations needed}
$\mu$ controls the degree of horizontal differentiation, where $\mu \rightarrow 0$ approximates perfect substitutability. While I forego to incorporate vertical differentiation throughout this study, it may be incorporated by choosing firm-specific quality parameters $a$. $a_0$ reflects the appeal of the outside good. It diminishes as $a_0 \rightarrow -\infty$. 

Profits of both agents $\pi_i$ are simply calculated as

\begin{gather}\label{profit}
\pi_{i,t} = (p_{i,t} - c) q_{i,t},
\end{gather}

where $c$ is the marginal cost.\footnote{Again, $c$ could be varied by adding suffixes accordingly.} \textcite{anderson_logit_1992} show that the multinominal logit demand model with symmetric firms entails a unqique one-shot equilibrium with best responses that solve:

\begin{gather}\label{best_response}
	p_n = p^* = c + \frac{\mu}{1 - (n + e^{\frac{a_0 - a + p^*}{\mu}})^{-1}}
\end{gather}

Naturally, the other extreme, a collusive (or monopoly) solution, is obtained by maximizing joint profits.\footnote{For this study, I approximate both cases using numerical optimization.} Both, the Nash outcomes characterized by $p_n$ and $\pi_n$ \emph{and} the fully collusive solution ($p_m$ and $\pi_m$) shall serve as benchmarks for the simulations.

Market entry and exit are not considered. The baseline parametrization is identical to \textcite{calvano_artificial_2019}:
$c = 1$,
$a = 2$,
$a_0 = 0$ and
$\mu = \frac{1}{4}$. These parameters give rise to a static Nash equilibrium with $p_n \approx 1.47$ and $\pi_n \approx 0.23$ per agent. The monopolist solution entails $p_m \approx 1.92$ with $\pi_m \approx 0.34$ for each product. Nevertheless, the following section covers the applied reinforcement learning methods for general parameters.

	
	\section{Reinforcement Learning with Function Approximation}
	
	
\section{Feature extraction}\label{feature_extraction}

\textbf{TBD: extended introduction (?)}

This section lays out the methods used in this study to map state-action combinations to a set of numerical values $\boldsymbol{x}$. I shall refer to them as \emph{feature extraction methods}. \footnote{The term \emph{feature} is borrowed from the computer science literature. It usually refers to a (transformed) input variable. It is common that the number of features dwarfs the number of original inputs.} As outlined in \autoref{value_approximation}, the state-action space contains just 3 variables ($p_{i,t-1}$, $p_{j,t-1}$ and $p_{i,t}$). To illustrate, consider a naive attempt of feature extraction where each $S_t$ and $A_t$ is converted to a feature without mathematical transformation, i.e.\ $x_1 = p_{i,t-1}$, $x_2 = p_{j,t-1}$ and $x_3 = p_{i,t}$. Every feature is assigned a single coefficient $w_d$ and the estimated value of any state-action combination would then be $\hat{q}(S_t, A_t, \boldsymbol{w}) = \sum_{d=1}^3 w_d x_d$. Obviously, this simplistic method fails to do justice to the complexity of the optimization problem. In particular, a \emph{reward-punishment} theme requires that actions are chosen conditional on past prices (i.e.\ the state space). Hence, it is imperative to consider interactions and non-linearities. \autoref{feature_extraction_summary} provides an overview of the 4 methods used in this study.

\begin{center}
	\begin{table}
		\begin{tabular}{|l|l|l|c|}
			\hline
			\textbf{Feature }&\textbf{Baseline}&\textbf{Length} $\boldsymbol{x}$&\textbf{factor when}\\
			\textbf{Extraction}&\textbf{Parametrization}&&\textbf{doubling} $m$\\
			\hline
			Tabular&-&$m^3 = 6,859$& x8\\
			\hline
			Tile Coding&$T = 5, \psi = 9$&$T~(\psi - 1)^3 = 2,560$& x1\\
			\hline
			Polynomial Tiles&$T = 5, \psi = 5, k = 4$&$T~(\psi - 1)^3 ({k + 3\choose3}  - 1) = 10880$& x1 \\
			\hline
			Sep. Polynomials&$k = 5$ &$m($ ${k+2}\choose{2}$ $-1) = 380$& x2 \\
			\hline
		\end{tabular}
		\caption{Feature extraction methods, number of parameters with $m=19$ and factor when doubling the number of available prices $m$.}
		\label{feature_extraction_summary}
	\end{table}
\end{center}

\textbf{1. finish table, 2. described columns in table briefly}

The remainder of this chapter describes the methods in more detail. Note that \emph{polynomial approximation}, as described in \autoref{polynomial}, is not directly used in this study but nevertheless introduced as a precursor to the final two methods \emph{polynomial tiles} and \emph{separate polynomials}.


\subsection{Tabular Learning}\label{tabular}

A natural way to represent the state-action space is to preserve a distinct feature (and coefficient) for every unique state-action combination. Features are binary, i.e.\ any feature is  $1$ if the associated state-action combination is selected and $0$ otherwise:

\begin{gather}\label{cell_activation}
x_d^{Tabular} = \begin{cases}
1 & \quad \text{if } \{p_{1, t-1}, p_{2, t-1}, p_{1, t}\} \text{~corresponds to cell}_d  \\
0 & \quad \text{if } \{p_{1, t-1}, p_{2, t-1}, p_{1, t}\} \text{~does not correspond to cell}_d \\ \end{cases} 
\end{gather}

The respective coefficient tracks the performance over time and directly represents the \emph{value} of that state-action combination. Accordingly, the length of $\boldsymbol{x}$ is $m^3$.\footnote{$3$ derives from the 2 prices from the previous episode plus the considered action of the current episode.} This approach is called \emph{tabular} because it is easy to imagine a table where every cell represents a unique state-action combination. Tabular methods have been used extensively in the simulations on algorithmic collusion that provided empirical evidence of collusive outcomes being possible in simple environments (\textbf(citations)). Their widespread application is justified by their conceptual simplicity and their historic usage in autonomous pricing of airline fares and electricity markets \parencite{ittoo_algorithmic_2017}. Moreover, tabular methods give rise to a family of robust learning algorithms with well-understood convergence guarantees (\textbf{citation}) \footnote{Q-Learning being just one particular application.}.

However, tabular methods are not necessarily the best or fastest way to learn an optimal policy. In real life markets, a salient factor may impede its effectiveness. Prices are (quasi-) continuous - a treat completely ignored by tabular methods. This has two major implications. First, the leeway of decision makers is artificially restricted. Second, due to a \emph{curse of dimensionality}, learning speed and success may deteriorate disproportionately with $m$. I will take a closer look at each of these points.

Obviously, any decision maker is only restricted by the currency's smallest feasible increment and can charge more than just a couple of prices. It is certainly conceivable, maybe even desirable, that a decision maker reduces the own number of considered prices to simplify the decision process. However, in most cases it will be impossible to impose such a restriction on competitors. As an extreme example, consider an opponent who never charges the same price twice. Whenever this opponent introduces a new price, a tabular learning agent is coerced to create a new cell in the state-action matrix that will never be revisited. Consequently, the agent continuously encounters new situations from which it can learn, but never utilizes the acquired knowledge.\footnote{One could attempt to circumvent this problem by discretizing the prices \emph{ex ante} (e.g. as in \autoref{available_prices}) and simply convert the real price to the closest available alternative. While this introduces some imprecision and it is unclear how to optimally discretize prices, it might constitute a practicable solution. In fact, that approach is a special case of \emph{tile coding}, the method I will introduce in \autoref{tile_coding}.}

More importantly, tabular learning does not scale well with $m$ and $n$. In the baseline specification, the number of features is $19^3 = 6859$. Doubling $m$ from $19$ to $38$ causes an eightfold increase of that number to $54,872$. Even worse, increasing the number of competitors alters the exponent. Changing $n$ from $2$ to $3$ entails an increase of features by the factor $m$, in the baseline specification from $6859$ to $130,321$.\footnote{A similar problem arises when the algorithm is supposed to account for cost and demand factors. Every added input, whether due to an additional opponent or any other profit-related variable, increases the table by a factor or $m$. While changes in costs and prices are not considered in this study, they obviously play an important role in reality.} It is easy to see that modest increases in complexity have the potential to evoke a disproportionate reduction in learning speed. Indeed, \textcite{calvano_algorithmic_2018} show that increasing $m$ and $n$ tends to reduce profits of tabular learning agents.

Another way of looking at the same issue is to consider the nature of the variable $p$. Prices are continuous and transforming them into a qualitative set of discrete actions disregards that fact. In particular, it prevents the opportunity to learn from the result of charging a particular price about the quality of \emph{similar} prices in the same situation. To illustrate with an inflated example, consider a manager who observes large profits after charging a price of $1000$. A human manager is able to infer that charging $1001$ instead would have yielded a similar profit. Tabular learning agents are not.


\textbf{point at tradeoff: precision vs. learning speed}
			* approximate continuous prices by reducing price intervals to arbitrary length



\subsection{Function approximation methods}

The function approximation methods considered in this study alleviate the \emph{curse of dimensionality}. In fact, length of the feature vector $\boldsymbol{x}$ in \emph{tile coding} and \emph{polynomial tiling} is unaffected by $m$. For \emph{separate polynomials}, it is proportional to $m$, i.e.\ doubling the number of feasible prices also doubles the number of features. Moreover, all methods augment learning in the sense that a particular state-action combination tends to evoke ampler parameter updates (in accordance with \autoref{update_rule}) that also change the future evaluation of \emph{similar} state-action combinations.

\subsubsection{Tile Coding}\label{tile_coding}
In reinforcement learning, \emph{tile coding} is a common way to extract binary features from a state-action space.\footnote{for an extensive introduction with helpful illustrations refer to \textcite{sutton_reinforcement_2018}} Its appeal stems partly from the fact that it is a generalization of tabular learning. The idea is that several \emph{tilings} superimpose the state-action space. The $\mathcal{T}$ tilings are offset but each tiling covers the entire state-action space:

\begin{gather}
	 T^L \leq A^L  ~ \text{and} ~ T^U \geq A^U ~~ \forall  ~~ T \in \{1, 2, ..., \mathcal{T} ~ \} ~~ \text{,}
\end{gather}

where $T^L$ and $T^U$, respectively, represent the lower and upper bound of tiling $T$. Each tiling is itself composed of uniformly spaced out \emph{tiles}.\footnote{With 2 dimensions, a tiling simply corresponds to a grid. In our case, the state-action space is 3-dimensional, so it may prove more intuitive to think of cubes instead of tilings and tiles.} Every tile is uniquely demarcated by a lower and an upper threshold for every dimension. Consequently, the number of tiles per tiling is controlled by the number of thresholds. For this simulation, it suffices to define a single set of thresholds per tiling that applies to all 3 dimensions. More specifically, the thresholds are spaced out evenly in the tiling-specific interval $[T^L, T^U]$:

\begin{gather}
(
T^L,
T^L + \frac{1(T^U - T^L)}{\psi - 1},
T^L + \frac{2(T^U - T^L)}{\psi - 1}~ , ... , ~
T^L + \frac{(\psi-2)(T^U - T^L)}{\psi - 1},
T^U) ~~ \text{,}
\end{gather}

where $\psi$ represents the number of thresholds. This gives rise to $(\psi-1)^3$ tiles per tiling. As indicated, tiles are binary, i.e.\ if a state-action observation falls into a particular demarcation, the corresponding tile is \emph{activated}:

\begin{gather}\label{tile_activation}
x_d^{Tiling} = \begin{cases}
1 & \quad \text{if } \{p_{i, t-1}, p_{j, t-1}, p_{i, t}\} \text{~in tile demaraction}_d  \\
0 & \quad \text{if } \{p_{i, t-1}, p_{j, t-1}, p_{i, t}\} \text{~not in tile demarcation}_d \\ \end{cases} 
\end{gather}

Since tiles within a tiling are non-overlapping, any state-action combination activates exactly $\mathcal{T}$ tiles, one per tiling. The total number of features is simply $T~(\psi - 1)^3$. Note that the tabular case can be recovered as a special case by setting $\mathcal{T}~ = 1$ and $\psi \geq m + 1$. In this case, every tile is activated by at most one feasible state-action combination which is equivalent to storing a dedicated coefficient for every state-action combination.\footnote{If $\psi > m + 1$, some tiles would never be activated. But again, every table entry would correspond to a unique tile.}


\subsubsection{Polynomials}\label{polynomial}

\emph{Polynomial approximation} applies polynomial transformations to its inputs. In order to keep this (and the upcoming) section brief, I will introduce the notation for the specific case of 3 variables.\footnote{for a more thorough treatment with variations, see e.g.\ \autoref{hastie}} Polynomial approximation of order $k$ maps $S_t$ and $A_t$ to a set of features, where a single feature corresponds to:


\begin{gather}\label{polynomial_extraction}
x_d^{Poly} = p_{i, t-1}^{\kappa_{d,1}} ~ p_{j, t-1}^{\kappa_{d,2}} ~ p_{i, t}^{\kappa_{d,3}}
\end{gather}


Every combination of exponents that adheres to the restrictions

\begin{itemize}
	\item $0 < \kappa_{d,1} + \kappa_{d,2} + \kappa_{d,3} \leq k  ~~ \forall ~ d$ and
	\item $\kappa_{d,1}, \kappa_{d,2}, \kappa_{d,3} \in \{0, 1, ..., k\} ~~  \forall ~ d$
\end{itemize}

constitutes one feature. Using polynomial approximation, the feature vector $\boldsymbol{x}$ contains ${k + 3\choose3}  - 1$ elements. I chose not to use a simple polynomial to approximate the valuation of the entire state-action space. Exploratory runs have shown that the method has some trouble converging and frequently produced unreasonable results in the provided environment. Perhaps, this is not surprising because every state-action combination will always produce non-zero values for \emph{all} features and change every single element in $\boldsymbol{w}$. This makes it difficult for the algorithm to develop different notions for \emph{different} prices.

\subsubsection{Polynomial Tiles}

What I call \emph{polynomial tiles} is a blend of \emph{tile coding} and \emph{polynomial approximation}. To be precise, just as in tile coding, the state-action space is divided into overlapping tiles. However, instead of a binary indication, every tile comprises a distinct polynomial. For the sake of notation, it is helpful to divide the index $d$ into a tiling component $e$ and a polynomial part $f$.  Hence:

\begin{gather}\label{poly_tiling_extraction}
x_d^{Poly~Tiling} = x_{e,f}^{Poly~Tiling} = \notag \\
\begin{cases}
p_{i, t-1}^{\kappa_{f,1}} ~ p_{j, t-1}^{\kappa_{f,2}} ~ p_{i, t}^{\kappa_{f,3}} & \quad \text{if } \{p_{i, t-1}, p_{j, t-1}, p_{i, t}\} \text{~in tile demaraction}_e  \\
0 & \quad \text{if } \{p_{i, t-1}, p_{j, t-1}, p_{i, t}\} \text{~not in tile demarcation}_e \\ \end{cases} 
\end{gather}

The restrictions on the exponents $\kappa$ from \autoref{polynomial} apply. The method accompanies $T~(\psi - 1)^3 ({k + 3\choose3}  - 1)$ features. As this method allows for a distinguished value estimation for different state-action combinations within a tile, it appears reasonable to increase the size of the tiles in order to decrease the number of coefficients and avoid overfitting. Specifically, I retain the number of tilings, but reduce the number of tiles per tiling from $512$ to $64$ by imposing $\mathcal{T} = 5$ and $\psi = 5$. Moreover, I allow for polynomial combinations up to degree $k=4$.

\subsubsection{Separated Polynomials}

\emph{Separated polynomials} maintain for every action a distinct set of parameters that apply \emph{polynomial approximation} to the state set. In reinforcement learning, it is common to store a separate set of coefficients for every feasible action \textbf{citation}.\footnote{This approach is best suited if the action space is qualitative and the state space continuous. In this simulation, only the latter is strictly true. Therefore, the two issues inherent to tabular learning I have outlined in \autoref{tabular}, also apply to the action space of \emph{separated polynomials}, but not to the state space.} Since $A_t$ is fixed within each set, the polynomial only considers $S_t$:\footnote{The restrictions on $k$ are adjusted accordingly:
	\begin{itemize}
		\item $0 < \kappa_{d,1} + \kappa_{d,2}  \leq k  ~~ \forall d$ and
		\item $\kappa_{d,1}, \kappa_{d,2} \in \{0, 1, ..., k\} ~~  \forall d$
	\end{itemize}
}


\begin{gather}\label{separated_poly_extraction}
x_d^{Separated~Poly} = \begin{cases}
p_{i, t-1}^{\kappa_{d,1}} ~ p_{j, t-1}^{\kappa_{d,2}} & \quad \text{if } a = A_t  \\
0 & \quad \text{if } a \ne A_t \\ \end{cases} 
\end{gather}

Note that the method models the value of an action as a function of $S_t$. The number of encompassed features arises naturally as $m($ $k+2\choose2$ $-1)$.


Perhaps, an inverse variation could be more intuitive from an economic perspective. Consider that every permutation of $S_t$ holds a distinct set of parameters. This approach is closer to the notion of selecting $a$ to optimize the reward \emph{given} a fixed state set $s$. I leave this variation open as a potential avenue for future research.

\subsection{Parameter Grid}

TBD
	
	
\section{Results}
This section reports on the simulation outcomes of the baseline specification. To foreshadow the results, profits mostly exceed Nash-predictions, but remain below monopoly profits. While agents learn to charge supra-competitive prices, they fail to incorporate \emph{reward-punishment} schemes consistently. Overall, the results crucially hinge on the combination of feature extraction method and selected parameters. Only tabular learning exhibits a clear tendency to punish deviations with lower prices in subsequent periods.

I report results for various specifications and will refer to every unique combination of feature extraction method and parameters as an \emph{experiment}. Every experiment consists of 48 \emph{runs}, i.e. repeated simulations with the exact same set of starting conditions. Lastly, within the scope of a particular \emph{run}, time steps are called \emph{periods}.

\subsection{Convergence}\label{convergence}

\textbf{TBD: As indicated,} convergence is not guaranteed in a non-stationary environment, much less so with function approximation. Notwithstanding the lack of a theoretical convergence guarantee, prior experiments have shown that simulation runs tend to approach a stable equilibrium in practice (\cite{calvano_artificial_2019} \textbf{and others}). Note that \emph{stability} simply refers to the observation that the same set of prices continuously recur over a longer time interval. The strategies upon convergence need not coincide with economic theory. In fact, at times the observed outcomes in this study contradict predictions from game theory. For instance, despite symmetric profit functions, the converged outcomes may display asymmetric prices. Moreover, price cycles, i.e.\ a recurring sequence of price combinations, occur frequently.\footnote{The model from \autoref{quantity} predicts symmetric outcomes without cycles. This is typical for simultaneous pricing games, but not universal across economic models. For instance, collusive outcomes in quantity competition (i.e.\ Cournot) may exhibit price asymmetries. The relevance of that prediction has been fortified in experimental settings, e.g.\ in \textcite{fischer_collusion_2019}. \textcite{maskine-tirole} pioneer a sequential pricing game that predicts \emph{Edgeworth price cycles} where agents successively undercut each other until one firm prefers to reset the cycle and increases its price. Based on their model, \textcite{klein_autonomous_2019} shows that \emph{Q-Learning} agents are indeed capable of learning those dynamic strategies.}


The following, arbitrary but practical, convergence rule was employed. If a price cycle recurred for 10,000 consecutive episodes, the algorithm is considered \emph{converged} and the simulation concludes. A price cycle requires both agents' adherence.\footnote{Of course it is possible that the cycle length differs between agents. For instance, one agent may continuously play the same price while the opponent keeps alternating between two prices. In this case, the cycle length is $1*2=2$.}

For efficiency reasons, price cycles up to a length of 10 are considered and a check for convergence is undertaken only every 2,000 episodes. If no convergence is achieved until 500,000 episodes, the simulation stops and the run is deemed \emph{not converged}. Furthermore, there are a number of runs that \emph{failed to complete} as a consequence of the program running into an error. Unfortunately, the program code does not allow to examine the exact cause of such occurrences in retrospect. However, by and large, the failed runs occurred with unsuitable specifications (see below for a detailed discussion).

\begin{figure}
	\includegraphics[width=\linewidth]{plots/converged.png}
	\caption{Number of runs per experiments that achieved convergence as a function of $\alpha$.}
	\label{converged}
\end{figure}

In accordance with the outlined convergence criteria above, \autoref{converged} displays the share of runs that, respectively, converged successfully, did not converge until the end of the simulation or failed to complete. Two main conclusions emerge. First, failed runs are mainly prevalent in specifications with a high value of $\alpha$ in conjunction with a polynomial feature method. Second, the tiling methods are more likely to converge. Both points deserve some further exploration.

Regarding the failed runs, \textbf{recall} from \autoref{feature_extraction} that features of polynomial extraction are not binary and warrant cautious adjustments of the coefficient vector. I suspect that with unreasonably large values of $\alpha$, the estimates of $\boldsymbol{w}$ overshoot early in the simulation, don't recover and at some point exceed the software's numerical limits.\footnote{Controlled runs where I could carefully monitor the development of the coefficient vector $\boldsymbol{w}$ seem to confirm the hypothesis. However, isolated errors \emph{with} reasonable parameter settings remain unexplained, see the top right panel in \autoref{converged}.} While important to acknowledge, the failed runs are largely an artifact of unreasonable specifications and I will \textbf{disregard them for the remainder of this chapter}. For instance, the percentages in the subsequent paragraph don't account for the failed runs.

Out of the completed runs without program failure, 95.4\% did converge. Though there are subtle differences between feature extraction methods. With only one exception, both tiling methods converged consistently for various $\alpha$. With only 85.4\% of runs converging, separate polynomials constitute the other extreme. The figure also indicates that convergence becomes less likely for low values of $\alpha$. With tabular learning, 92.9\% of runs converged without clear relation to different values of $\alpha$.

\autoref{convergence_at} displays a frequency polygon of the runs that achieved convergence within 500,000 episodes. Clearly, the distribution is fairly uniform across feature extraction methods. Most runs converged between 200,000 and 300,000 runs. This is an artifact of the decay in exploration as dictated by $\beta$. Before the focal point of 200,000 is reached, agents probabilistically experiment too frequently to observe 10,000 consecutive episodes without any deviation from the learned strategies. Thereafter, it becomes increasingly likely that both agents keep \emph{exploiting} their current knowledge and continuously play the same strategy for a sufficiently long time to trigger the convergence criteria. Note that the low quantity of runs converging between 300,000 and 500,000 suggests that increasing the maximum of allowed episodes would not necessarily entail a significantly higher portion of converged runs.

\begin{figure}
	\includegraphics[width=\linewidth]{plots/convergence_at.png}
	\caption{timing of convergence, runs that did not converge or failed to complete are excluded. Width of bins: 8,000}
	\label{convergence_at}
\end{figure}

\autoref{cycle_length} visualizes the distribution of cycle length and offers some interesting insights. Unsurprisingly, a first glance suggests that the frequency of runs decreases with cycle length. Not accounting for differences between selection methods, the bars appear similar to a geometric distribution with the largest bar corresponding to a 'cycle length of 1' (i.e.\ no cycle at all). Moving towards the right, the frequency of observed runs decreases with cycle length, though at a decreasing pace. In fact, there are even 7 runs with the largest considered cycle length of 10.

Again, there are substantial differences between the different feature extraction methods. Polynomial tiling largely follows the described decaying pattern. Similarly, simple tile coding rarely converges in long cycles, though its spike of 194 runs corresponds to a cycle length of 2. Contrary, almost all runs of the separated polynomials converged without cycles.\footnote{Though barely visible in \autoref{convergence_at}, there are 2 runs with a cycle length of 2.}. Lastly, the frequency of cycle length of converged tabular runs is distributed almost uniformly. This observation also suggests that the employed convergence rule may well have misclassified some of the runs in the top left panel of \autoref{converged} as \emph{not converged} where in reality the convergence cycle length simply exceeded the threshold arbitrarily set at 10. 

\begin{figure}
	\includegraphics[width=\linewidth]{plots/cycle_length.png}
	\caption{Number of converged runs with particular cycle length.}
	\label{cycle_length}
\end{figure}

\autoref{prices} unveils the ranges of prices within a cycle. For now, I proceed by examining profits upon convergence.

\subsection{Profits}

In order to benchmark the simulation profits, I normalize profits similar to \textcite{calvano_algorithmic_2018}:

\begin{gather}
\Delta = \frac{\bar{\pi} - p_n}{p_m - p_n} ~~ \text{,}
\end{gather}

where $\bar{\pi}$ represents profits averaged over the last 100 time steps upon convergence and over both agents in a single run. The normalization implies that $\Delta = 0$ and $\Delta = 1$ respectively reference the Nash and monopoly solution. Note that it is possible to obtain a $\Delta$ below $0$ (e.g. if both agents charge prices equal to marginal costs), but not above $1$.\footnote{Strictly speaking, exactly 1 is not attainable either. Recall that $m$ was chosen to allow for prices very close, but not equal to both benchmark prices. With $m = 19$, the highest feasible $\Delta$ is 0.9997.} \autoref{alpha} displays the convergence profits as a function of the feature extraction method and $\alpha$. Every data point represents one experiment, more specifically the mean of $\Delta$ across all runs making up the experiment.

\begin{figure}
	\includegraphics[width=\linewidth]{plots/alpha.png}
	\caption{average $\Delta$ for various experiments. Includes converged and non-converged runs. One data point (poly tiling, $\alpha = 0.0004$) is excluded for better presentability. Beware the logarithmic x-scale.}
	\label{alpha}
\end{figure}

First of all, note that average profits consistently remain between both benchmarks $p_m$ and $p_n$ across specifications.\footnote{There is one exception. One data point is hidden in the plot to preserve reasonable y axis limits. More specifically, for the polynomial tiles and $\alpha = 0.0001$, the average $\Delta$ is -1.73. This extends the observation in \autoref{convergence}. It appears that this particular $\alpha$ constitutes a critical point. While the program does not crash, agents only learn strategies void of any reasonableness. As \autoref{alpha} displays, outcomes within the benchmarks are obtained by further decreasing $\alpha$.} As with prior results, the plot unveils salient differences between feature extraction methods.  On average, polynomial tiling runs yield the highest profits. The average $\Delta$ peaks at 0.85 for $\alpha = 10^{-8}$. Higher values of $\alpha$ tend to progressively decrease profits. Moving downwards on the y-axis, both the tabular method and tile coding yield similar average values of $\Delta$. Furthermore, the level of $\alpha$ does not seem to impact $\Delta$ much. For both methods $\alpha = 10^{-4}$ induces the highest average $\Delta$ at 0.487 and 0.478 respectively. Similarly for separated polynomials, $\Delta$ does not seem to respond to variations in $\alpha$. The maximum $\Delta$ is 0.35.

Naturally, averaging $\Delta$ over all runs of an experiment, as done to create \autoref{alpha}, potentially hides subtleties in the distribution of $\Delta$. Therefore, \autoref{alpha_violin} displays a violin plot that shows the distribution of $\Delta$ per experiment. The distribution largely confirms the conclusion that most runs converge between $\Delta_m$ and $\Delta_n$. The only method with a significant quantity of runs with profits below the Nash benchmark are the separated polynomials. Overall, 19.8\% of runs converged with profits below the Nash equilibrium, though most of them ended up reasonably close. The percentage is largest for the experiment with $\alpha = 10^{-8}$: 27.1\%. While the other methods tend to elicit runs within the set up benchmarks, the variability remains quite high. This indicates a degree of path dependence and suggests that the algorithms are prone to stick to early explored strategies that are \emph{above-average}, but \emph{sub-optimal}. Polynomial tiles exhibit the narrowest $\Delta$ range, in particular for low $\alpha$.

\begin{figure}
	\includegraphics[width=\linewidth]{plots/alpha_violin.png}
	\caption{distribution of $\Delta$ for various experiments. Includes converged and non-converged runs. Violin widths are scaled to maximize width of single violins, comparisons of widths between violins are not meaningful. Violins are trimmed at smallest and largest observation respectively. One violin (poly tiling, $\alpha = 0.0004$) is excluded for better presentability. Horizontal lines represent the median. Beware the logarithmic x-scale.}
	\label{alpha_violin}
\end{figure}


\autoref{convergence} and \autoref{alpha} established that, what constitutes a sensible value of $\alpha$ clearly depends on the feature extraction method. Hence, for the remainder of this chapter, I will select an 'optimal' $\alpha$ for every feature extraction method and present further results only for these combinations. In determining \emph{optimality} of $\alpha$, I don't rely on a single hard criteria, rather I consider a number of factors including the percentage of converged runs, comparability with previous studies and prefer to select experiments with high average $\Delta$ as they are most central to the purpose of this study. \autoref{justifications} provides a justification for every experiments setting deemed \emph{optimal}. To get a sense of the variability of runs within the experiments and the price trajectory over time, \autoref{trajectory_Delta} displays the development of profits of all runs for the optimal values of $\alpha$. Moreover, \autoref{appendix} contains further trajectory visualizations of prices and profits.

\begin{center}
	\begin{table}
		
		\begin{tabular}{|l|c|l|}
			\hline
			\textbf{Feature Extraction Method}&$\boldsymbol{\alpha}$&\textbf{justification} \\
			\hline
			Tabular&0.1&- comparability with previous simulation studies \\
			&&- most pronounced response to price deviations \\
			&& \ \ (see \autoref{deviations}) \\
			\hline
			Tile Coding&0.001&- high $\Delta$ \\
			&&- most pronounced response to price deviations \\
			&&\ \ (see \autoref{deviations}) \\
			\hline
			Separated Polynomials&$10^{-6}$&- high percentage of converged runs \\
			\hline
			Polynomial Tiles&$10^{-8}$&- high $\Delta$ \\
			\hline
		\end{tabular}
		\caption{\emph{Optimized} values of $\alpha$ by feature extraction method}
		\label{justifications}
	\end{table}
\end{center}


\begin{figure}
	\includegraphics[width=\linewidth]{plots/trajectory_Delta.png}
	\caption{distribution of $\Delta$ over time in 'optimized' experiments. For individual runs, $\Delta$ is averaged over 50,000 periods apiece and both players. Plot includes converged and non-converged runs. Violin widths represent quantity of active runs at $t$ which enables comparisons between violins. As most runs converge after 200,000 to 300,000 episodes, violin widths decrease thereafter. Violins are trimmed at smallest and largest observation respectively. Horizontal lines represent the median.}
	\label{trajectory_Delta}
\end{figure}

\textbf{paragraph on differences between player profits?}

\subsection{Price Ranges}\label{prices}

As established in \autoref{convergence}, many simulations converge in price cycles of various lengths. \autoref{price_range} plots the range between the lowest and highest price a single agent charges in a cycle upon convergence. Naturally, the price range is null if no cycle is present. Perhaps unsurprisingly, the price range then tends to increase with cycle length, at times becoming remarkably high. The range of prices due to tabular learning frequently exceeds the range between collusive and Nash prices. This is a clear indication that price setting is occasionally irrational. Irrespective of agents competing or colluding, prices outside this range are not economically optimal. Recall from \autoref{convergence} that other feature extraction methods tend to yield lower cycle lengths. \autoref{price_range} extends that observation with the insight that those methods also produce lower price ranges. However, the inversion of the previous argument is dangerous. One should not deduct that behavior is \emph{closer to optimal} from the mere fact that prices appear more stable. In fact, the next section shows that the strategies learned with function approximation are often far from optimal and easy to exploit. Potential consequences of frequent price changes of significant magnitude are discussed \textbf{in section X}.

\begin{figure}
	\includegraphics[width=\linewidth]{plots/price_range.png}
	\caption{Price range of individual runs. Every point represents a particular run. Within groups, points are spaced out horizontally. Price range is defined as the difference between the highest and lowest price an agent charges within a cycle. Relationships to the opponent's prices are not examined. Only converged runs are considered (as cycle length is unavailable for other runs). Dashed line represents the difference between collusive and Nash outcome (i.e.\ $p_m - p_n$).}
	\label{price_range}
\end{figure}


\subsection{Deviations}\label{deviations}

This section examines whether the learned strategies are stable in the face of deviations from the learned behavior. There are at least two explanations for the existence of supra-competitive outcomes. First, agents simply fail to learn how to compete effectively and miss out on opportunities to undercut their opponent. Second, agents avoid deviating from the stable strategy because they fear retaliation and lower (discounted) profits in the long run. \textbf{As outlined before}, only the latter cause, some form of a \emph{reward punishment scheme}, allows to label the supra-competitive outcomes as \emph{collusive} and warrants attention from competition policy \parencite{assad_algorithmic_2020}. Therefore, the following \emph{deviation experiment} was conducted to scrutinize whether agents learn to actually punish deviations. Denote the period in which convergence was detected as $\tau = 0$. At this point, both agents played for 10,000 episodes an equilibrium strategy they mutually regard as optimal (on path). At $\tau = 1$, I force one agent to deviate from her learned strategy and play instead the short-term best response that mathematically maximizes profits. Subsequently, she reverts again to the learned strategy. In order to verify whether the non deviating agent proceeds to punish the cheater, he sticks to his learned behavior throughout the deviation experiment. The \emph{deviation experiment} lasts 10 episodes in total. Learning and exploration are disabled (i.e.\ $\alpha = \epsilon = 0$).\footnote{See \textbf{section TBD} for prolonged deviations and continued learning \emph{after} detected convergence.} In order to evaluate the deviation, it appears useful to define a \emph{counterfactual} situation where both agents stick to their learned strategies. Comparing (discounted) profits between the experiment and the counterfactual allows to assess the profitability of the deviation.

\textbf{words on punishment strategies: grim trigger vs. slow reversal }

\begin{figure}
	\includegraphics[width=\linewidth]{plots/average_intervention.png}
	\caption{Average price trajectory around deviation. }
	\label{average_intervention}
\end{figure}

As the responses to one agent's deviation vastly differ across feature extraction methods, it is natural to discuss them separately at first and contrast differences only thereafter.  It is difficult to summarize all information in a single graph or table, so I will consult \autoref{average_intervention}, \autoref{intervention_boxplot} and \autoref{share_deviation_profitability} simultaneously to describe the deviation and response patterns. Before that, a brief description of the plots is in line. \autoref{average_intervention} displays the price trajectory around the forced deviations averaged over all runs of an experiment.\footnote{To reiterate the result from the previous section, the plot reinforces that the price variation between periods is non-negligible even \emph{before} the deviation takes place - despite averaging over all runs of the optimal experiments.}  Since the average price trajectory might veil important differences between runs, \autoref{intervention_boxplot} illustrates the range of deviation and punishment prices compared to the counterfactual price that would have materialized if no deviation had taken place and agents kept following their learned strategies.\footnote{\autoref{intervention_profit_boxplot} in \autoref{appendix} shows a similar plot for profits.} Note that in the presence of price cycles, part of the variation can be explained by \emph{cycle shifting}, a phenomenon where the agents return to the learned cycle but the intervals are not aligned with the counterfactual path. These differences should even out over all runs of an experiment and therefore, not systematically bias the boxes in either direction. Similarly, the average price response in \autoref{average_intervention} is largely unaffected by this phenomenon. Finally, \autoref{share_deviation_profitability} reports the share of deviations that turned out to be profitable compared to the counterfactual.\footnote{\autoref{intervention_profitability_polygon} in \autoref{appendix} displays a more detailed frequency polygon to gauge how much more or less profitable the deviation is compared to the counterfactual of sticking to the learned strategy.}

\begin{figure}
	\includegraphics[width=\linewidth]{plots/intervention_boxplot.png}
	\caption{distribution of prices at and after deviation relative to alternative path \emph{without} forced deviation, i.e.\ the difference to the price at the same $\tau$ had no deviation taken place. Only includes converged runs because a clear counterfactual exists. Boxes demarcate 15th and 85th percentiles and are extended by whiskers that mark the entire range of price differences. Horizontal lines represent the group median.}
	\label{intervention_boxplot}
\end{figure}

Most importantly, only tabular learning evokes a conspicuous punishment from the non deviating agent. At $\tau = 2$, the non deviating agent tends to match, arguably even undercut, the deviation price whereas the deviating agent already begins reverting to pre-deviation prices. Though this result's general validity is qualified. \autoref{intervention_boxplot} unveils that the non deviating agent does not always reduce prices compared to the counterfactual. Despite the existence of punishment prices in some runs, agents are fairly quick to return to the price levels observed before the deviation was forced upon them. As early as $\tau = 4$ there is no visible difference between average pre- and post-deviation price levels.\footnote{Previous studies showcase a strong deviation is usually followed by a more gradual reversion to pre-deviation behavior (around 5-10 episodes), see in particular Figure 4 in \textcite{calvano_algorithmic_2018} and Figure 3 in \textcite{klein_autonomous_2019}.} This might partly follow from prices being relatively close to the Nash equilibrium in the first place. The punishments ensure that deviating is (strictly) profitable in only 18\% of runs. This suggests that, upon convergence, agents stick to a stable equilibrium, from which deviations tend to be unprofitable due to the cheated agent retaliating.

\begin{center}
	\begin{table}
		% latex table generated in R 3.6.1 by xtable 1.8-4 package
% Sat May 15 18:22:24 2021
\begin{tabular}{llrr}
  \hline
feature\_method & player & share profitable & share unprofitable \\ 
  \hline
tabular & deviating agent & 0.17 & 0.52 \\ 
  tabular & non deviating agent & 0.04 & 0.65 \\ 
  tiling & deviating agent & 0.56 & 0.33 \\ 
  tiling & non deviating agent & 0.06 & 0.83 \\ 
  poly-separated & deviating agent & 0.69 & 0.00 \\ 
  poly-separated & non deviating agent & 0.02 & 0.67 \\ 
  poly-tiling & deviating agent & 0.96 & 0.02 \\ 
  poly-tiling & non deviating agent & 0.00 & 0.98 \\ 
   \hline
\end{tabular}

		\caption[Share of profitable deviations]{Share of profitable and non-profitable deviations by agent and feature extraction method. Deviations are deemed \emph{profitable} if the discounted profits until $\tau = 10$ due to the deviation exceed cash flows from a counterfactual without deviation. Only includes converged runs because a clear counterfactual exists. Discounting is equivalent paramount to $\gamma$ in \autoref{td_error_expected}, i.e.\ 0.95. A significant number of 'deviations' are neither profitable nor unprofitable. In those runs, the learned strategy of the deviating agent is actually the best response at $\tau = 1$ and both agents keep following their respective price cycle.}
		\label{share_deviation_profitability}
	\end{table}
\end{center}

When examining the outcomes of the other feature extraction methods, different conclusions emerge. Recall from \autoref{alpha} that tile coding yielded convergence profits very similar to tabular learning. Yet, \autoref{average_intervention} and \autoref{intervention_boxplot} only hint at slight punishments in some runs. In fact, the median of the cheated agent's price at $\tau = 2$ is exactly 0, which amounts to a complete absence of a response. This lack of punishment renders 56\% of the cheater's deviations profitable. In light of that, it is surprising that the cheating agent tends to return to pre-intervention price levels instead of continuing to exploit her opponent's failure to punish deviations.

This is even more true for the feature extraction method utilizing separate polynomials. \autoref{convergence} outlined that this method typically did not converge in price cycles but a single, continuously played price instead. (In that sense, the opponent's behavior is more predictable and deviations easier to undertake.) The deviation impulse and responses are easy to summarize. In all runs, after the forced intervention at $\tau = 1$, both agents immediately return to the pre-deviation equilibrium. This is remarkable for two reasons. First, the non deviating agent completely fails to punish the cheater's behavior and does not respond to the price cut whatsoever. Consequently, 72\% of the deviations are profitable.\footnote{The remaining 31\% comprise runs where the deviating agent was already playing the short-term best response. Remember that the separated polynomial method tends to result in prices at or close to the Nash equilibrium.} This leads to the second point. Despite the obvious advantage of cheating, the deviating agent returns to the pre-deviation price without exception, thus failing to exploit her opponent's weakness. To be fair, the initial price levels are fairly close to the Nash equilibrium and the deviation's profitability is relatively small compared to the potential gains realizable in other experiments (see also \autoref{intervention_profitability_polygon} in \autoref{appendix}). Still, it is puzzling that such a simple strategy improvement remains consistently untapped. In conclusion, the agents' failure to play economically sound strategies casts doubts on the viability of the feature extraction method in reality.

Finally, turn your attention to the experiment with polynomial tiling. Recall that this experiment generated outcomes closest to perfect collusion. By and large, the deviation experiment for polynomial tiling is similar to the one with separated polynomials, but there are some variations between runs that warrant detailed examination. Consider first the non deviating agent. Again, the majority of runs exhibits a failure to respond to the price cut. However, selected runs show a \emph{matching} strategy where the cheated agent meets the price cut with a similar price. Notably, in those circumstances, agents \emph{do not return} to the previously learned path but quickly establish a new equilibrium. Moreover, note that \autoref{intervention_boxplot} displays a slight bias downwards over all periods. This is indicative of \emph{continued cheating} of the deviating agent. After being forced to undercut the price, she proceeds to set prices below pre-deviation levels without getting punished. This, too, results in a new equilibrium. \autoref{intervention_poly_tiling} in \autoref{appendix} illustrates both phenomena (price matching and continued cheating) through the exact price sequence of exemplary runs. In light of high pre-deviation prices and the lack of retaliatory prices, it is unsurprising that 96\% of deviations are profitable. The conclusions for polynomial tiling are similar to the separated polynomials. Baring a few exceptions, the non deviating agent fails to respond to a price cut and is easy to exploit. On the other hand, the deviating agent tends to leave that weakness unexploited.
Overall the deviation exercise suggests that while algorithmic agents manage to sustain high prices when playing each other, their strategies are incomplete and easy to exploit.

\textbf{summary of all methods}

Evidently, under the regime of this study's simulations, tabular learning is better in producing stable supra-competitive outcomes than the functional approximation methods. To illustrate the notion of \emph{stability} in this context, consider the following thought experiment of a \emph{superagent}. Upon convergence, a rational player with perfect information on the economic environment and the learned best responses of both agents enters the game and takes over pricing authority from one of them.\footnote{For the sake of the argument it is irrelevant whether this superagent is human or not.} Importantly, the superagent could anticipate the opponent's price in the next period and calculate the short-term maximizing response as well as the opponent's reaction to the deviation and so on. When playing against a tabular learning agent, the superagent would deliberately stick to the convergence pricing scheme as a cheating surely evokes a retaliation rendering a deviation unprofitable (see \autoref{share_deviation_profitability}). Contrary, when facing an opponent who learned utilizing a functional approximation method, the superagent could easily cheat on the opponent to increase short-term profits without being punished in subsequent periods.

The evidence also suggests that functional approximation methods create hesitation in the agents to change best responses. Probabilistically, \emph{exploration} ensures that both agents will undercut the price of their opponent and realize excess profits similar to those in the forced deviation experiment. However, it appears that agents fail to learn (enough) from such \emph{explored cheating}. As evidenced by the undertaken deviation experiment, they typically return to the prior price (cycle) immediately. This rigidity in adjusting strategies potentially points to a problem with the specific algorithm or the tuning of its parameters. For instance, a higher $\alpha$ could enable the cheater to learn faster that an unpunished deviation is more profitable than adhering to the learned strategy. 

Recall that learning and exploration were turned off for the deviation experiment. This enables an objection to the presented results. The non deviating agent, stripped of his ability to adjust his strategy, might only be exploitable for a finite number of episodes until he adjusts his strategy. In fact, a generosity to condone isolated price cuts might be conducive to establishing high price levels early in the simulation runs. However, \autoref{prolonged deviation} demonstrates that the lack of punishment in response to a deviation remains ubiquitous in prolonged deviation experiments with enabled learning ($\alpha > 0$).

\subsection{responses off equilibrium}

\textbf{TBD}




	
	\section{Robustness and variations}\label{robustness}


\subsection{Prolonged deviations}

To extend the mixed results from \autoref{deviations}, I conducted another \emph{prolonged deviation} experiment with continued learning. As I will show, the previously drawn conclusions remain intact. However, first I briefly explain why a continued intervention could theoretically be different from a one-time deviation. The critical components are continued learning and the eligibility trace vector $\boldsymbol{z}$ that make conceivable an agent tolerating isolated deviations but punishing longer price cuts.

Without eligibility traces and the ability to learn, a one-time deviation suffices to assess retaliatory behavior because the memory is too short to \emph{remember} that the opponent cheated for longer than a single period. Likewise, stripping the non deviating agent of his ability to update $\boldsymbol{w}$ renders him unable to learn that tolerating deviations is exploitable and culminates in low rewards. Consequently, if he failed to punish a deviation at $\tau = 2$, he will not react at $\tau = 3$ either. On the contrary, with the ability to learn enabled, both agents can readjust the parameter updates. For instance, after discovering that tolerating a one-time deviation yields a low reward, the non deviating agent might adjust $\boldsymbol{w}$ and decide to play a different action next time he is cheated (e.g.\ match or punish the price cut). This is augmented by the length of deviation episodes and the existence of eligibility traces. If the deviating agent \emph{continues} to cheat, the opponent should continue to decrease the valuation of the \emph{tolerating} strategy and could ultimately fall back to the next best action (which might be a price cut).\footnote{Furthermore, remember that the deviation experiment is conducted right after convergence was detected. Consequently, the algorithm was \emph{on-path} for a large number of episodes and the eligibility traces have not been reset recently (see \autoref{eligibility_trace_update}). Therefore, after the deviation experiment concludes, the convergence equilibrium might not be feasible anymore because its valuation by the agents changed.}

The \emph{prolonged deviation} lasts 20 episodes in total. It was set up as follows. The deviating agent anticipates the price of her opponent perfectly and continuously plays the best response for a total of 10 periods of cheating. This relies on the assumption that she is capable of perfectly predicting her opponent's response to her initial deviation. Exploration remains disabled ($\epsilon = 0$) but both agents continue learning from their actions.\footnote{I also prescribe that the forced deviation is considered \emph{on-policy}. Since $\epsilon = 0$, this is most natural to incorporate.} After I stop forcing the deviating agent to play the best response, both agents play another 10 episodes adhering to their learned strategies. Exploration remains disabled and learning continues until the very last episode.

\begin{figure}
	\includegraphics[width=\linewidth]{plots/average_prolonged_intervention.png}
	\caption{Average price trajectory around prolonged deviation.}
	\label{average_prolonged_intervention}
\end{figure}

\autoref{average_prolonged_intervention} displays the average price trajectory around the prolonged deviation and confirms the previous deductions. Only with tabular learning does the non deviating agent matches the price cuts systematically. The top left panel shows both agents hover around the Nash benchmark for the deviation period. Clearly, this is unprofitable for both agents but a necessary punishment to sustain supra-competitive prices in the first place. Note also the quick return to pre-deviation levels as soon as the deviating agent returns to her learned behavior. This illustrates that the supra-competitive outcomes remain sustainable in the face of persistent interruptions.

With regard to the three function approximation methods, the deviating agent appears to systematically exploit her opponent who fails to punish the price cut. The subtle differences between methods extend to the prolonged deviation. Separated polynomials evoke no response at all from the non deviating agent. Both tiling methods show a small \emph{average} price cut over the duration of the prolonged deviation. Again, averaging over all runs of an experiment veils important subtleties. Indeed, \autoref{prolonged_intervention_boxplot} uncovers that only isolated runs exhibit the non deviating agent cutting the pre-deviation price levels. A reaction is absent in most runs which enables continuous exploitation by the deviating agent. Despite that, the latter tends to return to pre-deviation price levels. Therefore, both agents act far from optimal (in the economic sense of the word) and fail to learn (enough) from the prolonged deviation experiment. Lastly, note the difference between pre- and post-deviation price levels at the bottom right panel, representing polynomial tiles. As noted previously, this suggests that the agents proceed to play a different, less profitable equilibrium after the deviation. This easy switch to a new strategy impugns the viability of the pre-deviation equilibrium in the first place.

It is conceivable, maybe even likely, that the non deviating agent does alter its strategy after a time frame much longer than 10 episodes. However, this is of no importance for this study because the agent's strategy is easy to exploit in the short term and, due to discounting, the deviations are profitable (see \autoref{share_deviation_profitability}).


\subsection{Parameter variations}\label{vary_parameter}

Besides the learning rate $\alpha$, the exploration strategy is arguably the most important steering choice in reinforcement learning. As discussed, $\beta$ controls the decay in exploration over time. I run a number of experiments varying $\beta$ while keeping the manually optimized values of $\alpha$ constant (see \autoref{justifications}).\footnote{Note that these values are not necessarily 'optimized' for alternative $\beta$. Ideally, exploration rate and learning speed should not be considered in isolation. Indeed, \textcite{calvano_algorithmic_2018} show that lower values of $\alpha$ perform better if exploration is extensive. However, the scope of this study does not allow to systematically search over a 2-dimensional grid of $\alpha$ and $\beta$.} \autoref{beta} displays that the impact of exploration on $\Delta$ is relatively small across feature extraction methods. However, applying the deviation routine described in \autoref{deviations} uncovers notable differences with respect to \emph{incentive compatibility}. \autoref{share_deviation_profitability_beta} clearly shows that cheating becomes less profitable when the opponent had more opportunities to explore actions. This is most apparent with tabular learning where the share of profitable deviations ranges from 39\% at $\beta = 0.00016$ to 6\% at $\beta= 2*10^{-5}$. \autoref{average_intervention_beta_tabular} in \autoref{appendix_2} extends the observation and shows that punishment severity and length increase with extended exploration.

\begin{figure}
	\includegraphics[width=\linewidth]{plots/beta.png}
	\caption{average $\Delta$ for various experiments. Includes converged and non-converged runs.}
	\label{beta}
\end{figure}


\begin{center}
	\begin{table}
		% latex table generated in R 3.6.1 by xtable 1.8-4 package
% Tue Jun 01 16:42:24 2021
\begin{tabular}{llrrrr}
  \hline
FEM & agent & $\beta =$ 0.00016 & $\beta =$ 8e-05 & $\beta =$ 2e-05 & $\beta =$ 1e-05 \\ 
  \hline
Tabular & deviating & 0.39 & 0.38 & 0.06 & 0.08 \\ 
  Tabular & non deviating & 0.20 & 0.09 & 0.00 & 0.04 \\ 
  Tile Coding & deviating & 0.50 & 0.54 & 0.56 & 0.67 \\ 
  Tile Coding & non deviating & 0.04 & 0.08 & 0.04 & 0.00 \\ 
  Separate Polynomials & deviating & 0.58 & 0.67 & 0.89 & 0.90 \\ 
  Separate Polynomials & non deviating & 0.00 & 0.00 & 0.00 & 0.00 \\ 
  Polynomial Tiles & deviating & 1.00 & 0.92 & 0.92 & 0.91 \\ 
  Polynomial Tiles & non deviating & 0.02 & 0.02 & 0.02 & 0.00 \\ 
   \hline
\end{tabular}

		\caption{Share of profitable deviations by agent and feature extraction method. Deviations are deemed \emph{profitable} if the discounted ($\gamma = 0.95$) profits due to the deviation until $\tau = 10$  exceed cash flows from a counterfactual without deviation. Only includes converged runs because a clear counterfactual exists.}
		\label{share_deviation_profitability_beta}
	\end{table}
\end{center}

I briefly explore how the choice of $\lambda$ affects the simulation outcomes in \autoref{appendix_2}.

\subsection{Price grid}

\autoref{feature_extraction_summary} emphasized that the length of the parameter vector $\boldsymbol{w}$ with tabular learning increases disproportionately with $m$. Likewise, the optimization problem is likely to become more complex. On the other hand, the feature extraction mechanisms of tile coding and polynomial tiles are largely unaffected by $m$. To gauge the effect on outcomes, I executed experiments with additional variations of $m=10$, $m = 39$ and $m = 63$.\footnote{As before, the odd numbers are chosen to enable prices close to $p_m$ and $p_n$} Due to computational restrictions, these experiments  only comprise 16 runs. Accordingly, inference should be treated with care.

\begin{figure}
	\includegraphics[width=\linewidth]{plots/m.png}
	\caption{average $\Delta$ for various experiments. Includes converged and non-converged runs.}
	\label{m}
\end{figure}

Unsurprisingly, convergence becomes less likely when $m$ increases. While all runs with $m=10$ converged, the percentage for $m=39$ and $m=63$ is only  67.2\% and 57.8\% respectively. This is driven mainly by less converged runs in \emph{tabular learning} and \emph{tile coding}.\footnote{See \autoref{converged_m} in \autoref{appendix_2}.} \autoref{m} illustrates that varying $m$ does not seem to have much of an impact on the average of $\Delta$. 
\textbf{a little surprising?}. 

However, the number of feasible prices appears to impact the stability of the equilibrium. \autoref{share_deviation_profitability_m} suggests that the share of profitable deviations increases with $m$ for tabular learning and the polynomial tiles method.\footnote{Regarding polynomials tiles, the portion might actually be higher as some deviations lead to new equilibria with lower prices and profits, but the counterfactual comparison only takes into account 10 episodes.} This is supported further by the fact that punishments seem to be strongest with $m = 10$ (see \autoref{average_intervention_m_10} in \autoref{appendix_2}).

\begin{center}
	\begin{table}
		% latex table generated in R 3.6.1 by xtable 1.8-4 package
% Fri May 28 22:34:58 2021
\begin{tabular}{llrrrr}
  \hline
feature\_method & agent & m = 10 & m = 19 & m = 39 & m = 63 \\ 
  \hline
tabular & deviating & 0.12 & 0.18 & 0.42 & 0.67 \\ 
  tabular & non deviating & 0.00 & 0.04 & 0.08 & 0.33 \\ 
  tiling & deviating & 0.69 & 0.56 &  &  \\ 
  tiling & non deviating & 0.00 & 0.06 &  &  \\ 
  poly-separated & deviating & 0.88 & 0.72 & 0.93 & 0.87 \\ 
  poly-separated & non deviating & 0.00 & 0.02 & 0.00 & 0.20 \\ 
  poly-tiling & deviating & 0.69 & 0.96 & 1.00 & 0.94 \\ 
  poly-tiling & non deviating & 0.06 & 0.00 & 0.00 & 0.00 \\ 
   \hline
\end{tabular}

		\caption{Share of profitable deviations by agent and feature extraction method. Deviations are deemed \emph{profitable} if the discounted ($\gamma = 0.95$) profits due to the deviation until $\tau = 10$  exceed cash flows from a counterfactual without deviation. Only includes converged runs because a clear counterfactual exists.}
		\label{share_deviation_profitability_m}
	\end{table}
\end{center}



Recall from \autoref{price_grid_formula} that $\zeta$ controls the available excess range above the fully collusive price $p_m$. These prices are inferior to $p_m$ in almost any situation and the simulations confirm that few runs converge in \emph{supra-collusive} prices. Still, \emph{ex ante} the effect of $\zeta$ on outcomes is ambiguous and may differ between feature extraction methods. I will briefly sketch some of the possible effects.

Most importantly, an increase in $\zeta$ increases the share of available prices above $p_m$ and decreases the share of \emph{viable} prices within the range of $p_m$ and $p_n$. Consequently, the agents may quickly discard a larger share of actions engendering low (or negative) rewards and \emph{narrow down} the range of reasonable actions between $p_m$ and $p_n$. Then, with fewer available actions, the optimization within that range might be facilitated. This might be particularly important with the separated polynomials method because agents could learn that certain polynomials associated with actions above $p_m$ (or below $p_n$) consistently yield low rewards - irrespective of the preceding state set, refrain from playing them early in the simulation and focus on refining the polynomials of actions within the range of $p_n$ and $p_m$.

There is an additional effect on the both tiling methods. The thresholds of tiles derive from the size of the action space. Consequently, all tiles are relocated and some actions will be associated with different tiles. \emph{A priori}, the effect on outcomes is hard to predict.

 \begin{figure}
	\includegraphics[width=\linewidth]{plots/zeta.png}
	\caption{average $\Delta$ for various experiments. Includes converged and non-converged runs.}
	\label{zeta}
\end{figure}

I conducted three additional experiments with $\zeta \in \{0.1, 0.5, 1.5\}$ to assess the impact of varying $zeta$ while keeping $m=19$ to ensure comparability between experiments. \footnote{Note however, different $zeta$ may prohibit playing actions very close to $p_n$ or $p_m$. For instance, with $\zeta = 1.5$, the price closest to $p_n = 1.473$ ($p_m = 1.925$) is $1.454$ ($1.908$). The distances are quite a bite higher than in the default specification.} \autoref{zeta} illustrates that $\zeta$ significantly influences profits upon convergence. Across feature extraction methods, the average $\Delta$ increases with $\zeta$. Moreover, \autoref{zeta_violin_prices} confirms that average prices upon convergence largely remain within the Nash and collusive benchmarks. Polynomial tiles constitute the only exception which further discredits the method as appropriate for the learning task.

\begin{figure}
	\includegraphics[width=\linewidth]{plots/zeta_violin_prices.png}
	\caption{distribution of average prices (over both players and cycle steps) for various experiments. Includes converged and non-converged runs. Violin widths are scaled to maximize width of single violins, comparisons of widths between violins are not meaningful. Violins are trimmed at smallest and largest observation respectively. Horizontal lines represent the median.}
	\label{zeta_violin_prices}
\end{figure}

To reiterate, prices close to the collusive solution are not necessarily evidence of a stable equilibrium with a \emph{reward-punishment} scheme. If anything, the simulation runs in this study have suggested the opposite. It turns out that despite the differences in $\Delta$, the stability of the learned strategies does is not heavily influenced by $\zeta$. \autoref{share_deviation_profitability_zeta} does not show prominent trends in the share of profitable deviations. As further evidence, \autoref{average_intervention_zeta_tabular} in \autoref{appendix_2} shows that tabular learning agents tend to punish deviations with retaliatory prices for all considered variations of $\zeta$. Similarly, the absence of competitive reactions to the punishment for the other feature extraction methods does not seem to hinge on $\zeta$.

\begin{center}
	\begin{table}
		% latex table generated in R 3.6.1 by xtable 1.8-4 package
% Fri May 28 22:35:14 2021
\begin{tabular}{llrrrr}
  \hline
FEM & agent & $\zeta =$ 0.1 & $\zeta =$ 0.5 & $\zeta =$ 1.0 & $\zeta =$ 1.5 \\ 
  \hline
tabular & deviating & 0.24 & 0.21 & 0.18 & 0.31 \\ 
  tabular & non deviating & 0.07 & 0.12 & 0.04 & 0.04 \\ 
  tiling & deviating & 0.38 & 0.42 & 0.56 & 0.44 \\ 
  tiling & non deviating & 0.04 & 0.02 & 0.06 & 0.08 \\ 
  poly-separated & deviating & 0.46 & 0.57 & 0.72 & 0.60 \\ 
  poly-separated & non deviating & 0.00 & 0.00 & 0.02 & 0.00 \\ 
  poly-tiling & deviating & 0.83 & 0.96 & 0.96 & 0.84 \\ 
  poly-tiling & non deviating & 0.10 & 0.08 & 0.00 & 0.09 \\ 
   \hline
\end{tabular}

		\caption{Share of profitable deviations by agent and feature extraction method. Deviations are deemed \emph{profitable} if the discounted ($\gamma = 0.95$) profits due to the deviation until $\tau = 10$  exceed cash flows from a counterfactual without deviation. Only includes converged runs because a clear counterfactual exists.}
		\label{share_deviation_profitability_zeta}
	\end{table}
\end{center}




\subsection{Discount factor}

In dynamic oligopolies, theory ascribes great importance to the discount factor $\gamma$. Typically, there exists a critical value below which the weight on future profits becomes too low to sustain any collusive behavior. Likewise, if $\gamma$ is sufficiently high, rational actors with full information will collude on the monopoly solution. In reality, there are various reasons why decision makers may end up charging prices between both extremes. For instance, they might not be fully aware of what exactly the benchmark prices are and might struggle to communicate and agree on a joint action (explicitly or tacitly). Similarly, in reinforcement learning, it is unlikely that there exists a strict dichotomy between fully collusive and perfectly competitive agents. Indeed, the results so far suggests that many intermediate levels are realistic. Nevertheless, with lower values of $\gamma$, less weight is put on the (expected) value of the future state in \autoref{td_error_expected} and the immediate reward $R_t$ gains relative importance.

To gauge the effect of $\gamma$ on outcomes, I conducted a series of experiments ranging from perfectly myopic ($gamma =0$) to infinitely patient ($\gamma = 1$) agents.\footnote{While $\gamma = 1$ is usually easy to model in economics, it is highly problematic in continuing learning tasks due to its infinite sum property (this is the main reason why discounting is commonly utilized in reinforcement learning in the first place, see e.g.\ \textcite{schwartz_reinforcement_1993}). Consider the following example. An agent with no time preference ($\gamma = 1$) in a continuous task explores  early that a particular action consistently yields positive rewards. When \emph{exploiting}, the agent keeps playing that action and the value estimate accumulates to infinity. This results in a significant bias towards actions that have been explored early and at some point becomes computationally infeasible. Similarly, values marginally below $1$ are known to be unstable \parencite{naik_discounted_2019}.} \autoref{gamma} summarizes the average $\Delta$. The curves of tabular learning and tile coding indicate the hypothesized pattern. though the relationship is not as clear as one might expect, with $\gamma = 0$, the average profits are much closer to the Nash benchmark.\footnote{The relationship is more pronounced with prices, see \autoref{gamma_violin_price} in \autoref{appendix_2}.}

\textbf{another subtlety: outcomes highest at $\gamma = 0.9$?, might be random variation or an artifact of the convergence problems with large $\gamma$. }

\begin{figure}
	\includegraphics[width=\linewidth]{plots/gamma.png}
	\caption{average $\Delta$ for various experiments. Includes converged and non-converged runs.}
	\label{gamma}
\end{figure}

With regard to the polynomial feature extraction methods, the figure serves as further evidence of their ineptness for the considered learning task. Even without discounting ($\gamma = 0$), the outcomes remain high. In fact, they are even higher with the separated polynomial method. This clearly hints at a failure to learn how to compete when \emph{only} the immediate reward matters.\footnote{I interpret this similar to the results in \textcite{waltman_q-learning_2008} where \emph{memoryless} agents without the ability to assert whether the opponent cheated still learn to charge supra-competitive prices.}

\subsection{Algorithm Variations}\label{vary_algorithm}

Of course, the specific algorithm described in \autoref{expected SARSA} is only one of many ways to use function approximation in learning tasks. I will consider two variations: \emph{Tree backup} and \emph{on-policy SARSA}.

\subsubsection{Tree backup}

\textcite{precup} suggest the \emph{tree backup} algorithm as a successor to Q-Learning. Compared to the \emph{expected SARSA} algorithm, \autoref{eligibility_trace_update} is updated in a slightly different way:

\begin{gather}\label{eligibility_traces_tree_backup}
\gamma \lambda \kappa(A_t | S_t) \boldsymbol{z}_{t-1} + \frac{\Delta \hat{q}}{\Delta \boldsymbol{w}_t} ~~~~~ \text{,}
\end{gather}

where $\kappa(A_t | S_t)$ represents the probability of choosing $A_t$ if the agent were to follow a hypothetical target policy with $\epsilon= 0$. As with the trace in expected SARSA, the idea is that the trace resets to 0 as soon as a non-greedy action is played. Unsurprisingly, applying the tree backup with algorithm to the environment yields similar results. \autoref{tb_violin} displays the distribution of $\Delta$ which is reminiscent of the violins for optimized values of $\alpha$ in \autoref{alpha_violin}. Similarly,  \autoref{average_intervention_tb} reiterates that only tabular learning agents show a consistent punishment in response to a deviation and the cheated agents learning with feature approximation methods fail to respond. The bottom right panel, representing the polynomial tile method, hints at a vague \emph{matching strategy} leading to new equilibria again, but here averaging turns out to be deceptive. In fact, only in 12.5\% of the runs does the non deviating agent respond with a price cut. The distribution of deviation prices and responses are displayed as boxplots in \autoref{appendix_2}.

\begin{figure}
	\includegraphics[width=\linewidth]{plots/tb_violin.png}
	\caption{distribution of $\Delta$ for \emph{tree backup} algorithm. $\alpha$ is optimized in accordance with \autoref{justifications}. Includes only converged runs for better presentability. Violin widths are scaled to maximize width of single violins, comparisons of widths between violins are not meaningful. Violins are trimmed at smallest and largest observation respectively. Horizontal lines represent the median.}
	\label{tb_violin}
\end{figure}

\begin{figure}
	\includegraphics[width=\linewidth]{plots/average_intervention_tb.png}
	\caption{Average price trajectory around deviation.}
	\label{average_intervention_tb}
\end{figure}

\subsubsection{On policy SARSA}
\emph{Q-Learning}, \emph{tree backup} and \emph{expected SARSA} all belong to the family of \emph{off-policy} methods. This stems from the simple fact that the (discounted) value estimation of the state-action combination at $t+1$ is not always based on the actually chosen action $A_t{t+1}$ (see \autoref{td_error_expected} and \textbf{TBD} (?) ).\footnote{It does if $\epsilon = 0$.}  So, it is \emph{off-path} of the actually pursued policy. Off-policy methods tend to provide sophisticated learning but convergence guarantees for them are generally weaker than for \emph{on-policy} algorithms \parencite{sutton_reinforcement_2018} \textbf{other source?}.\footnote{The main reason why I haven't put much consideration into this is that convergence is that due to the \emph{moving target problem} described in \autoref{convergence_considerations}, convergence is not guaranteed anyway. Moreover, \autoref{hettich} shows that off-policy methods can work well with function approximation.}. As their name suggests, \emph{on-policy} methods wait with valuation of the future state-action combination until it is actually known. The straight forward-adaption of calculating the TD error is:

\begin{gather}\label{td_error_on_policy}
\delta_t^{SARSA} = r_t + \gamma \hat{q}(S_{t+1}, A_{t+1}, \boldsymbol{w}_t) - \hat{q}(S_t, A_t, \boldsymbol{w}_t) ~~ \text{,} \\
\end{gather}

Note that learning is delayed in the sense that $\delta_t^{SARSA}$ can only be calculated after the action in the next period has been taken. Using the optimized values of $\alpha$, I conducted one experiment per feature extraction method. \autoref{appendix_2} documents the exact algorithm. \autoref{op_violin} illustrates that the distribution of outcomes per experiment are similar to the two \emph{off-policy} algorithms. overall, the conclusions drawn in the previous section also apply to the \emph{on-policy} algorithm.

\begin{figure}
	\includegraphics[width=\linewidth]{plots/op_violin.png}
	\caption{distribution of $\Delta$ for \emph{on-policy} algorithm. $\alpha$ is optimized in accordance with \autoref{justifications}. Includes only converged runs for better presentability. Violin widths are scaled to maximize width of single violins, comparisons of widths between violins are not meaningful. Violins are trimmed at smallest and largest observation respectively. Horizontal lines represent the median.}
	\label{op_violin}
\end{figure}

		
\subsection{Differential Reward Setting}

In reinforcement learning, discounting is commonly used to avoid infinite value accumulation (e.g.\ in \autoref{td_error_expected}), but rarely has a practical interpretation \parencite{schwartz_reinforcement_1993}. Therefore, the blend with an economic task seems natural. However, despite wide usage, \textcite{naik_discounted_2019} argue that discounting is fundamentally incompatible in combination with function approximation in infinite sequences. They suggest an alternative \emph{differential reward} setting, where \autoref{dt_error_expected} is replaced by:\footnote{See chapter 10 in \textcite{sutton_reinforcement_2018} for a rigorous treatment formulation. \autoref{hettich} shows that the differential reward setting works well with agents in a Bertrand environment. He also compares both settings and finds a tendency to oscillating behavior with discounting.}


\begin{gather}\label{differential_reward}
\delta_t^{differential} = r_t - \widetilde{R}_{t} + \hat{q}(S_{t+1}, A_{t+1}, \boldsymbol{w}_t) - \hat{q}(S_t, A_t, \boldsymbol{w}_t) ~~  \text{,}
\end{gather}

where $\widetilde{r}_{t-1}$ is a (weighted) average reward periodically updated according to

\begin{gather}
	\widetilde{r}_{t+1} = \widetilde{R}_t + \upsilon r_t\text{.}
\end{gather}

The formulation ensures that recent rewards are weighted higher. $\upsilon$ is a parameter controlling the speed of adjustment. Note that the differential reward setting does not involve any discounting. At first glance, this clashes with the economic understanding of time preferences.\footnote{This is the main reason why I have not utilized the differential setting in the main part of this study.} However, there are two arguments why the differential reward setting might still be well suited. First, \textcite{sutton_reinforcement_2018} proof that, due to the infinite nature of the Bertrand environment, the ordering of policies in the discounted value setting and the setting with average rewards are equivalent (irrespective of $\gamma$). Second, pricing algorithms tend to be used in markets with frequent price changes where it is less important whether a profit is realized immediately or in the next period.


\begin{figure}
	\includegraphics[width=\linewidth]{plots/converged_upsilon.png}
	\caption{Number of runs per experiment that achieved convergence as a function of $\lambda$.}
	\label{converged_upsilon}
\end{figure}

I conducted a series of experiments varying over the following values for $\upsilon$: $0.001$, $0.005$, $0.01$, $0.025$, $0.05$ and $0.1$. As with the other variations, $\alpha$ is fixed at values deemed optimal. \autoref{converged_upsilon} shows the convergence tendencies as a function of $\upsilon$ and the feature extraction methods. Disregarding two runs that failed to complete, convergence is consistently achieved for tabular learning, tile coding and separated polynomials. Contrary, only 74.2\% of polynomial tiles runs converged. This starkly contrasts the observation made in the experiments using the discounted reward setting. There, all runs with polynomial tiles converged for various values of $\alpha$.\footnote{The statements disregards runs that failed to complete. Refer back to the bottom right panel in \autoref{converged}. \autoref{convergence_at_upsilon} in \autoref{appendix_2} shows that polynomial tiles in the differential reward setting also tend to converge later than the other methods.} Moreover, the plot suggests that low values of $\upsilon$ impede convergence for this method.


 \begin{figure}
	\includegraphics[width=\linewidth]{plots/upsilon.png}
	\caption{average $\Delta$ for various values of $\upsilon$ in the differential reward setting. Includes converged and non-converged runs. Beware the logarithmic x-scale.}
	\label{upsilon}
\end{figure}


\autoref{upsilon} displays how the average profits relative to $p_n$ and $p_m$ change with $\upsilon$. The overall impact is small. However, tabular learning and, to a lesser extent, tile coding seem to be sensitive to very low values of $\upsilon$.\footnote{\autoref{upsilon_violin} reveals that the variability in average $\Delta$ is higher than in the discounted setting.} With respect to punishment of price cuts, the results are similar to the discounted setting. Irrespective of $\upsilon$, the majority of deviations in experiments with separated polynomials and polynomial tiles is profitable and evokes no retaliation. With tabular learning, the share of profitable deviations is 24.7\% over all runs. There is slight evidence that the hint at some sort of punishment in tile coding is more pronounced in the differential reward setting. Only 48.8\% of deviations are strictly profitable.\footnote{The percentage drops to only 45.8\% if only runs with $\upsilon = 0.1$ or $\upsilon = 0.005$ are considered.} \autoref{intervention} in \autoref{appendix_2} shows that on average, the non deviating agent retaliates with a slight price cut at $\tau = 2$ for various $\upsilon$.


\pagebreak
	
	\section{Conclusions}\label{conclusions}

I argued that despite the success of Q-Learning in achieving collusion in repeated games of price competition, tabular learning methods might face practical challenges when applied to real markets. Therefore, I developed three linear function approximation methods that scale better with the learning task's complexity and combined them with suitable reinforcement learning algorithms. To assess their merit, I deployed them to a simultaneous pricing environment and compared their performance to tabular learning.

The simulations have shown that all \gls{fem}s tend to converge in supra-competitive profits as long as parameter specifications remain reasonable.  As is shown in other studies, tabular learning agents acquire truly collusive strategies and show a stubborn resilience to return to the convergence equilibrium after episodes of forced deviation. On the contrary, the convergence equilibra arising in simulations with function approximation \gls{fem}s are not supported by a reward-punishment scheme. I show in various deviation exercises that agents have not learned to systematically punish price cuts. Thus, the simulation \emph{failed to provide evidence of collusion with function approximation \gls{fem}s}. Furthermore, despite the obvious lack of a credible deterrent, deviating agents are unable to exploit that weakness and return to pre-deviation prices. This is clear evidence of irrational behavior on the side of the agents. These findings also apply when using different algorithms and are robust to variations in learning and environment parameters. In particular, the introduction of eligibility traces does not qualitatively change the conclusions.

I stress that the mere existence of supra-competitive prices in the simulations does not make the \gls{fem}s viable. In fact, the only reason supra-competitive prices arise in settings with function approximation is that \emph{both} agents fail to compete efficiently. Indeed, their success hinges on the other agent also playing an inferior strategy. It is easy to see that playing such exploitable strategies are unlikely to succeed in real market settings. A potential exception is the \emph{hub and spoke} scenario envisioned by \textcite{ezrachi_algorithmic_2017}. The exploitable algorithms could prove successful if a vendor was able to supply it to \emph{all} competitors in an industry.

It is hard to pinpoint the exact causes of these failures. An obvious lever for improvement is the parametrization. The default specification was largely arbitrary and I did not systematically optimize parameters for computational reasons. The most obvious candidate for improving strategies is the exploration rate. But since the results are similar for a number of specifications, I am doubtful that optimizing parameters would make much of a difference. Alternatively, one could trial other, more sophisticated \gls{fem}s. However, linear function approximation might be generally inadequate to learn collusive strategies precisely because stable strategies require non-linear responses. I suspect that \emph{linear} function approximation could be a dead end in the realm of multi-agent reinforcement learning in economic environments. Nevertheless, absence of evidence does not equate evidence of absence. Indeed, \textcite{hettich_algorithmic_2021} proves that agents learning with \emph{non-linear} function approximation can be very successful in forging collusion.

I leave open two avenues for future research. First, further simulation studies could prove instrumental to understand under which conditions algorithmic collusion is likely. Most considered environments (including the one in this study) are rather simple and prefabricated. It would be interesting to see how algorithms behave in more challenging conditions (e.g.\ many players, dynamic demand, multi-sided markets). Possible extensions include \emph{actor-critic models} that allow to incorporate continuous action spaces.\footnote{\textcite[p.16]{hettich_algorithmic_2021} makes the same suggestion.} Second, empirical studies on real markets are imperative to get a refined understanding of how real the threat of algorithmic collusion is. \textcite{assad_algorithmic_2020} show that increased price margins in the wake of independently acquired algorithms are \emph{possible}. Whether this results holds for other industries and over time remains to be seen.

	\pagebreak
	\printbibliography
	
	\begin{appendices}
	\section{Results Appendix}\label{appendix}
	
		\autoref{trajectory_price} displays the distribution of average prices in various runs of the optimized experiments. \autoref{all_runs} displays the price and profit trajectory of single runs over time. Only runs of the \emph{optimal} runs are printed, as explained in \autoref{justifications}. Both metrics are averaged over 50,000 episodes apiece and over both players. Again, note that, by and large, prices and profits remain within the benchmarks of Nash competition and the cartel case.


\begin{figure}
	\includegraphics[width=\linewidth]{plots/trajectory_price.png}
	\caption{distribution of $p$ over time of 'optimized' experiments. For individual runs, $p$ is averaged over 50,000 periods apiece and both players. Plot includes converged and non-converged runs. Violin widths represent quantity of active runs at $t$ which enables comparisons between violins. As most runs converge after 200,000 to 300,000 episodes, violin widths decrease thereafter. Violins are trimmed at smallest and largest observation respectively. Horizontal lines represent the median.}
	\label{trajectory_price}
\end{figure}

\begin{figure}
	\includegraphics[width=\linewidth]{plots/all_runs.png}
	\caption{all runs for manually optimized $\alpha$}
	\label{all_runs}
\end{figure}

	
	\pagebreak
	\section{Robustness and variations Appendix}\label{appendix_2}
	

	
		\subsection{Prolonged deviations}\label{prolonged_deviations_appendix}

\autoref{prolonged_intervention_boxplot} displays the distribution of prices relative to the counterfactual during and after the prolonged deviation experiment. The deviating agent sustains deviation prices for 10 periods. With tabular learning, the cheated agent matches the price cuts on average. This means that both agents are likely to make lower profits. After the deviation, they return to pre-deviation levels. The responses in tile coding and polynomial tiles are less pronounced. In fact, the median indicates that the non deviating agents does not respond at all. 

\begin{figure}
		\includegraphics[width=\linewidth]{plots/prolonged_intervention_boxplot.png}
		\caption[Distribution of price differences around prolonged deviation by \gls{fem}]{Distribution of price differences around prolonged deviation relative to counterfactual path \emph{without} forced deviation, i.e.\ the difference to the price had no deviation taken place, by \gls{fem}. Only includes converged runs because a clear counterfactual exists. Boxes demarcate 15th and 85th percentiles. They are extended by whiskers that mark the entire range of price differences. Horizontal lines represent the group median.}
		\label{prolonged_intervention_boxplot}
\end{figure}


\clearpage

\subsection{Exploration ($\beta$)}\label{beta_appendix}

Section \ref{vary_parameter} showed that deviations in tabular learning environments were less likely to be profitable if exploration was extensive. \autoref{average_intervention_beta_tabular} confirms that the convergence equilibria are more likely to be underpinned by severe punishment strategies if $\beta$ decreases (i.e.\ exploration becomes more extensive). The immediate response of the non deviating agent is harshest with $\beta = 10^{-5}$.

\begin{figure}
	\includegraphics[width=\linewidth]{plots/average_intervention_beta_tabular.png}
	\caption[Average price trajectory around deviation by $\beta$]{Average price trajectory around deviation by $\beta$ (values on strip). Includes only tabular learning experiments. Points represent the average price over all runs of an experiment. Dashed horizontal lines represent the fully collusive price $p_m$ and the static Nash solution $p_n$. Dotted vertical line reflects time of convergence, i.e.\ the period immediately before the forced deviation.}
	\label{average_intervention_beta_tabular}
\end{figure}


\clearpage

\subsection{Memory ($\lambda$)}\label{lambda_appendix}

Recall that high values of $\lambda$ increase the algorithm's hindsight but also the variance. This is reflected in both convergence rates and outcomes. \autoref{converged_lambda} clearly indicates that high values of $\lambda$ impede convergence for tabular learning and the separate polynomials \gls{fem}. Similarly, \autoref{lambda_violin} exhibits greater variability in profits with increasing $\lambda$. This holds true for all feature extraction methods, but is most salient for separate polynomials where a significant number of runs end in profits below the Nash equilibrium once $\lambda$ exceeds $0.6$.\footnote{The runs with $\Delta <0$ are mainly runs where convergence was not achieved.}

\begin{figure}
	\includegraphics[width=\linewidth]{plots/converged_lambda.png}
	\caption[Converged runs by \gls{fem} and $\lambda$]{Number of runs per experiments that (i) achieved convergence, (ii) did not converge or (iii) failed to complete as a function of \gls{fem} and $\lambda$.}
	\label{converged_lambda}
\end{figure}


\begin{figure}
	\includegraphics[width=\linewidth]{plots/lambda_violin.png}
	\caption[Distribution of $\Delta$ by \gls{fem} and $\lambda$]{Distribution of $\Delta$ by \gls{fem} and $\lambda$. Includes converged and non-converged runs. Violin widths are scaled to maximize width of individual violins, comparisons of widths between violins are not meaningful. Violins are trimmed at smallest and largest observation respectively.}
	\label{lambda_violin}
\end{figure}

\clearpage

\subsection{Price grid}\label{price_grid_appendix}

\begin{figure}
	\includegraphics[width=\linewidth]{plots/converged_m.png}
	\caption[Converged runs by \gls{fem} and $m$]{Number of runs per experiments that (i) achieved convergence, (ii) did not converge or (iii) failed to complete as a function of \gls{fem} and $m$. $m = 19$ is not plotted because the number of runs is not comparable (refer back to \autoref{converged}).}
	\label{converged_m}
\end{figure}

Recall that $m$ determine the number of feasible prices and therefore increases the complexity of the learning task. This holds particularly true for tabular learning and, to a lesser extent, for separate polynomials (refer back to \autoref{feature_extraction_summary}). Against that backdrop, it is unsurprising that \autoref{converged_m} shows less runs converging if $m$ increases. The effect is most pronounced for tabular learning and tile coding. \autoref{average_intervention_m_10} shows the average response to a deviation for $m=10$. Compared to the baseline parametrization, the punishment of the cheated agent seems more severe with tabular learning and polynomial tiles. However, the low sample size of runs (16 per experiment) warrants cautious interpretation.

\begin{figure}
	\includegraphics[width=\linewidth]{plots/average_intervention_m_10.png}
	\caption[Average price trajectory around deviation by \gls{fem}with  $m=10$]{Average price trajectory around deviation by \gls{fem} with $m=10$. Points represent the average price over all runs of an experiment. Dashed horizontal lines represent the fully collusive price $p_m$ and the static Nash solution $p_n$. Dotted vertical line reflects time of convergence, i.e.\ the period immediately before the forced deviation.}
	\label{average_intervention_m_10}
\end{figure}

Moving to variations in $\zeta$, the parameter controlling the excess price range above $p_m$, \autoref{zeta_violin_prices} confirms that average prices upon convergence largely remain within the Nash and collusive benchmarks. Polynomial tiles constitute the only exception. With $\zeta = 1$, a significant share of runs displays prices above $p_m$. This finding further discredits the \gls{fem} as appropriate for the learning task. \autoref{average_intervention_zeta_tabular} displays the price trajectory during the forced deviation episode for tabular learning. Retaliatory pricing is visible in all variations.

\begin{figure}
	\includegraphics[width=\linewidth]{plots/zeta_violin_prices.png}
	\caption[Distribution of average prices by \gls{fem} and $\zeta$]{Distribution of average prices upon convergence by \gls{fem} and $\zeta$. Includes converged and non-converged runs. Violin widths are scaled to maximize width of individual violins, comparisons of widths between violins are not meaningful. Violins are trimmed at smallest and largest observation respectively. Horizontal lines represent the median.}
	\label{zeta_violin_prices}
\end{figure}


\begin{figure}
	\includegraphics[width=\linewidth]{plots/average_intervention_zeta_tabular.png}
	\caption[Average price trajectory around deviation by $\zeta$]{Average price trajectory around deviation by $\zeta$. Numbers on strip represent values of $\zeta$. Includes only tabular learning experiments. Points represent the average price over all runs of an experiment. Dashed horizontal lines represent the fully collusive price $p_m$ and the static Nash solution $p_n$. Dotted vertical line reflects time of convergence, i.e.\ the period immediately before the forced deviation.}
	\label{average_intervention_zeta_tabular}
\end{figure}

\clearpage

\subsection{Discount factor}\label{discounting_appendix}

The expectation with low values of $\gamma$ is that agents value future profits less and price closer to the competitive benchmark $p_n$ in order to increase immediate profits. Indeed, \autoref{gamma_violin_price} shows that the distribution of average prices clearly shifts downwards for three of the four \gls{fem}s. The effect is very clear for tabular learning and polynomial tiles. With regard to the latter, note that prices are still far above $p_n$. This is puzzling. Without regard for future profits one would expect agents to end up very close to the Nash solution. Likewise, $\gamma$'s effect on average prices with tile coding points in the expected direction but is unexpectedly subtle. With separate polynomials, the impact of $\gamma$ is not obvious. If anything, the plot suggests that low discount factors increase profits.

\begin{figure}
	\includegraphics[width=\linewidth]{plots/gamma_violin_price.png}
	\caption[Distribution of average prices by \gls{fem} and $\gamma$]{Distribution of average prices upon convergence by \gls{fem} and $\gamma$. Includes converged and non-converged runs. Violin widths are scaled to maximize width of individual violins, comparisons of widths between violins are not meaningful. Violins are trimmed at smallest and largest observation respectively. Horizontal lines represent the median.}
	\label{gamma_violin_price}
\end{figure}

\clearpage

\subsection{Alternative algorithms}\label{vary_algorithm_appendix}

\autoref{average_intervention_tb} displays the average price trajectory around the deviation episode due to runs utilizing the \emph{tree-backup} algorithm from \autoref{tree_backup}. As with the default algorithm described in \autoref{parameter_update}, the panels reiterates that only tabular learning agents show a consistent punishment in response to the forced deviation and the cheated agents learning through function approximation \gls{fem}s fail to respond in a compelling way. The bottom right panel, representing the polynomial tiles \gls{fem}, hints at a vague \emph{matching strategy} culminating in new equilibria. But averaging turns out to be deceptive here. In fact, only in 12.5\% of the runs does the non deviating agent respond with a price cut. \autoref{intervention_boxplot_tb} displays the distribution of prices around the forced deviation. With regard to polynomial tiles, \emph{some} runs show a sort of punishment or matching behavior in the wake of a price cut, but the vast majority (87.5\%) of runs show no response.

\begin{figure}
	\includegraphics[width=\linewidth]{plots/average_intervention_tb.png}
	\caption[Average price trajectory around deviation by \gls{fem} with \emph{tree backup} algorithm]{Average price trajectory around deviation by \gls{fem} with \emph{tree backup} algorithm. Points represent the average price over all runs of an experiment. Dashed horizontal lines represent the fully collusive price $p_m$ and the static Nash solution $p_n$. Dotted vertical line reflects time of convergence, i.e.\ the period immediately before the forced deviation.}
	\label{average_intervention_tb}
\end{figure}


\begin{figure}
	\includegraphics[width=\linewidth]{plots/intervention_boxplot_tb.png}
	\caption[Distribution of price differences around deviation by \gls{fem} with \emph{tree backup} algorithm]{Distribution of price differences around deviation relative to counterfactual path \emph{without} forced deviation, i.e.\ the difference to the price had no deviation taken place by \gls{fem} with \emph{tree backup} algorithm. Only includes converged runs because a clear counterfactual exists. Boxes demarcate 15th and 85th percentiles. They are extended by whiskers that mark the entire range of price differences. Horizontal lines represent the group median.}
	\label{intervention_boxplot_tb}
\end{figure}

\autoref{SARSA} describes the \emph{on-policy} algorithm used in \autoref{vary_algorithm}. Note that $\rho$ does not appear in the update system anymore. This is precisely because the algorithm only takes into account the actually selected action at $t+1$.

\begin{algorithm}
	\caption{SARSA (on policy)}
	\begin{algorithmic}[]
		\label{SARSA}
		\small
		\STATE input feasible prices via $m \in \mathbb{N}$ and $\zeta > 0$
		\STATE configure static algorithm parameters $\alpha > 0$, $\beta > 0$, and $\lambda \in [0, 1]$
		\STATE initialize parameter vector and eligibility trace $\boldsymbol{w} = \boldsymbol{z} = \boldsymbol{0}$
		\STATE declare convergence rule (see \autoref{convergence})
		\STATE start tracking time: $t = 1$
		\STATE randomly initialize state $S_t$
		\STATE choose initial action $A_t$
		\WHILE{convergence is not achieved,}
		\STATE observe profit $\pi$, adjust to reward $r$
		\STATE move to next state: $t \leftarrow t+1$ and $S_{t+1} \leftarrow A_t$
		\STATE select action $A_{t+1}$ according to \autoref{action_selection}
		\STATE calculate TD-error: $\delta \leftarrow r + \gamma \hat{q}(S_{t+1}, A_{t+1}) - \hat{q}(S_t, A_t)$ (\autoref{td_error_on_policy})
		\STATE update eligibility trace: $\boldsymbol{z} \leftarrow \gamma \lambda \boldsymbol{z} + \boldsymbol{x}$
		\STATE update parameter vector: $\boldsymbol{w} \leftarrow \boldsymbol{w} + \alpha  \delta  \boldsymbol{z}$ (\autoref{update_rule})
		\STATE $S \leftarrow S_{t+1}$ and $A \leftarrow A_{t+1}$
		\ENDWHILE
	\end{algorithmic}
\end{algorithm}

\clearpage

\subsection{Differential reward setting}\label{differential_appendix}

Section \ref{differential} described the \emph{differential reward} setting, an alternative method to incorporate rewards into the learning process. I also described how the separated polynomial method struggled to achieve convergence in the alternative setting (refer back to \autoref{converged_upsilon}). \autoref{convergence_at_upsilon} emphasizes that point. A surprisingly large number of runs converges at a stage where exploration is incredibly rare. This suggests that agents, despite continuous exploitation, frequently change their evaluation of what the optimal action is.

\begin{figure}
	\includegraphics[width=\linewidth]{plots/convergence_at_upsilon.png}
	\caption[Timing of convergence in \emph{differential reward} setting by \gls{fem}]{Timing of convergence in \emph{differential reward} setting by \gls{fem}. only includes converged runs. Width of bins: 8,000.}
	\label{convergence_at_upsilon}
\end{figure}


\autoref{upsilon_violin} displays for every experiment in the differential reward setting the range of $\Delta$ upon convergence. It appears that the considered values of $\upsilon$ do not impact the outcomes much. In comparison to the baseline runs, tabular learning and tile coding exhibit larger variation. In the case of tabular learning, some runs hover around Nash profits while others converge in equilibria close to the perfectly collusive benchmark.


\begin{figure}
	\includegraphics[width=\linewidth]{plots/upsilon_violin.png}
	\caption[Distribution of $\Delta$ by \gls{fem} and $\upsilon$]{Distribution of $\Delta$ by \gls{fem} and $\upsilon$. Includes converged and non-converged runs from experiments employing the differential reward setting. Violin widths are scaled to maximize width of individual violins, comparisons of widths between violins are not meaningful. Violins are trimmed at smallest and largest observation respectively.}
	\label{upsilon_violin}
\end{figure}

\autoref{intervention_boxplot_upsilon_005} illustrates the charged prices around the intervention relative to a counterfactual without a forced deviation for experiments with $\upsilon = 0.005$. Tabular learning shows a clear tendency to punish price cuts at $\tau = 2$. For tile coding and polynomial tiles, a price cut in response to the deviation occurs in \emph{some} runs.

\begin{figure}
	\includegraphics[width=\linewidth]{plots/intervention_boxplot_upsilon_005.png}
	\caption[Distribution of price differences around deviation by \gls{fem} in differential reward setting with $\upsilon = 0.005$]{Distribution of price differences around deviation by \gls{fem} relative to counterfactual path \emph{without} forced deviation, i.e.\ the difference to the price had no deviation taken place in the differential reward setting with $\upsilon = 0.005$. Only includes converged runs because a clear counterfactual exists. Boxes demarcate 15th and 85th percentiles. They are extended by whiskers that mark the entire range of price differences. Horizontal lines represent the group median.}
	\label{intervention_boxplot_upsilon_005}
\end{figure}
	\end{appendices}
\end{document}
