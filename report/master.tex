\documentclass[a4paper]{scrartcl}
%alternatives to classic article are: book, report, scrartcl (recommended), scrreprt, scrbook
%more options Optionen: 11pt, 12pt, twoside, twocolumn

\usepackage[english]{babel}  %for German texts: ngerman
%\usepackage[utf8]{inputenc}   %encode in UTF-8 (I think TexStudio does this by default)


\usepackage[backend = biber, citestyle=authoryear, bibstyle = authoryear, doi=false,isbn=false,url=false,eprint=false]{biblatex} %more options, see e.g. https://www.overleaf.com/learn/latex/Biblatex_citation_styles

\AtEveryBibitem{
	\clearlist{language} % clears language
	
} 

\renewbibmacro{in:}{}

\usepackage{xpatch}

% No dot before number of articles
\xpatchbibmacro{volume+number+eid}{%
	\setunit*{\adddot}%
}{%
}{}{}

% Number of articles in parentheses
\DeclareFieldFormat[article]{number}{\mkbibparens{#1}}


\addbibresource{zotero_refs.bib} %pass the name of the bib file


% define glossary
\usepackage[nopostdot,nogroupskip,style=super,nonumberlist,automake]{glossaries}

\glstocfalse
\makeglossaries

\newglossaryentry{mdp}{name={MDP},description={Markov Decision Process}}
\newglossaryentry{fem}{name={FEM},description={Feature Extraction Method}}
\newglossaryentry{td_error}{name={TD error},description={temporal-difference error}}


%Formatting and Layout
\usepackage[left = 4cm, right = 2cm, top = 2.5cm, bottom = 2cm]{geometry}

% line spacing: 1.5
\usepackage[onehalfspacing]{setspace}
\renewcommand{\baselinestretch}{1.5}

%no indentation in captions of figures and tables
\setcapindent{0pt} 


% Package for algorithm boxes
\usepackage{algorithm,algorithmic}

% section name is printed on the head of each page
\pagestyle{headings}   


\usepackage{graphicx}  %for graphics
\usepackage{amsmath,amsfonts,amssymb,amsthm,mathtools}  %math packages



% abstract and appendices
\usepackage{abstract}
\usepackage[title]{appendix}
\newcommand*{\Appendixautorefname}{appendix}

%References
\usepackage[hidelinks]{hyperref}
\usepackage{cleveref}


\begin{document}
	
	% overwrite autoreference labels
	\def\sectionautorefname{section}
	\def\subsectionautorefname{section}
	\def\subsubsectionautorefname{section}
	\def\equationautorefname{equation}
	\def\algorithmautorefname{Algorithm}
	

	
	% start roman page numbers
	\pagenumbering{Roman}
	
	% center front page, no page number
	\newgeometry{left = 2cm, right = 2cm}
	\thispagestyle{empty}
	
	\begin{titlepage}
		\centering
		\vspace{1cm}
		{\Large\bfseries \LaTeX Template \par}
		\vspace{4cm}
		{\large\itshape Heinrich-Heine-University Düsseldorf\par}
		\vspace{0.5cm}
		{\large\itshape Faculty of Business Administration and Economics\par}
		\vspace{0.5cm}
		{\large\itshape MW70 Competition Law and Policy\par}
		\vspace{0.5cm}
		{\large\itshape Summer Term 2018\par}
		{\large\itshape \par}
		\vfill
		by\par
		\vspace{0.5cm}
		Malte Jeschonneck\\
		Malte.Jechonneck@uni-duesseldorf.de\\
		Matriculation Number: 2307497\\
		Mörsenbroicher Weg 179, 40470 Düsseldorf\\
		Program: Economics, M. Sc.\\
		First Term
		
		
		\vfill
		
		% Bottom of the page
		{\large \today\par}
	\end{titlepage}

\restoregeometry % go back to default layout

\newpage
	\begin{abstract}
		 The increased prevalence of pricing algorithms incited an ongoing debate about new forms of collusion. The concern is that intelligent algorithms may be able to forge collusive schemes without being explicitly instructed to do so. I attempt to examine the ability of competing \emph{reinforcement learning} algorithms to maintain collusive prices in a simulated oligopoly of price competition. To my knowledge, this study is the first to use a reinforcement learning system with linear function approximation and eligibility traces in an economic environment. I show that the deployed agents sustain supra-competitive prices, but tend to be exploitable in the short-term. The degree of collusion crucially hinges on the utilized method to estimate the qualities of actions. This finding is robust to variations of parameters that control the learning process.
	\end{abstract}
	
\newpage	
	
	% TOC, LOF, LOT, Abbreviations
	\tableofcontents
	\newpage
	\listoffigures
	\newpage
	\listoftables
	\newpage
	\printglossary[title={List of Abbreviations}] %Generate List of Abbreviations
	\newpage
	\pagenumbering{arabic}
	
	\section{Introduction}

There is little doubt that algorithms will play an increasingly important role in economic life. Dynamic pricing software is frequently used in online retail markets \parencite{chen_empirical_2016}, the tourist industry \parencite[p.4]{den_boer_dynamic_2015} and at petrol stations \parencite[pp.7-9]{assad_algorithmic_2020}. As with many other technological advances, the economic advantages are conspicuous. Not only does automating pricing decisions cut costs and free up resources, algorithms may also be better at predicting demand and react faster to changing market conditions \parencite[p. 15]{oecd_algorithms_2017}. Overall, there is little doubt that pricing algorithms may be used as a tool by companies to gain competitive advantages. It is worth pointing out that thereby intensified competition also benefits consumers.\footnote{Of course other types of algorithms that benefit consumers exist. Price comparison tools have been around for a while but applications extend beyond mere reduction of search costs. \textcite{gal_algorithmic_2017} champion \emph{algorithmic consumers}, electronic assistants that compare product characteristics at low transaction costs enabling humans to completely outsource their purchase decisions. Moreover, algorithmic consumers may challenge market power of suppliers by bundling consumer interests.}

Nevertheless, concerns have been raised that ceding pricing authority to algorithms has the potential to create new forms of collusion that contemporaneous competition policy is not well equipped to deal with. The main issue is that the traditional dichotomy between \emph{explicit} and \emph{tacit} collusion is potentially unsuitable in the case of pricing software. Traditionally, competition authorities only prohibit and punish explicit pricing agreements. On the contrary, tacit collusion (e.g.\ \emph{intelligent market adaption}) is typically tolerated despite the economic effect on consumers being equally detrimental \parencite[p. 141]{motta_competition_2004}. In practice, the distinction is sometimes vague and the advent of pricing algorithms is believed to blur the line. It is still unclear when \emph{algorithmic collusion} could elude competition enforcement.\footnote{A different issue is that pricing algorithms with information on consumer characteristics may be able to augment the scope of \emph{price discrimination}, i.e.\ companies extracting rent by charging to every consumer the highest price he is willing to pay. Under which circumstances competition authorities should be concerned with this possibility is outlined in \textcite{oecd_price_2016}. \textcite{ezrachi_algorithmic_2017} develop a scenario where discriminatory pricing and tacit collusion occur simultaneously. Both issues remain outside the scope of this study.}

Unfortunately, there is a lack of empirical studies assessing the effects of autonomous pricing software in the real world. A notable study of the German retail gasoline market by \textcite{assad_algorithmic_2020} documents that margins in duopoly markets increased substantially after both duopolists switched from manual pricing to algorithmic-pricing software. Further field studies could prove instrumental to confirm and refine these findings. As a substitute, there is a growing number of simulation studies that show the capacity of \emph{reinforcement learning} algorithms to create and sustain collusive equilibria in repeated games of competition (see \autoref{literature review}). However, the direct transferability of these findings to real markets is questionable. Most studies use a simple tabular learning method, called \emph{Q-Learning}, that requires discretizing prices and does not scale well if the complexity of the environment increases.

This study attempts to mitigate these problems by employing \emph{linear function approximation} to estimate the value of actions. More specifically, I develop three methods of function approximation and run a series of experiments to assess how they compare to tabular learning. Moreover, I utilize \emph{eligibility traces} as an efficient way to increase the memory of agents interacting in the environment.\footnote{Neither linear function approximation nor eligibility traces are new concepts in reinforcement learning. However, to my knowledge, this is the first study to apply them to a repeated pricing game.}

To foreshadow the results, the simulations show that the developed function approximation methods, like tabular learning, result in supra-competitive prices upon convergence. However, \emph{unlike} tabular learning, the learned strategies are easy to exploit. By forcing one of the agents to diverge from the convergence equilibrium, I show that the cheated agent fails to punish that deviation. This indicates that the learned equilibrium strategies are unstable vis-à-vis rational agents with full information. This observation is robust to a number of variations and extensions. Also, with respect to eligibility traces, excessively increasing memory tends to destabilize the learning process, but the overall impact for reasonable parametrization appears small.

The remainder of this paper is organized as follows. The next section briefly surveys simulation studies similar to this one. I also review the scholarly literature on algorithmic competition and contemporaneous regulation. Section \ref{enironment} introduces the repeated pricing environment in which the artificial competitors interact. Section \ref{algorithm} presents in detail the deployed learning algorithm with its parametrization and \autoref{feature_extraction} discusses the developed methods to estimate action values with function approximation. I present the results in \autoref{results} and consider variations and extensions in \autoref{robustness}. Section \ref{conclusions} concludes.
	
	\section{Literature review}\label{literature review}

This study concerns itself with the ability of algorithms to forge collusion without being explicitly instructed. Situations in which  humans would be unable to achieve such schemes are of special interest. This section provides an overview of academic and institutional assessments of the controversial topic.

First, it is helpful to define collusion. In this text, I will follow \textcite[pp.334-336]{harrington_developing_2018} who recognizes that \emph{supra-competitive} prices, i.e.\ prices above a competitive benchmark, must be underpinned by a \emph{reward-punishment scheme} to be labeled as \emph{collusive}. The scheme is maintained by a mutual understanding that a firm's current behavior affects its competitors' future conduct. Specifically, a participant's adherence to high prices is rewarded with high future prices of competitors ensuring high industry margins. Conversely, deviations are punished with price cuts.\footnote{Of course, competitors can collude not only on prices, but in a variety of ways. The concept easily extends to other dimensions, e.g.\ investment or product quality.} Naturally, it is yet to be seen how likely the scenario of algorithms achieving collusion in real markets is. It is also unclear whether the scenario constitutes illicit behavior and warrants intervention from competition authorities. This question is not a primary concern of this paper. Nevertheless, I will sketch some of the main positions before moving on.

The central issue is that contemporaneous competition policy does not consider collusion itself as illicit behavior. Rather, it is the process, by which it is achieved, that determines legality \parencite[p.339-341]{harrington_developing_2018}. Explicit communication among competitors who consciously agree on price levels is clearly illegal. Smart adaption to market conditions by individual agents is not. These distinct cases are often referred to as \emph{explicit} and \emph{tacit} collusion. Whether to put cooperating algorithms in the former or the latter category is subject to ongoing controversy. A binary answer probably does not do justice to the problem's complexity.  

At the very least, academics consent that algorithms could be utilized to facilitate \emph{existing} collusive agreements \parencite[p.219]{ezrachi_sustainable_2018}. For instance, cartel members could automate detection and punishment of deviations from an agreement through an algorithm. Other conceivable schemes include facilitated market segmentation and price \emph{signalling} \parencite[p.29]{oecd_price_2016}. While these scenarios may alter the operational scope of market investigations to account for the role of deployed algorithms, they are well covered by contemporary competition practices.\footnote{See e.g.\ a joint statement by the German and French federal cartel authorities \parencite{bundeskartellamt_working_nodate} and \textcite{cma_case_2016}, \textcite{oefgen_decision_2019} for two exemplary cases with algorithms \emph{facilitating} collusive agreements.} In fact, the specifics of \emph{facilitating} algorithms might not be highly important because the mere \emph{intention} to collude suffices to invoke competition laws \parencite[p.29]{bundeskartellamt_working_nodate}.

The most interesting scenario concerns independently developed or acquired algorithms that align pricing behavior. The US Department of Justice states that competitors are unlikely to be held liable if they independently adopt similar pricing software \parencite[p.6]{doj_algorithms_2017}. A note from the European Union indicates that companies can not expect to completely avoid liability by referring to their pricing algorithms. Moreover, they do not rule out the option that algorithms \emph{decoding} each other may be within the scope of explicit communication \parencite[p.8-9]{eu_algorithms_2017}. 

\textcite[pp.105-106]{gal_algorithms_2019} argues that in the special case of \emph{rule based} algorithms, a programmer's intent to create coordination could in principal be derived from her developed code. For instance, the conscious decision to include punishment mechanisms if a competitor's price falls below a certain threshold seems incriminating. However, many algorithms do not follow a rigid set of rules. Rather, its programmer defines higher-level objectives such as \emph{profit maximization} and the algorithm itself figures out how to act in a specific situation based on data and experience.\footnote{Every form of reinforcement learning falls into this second category.} Obviously, it is hard to infer intent from these types of algorithms. On a similar note, \textcite[p.350-351]{harrington_developing_2018} emphasizes that determining intentions from program code is conceptually appealing, but costly to implement.

\textcite{mehra_antitrust_2015} takes a more controversial view. He points out that the traditional distinction between explicit and tacit collusion emerged with human agents in mind and competition laws did not foresee pricing algorithms that are more likely to achieve cartel solutions in oligopolistic settings due to superior speed, accuracy and even rationality when analyzing and adjusting prices. Consequently, he argues that the increasing prevalence of automated pricing software warrants a reassessment of current competition law and enforcement. Moving forward, I will discuss the likelihood of collusion among algorithms arising in real markets. Due to a lack of empirical evidence, I will focus on theoretical considerations and simulation studies.

As pointed out earlier, the scarcity of field studies on pricing algorithms in real markets prohibits generalized conclusions. However, theoretical considerations might shed light on when algorithmic coordination is a valid concern. Any form of collusion requires timely detection of deviations and a credible threat of punishment \parencite[pp.48-56]{stigler_theory_1964}.\footnote{Naturally, industry characteristics play an important role (e.g.\ number of firms, market entry barriers or product homogeneity). See \textcite[p.142-149]{motta_competition_2004} for an extensive list of structural factors and their impact on the likelihood of collusion arising.}  Surely, algorithms are able to fulfill these conditions but so do humans. \textcite{schwalbe_algorithms_2018} argues that the advent of algorithms does not raise \emph{novel} competition problems. Indeed, humans might be replaced by pricing software but this does not inevitably make collusion more likely. The argument is strengthened by a list of experimental settings where algorithms fail to cooperate. Schwalbe stresses that the ability to communicate is vital to achieve collusive outcomes in markets with more than two participants and raises the question whether algorithms are better at communicating than humans.

\textcite[p.10-13]{ittoo_algorithmic_2017} emphasize the challenges associated with applying \emph{reinforcement learning} algorithms to real markets. First, there are practical implementation issues and mapping real market conditions to a reinforcement learning data problem is not always natural. Second, convergence guarantees break down as soon as the market is subject to changing conditions.\footnote{Technically, guarantees of convergence in reinforcement learning tasks are only valid if the environment is stationary, an assumption that is violated if demand conditions change or competitors price dynamically (see \autoref{convergence_considerations}). However, absence of convergence guarantees does not render convergence impossible.} Third, tabular learning methods do not scale well with the complexity of learning tasks. Consequently, mastering collusion might take a long time.\footnote{I will revisit this point in \autoref{tabular}.} They conclude that the deployment of pricing algorithms possibly, but not inevitably, leads to collusive outcomes.

\textcite[pp.6-17]{ezrachi_algorithmic_2017} argue that algorithms have the potential to establish tacit collusion in markets where conscious parallelism among humans is unrealistic. For instance, they develop a \emph{hub and spoke} scenario in which a third party software vendor provides the same or a similar pricing algorithm to competing sellers. The single algorithm could then align the pricing behavior of competitors resulting in conditions conducive to collusion. The authors suggest counter measures such as imposing restrictions on the allowed frequency of price changes or artificially reducing price transparency.


While there are numerous studies on the behavior of learning algorithms in cooperative and competitive multi-agent games\footnote{See e.g.\ \textcite{leibo_multi-agent_2017} and \textcite{crandall_cooperating_2018} for recent large-scale experimental studies.}, their application in oligopolistic environments has been rare and the trialed algorithms have been relatively simple. A seminal study by \textcite{waltman_q-learning_2008} examines two \emph{Q-Learning} agents in a \emph{Cournot} environment. Their simulations result in supra-competitive outcomes. However, even \emph{memoryless} agents without knowledge of past outcomes manage to attain quantities below the one-shot Nash equilibrium. This casts doubt on the viability of the learned strategies vis-à-vis rational agents. Truly memoryless agents can not pursue \emph{punishment strategies} because they are unable to even detect them. Thus, constantly playing the one-shot solution \emph{should} be the only rational strategy. It appears, the agents do not \emph{learn how to collude}, but rather \emph{fail to learn how to compete}.

Two further studies that model agents in games of infinitely repeated quantity competition should be mentioned. Inspired by the management literature, \textcite{kimbrough_learning_2009} trial a \emph{probe and adjust} algorithm.\footnote{To my knowledge, \emph{probe and adjust} is the only algorithm that explored continuous price setting in repeated games of competition to date.} Their agents repeatedly draw prices from a continuous price range. After some time, they assess whether low or high prices yielded better rewards and adjust the range of prices accordingly. They find that agents end up playing one-shot Nash prices unless industry profits enter the reward function in some way. \textcite{siallagan_aspiration-based_2013} propose \emph{aspiration based} learning where agents are allowed to communicate expectations to each other. They find that supra-competitive prices are attainable if the number of available options does not exceed 3.

Recent studies have focused on price instead of quantity competition. \textcite{klein_autonomous_2019} shows that \emph{Q-Learning} algorithms in a sequential price setting environment maintain a supra-competitive price level. He reports two types of equilibria: constant market prices and \emph{Edgeworth price cycles} where competitors sequentially undercut each other until profits become low and one firm resets the cycle by increasing its price significantly.\footnote{\textcite{noel_edgeworth_2008} considers a similar environment. However, he uses \emph{dynamic programming} for learning. His deployed agents \emph{know} their environment in detail, an assumption unlikely to hold in real markets. With \emph{Q-Learning}, agents estimate the action values based on past experiences.} Importantly, the high price levels are underpinned by a \emph{reward-punishment scheme}, i.e.\ a price cut of one agent evokes punishment prices by the opponent. Interestingly, the agents return to pre-deviaton levels within a couple of periods.

This study is closest to \textcite{calvano_artificial_2020} and \textcite{hettich_algorithmic_2021}. The former authors show that Q-Learning agents learn to sustain collusion through a \emph{reward-punishment} scheme in a simultaneous pricing environment. These findings are remarkably robust to variations and extensions. Furthermore, they find that agents learn to price competitively if they are memoryless (i.e.\ can not remember past prices) or short-sighted (i.e.\ do not value future profits). This coincides with predictions from economic theory. An important extension comes from \textcite{hettich_algorithmic_2021}. As in the present study, he utilizes function approximation, specifically a \emph{deep Q-Network algorithm} originally due to \textcite{mnih_human-level_2015}. He shows that the method converges much faster than \emph{Q-Learning}. The importance of that finding is augmented by the fact that the algorithm is much easier to scale to real applications.\footnote{\textcite{johnson_platform_2020} provide another extension. Introducing a \emph{multi-agent reinforcement learning} approach, they show that collusion arises even when the number of agents, often regarded a main inhibitor of cooperative behavior, is significantly increased. Moreover, they show that market design can significantly disturb collusion.}

To summarize, recent simulation studies show that reinforcement learning algorithms are capable of colluding in prefabricated environments. This paper tries to extend those findings by trialing \emph{linear} function approximation and eligibility traces.






	
	\section{Environment}\label{enironment}

This section presents the simulated economic environment that the autonomous pricing agents interact with. I consider an infinitely repeated pricing game with a logit demand as in \textcite{calvano_artificial_2019}. Restricting the analysis to the oligopoly case with $n=2$ agents (where $i = 1,2$), the market comprises \emph{2} differentiated products and an outside option. In every period $t$, both agents simultaneously pick a price $p_i$. Demand for agent $i$ is then determined\footnote{Generalization to a model with \emph{n} agents is straightforward. In fact, the demand formula remains the same. The limitation to 2 agents is merely chosen for computational efficiency and the (intuitive) conjecture that the simulation results generalize to more players provided learning time is sufficiently high.}:

\begin{gather}\label{quantity}
q_{i,t}=\frac{e^{\frac{a_i - p_{i,t}}{\mu}}}{\sum_{j=1}^{n}~ e^{\frac{a_j-p_{j,t}}{\mu}}+e^{\frac{a_0}{\mu}}}
\end{gather}

\textbf{Citations needed}
$\mu$ controls the degree of horizontal differentiation, where $\mu \rightarrow 0$ approximates perfect substitutability. Vertical differentiation is incorporated through the quality parameters $a_i$. $a_0$ reflects the appeal of the outside good. It diminishes as $a_0 \rightarrow -\infty$ \parencite{anderson_logit_1992}. 

Profits of both agents $\pi_i$ are simply calculated as

\begin{gather}\label{profit}
\pi_{i,t} = (p_{i,t} * q_{i,t}) - c_i,
\end{gather}

where $c_i$ is a firm-specific marginal cost. Market entry and exit are not considered. The baseline parametrization emulates \textcite{calvano_artificial_2019}:
$c_i = 1$,
$a_i = 2$,
$a_0 = 0$ and
$\mu = \frac{1}{4}$. These parameters give rise to a static Nash equilibrium with $p_n \approx 1.47$ and $\pi_n \approx 0.23$ per agent. The monopolist solution entails $p_m \approx 1.92$ with $\pi_m \approx 0.34$ for each product. Nevertheless, the following section covers the applied reinforcement learning methods for general parameters.

	
	\section{Reinforcement Learning with Function Approximation}

if the system were stationary --> convergence guarantee (e.g. Jaakoola et al. 1994). However, non-stationarity induced by multi-agent learning breaks that guarantee. To my knowledge, no guarantees, but empirically strong results

Though both agents repeatedly face the environment as presented in \autoref{enironment}, I will present this section from the vantage point of a single agent. Accordingly, the subscript {i} is dropped when appropriate.

\subsection{value approximation}\label{value_approximation}

The agent learns to approximate the value of an action given the available information. The potential actions reflect the available prices in the current period. It is useful to discretize the action space.\footnote{This discretization usually implies that agents will not charge exactly $p_n$ or $p_m$.} Compared to the baseline specification in \textcite{calvano_artificial_2019}, I consider a wider price range confined by a lower bound $A^L$ and an upper bound $A^U$:

\begin{gather}
A^{L} = c
\end{gather}

\begin{gather}
A^{U} = p_m + \zeta (p_n - c)
\end{gather}

The lower bound ensures positive margins. It is conceivable that a human manager could implement a sanity restriction like that before ceding pricing authority to an algorithm. The parameter $\zeta$ controls the extent to which the upper bound $A^U$ exceeds the monopoly price. With $\zeta = 1$, the difference between $A^{L}$ and $p_n$ is equal to the difference between $A^{U}$ and $p_m$. The available set of prices $\mathcal{A}$ is then evenly spaced out in the interval $[A^L, A^U]$:

\begin{gather}
	\mathcal{A} = (A^L, A^L + \frac{1(A^U - A^L)}{m-1}, A^L + \frac{2(A^U - A^L)}{m-1}~ , ... , ~ A^L + \frac{(m-2)(A^U - A^L)}{m-1}, A^U)
\end{gather}

$m$ determines the number of feasible prices. Following \textcite{sutton_reinforcement_2018}, I denote any possible action as $a \in \mathcal{A}$ and the actual realization at time $t$ as $A_t$.

In this simulation, the state set $S_t$ comprises merely the prices of the previous period $t-1$:

\begin{gather}
S_t = \{ p_{i, t-1}, p_{j, t-1} \}
\end{gather}


Accordingly, for every state variable, the set of possible states is identical to the feasible actions, i.e. $\mathcal{A}$. However, this is not required with function approximation methods. Theoretically, any state variable could be continuous and unbounded. Similarly to actions, $s \in \mathcal{S}$ denotes any possible state set and $S_t$ refers to the actual states at $t$.






Lastly, a set of parameters $\boldsymbol{w} = \{w_1, w_2, ..., w_D\}$, where $d \in \{1, 2, ..., D\}$, maps any combination of $S_t$ and $A_t$ to a value estimate $\hat{q-}_t$.\footnote{In the computer science literature, $\boldsymbol{w}$ is typically referred to as \emph{weights}. I will stick to the economic vocabulary and declare $\boldsymbol{w}$ parameters.} Hence:

\begin{gather}\label{q_estimation}
	\hat{q-}_t = \hat{q}(S_t,A_t,\boldsymbol{w}_t) = \hat{q}(p_{i, t-1}, p_{j, t-1}, p_{i, t}, \boldsymbol{w}_t)
\end{gather}

More specifically, any state-action combination is represented by a \emph{feature vector} $\boldsymbol{x}_t = \boldsymbol{x}(S_t, A_t) = \{x_1(S_t, A_t), x_2(S_t, A_t), ..., x_D(S_t, A_t)\}$ and every \emph{feature} $x_d = x_d(S_t, A_t)$, where every element is derived from a state, an action or a combination thereof. Moreover each $x_d$ is associated with a counterpart $w_d$. In section \ref{feature_extraction} the mechanisms to extract features are outlined in more detail. For now, note that I will only consider linear functions of $\hat{q}$. In this case, \autoref{q_estimation} can be written as the inner product of the feature vector and the set of parameters, i.e.\ $\boldsymbol{x}_t \top \boldsymbol{w} = \sum_{d=1}^{D} x_d * w_d$

Two elements are required for the algorithms to work successfully. First, the agents must mix between \emph{exploration and exploitation}. Second, the set of parameters $\boldsymbol{w}$ must be continuously optimized.

\paragraph{Exploration and Exploitation} 
In every period, the agent chooses either to \emph{exploit} its current knowledge and pick the supposedly optimal action or to \emph{explore} in order to test the merit of alternative choices that are perceived sub-optimal but may turn out to be superior. As is common, I use a simple $\epsilon$-greedy policy to steer this tradeoff:

\begin{gather}\label{action_selection}
 A_t = \begin{cases} arg ~\underset{a}{max} ~ \hat{q}(S_t,a,\boldsymbol{w}_t) & \quad \text{with probability } 1 - \epsilon_t\\
\text{randomize over } \mathcal{A} & \quad \text{with probability } \epsilon_t\\ \end{cases} 
\end{gather}

In words, the agent chooses to play the action that is regarded optimal with probability $1-\epsilon_t$ and randomizes over all prices with probability $\epsilon_t$.\footnote{If more than one $a$ maximizes $\hat{q}$, ties are broken randomly.} The subscript suggests that exploration varies over time. The explicit definition is given by:

\begin{gather}
	\epsilon_t = \psi e^{-\beta t}~ \text{, where}
\end{gather}

$\psi \in [0, 1]$ defines the initial exploration rate at $t = 0$ and $\beta$ controls the speed of its decay. This \emph{time-declining} exploration rate ensures that the agent randomizes actions frequently at the beginning of the simulation and stabilizes its behavior over time. 

After both agents selected an action, the quantities and profits are realized in accordance with equations \ref{quantity} \& \ref{profit}. The agents' actions in period $t$ become the state set in $t+1$ and new actions are chosen as dictated by equations \ref{q_estimation} \& \ref{action_selection}.

Irrespective of the \emph{exploit vs explore} decision, the agent proceeds to leverage the observed outcomes to refine $\boldsymbol{w}$.



\paragraph{Update}

After observing the opponent's price and the own profits, the agent exploits this new information to improve $\boldsymbol{w}$. A good starting point to introduce the utilized update rules is the so called \emph{TD error}, denoted $\delta_t$ (\textbf{finalize footnote}).\footnote{Without function approximation, versions of the \emph{TD error} usually encompass a discount factor $gamma$, such as:
	\begin{center}
		$\delta_t^{SARSA} = \pi_t + \gamma \hat{q}(S_{t+1}, A_{t+1}, \boldsymbol{w}) - \hat{q}(S_t, A_t, \boldsymbol{w})$
		
		$\delta_t^{Q-Learning} = \pi_t + \gamma ~ \underset{a}{max} ~ \hat{q}(S_{t+1}, a, \boldsymbol{w}) - \hat{q}(S_t, A_t, \boldsymbol{w})$
\end{center}
While they come with a meaningful economic interpretation, \textcite{sutton_reinforcement_2018} and \textcite{naik_discounted_2019} show that their use is inappropriate in infinite sequences with function approximation settings. Moreover, a policy maximizing average rewards is equivalent to a policy maximizing the average of discounted future values - irrespective of the particular discount factor.}

\textbf{average setting update}

\begin{gather}
	\delta_t = r_t - \widetilde{R}_{t-1} + \hat{q}(S_{t+1}, A_{t+1}, \boldsymbol{w}_t) - \hat{q}(S_t, A_t, \boldsymbol{w}_t) ~~   \text{,}
\end{gather}

where the reward $r_t = \pi_t - p_n$ reflects the profits relative to the Nash solution and $\widetilde{R}_{t-1}$ is a (weighted) average reward.\footnote{Please note that I explicitly distinguish profits and rewards. Profits, $\pi$, represent the monetary remuneration from operating in the environment and can be interpreted economically. However, profits do not enter the learning algorithm directly. Instead, rewards, $r$, immediate successors of profits, constitute the signal that is utilized as feedback by the agents to refine their algorithms.} $\delta_t$ measures the difference between the \emph{ex ante} ascribed value to the selected state-action combination in $t$ and the \emph{ex post} \emph{differential} profit $\pi_t - \widetilde{R}_{t-1}$ in conjunction with the estimated value of the newly arising state-action combination in $t+1$. A positive $\delta_t$ indicates that the actual realization turned out to exceed the original expectation. Likewise, a negative $\delta_t$ suggests that the realization failed short of the expected reward of playing the particular state-action combination. In both instances, $\boldsymbol{w}$ will be adjusted accordingly, such that the state-action combination is valued respectively higher or lower next time. Note that $\delta_t$ can only be calculated after the action in the next period has been taken.\footnote{This is often referred to as \emph{SARSA}, abbreviating a state-action-reward-state-action sequence.}


Surprisingly, this system 
* discount factor can be = 1






\emph{Semi-gradient} methods constitute a basic procedure for such continuous optimization. They serve as a good benchmark before developing more complex algorithms. Essentially, the direction and magnitude of updating parameters is driven by the \emph{TD error} $\delta_t$ and the gradient of $\hat{q}_t(S_t, A_t, \boldsymbol{w})$ with respect to $\boldsymbol{w}$:
$\frac{\Delta \hat{q}}{\Delta \boldsymbol{w}} =
\{ \frac{\Delta \hat{q}}{\Delta w_1},
\frac{\Delta \hat{q}}{\Delta w_2},
...,
\frac{\Delta \hat{q}}{\Delta w_d}  \}$. The update rule is:

\begin{gather}
 \boldsymbol{w}_{t+1} \leftarrow \boldsymbol{w}_t +
 	\alpha \delta_t
 	\frac{\Delta \hat{q}}{\Delta \boldsymbol{w}} ~~ \text{,}
\end{gather}

where $\alpha$ steers the speed of learning. As indicated earlier, I will only consider approximations that are linear in parameters. Thus, $\frac{\Delta \hat{q}}{\Delta \boldsymbol{w}}$ simplifies to the \emph{feature vector} $\boldsymbol{x}_t = \{x_1, x_2, ..., x_d\}$.

\textcite{seijen_true_2014} showed that the performance of that algorithm can be improved by keeping track of an eligibility vector $\boldsymbol{z} = \{z_1, z_2, ..., z_d\}$, called the \emph{Dutch Trace}. Like $\boldsymbol{w}$, this trace vector is updated at every time step. The trick is that the eligibility trace controls the magnitude by which individual parameters are updated, prioritizing those that contributed to producing an estimate of $hat{q}$. The update rule for the eligibility trace is:

\begin{gather}
\boldsymbol{z}_{t} \leftarrow \boldsymbol{z}_{t-1} \gamma \lambda + \boldsymbol{x_t} (1 - \alpha \gamma \lambda  {  \boldsymbol{x_t} \intercal \boldsymbol{z}_{t-1} }) ~~ \text{,}
\end{gather}

where $ \boldsymbol{x}_t \top \boldsymbol{z}_t $ denotes the inner product of $\boldsymbol{x}_t$ and $\boldsymbol{z}_t$, i.e. $ \boldsymbol{x}_t \top \boldsymbol{z}_t  = \sum_{i}^{d} x_i z_i$. $\boldsymbol{z}$ is then used in the refined parameter update:

\begin{gather}
	\boldsymbol{w}_{t+1} \leftarrow
		\boldsymbol{w}_{t} +
		\alpha \delta \boldsymbol{z}_t +
		\alpha ( \boldsymbol{w}_t \intercal \boldsymbol{x}_t  -
				 \boldsymbol{w}_{t-1} \intercal \boldsymbol{x}_t)
				(\boldsymbol{z}_t - \boldsymbol{x}_t)
\end{gather}





\subsection{Feature Extraction}\label{feature_extraction}

\textbf{TBD: introduction}
As outlined in \autoref{value_approximation}, the state-action space contains just 3 variables. Assigning a single coefficient to each variable certainly fails to do justice to the complexity of the optimization problem. In particular, a \emph{reward-punishment} theme requires that actions are chosen conditional on past prices (i.e.\ the state space). Hence, it is imperative to consider interactions and non-linearities. Therefore, I utilize various methods to extract features form the state-action space.

In reinforcement learning, a common approach is to store a distinct set of coefficients for every feasible action.\footnote{In this case, the vector of coefficients contains \emph{m} times features components} This is a sensible approach with qualitative action spaces. However, very much like tabular learning, a separate set of coefficients neglects the (quasi-) continuous nature of prices. Therefore two issues arise. First, discretizing the action space doesn't scale well if the number of feasible prices increases. Consequently, learning requires relatively many periods with large $m$. Second, observing a particular reward may not only constitute an informative feedback for the particular action undertaken, but also for 'similar' prices. Using and updating coefficients valid for (a subset of) all feasible prices exploits this.

For this simulation, I use \emph{polynomials}, \emph{polynomial splines} and \emph{tile coding} to extract features from the state-action space.

\subsubsection{Polynomials}

\emph{Polynomial approximation} of order $k$ maps states and action to a set of features, where a single feature corresponds to:



\begin{gather}
x_i^{Poly} = p_{1, t-1}^{\kappa_1} ~ p_{2, t-1}^{\kappa_2} ~ p_{1, t}^{\kappa_3}
\end{gather}


Every combination of exponents that adheres to the restrictions

\begin{itemize}
	\item $0 < \kappa_1 + \kappa_2 + \kappa_3 \leq k$ and
	\item $\kappa_1, \kappa_2, \kappa_3 \in \{0, 1, ..., k\}$
\end{itemize}

constitutes one feature. Using polynomial approximation, the feature vector $\boldsymbol{x}$ contains ${k + 3\choose3}  - 1$ elements.

\subsubsection{Normalized Polynomials}

\textbf{TBD}

\subsubsection{Polynomial Splines}

\textbf{TBD}

\subsubsection{Tile Coding}

In reinforcement learning, \emph{Tile Coding} is a common way to extract linear, in fact binary, features from a state-action space.\footnote{for an extensive introduction with instructive illustrations refer to \textcite{sutton_reinforcement_2018}} The idea is that several \emph{tilings} superimpose the state-action space. The $\mathcalligra{T}$ \ tilings are offset but each tiling covers the entire state-action space:

\begin{gather}
	 \mathcal{T}^L \leq A^L  ~ \& ~ \mathcal{T}^U \geq A^U    \text{for } \mathcal{T} \in \{1, 2, ..., \mathcalligra{T} ~ \}
\end{gather}

Each tiling is itself composed of uniformly spaced out \emph{tiles}.\footnote{With 2 dimensions, a tiling simply corresponds to a grid. In our case, the state-action space is 3-dimensional, so it may prove more intuitive to think of cubes instead of tilings and tiles.} Every tile is uniquely demarcated by a lower and an upper threshold for every dimension. Consequently, the number of tiles per tiling is controlled by the number of thresholds. For this simulation, it suffices to define a single set of thresholds per tiling that applies to all 3 dimensions. More specifically, the thresholds are spaced out evenly in the tiling-specific interval $[\mathcal{T}^L, \mathcal{T}^U]$:

\begin{gather}
\mathcal{T} = (
\mathcal{T}^L,
\mathcal{T}^L + \frac{1(\mathcal{T}^U - \mathcal{T}^L)}{\tau},
\mathcal{T}^L + \frac{2(\mathcal{T}^U - \mathcal{T}^L)}{\tau}~ , ... , ~
\mathcal{T}^L + \frac{(\tau-1)(\mathcal{T}^U - \mathcal{T}^L)}{\tau},
\mathcal{T}^U)
\end{gather}

This gives rise to $\tau^3$ tiles per tiling. Tiles are binary, i.e.\ if a state-action observation falls into a particular demarcation, the corresponding tile is \emph{activated}:

\begin{gather}\label{tile_activation}
x_i^{Tiling} = \begin{cases}
1 & \quad \text{if } \{p_{1, t-1}, p_{2, t-1}, p_{1, t}\} \text{~in tile demaraction}_i  \\
0 & \quad \text{if } \{p_{1, t-1}, p_{2, t-1}, p_{1, t}\} \text{~not in tile demarcation}_i \\ \end{cases} 
\end{gather}

Since tiles within a tiling are non-overlapping, any state-action combination activates exactly $\mathcal{T}$ tiles, one per tiling. The total number of features is simply $\mathcalligra{T}~\tau^3$. Note that the tabular case can be recovered as a special case by setting $\mathcalligra{T}~ = 1$ and $\tau \leq m$. In this case, every tile is activated by at most one feasible state-action combination which is equivalent to storing a dedicated coefficient for every state-action combination.\footnote{If $\tau > m$, some tiles would never be activated. But again, every table entry would correspond to a unique tile.}

\textbf{TBD: This feature exhibits the advantage of function approximation in large state spaces.  dimensionality problem when increasing the exponent, or $\tau$ not so much when increasing the number of tilings}

	
	
\section{Feature extraction}\label{feature_extraction}

\textbf{TBD: extended introduction (?)}

This section lays out the methods used in this study to map state-action combinations to a set of numerical values $\boldsymbol{x}$. I shall refer to them as \emph{feature extraction methods}. \footnote{The term \emph{feature} is borrowed from the computer science literature. It usually refers to a (transformed) input variable. It is common that the number of features dwarfs the number of original inputs.} As outlined in \autoref{value_approximation}, the state-action space contains just 3 variables ($p_{i,t-1}$, $p_{j,t-1}$ and $p_{i,t}$). To illustrate, consider a naive attempt of feature extraction where each $S_t$ and $A_t$ is converted to a feature without mathematical transformation, i.e.\ $x_1 = p_{i,t-1}$, $x_2 = p_{j,t-1}$ and $x_3 = p_{i,t}$. Every feature is assigned a single coefficient $w_d$ and the estimated value of any state-action combination would then be $\hat{q}(S_t, A_t, \boldsymbol{w}) = \sum_{d=1}^3 w_d x_d$. Obviously, this simplistic method fails to do justice to the complexity of the optimization problem. In particular, a \emph{reward-punishment} theme requires that actions are chosen conditional on past prices (i.e.\ the state space). Hence, it is imperative to consider interactions and non-linearities. \autoref{feature_extraction_summary} provides an overview of the 4 methods used in this study.

\begin{center}
	\begin{table}
		\begin{tabular}{|l|l|l|c|}
			\hline
			\textbf{Feature Extraction}&\textbf{Baseline}&\textbf{Length} $\boldsymbol{x}$&\textbf{factor when}\\
			\textbf{Method}&\textbf{Parametrization}&&\textbf{doubling} $m$\\
			\hline
			Tabular&-&$m^3 = 6,859$& x8\\
			\hline
			Tile Coding&$T = 5, \psi = 9$&$T~(\psi - 1)^3 = 2,560$& x1\\
			\hline
			Polynomial Tiles&$T = 5, \psi = 5, k = 4$&$T~(\psi - 1)^3 ({k + 3\choose3}  - 1) = 10880$& x1 \\
			\hline
			Separated Polynomials&$k = 5$ &$m($ ${k+2}\choose{2}$ $-1) = 380$& x2 \\
			\hline
		\end{tabular}
		\caption{Feature extraction methods, number of parameters with $m=19$ and complexity.}
		\label{feature_extraction_summary}
	\end{table}
\end{center}

\textbf{1. finish table, 2. described columns in table briefly}

The remainder of this chapter describes the methods in more detail. Note that \emph{polynomial approximation}, as described in \autoref{polynomial}, is not directly used in this study but nevertheless introduced as a precursor to the final two methods \emph{polynomial tiles} and \emph{separate polynomials}.


\subsection{Tabular Learning}\label{tabular}

A natural way to represent the state-action space is to preserve a distinct feature (and coefficient) for every unique state-action combination. Features are binary, i.e.\ any feature is  $1$ if the associated state-action combination is selected and $0$ otherwise:

\begin{gather}\label{cell_activation}
x_d^{Tabular} = \begin{cases}
1 & \quad \text{if } \{p_{1, t-1}, p_{2, t-1}, p_{1, t}\} \text{~corresponds to cell}_d  \\
0 & \quad \text{if } \{p_{1, t-1}, p_{2, t-1}, p_{1, t}\} \text{~does not correspond to cell}_d \\ \end{cases} 
\end{gather}

The respective coefficient tracks the performance over time and directly represents the \emph{value} of that state-action combination. Accordingly, the length of $\boldsymbol{x}$ is $m^3$.\footnote{$3$ derives from the 2 prices from the previous episode plus the considered action of the current episode.} This approach is called \emph{tabular} because it is easy to imagine a table where every cell represents a unique state-action combination. Tabular methods have been used extensively in the simulations on algorithmic collusion that provided empirical evidence of collusive outcomes being possible in simple environments (\textbf(citations)). Their widespread application is justified by their conceptual simplicity and their historic usage in autonomous pricing of airline fares and electricity markets \parencite{ittoo_algorithmic_2017}. Moreover, tabular methods give rise to a family of robust learning algorithms with well-understood convergence guarantees (\textbf{citation}) \footnote{Q-Learning being just one particular application.}.

However, tabular methods are not necessarily the best or fastest way to learn an optimal policy. In real life markets, a salient factor may impede its effectiveness. Prices are (quasi-) continuous - a treat completely ignored by tabular methods. This has two major implications. First, the leeway of decision makers is artificially restricted. Second, due to a \emph{curse of dimensionality}, learning speed and success may deteriorate disproportionately with $m$. I will take a closer look at each of these points.

Obviously, any decision maker is only restricted by the currency's smallest feasible increment and can charge more than just a couple of prices. It is certainly conceivable, maybe even desirable, that a decision maker reduces the own number of considered prices to simplify the decision process. However, in most cases it will be impossible to impose such a restriction on competitors. As an extreme example, consider an opponent who never charges the same price twice. Whenever this opponent introduces a new price, a tabular learning agent is coerced to create a new cell in the state-action matrix that will never be revisited. Consequently, the agent continuously encounters new situations from which it can learn, but never utilizes the acquired knowledge.\footnote{One could attempt to circumvent this problem by discretizing the prices \emph{ex ante} (e.g. as in \autoref{available_prices}) and simply convert the real price to the closest available alternative. While this introduces some imprecision and it is unclear how to optimally discretize prices, it might constitute a practicable solution. In fact, that approach is a special case of \emph{tile coding}, the method I will introduce in \autoref{tile_coding}.}

More importantly, tabular learning does not scale well with $m$ and $n$. In the baseline specification, the number of features is $19^3 = 6859$. Doubling $m$ from $19$ to $38$ causes an eightfold increase of that number to $54,872$. Even worse, increasing the number of competitors alters the exponent. Changing $n$ from $2$ to $3$ entails an increase of features by the factor $m$, in the baseline specification from $6859$ to $130,321$.\footnote{A similar problem arises when the algorithm is supposed to account for cost and demand factors. Every added input, whether due to an additional opponent or any other profit-related variable, increases the table by a factor or $m$. While changes in costs and prices are not considered in this study, they obviously play an important role in reality.} It is easy to see that modest increases in complexity have the potential to evoke a disproportionate reduction in learning speed. Indeed, \textcite{calvano_algorithmic_2018} show that increasing $m$ and $n$ tends to reduce profits of tabular learning agents.

Another way of looking at the same issue is to consider the nature of the variable $p$. Prices are continuous and transforming them into a qualitative set of discrete actions disregards that fact. In particular, it prevents the opportunity to learn from the result of charging a particular price about the quality of \emph{similar} prices in the same situation. To illustrate with an inflated example, consider a manager who observes large profits after charging a price of $1000$. A human manager is able to infer that charging $1001$ instead would have yielded a similar profit. Tabular learning agents are not.


\textbf{point at tradeoff: precision vs. learning speed}
			* approximate continuous prices by reducing price intervals to arbitrary length



\subsection{Function approximation methods}

The function approximation methods considered in this study alleviate the \emph{curse of dimensionality}. In fact, length of the feature vector $\boldsymbol{x}$ in \emph{tile coding} and \emph{polynomial tiling} is unaffected by $m$. For \emph{separate polynomials}, it is proportional to $m$, i.e.\ doubling the number of feasible prices also doubles the number of features. Moreover, all methods augment learning in the sense that a particular state-action combination tends to evoke ampler parameter updates (in accordance with \autoref{update_rule}) that also change the future evaluation of \emph{similar} state-action combinations.

\subsubsection{Tile Coding}\label{tile_coding}
In reinforcement learning, \emph{tile coding} is a common way to extract binary features from a state-action space.\footnote{for an extensive introduction with helpful illustrations refer to \textcite{sutton_reinforcement_2018}} Its appeal stems partly from the fact that it is a generalization of tabular learning. The idea is that several \emph{tilings} superimpose the state-action space. The $\mathcal{T}$ tilings are offset but each tiling covers the entire state-action space:

\begin{gather}
	 T^L \leq A^L  ~ \text{and} ~ T^U \geq A^U ~~ \forall  ~~ T \in \{1, 2, ..., \mathcal{T} ~ \} ~~ \text{,}
\end{gather}

where $T^L$ and $T^U$, respectively, represent the lower and upper bound of tiling $T$. Each tiling is itself composed of uniformly spaced out \emph{tiles}.\footnote{With 2 dimensions, a tiling simply corresponds to a grid. In our case, the state-action space is 3-dimensional, so it may prove more intuitive to think of cubes instead of tilings and tiles.} Every tile is uniquely demarcated by a lower and an upper threshold for every dimension. Consequently, the number of tiles per tiling is controlled by the number of thresholds. For this simulation, it suffices to define a single set of thresholds per tiling that applies to all 3 dimensions. More specifically, the thresholds are spaced out evenly in the tiling-specific interval $[T^L, T^U]$:

\begin{gather}
(
T^L,
T^L + \frac{1(T^U - T^L)}{\psi - 1},
T^L + \frac{2(T^U - T^L)}{\psi - 1}~ , ... , ~
T^L + \frac{(\psi-2)(T^U - T^L)}{\psi - 1},
T^U) ~~ \text{,}
\end{gather}

where $\psi$ represents the number of thresholds. This gives rise to $(\psi-1)^3$ tiles per tiling. As indicated, tiles are binary, i.e.\ if a state-action observation falls into a particular demarcation, the corresponding tile is \emph{activated}:

\begin{gather}\label{tile_activation}
x_d^{Tiling} = \begin{cases}
1 & \quad \text{if } \{p_{i, t-1}, p_{j, t-1}, p_{i, t}\} \text{~in tile demaraction}_d  \\
0 & \quad \text{if } \{p_{i, t-1}, p_{j, t-1}, p_{i, t}\} \text{~not in tile demarcation}_d \\ \end{cases} 
\end{gather}

Since tiles within a tiling are non-overlapping, any state-action combination activates exactly $\mathcal{T}$ tiles, one per tiling. The total number of features is simply $T~(\psi - 1)^3$. Note that the tabular case can be recovered as a special case by setting $\mathcal{T}~ = 1$ and $\psi \geq m + 1$. In this case, every tile is activated by at most one feasible state-action combination which is equivalent to storing a dedicated coefficient for every state-action combination.\footnote{If $\psi > m + 1$, some tiles would never be activated. But again, every table entry would correspond to a unique tile.}


\subsubsection{Polynomials}\label{polynomial}

\emph{Polynomial approximation} applies polynomial transformations to its inputs. In order to keep this (and the upcoming) section brief, I will introduce the notation for the specific case of 3 variables.\footnote{for a more thorough treatment with variations, see e.g.\ \autoref{hastie}} Polynomial approximation of order $k$ maps $S_t$ and $A_t$ to a set of features, where a single feature corresponds to:


\begin{gather}\label{polynomial_extraction}
x_d^{Poly} = p_{i, t-1}^{\kappa_{d,1}} ~ p_{j, t-1}^{\kappa_{d,2}} ~ p_{i, t}^{\kappa_{d,3}}
\end{gather}


Every combination of exponents that adheres to the restrictions

\begin{itemize}
	\item $0 < \kappa_{d,1} + \kappa_{d,2} + \kappa_{d,3} \leq k  ~~ \forall ~ d$ and
	\item $\kappa_{d,1}, \kappa_{d,2}, \kappa_{d,3} \in \{0, 1, ..., k\} ~~  \forall ~ d$
\end{itemize}

constitutes one feature. Using polynomial approximation, the feature vector $\boldsymbol{x}$ contains ${k + 3\choose3}  - 1$ elements. I chose not to use a simple polynomials to approximate the valuation of the entire state-action space. Exploratory runs have shown that the method has some trouble converging and frequently produced unreasonable results in the provided environment. Perhaps, this is not surprising because every state-action combination will always produce non-zero values for \emph{all} features and change every single element in $\boldsymbol{w}$. This makes it difficult for the algorithm to develop different notions for \emph{different} prices.

\subsubsection{Polynomial Tiles}

What I call \emph{polynomial tiles} is a blend of \emph{tile coding} and \emph{polynomial approximation}. To be precise, just as in tile coding, the state-action space is divided into overlapping tiles. However, instead of a binary indication, every tile comprises a distinct polynomial. For the sake of notation, it is helpful to divide the index $d$ into a tiling component $e$ and a polynomial part $f$.  Hence:

\begin{gather}\label{poly_tiling_extraction}
x_d^{Poly~Tiling} = x_{e,f}^{Poly~Tiling} = \notag \\
\begin{cases}
p_{i, t-1}^{\kappa_{f,1}} ~ p_{j, t-1}^{\kappa_{f,2}} ~ p_{i, t}^{\kappa_{f,3}} & \quad \text{if } \{p_{i, t-1}, p_{j, t-1}, p_{i, t}\} \text{~in tile demaraction}_e  \\
0 & \quad \text{if } \{p_{i, t-1}, p_{j, t-1}, p_{i, t}\} \text{~not in tile demarcation}_e \\ \end{cases} 
\end{gather}

The restrictions on the exponents $\kappa$ from \autoref{polynomial} apply. The method accompanies $T~(\psi - 1)^3 ({k + 3\choose3}  - 1)$ features. As this method allows for a distinguished value estimation for different state-action combinations within a tile, it appears reasonable to increase the size of the tiles in order to decrease the number of coefficients and avoid overfitting. Specifically, I retain the number of tilings, but reduce the number of tiles per tiling from $512$ to $64$ by imposing $\mathcal{T} = 5$ and $\psi = 5$. Moreover, I allow for polynomial combinations up to degree $k=4$.

\subsubsection{Separated Polynomials}

\emph{Separated polynomials} maintain for every action a distinct set of parameters that apply \emph{polynomial approximation} to the state set. In reinforcement learning, it is common to store a separate set of coefficients for every feasible action \textbf{citation}.\footnote{This approach is best suited if the action spaces is qualitative and the state space continuous. In this simulation, only the latter is strictly true. Therefore, the two issues inherent to tabular learning I have outlined in \autoref{tabular}, also apply to the action space of \emph{separated polynomials}, but not to the state space.} Since $A_t$ is fixed within each set, the polynomial only considers $S_t$:\footnote{The restrictions on $k$ are adjusted accordingly:
	\begin{itemize}
		\item $0 < \kappa_{d,1} + \kappa_{d,2}  \leq k  ~~ \forall d$ and
		\item $\kappa_{d,1}, \kappa_{d,2} \in \{0, 1, ..., k\} ~~  \forall d$
	\end{itemize}
}


\begin{gather}\label{separated_poly_extraction}
x_d^{Separated~Poly} = \begin{cases}
p_{i, t-1}^{\kappa_{d,1}} ~ p_{j, t-1}^{\kappa_{d,2}} & \quad \text{if } a = A_t  \\
0 & \quad \text{if } a \ne A_t \\ \end{cases} 
\end{gather}

Note that the method models the value of an action as a function of $S_t$. The number of encompassed features arises naturally as $m($ $k+2\choose2$ $-1)$.


Perhaps, an inverse variation could be more intuitive from an economic perspective. Consider a case where every permutation of $S_t$ holds a distinct set of parameters. This approach is closer to the notion of selecting $a$ to optimize the reward \emph{given} a fixed state set $s$. I leave this variation open as a potential avenue for future research.

\subsection{Parameter Grid}

TBD
	
	\section{Results}\label{results}
This section reports on the simulation outcomes of the baseline specification. To foreshadow the results, profits mostly exceed Nash-predictions, but remain below monopoly profits. While agents learn to charge supra-competitive prices, they fail to incorporate \emph{reward-punishment} schemes consistently. Overall, the results crucially hinge on the combination of \gls{fem} and selected parameters. Only tabular learning exhibits a clear tendency to punish deviations with lower prices in subsequent periods.

I report results for various specifications and will refer to every unique combination of \gls{fem} and parameters as an \emph{experiment}. Every experiment consists of 48 \emph{runs}, i.e. repeated simulations with the exact same set of starting conditions. Lastly, within the scope of a particular \emph{run}, time steps are called \emph{periods}.\footnote{The simulations are run in \emph{R}. The code is available on github (\url{https://github.com/MalteJe/ai_collusion}). Another technical note: The program seed does not vary between experiments, i.e.\ for every run there is a \emph{sibling run} with an equivalent initialization of the random number generator in every other experiment. This affects the initial state and the decision in which periods the agents decide to explore.}

\subsection{Convergence}\label{convergence}

\textbf{TBD: As indicated,} convergence is not guaranteed in a non-stationary environment, much less so with function approximation. Notwithstanding the lack of a theoretical convergence guarantee, prior experiments have shown that simulation runs tend to approach a stable equilibrium in practice (\cite{calvano_artificial_2020} \textbf{and others}). In the context of diagnosing convergence, \emph{stability} simply refers to the observation that the same set of prices continuously recur over a longer time interval. The strategies upon convergence need not coincide with economic theory. In fact, at times the observed outcomes in this study contradict predictions from game theory. For instance, despite symmetric profit functions, the converged outcomes may display asymmetric prices. Moreover, price cycles, i.e.\ a recurring sequence of price combinations, occur frequently.\footnote{The model from \autoref{quantity} predicts symmetric outcomes without cycles. This is typical for simultaneous pricing games, but not universal across economic models. For instance, collusive outcomes in quantity competition (i.e.\ Cournot) may exhibit price asymmetries. The relevance of that prediction has been fortified in experimental settings, e.g.\ in \textcite{fischer_collusion_2019}. \textcite{maskin_theory_1988} pioneer a sequential pricing game that predicts \emph{Edgeworth price cycles} where agents successively undercut each other until one firm prefers to reset the cycle and increases its price. Based on their model, \textcite{klein_autonomous_2019} shows that \emph{Q-Learning} agents are indeed capable of learning those dynamic strategies.}

The following, arbitrary but practical, convergence rule was employed. If a price cycle recurred for 10,000 consecutive episodes, the algorithm is considered \emph{converged} and the simulation concludes. A price cycle requires both agents' adherence.\footnote{Of course it is possible that the cycle length differs between agents. For instance, one agent may continuously play the same price while the opponent keeps alternating between two prices. In this case, the cycle length is $1*2=2$.} For efficiency reasons, price cycles up to a length of 10 are considered and a check for convergence is undertaken only every 2,000 episodes. If no convergence is achieved until 500,000 episodes, the simulation stops and the run is deemed \emph{not converged}. Furthermore, there are a number of runs that \emph{failed to complete} as a consequence of the program running into an error. Unfortunately, the program code does not allow to examine the exact cause of such occurrences in retrospect. However, by and large, the failed runs occurred with unsuitable specifications (see below for a detailed discussion).

\begin{figure}
	\includegraphics[width=\linewidth]{plots/converged.png}
	\caption[Converged runs by \gls{fem} and $\alpha$]{Number of runs per experiments that (i) achieved convergence, (ii) did not converge or (iii) failed to complete as a function of \gls{fem} and $\alpha$.}
	\label{converged}
\end{figure}

In accordance with the outlined convergence criteria above, \autoref{converged} displays the share of runs that, respectively, converged successfully, did not converge until the end of the simulation or failed to complete. Two main conclusions emerge. First, failed runs are mainly prevalent in specifications with a high value of $\alpha$ in conjunction with a polynomial \gls{fem}. Second, the tiling methods are more likely to converge. Both points deserve some further elucidation.

Regarding the failed runs, recall from \autoref{learning_speed_considerations} that features of polynomial extraction are not binary and warrant cautious adjustments of the coefficient vector. I suspect that with unreasonably large values of $\alpha$, the estimates of $\boldsymbol{w}$ overshoot early in the simulation, don't recover and at some point exceed the software's numerical limits.\footnote{Controlled runs where I could carefully monitor the development of the coefficient vector $\boldsymbol{w}$ seem to confirm the hypothesis. However, isolated errors \emph{with} small values of $\alpha$ remain unexplained, see in particular the top right panel in \autoref{converged}.} While important to acknowledge, the failed runs are largely an artifact of unreasonable specifications and I will \textbf{disregard them for the remainder of this chapter}. For instance, the percentages in the subsequent paragraph don't account for the failed runs.

Out of the completed runs without program failure, 95.4\% did converge. Interestingly, there are subtle differences between \gls{fem}s. With only one exception, both tiling methods converged consistently for various $\alpha$. With only 85.4\% of runs converging, separate polynomials constitute the other extreme. The figure also indicates that convergence becomes less likely for low values of $\alpha$. With tabular learning, 92.9\% of runs converged without clear relation to different values of $\alpha$.

\autoref{convergence_at} displays a frequency polygon of the runs that achieved convergence within 500,000 episodes. Clearly, the distribution is fairly uniform across \gls{fem}s. Most runs converged between 200,000 and 300,000 runs. This is an artifact of the decay in exploration as dictated by $\beta$. Before the focal point of 200,000 is reached, agents probabilistically experiment too frequently to observe 10,000 consecutive episodes without any deviation from the learned strategies. Thereafter, it becomes increasingly likely that both agents keep \emph{exploiting} their current knowledge and continuously play the same strategy for a sufficiently long time to trigger the convergence criteria. Note that the low quantity of runs converging between 300,000 and 500,000 suggests that increasing the maximum of allowed episodes would not necessarily produce a significantly higher share of converged runs.

\begin{figure}
	\includegraphics[width=\linewidth]{plots/convergence_at.png}
	\caption[Timing of convergence]{Timing of convergence, only includes converged runs. Width of bins: 8,000.}
	\label{convergence_at}
\end{figure}

\autoref{cycle_length} visualizes the distribution of cycle length and offers some interesting insights. Unsurprisingly, a first glance suggests that the frequency of runs decreases with cycle length. Not accounting for differences between selection methods, the bars appear similar to a geometric distribution with the largest bar corresponding to a 'cycle length of 1' (i.e.\ no cycle at all). Moving towards the right, the frequency of observed runs decreases with cycle length, though at a decreasing pace. In fact, there are even 7 runs with the largest considered cycle length of 10.

This time, the differences between \gls{fem}s are substantial. Polynomial tiles largely follows the described decaying pattern. Similarly, tile coding rarely converges in long cycles, though the most prevalent cycle length is 2 (with a total of 194 runs). Contrary, almost all runs of the separate polynomials converged without cycles.\footnote{Though barely visible in \autoref{convergence_at}, there are 2 runs with a cycle length of 2.} Lastly, the frequency of cycle length of converged tabular runs is distributed almost uniformly. This observation also suggests that the employed convergence rule may well have misclassified some of the runs in the top left panel of \autoref{converged} as \emph{not converged} where in reality the convergence cycle length simply exceeded the threshold arbitrarily set at 10. 

\begin{figure}
	\includegraphics[width=\linewidth]{plots/cycle_length.png}
	\caption{Number of converged runs by \gls{fem} and cycle length.}
	\label{cycle_length}
\end{figure}


It is not obvious why there exist such differences between \gls{fem}s, especially since there is no economic justification for price cycles in the simultaneous pricing environment. Nevertheless, it is important to recognize those differences.  Appendix \ref{prices} unveils that the cycle length is also positively related to the range of prices agents charge upon convergence. Next, I proceed by examining profits.

\subsection{Profits}\label{profits}

In order to benchmark the simulation profits, I normalize profits as in \textcite{calvano_artificial_2020} and \textcite{hettich_algorithmic_2021}:

\begin{gather}
\Delta = \frac{\bar{\pi} - \pi_n}{\pi_m - \pi_n} ~~ \text{,}
\end{gather}

where $\bar{\pi}$ represents profits averaged over the last 100 time steps upon convergence and over both agents in a single run. The normalization implies that $\Delta = 0$ and $\Delta = 1$ respectively reference the Nash and monopoly solution.\footnote{I will denote these special cases as $\Delta_n$ and $\Delta_m$.} Note that it is possible to obtain a $\Delta$ below $0$ (e.g. if both agents charge prices equal to marginal costs), but not above $1$.\footnote{Strictly speaking, exactly 1 is not attainable either. Recall that $m$ was chosen to allow for prices very close, but not equal to both benchmark prices. With $m = 19$, the highest feasible $\Delta$ is 0.9997.}

\begin{figure}
	\includegraphics[width=\linewidth]{plots/alpha.png}
	\caption[average $\Delta$ by \gls{fem} and $\alpha$]{average $\Delta$ by \gls{fem} and $\alpha$. Includes converged and non-converged runs. One data point (\textbf{polynomial tiles}, $\alpha = 0.0004$) is excluded for better presentability. Beware the logarithmic x-scale.}
	\label{alpha}
\end{figure}

\autoref{alpha} displays the convergence profits as a function of \gls{fem} and $\alpha$. Every data point represents one experiment, more specifically the mean of $\Delta$ across all runs making up the experiment. First of all, note that average profits consistently remain between both benchmarks $p_m$ and $p_n$ across specifications.\footnote{There is one exception. One data point is hidden in the plot to preserve reasonable y axis limits. More specifically, for the \gls{fem} polynomial tiles with $\alpha = 0.0001$, the average $\Delta$ is -1.73. This extends the observation in \autoref{convergence}. It appears that this particular $\alpha$ constitutes a critical point. While the program does not crash, agents only learn strategies void of any reasonableness.} As with prior results, the plot unveils salient differences between \gls{fem}s. On average, polynomial tiles runs yield the highest profits. The average $\Delta$ peaks at 0.85 for $\alpha = 10^{-8}$. Higher values of $\alpha$ tend to progressively decrease profits. Moving downwards on the y-axis, both tabular learning and tile coding yield similar average values of $\Delta$. Furthermore, the level of $\alpha$ does not seem to impact $\Delta$ much. For both methods $\alpha = 10^{-4}$ induces the highest average $\Delta$ at 0.487 and 0.478 respectively. Similarly for separate polynomials, $\Delta$ does not seem to respond to variations in $\alpha$. The maximum $\Delta$ is 0.35.

\begin{figure}
	\includegraphics[width=\linewidth]{plots/alpha_violin.png}
	\caption[Distribution of $\Delta$ by \gls{fem} and $\alpha$]{Distribution of $\Delta$ by \gls{fem} and $\alpha$. Includes converged and non-converged runs. Violin widths are scaled to maximize width of individual violins, comparisons of widths between violins are not meaningful. Violins are trimmed at smallest and largest observation respectively. One experiment (polynomial tiles, $\alpha = 0.0004$) is excluded for better presentability. Horizontal lines represent the median. Beware the logarithmic x-scale.}
	\label{alpha_violin}
\end{figure}

Naturally, averaging $\Delta$ over all runs of an experiment, as done to create \autoref{alpha},  has the potential to hide subtleties in the distribution of $\Delta$. Therefore, \autoref{alpha_violin} displays a violin plot illustrating the distribution of $\Delta$ per experiment. The distribution largely confirms the conclusion that most runs converge between $\Delta_m$ and $\Delta_n$. The only \gls{fem} that generated a significant quantity of runs with profits below the Nash benchmark are separate polynomials where 19.8\% of runs converged with profits below the Nash equilibrium, though most of them ended up reasonably close. The percentage is largest for the experiment with $\alpha = 10^{-8}$: 27.1\%. While the other methods tend to elicit runs within the benchmarks, the variability remains quite high. This indicates a degree of path dependence and suggests that the algorithms are prone to stick to early explored strategies that are \emph{above average}, but \emph{sub-optimal}. Polynomial tiles exhibit the narrowest range of $\Delta$, in particular for low $\alpha$.

Section \ref{convergence} and \autoref{alpha} established that, what constitutes a sensible value of $\alpha$ clearly depends on the \gls{fem}. Therefore, for the remainder of this chapter, I will select an 'optimal' $\alpha$ for every \gls{fem} and present further results only for these combinations. In determining \emph{optimality} of $\alpha$, I don't rely on a single hard criteria, rather I consider a number of factors including the percentage of converged runs, comparability with previous studies and prefer to select experiments with high average $\Delta$ as they are most central to the purpose of this study. \autoref{justifications} provides a justification for every experiments setting deemed \emph{optimal}. To get a sense of the variability of runs within the optimized experiments and the price trajectory over time, \autoref{trajectory_Delta} displays the development of profits of all runs for the optimal values of $\alpha$. Moreover, \autoref{appendix} contains further trajectory visualizations of prices and profits.


	\begin{table}
		\centering
		\begin{tabular}{|l|c|l|}
			\hline
			\textbf{\gls{fem}}&$\boldsymbol{\alpha}$&\textbf{justification} \\
			\hline
			Tabular&0.1&- comparability with previous simulation studies \\
			&&- most pronounced response to price deviations \\
			&& \ \ (see \autoref{deviations}) \\
			\hline
			Tile Coding&0.001&- high $\Delta$ \\
			&&- most pronounced response to price deviations \\
			&&\ \ (see \autoref{deviations}) \\
			\hline
			Separate Polynomials&$10^{-6}$&- high percentage of converged runs \\
			\hline
			Polynomial Tiles&$10^{-8}$&- high $\Delta$ \\
			\hline
		\end{tabular}
		\caption{\emph{Optimized} values of $\alpha$ by \gls{fem}}
		\label{justifications}
	\end{table}



\subsection{Deviations}\label{deviations}

This section examines whether the learned strategies are stable in the face of deviations from the learned behavior. There are at least two explanations for the existence of supra-competitive outcomes. First, agents simply fail to learn how to compete effectively and miss out on opportunities to undercut their opponent. Second, agents avoid deviating from the stable strategy because they fear retaliation and lower (discounted) profits in the long run. Importantly, only the latter cause, supra-competitive prices underpinned by some form of a \emph{reward-punishment scheme}, allows to label the outcomes as \emph{collusive} and warrants attention from competition policy \parencite{assad_algorithmic_2020}. Therefore, the following \emph{deviation experiment} was conducted to scrutinize whether agents learn to actually retaliate in the wake of a deviation. Denote the period in which convergence was detected as $\tau = 0$. At this point, both agents played for 10,000 episodes an equilibrium strategy they mutually regard as optimal. At $\tau = 1$, I force one agent to deviate from her learned strategy and play instead the short-term best response that mathematically maximizes profits. Subsequently, she reverts to the learned strategy. In order to verify whether the non deviating agent proceeds to punish the cheater, he sticks to his learned behavior throughout the \emph{deviation experiment}. In total, the deviation episode lasts 10 periods. Learning and exploration are disabled (i.e.\ $\alpha = \epsilon = 0$).\footnote{See \autoref{prolonged_deviations} for prolonged deviations and continued learning \emph{after} detected convergence.} In order to evaluate the deviation, it appears useful to define a \emph{counterfactual} situation where both agents stick to their learned strategies for another 10 episodes. Comparing (discounted) profits between the experiment and the counterfactual allows to assess the profitability of the deviation.

\textbf{words on punishment strategies?: grim trigger vs. slow reversal }

\begin{figure}
	\includegraphics[width=\linewidth]{plots/average_intervention.png}
	\caption[Average price trajectory around deviation by \gls{fem}]{Average price trajectory around deviation by \gls{fem}. Points represent the average price over all runs of an experiment. Dashed horizontal lines represent the fully collusive price $p_m$ and the static Nash solution $p_n$. Dotted vertical line reflects time of convergence, i.e.\ the period immediately before the forced deviation.}
	\label{average_intervention}
\end{figure}

As the responses to one agent's deviation vastly differ across \gls{fem}s, it is natural to discuss them separately at first and contrast differences only thereafter. It is difficult to summarize all information in a single graph or table, so I will consult \autoref{average_intervention}, \autoref{intervention_boxplot} and \autoref{share_deviation_profitability} simultaneously to describe the deviation and response patterns. Before that, a brief description of these plots is in line. \autoref{average_intervention} displays the price trajectory around the forced deviations averaged over all runs of an experiment.\footnote{To reiterate the result from the previous section, the plot reinforces that the price variation between periods is non-negligible \emph{before} the deviation even takes place - despite averaging over all runs of the optimal experiments.}  Since the average price trajectory might veil important differences between runs, \autoref{intervention_boxplot} illustrates the range of deviation and punishment prices compared to the counterfactual price that would have materialized if no deviation had taken place and agents kept following their learned strategies. Note that in the presence of price cycles, part of the variation can be explained by \emph{cycle shifting}, a phenomenon where the agents return to the learned cycle but the intervals are not aligned with the counterfactual path. These differences should even out over all runs of an experiment and therefore, not systematically bias the boxes in either direction. Similarly, the average price response in \autoref{average_intervention} is largely unaffected by this phenomenon. Finally, \autoref{share_deviation_profitability} reports the share of deviations that turned out to be profitable compared to the counterfactual.\footnote{Appendix \ref{deviations_appendix} contains further visualizations of the deviation experiment.}

\begin{figure}
	\includegraphics[width=\linewidth]{plots/intervention_boxplot.png}
	\caption[Distribution of price differences around deviation by \gls{fem}]{Distribution of price differences around deviation by \gls{fem} relative to counterfactual path \emph{without} forced deviation, i.e.\ the difference to the price had no deviation taken place. Only includes converged runs because a clear counterfactual exists. Boxes demarcate 15th and 85th percentiles. They are extended by whiskers that mark the entire range of price differences. Horizontal lines represent the group median.}
	\label{intervention_boxplot}
\end{figure}

Most importantly, only tabular learning evokes a conspicuous punishment from the non deviating agent. At $\tau = 2$, the non deviating agent tends to match, arguably even undercut, the deviation price whereas the deviating agent already begins reverting to pre-deviation prices. Though this result's general validity is qualified. \autoref{intervention_boxplot} unveils that the non deviating agent does not always reduce prices compared to the counterfactual. Despite the existence of punishment prices in some runs, agents are fairly quick to return to the price levels observed before the deviation was forced upon them. As early as $\tau = 4$ there is no visible difference between average pre- and post-deviation price levels.\footnote{Previous studies showcase a strong deviation is usually followed by a more gradual reversion to pre-deviation behavior (around 5-10 episodes), see in particular Figure 4 in \textcite{calvano_algorithmic_2018} and Figure 3 in \textcite{klein_autonomous_2019}.} This might partly follow from prices being relatively close to the Nash equilibrium in the first place. The punishments ensure that deviating is (strictly) profitable in only 18\% of runs. This suggests that, upon convergence, agents stick to a stable equilibrium, from which deviations tend to be unprofitable due to the cheated agent retaliating.


	\begin{table}
		\centering
		% latex table generated in R 3.6.1 by xtable 1.8-4 package
% Thu May 27 09:48:14 2021
\begin{tabular}{llrr}
  \hline
feature\_method & agent & share profitable & share unprofitable \\ 
  \hline
tabular & deviating & 0.18 & 0.56 \\ 
  tabular & non deviating & 0.04 & 0.69 \\ 
  tiling & deviating & 0.56 & 0.33 \\ 
  tiling & non deviating & 0.06 & 0.83 \\ 
  poly-separated & deviating & 0.72 & 0.00 \\ 
  poly-separated & non deviating & 0.02 & 0.70 \\ 
  poly-tiling & deviating & 0.96 & 0.02 \\ 
  poly-tiling & non deviating & 0.00 & 0.98 \\ 
   \hline
\end{tabular}

		\caption[Share of profitable deviations by \gls{fem} and agent]{Share of profitable and non-profitable deviations by \gls{fem} and agent. Deviations are deemed \emph{profitable} if the discounted profits until $\tau = 10$ due to the deviation exceed cash flows from a counterfactual without deviation. Only includes converged runs because a clear counterfactual exists. Discounting is equivalent to $\gamma$ in \autoref{td_error_expected}, i.e.\ 0.95. A significant number of 'deviations' are neither profitable nor unprofitable. In those runs, the learned strategy of the deviating agent is actually the best response at $\tau = 1$ and both agents keep following their respective price cycle.}
		\label{share_deviation_profitability}
	\end{table}

When examining the outcomes of the other \gls{fem}, different conclusions emerge. Recall from \autoref{alpha} that tile coding yielded convergence profits very similar to tabular learning. Yet, \autoref{average_intervention} and \autoref{intervention_boxplot} only hint at slight punishments in some runs. In fact, the median of the cheated agent's price at $\tau = 2$ is exactly 0, which amounts to a complete absence of a response. This lack of punishment renders 56\% of the cheater's deviations profitable. In light of that, it is surprising that the cheating agent tends to return to pre-intervention price levels instead of continuing to exploit her opponent's failure to punish deviations.

This is even more true for the separate polynomials \gls{fem}. Section \ref{convergence} outlined that this method typically did not converge in price cycles but a single, continuously played price instead. The deviation responses are easy to summarize. In all runs, after the forced intervention at $\tau = 1$, both agents immediately return to the pre-deviation equilibrium. This is remarkable for two reasons. First, the non deviating agent completely fails to punish the cheater's behavior and does not respond to the price cut whatsoever. Consequently, 72\% of the deviations are profitable.\footnote{The remaining 31\% comprise runs where the deviating agent was already playing the short-term best response. Remember that the separate polynomials \gls{fem} tends to result in prices at or close to the Nash equilibrium.} This leads to the second point. Despite the obvious advantage of cheating, the deviating agent returns to the pre-deviation price without exception, thus failing to exploit her opponent's weakness. To put this in the right context, remember that the initial price levels are fairly close to the Nash equilibrium and the deviation's profitability is relatively small compared to the potential gains realizable in other experiments (see also Appendix \ref{deviations_appendix}). Still, it is puzzling that such a simple strategy improvement remains consistently untapped. In conclusion, the agents' failure to play economically sound strategies casts doubts on the viability of the \gls{fem} in reality.

Finally, turn your attention to the experiment with polynomial tiles. Recall that this experiment generated outcomes closest to perfect collusion. Despite that, the deviation experiment for polynomial tiles is similar to the one with separate polynomials. Yet there are some variations between runs that warrant detailed examination. Consider first the non deviating agent. Again, the majority of runs exhibits a failure to respond to the price cut. However, selected runs show a \emph{matching} strategy where the cheated agent meets the price cut with a similar price. Notably, in those circumstances, agents \emph{do not return} to the previously learned path but quickly establish a new equilibrium. Moreover, note that \autoref{intervention_boxplot} displays a slight bias downwards over all periods. This is indicative of \emph{continued cheating} of the deviating agent. After being forced to undercut the price, she proceeds to set prices below pre-deviation levels without getting punished. This, too, results in a new equilibrium.\footnote{\autoref{intervention_poly_tiling} in \autoref{appendix} illustrates both phenomena (price matching and continued cheating) through the exact price sequence of exemplary runs.} In light of high pre-deviation prices and the lack of retaliatory prices, it is unsurprising that 96\% of deviations are profitable. The conclusions for polynomial tiles are similar to separate polynomials. Baring a few exceptions, the non deviating agent fails to respond to a price cut and is easy to exploit. On the other hand, the deviating agent tends to leave that weakness unexploited.
Overall the deviation exercise suggests that while algorithmic agents manage to sustain high prices when playing each other, their strategies are incomplete and easy to exploit.

\textbf{summary of all methods}

Evidently, under the regime of this study's simulations, tabular learning is better in producing stable supra-competitive outcomes than the function approximation \gls{fem}s. To illustrate the notion of \emph{stability} in this context, consider the following thought experiment of a \emph{superagent}. Upon convergence, a rational player with perfect information about the economic environment and the learned policies of both agents enters the game and takes over pricing authority from one of them.\footnote{For the sake of the argument it is irrelevant whether this superagent is human or not.} Importantly, the superagent could anticipate the opponent's price in the next period and calculate the short-term maximizing response as well as the opponent's reaction to the deviation and so on. When playing against a tabular learning agent, the superagent would deliberately stick to the convergence pricing scheme as cheating is sure to evoke a retaliation rendering a deviation unprofitable (see \autoref{share_deviation_profitability}). Contrary, when facing an opponent who learned its strategy through a function approximation \gls{fem}, the superagent could easily cheat on the opponent to increase short-term profits without being punished in subsequent periods.

The evidence also suggests that function approximation creates hesitation in the agents to change best responses. Probabilistically, \emph{exploration} ensures that both agents will undercut the price of their opponent and realize excess profits similar to those in the forced deviation experiment. However, it appears that agents fail to learn (enough) from such \emph{explored cheating}. As evidenced by the undertaken deviation experiment, they typically return to the pre-deviation price (cycle) immediately. This rigidity in adjusting strategies potentially points to a problem with the specific algorithm or the tuning of its parameters. For instance, a higher $\alpha$ could enable the cheater to learn faster that an unpunished deviation is more profitable than adhering to the learned strategy. 

Recall that learning and exploration were turned off for the deviation experiment. This gives rise to an objection to the presented results. The non deviating agent, stripped of the ability to adjust his strategy, might only be exploitable for a finite number of episodes until he adjusts his strategy. In fact, a generosity to condone isolated price cuts might be conducive to establishing high price levels early in the simulation runs. However, \autoref{prolonged_deviations} demonstrates that the lack of punishment in response to a deviation remains ubiquitous in prolonged deviation experiments with enabled learning ($\alpha > 0$).

\textbf{conclude and refer to appendix for strategies off-path}

	
	\section{Robustness and variations}\label{robustness}

To show that the presented findings are not an artifact of specific experiment design choices, this section performs are variety of robustness checks and reports results on slightly altered learning schemes. In all additional experiments I hold $\alpha$ fixed at its optimal values (see \autoref{justifications}). I will show that the considered variations do not impact the simulation results much. Rather, they reinforce the conclusions from the last section. First, I consider an alternative deviation experiment. In sections \ref{vary_parameter} and \ref{price_grid_section} I vary the parameters controlling the learning process and the price grid that is available to both agents. Variations to the discount factor are presented in \autoref{discounting}. Finally, I consider slight alterations to, respectively, the algorithm and the reward setting in sections \ref{vary_algorithm} and \ref{differential}.

\subsection{Prolonged deviations}\label{prolonged_deviations}

To extend the mixed results from the deviation experiments in \autoref{deviations}, I conducted another \emph{prolonged deviation} experiment with continued learning. As I will show, the previously drawn conclusions remain intact. However, first I briefly explain why a continued intervention could theoretically be different from a one-time deviation. The critical components are continued learning and the eligibility trace vector $\boldsymbol{z}$ that make conceivable an agent tolerating isolated deviations but punishing longer price cuts.

Without eligibility traces and the ability to learn, a one-time deviation suffices to assess retaliatory behavior because the memory is too short to \emph{remember} that the opponent cheated for longer than a single period. Likewise, stripping the non deviating agent of his ability to update $\boldsymbol{w}$ renders him unable to learn that tolerating deviations is exploitable and can culminate in continuous low rewards. Consequently, if he failed to punish a deviation at $\tau = 2$, he will not react at $\tau = 3$ either. On the contrary, with the ability to learn enabled, both agents can readjust the parameter updates. For instance, after discovering that tolerating a one-time deviation yields a low reward, the non deviating agent might adjust $\boldsymbol{w}$ and decide to play a different action next time he is cheated (e.g.\ match the price cut). This is augmented by the length of deviation episodes and the existence of eligibility traces. If the deviating agent \emph{continues} to cheat, the opponent should continue to decrease the valuation of the \emph{tolerating} strategy and could ultimately fall back to the next best action (which might be a price cut).\footnote{Furthermore, remember that the deviation experiment is conducted right after convergence was detected. Consequently, the algorithm was \emph{on-path} for a large number of periods and the eligibility traces have not been reset recently (see \autoref{eligibility_trace_update}). Therefore, updates are large in magnitude and after the deviation experiment concludes, the convergence equilibrium might not be feasible anymore because its valuation by the agents changed.}

The \emph{prolonged deviation} experiment lasts 20 periods in total. It was set up as follows. The deviating agent anticipates the price of her opponent perfectly and continuously plays the best response for a total of 10 periods of cheating. This imposes the assumption that she is capable of perfectly predicting her opponent's response to the initial deviation. Exploration remains disabled ($\epsilon = 0$) but both agents continue learning from their actions.\footnote{I also prescribe that the forced deviation is considered \emph{on-policy}. Since $\epsilon = 0$, this is most natural to incorporate.} After I stop forcing the deviating agent to play the best response, both agents play another 10 periods adhering to their learned strategies.

\begin{figure}
	\includegraphics[width=\linewidth]{plots/average_prolonged_intervention.png}
	\caption[Average price trajectory around prolonged deviation by \gls{fem}]{Average price trajectory around prolonged deviation by \gls{fem}. Points represent the average price over all runs of an experiment. Dashed horizontal lines represent the fully collusive price $p_m$ and the static Nash solution $p_n$. Dotted vertical line reflects time of convergence, i.e.\ the period immediately before the forced deviation.}
	\label{average_prolonged_intervention}
\end{figure}

The additional deviation experiments were conducted with the optimal values of $\alpha$ in accordance with \autoref{justifications}. \autoref{average_prolonged_intervention} displays the average price trajectory around the prolonged deviation and confirms the previous observations. Only with tabular learning does the non deviating agent match the price cuts systematically. The top left panel shows both agents hover around the Nash benchmark for the entire duration of the deviation episode. Clearly, this is unprofitable for both agents but the punishment is necessary to sustain supra-competitive prices in the first place. Note also the quick return to pre-deviation levels as soon as the deviating agent returns to her learned behavior. Since the opponent follows immediately, it appears that the return must be initiated be the original cheater. The experiment illustrates that the supra-competitive outcomes remain sustainable in the face of persistent interruptions.

With regard to the three function approximation methods, the deviating agent appears to systematically exploit her opponent who fails to punish the price cut. The subtle differences between \gls{fem}s extend to the prolonged deviation. Separate polynomials evoke no response from the non deviating agent. Both tiling methods show a small \emph{average} price cut over the duration of the prolonged deviation, but this response falls short of a reliable mechanism that consistently deters deviation across runs. Indeed, Appendix \ref{prolonged_deviations_appendix} shows that only isolated runs exhibit the non deviating agent cutting the pre-deviation price levels. Most runs show no reaction which is veiled by averaging over all runs of the experiment. This absence of a retaliation opens up the opportunity for continuous exploitation by the deviating agent. Despite that, the latter tends to return to pre-deviation price levels. Therefore, both agents act far from optimal (in the economic sense of the word) and fail to learn (enough) from the prolonged deviation experiment. Lastly, note the difference between pre- and post-deviation price levels at the bottom right panel, representing polynomial tiles. As noted previously, this suggests that the agents proceed to play a different, less profitable equilibrium after the deviation. This easy switch to a new strategy further challenges the viability of the pre-deviation equilibrium in the first place.

It is conceivable, maybe even likely, that the non deviating agent does alter its strategy after a time frame much longer than 10 periods. However, this is not important for this study because the agent's strategy is easy to exploit in the short term and the deviations are clearly profitable (refer back to \autoref{share_deviation_profitability}).


\subsection{Learning parameters}\label{vary_parameter}

Besides the learning rate $\alpha$, the exploration strategy is arguably the most important steering choice in reinforcement learning. As discussed, $\beta$ controls the decay in exploration over time. To assess its impact on the sensitivity of outcomes, I run a number of experiments varying $\beta$ while keeping the manually optimized values of $\alpha$ constant (see \autoref{justifications}).\footnote{Note that these values are not necessarily 'optimized' for alternative $\beta$. Ideally, exploration rate and learning speed should not be considered in isolation. Indeed, \textcite{calvano_artificial_2020} show that lower values of $\alpha$ perform better if exploration is extensive. However, the scope of this study does not allow to systematically search over a 2-dimensional grid of $\alpha$ and $\beta$.} Naturally, I adjust  proportionately the number of maximal periods before a run is forced to terminate.\footnote{With the lowest and highest values of $\beta$ ($0.00016$ and $10^-{5}$), the maximum number of periods is adjusted to $125,000$ and $2,000,000$ respectively.}

\begin{figure}
	\includegraphics[width=\linewidth]{plots/beta.png}
	\caption[Average $\Delta$ by \gls{fem} and $\beta$]{Average $\Delta$ by \gls{fem} and $\beta$. Includes converged and non-converged runs.}
	\label{beta}
\end{figure}

\autoref{beta} displays that the impact of exploration on average $\Delta$ is relatively small across \gls{fem}s. Interestingly, applying the deviation routine described in \autoref{deviations} uncovers that extended exploration supports the stability of the convergence equilibrium only in the case of tabular learning. \autoref{share_deviation_profitability_beta} clearly shows that cheating becomes less profitable when the non-deviating agent utilizing tabular learning had more opportunities to explore reactions to a deviation. The share of profitable deviations ranges from 39\% at $\beta = 0.00016$ to 6\% at $\beta= 2*10^{-5}$. Strangely, the separate polynomials \gls{fem} shows the opposite pattern. Less exploration makes deviations less attractive. My interpretation is that the overall lower prices levels associated with less exploration indicate that the deviating agent already plays the best response in many runs. Lastly, the value of $\beta$ does not seem to have a large impact on either of the tiling \gls{fem}s. Their convergence equilibria are unstable for all trialed values of $\beta$.

\begin{table}
	\centering
	% latex table generated in R 3.6.1 by xtable 1.8-4 package
% Mon May 24 14:51:47 2021
\begin{tabular}{llrrrrr}
  \hline
feature method & agent & $\beta =$ 0.00016 & $\beta =$ 8e-05 & $\beta =$ 4e-05 & $\beta =$ 2e-05 & $\beta =$ 1e-05 \\ 
  \hline
tabular & deviating & 0.39 & 0.38 & 0.18 & 0.06 & 0.08 \\ 
  tabular & non deviating & 0.20 & 0.09 & 0.04 & 0.00 & 0.04 \\ 
  tiling & deviating & 0.50 & 0.54 & 0.56 & 0.56 & 0.67 \\ 
  tiling & non deviating & 0.04 & 0.08 & 0.06 & 0.04 & 0.00 \\ 
  poly-separated & deviating & 0.58 & 0.67 & 0.72 & 0.89 & 0.90 \\ 
  poly-separated & non deviating & 0.00 & 0.00 & 0.02 & 0.00 & 0.00 \\ 
  poly-tiling & deviating & 1.00 & 0.92 & 0.96 & 0.92 & 0.91 \\ 
  poly-tiling & non deviating & 0.02 & 0.02 & 0.00 & 0.02 & 0.00 \\ 
   \hline
\end{tabular}

	\caption[Share of profitable deviations by \gls{fem}, agent and $\beta$]{Share of profitable deviations by \gls{fem}, agent and $\beta$. Annotations from \autoref{share_deviation_profitability} apply.}
	\label{share_deviation_profitability_beta}
\end{table}

Appendix \ref{beta_appendix} supplements the observation of this section by showing that punishment severity and length also increase with extended exploration. Moreover, in Appendix \ref{lambda_appendix} I briefly discuss that the choice of $\lambda$ does have little impact on deviation behavior. But the variance in the distribution of profits between runs seems to increase with high values of $\lambda$.

\subsection{Price grid}\label{price_grid_section}

\autoref{feature_extraction_summary} emphasized that the length of the parameter vector $\boldsymbol{w}$ with tabular learning increases disproportionately with $m$. Likewise, the optimization problem is likely to become more complex. On the other hand, the feature extraction mechanisms of tile coding and polynomial tiles are largely unaffected by $m$. To gauge the effect on outcomes, I executed experiments with additional variations, specifically $m=10$, $m = 39$ and $m = 63$.\footnote{As before, the odd numbers are chosen to enable prices close to $p_m$ and $p_n$} Due to computational restrictions, these experiments  only comprise 16 runs. Accordingly, inference should be treated with care.

\begin{figure}
	\includegraphics[width=\linewidth]{plots/m.png}
	\caption[Average $\Delta$ by \gls{fem} and $m$]{Average $\Delta$ by \gls{fem} and $m$. Includes converged and non-converged runs.}
	\label{m}
\end{figure}

Unsurprisingly, convergence becomes less likely when $m$ increases. While all runs with $m=10$ converged, the percentage for $m=39$ and $m=63$ is only  67.2\% and 57.8\% respectively.\footnote{ This is driven mainly by less converged runs with the \gls{fem}s tabular learning and tile coding. See Appendix \ref{price_grid_appendix}.} Despite that, \autoref{m} indicates that varying $m$ does not seem to have much of an impact on the average $\Delta$. On first glance, this seems confusing. As the complexity of the problem increases, one would expect agents to struggle with optimizing their strategy. However, the puzzle is partly solved by taking into account the stability of the equilibrium. \autoref{share_deviation_profitability_m} suggests that the share of profitable deviations increases with $m$ for tabular learning and the polynomial tiles \gls{fem}. Most notably, the share of profitable punishments in runs with tabular learning increases from only 12\% when $m=10$ to 67\% when $m=63$. Interestingly, the share of profitable deviation in runs with polynomial tiles is also significantly smaller when $m=10$. Recall that the \gls{fem} evoked a weak punishment and gave rise to new equilibria in \emph{some} runs. This tendency seems to be reinforced when $m$ is low.

To summarize, increasing the environment's complexity through $m$ makes supra-competitive outcomes not less likely, but less stable in the face of deviations. Appendix \ref{price_grid_appendix} supports this hypothesis by showing that punishments seem to be strongest with $m = 10$.

\begin{table}
	\centering
	% latex table generated in R 3.6.1 by xtable 1.8-4 package
% Sun May 23 20:24:42 2021
\begin{tabular}{llrrrr}
  \hline
feature\_method & player & m = 10 & m = 19 & m = 39 & m = 63 \\ 
  \hline
tabular & deviating agent & 0.12 & 0.18 & 0.42 & 0.67 \\ 
  tabular & non deviating agent & 0.00 & 0.04 & 0.08 & 0.33 \\ 
  tiling & deviating agent & 0.69 & 0.56 &  &  \\ 
  tiling & non deviating agent & 0.00 & 0.06 &  &  \\ 
  poly-separated & deviating agent & 0.88 & 0.72 & 0.93 & 0.87 \\ 
  poly-separated & non deviating agent & 0.00 & 0.02 & 0.00 & 0.20 \\ 
  poly-tiling & deviating agent & 0.69 & 0.96 & 1.00 & 0.94 \\ 
  poly-tiling & non deviating agent & 0.06 & 0.00 & 0.00 & 0.00 \\ 
   \hline
\end{tabular}

	\caption[Share of profitable deviations by \gls{fem}, agent and $m$]{Share of profitable deviations by \gls{fem}, agent and $m$. Annotations from \autoref{share_deviation_profitability} apply. Empty cells are a consequence of no converged runs for the particular experiment.}
	\label{share_deviation_profitability_m}
\end{table}


Next, I will consider variations in $\zeta$. Recall from \autoref{price_grid_formula} that $\zeta$ controls the available excess range above the fully collusive price $p_m$. These prices are inferior to $p_m$ in almost any situation and the simulations confirm that few runs converge in \emph{supra-monopoly} prices. So how may a change in $\zeta$ affect the learning behavior? Most importantly, large values of $\zeta$ increase the share of available prices above $p_m$ and decreases the share of \emph{viable} prices within the range of $p_m$ and $p_n$. Consequently, the agents may quickly discard a larger share of actions engendering low (or negative) rewards and \emph{narrow down} the range of reasonable actions between $p_n$ and $p_m$. Then, with fewer available actions, the optimization within that range might be facilitated. This might be particularly important with the separate polynomials \gls{fem} because agents could learn that certain polynomials associated with actions above $p_m$ (or below $p_n$) consistently yield low rewards - irrespective of the preceding state set, refrain from playing them early in the simulation and focus on refining the polynomials of actions within the range of $p_n$ and $p_m$.

There is an additional effect on both tiling methods. The thresholds of all tiles derive from the size of the action space. Therefore, tiles are resized and relocated. The natural consequence is that some state-actions combinations will be associated with different tiles. \emph{A priori}, the effect on outcomes is hard to predict.

\begin{figure}
	\includegraphics[width=\linewidth]{plots/zeta.png}
	\caption[Average $\Delta$ by \gls{fem} and $\zeta$]{Average $\Delta$ by \gls{fem} and $\zeta$. Includes converged and non-converged runs.}
	\label{zeta}
\end{figure}

I conducted three additional experiments with $\zeta \in \{0.1, 0.5, 1.5\}$ to assess the impact of varying $\zeta$ while keeping $m$ constant at $19$ to ensure comparability between experiments.\footnote{Note however, other $\zeta$ may prohibit playing actions very close to $p_n$ or $p_m$. For instance, with $\zeta = 1.5$, the price closest to $p_n = 1.473$ ($p_m = 1.925$) is $1.454$ ($1.908$). These gaps are quite a bit higher than in the default specification.} \autoref{zeta} illustrates that $\zeta$ significantly influences profits upon convergence. Across \gls{fem}s, the average $\Delta$ increases with $\zeta$. This trend is most pronounced with polynomial tiles. To reiterate, prices close to the collusive solution are not necessarily evidence of a stable equilibrium with a \emph{reward-punishment} scheme. If anything, the simulation runs in this study have suggested the opposite and it turns out that despite the differences in $\Delta$, the stability of the learned strategies is not heavily influenced by $\zeta$. \autoref{share_deviation_profitability_zeta} does not show prominent trends in the share of profitable deviations. As further evidence, Appendix \ref{price_grid_appendix} shows that tabular learning agents tend to punish deviations with retaliatory prices in all considered variations of $\zeta$. Similarly, the absence of retaliatory prices in experiments with function approximation \gls{fem}s does not seem to hinge on $\zeta$.

\begin{table}
	\centering
	% latex table generated in R 3.6.1 by xtable 1.8-4 package
% Tue Jun 01 16:43:02 2021
\begin{tabular}{llrrrr}
  \hline
FEM & agent & $\zeta =$ 0.1 & $\zeta =$ 0.5 & $\zeta =$ 1.0 & $\zeta =$ 1.5 \\ 
  \hline
Tabular & deviating & 0.24 & 0.21 & 0.24 & 0.31 \\ 
  Tabular & non deviating & 0.07 & 0.12 & 0.07 & 0.04 \\ 
  Tile Coding & deviating & 0.38 & 0.42 & 0.56 & 0.44 \\ 
  Tile Coding & non deviating & 0.04 & 0.02 & 0.04 & 0.08 \\ 
  Separate Polynomials & deviating & 0.46 & 0.57 & 0.69 & 0.60 \\ 
  Separate Polynomials & non deviating & 0.00 & 0.00 & 0.00 & 0.00 \\ 
  Polynomial Tiles & deviating & 0.83 & 0.96 & 0.85 & 0.84 \\ 
  Polynomial Tiles & non deviating & 0.10 & 0.08 & 0.04 & 0.09 \\ 
   \hline
\end{tabular}

	\caption[Share of profitable deviations by \gls{fem}, agent and $\zeta$]{Share of profitable deviations by \gls{fem}, agent and $\zeta$. Annotations from \autoref{share_deviation_profitability} apply.}
	\label{share_deviation_profitability_zeta}
\end{table}

To summarize, the grid of available prices does have some impact on the outcomes. Perhaps unsurprisingly, by increasing the complexity of the problem through the number of available prices $m$, tabular learning agents are less likely to support their convergence equilibria with punishment strategies. Similarly, an increase in the share of \emph{viable} prices in the range of $p_m$ and $p_n$ seems to lower convergence profits. Despite these subtleties, the qualitative conclusions from \autoref{results} remain intact.

\subsection{Discount factor}\label{discounting}

In dynamic oligopolies, theory ascribes great importance to the discount factor $\gamma$. Typically, there exists a critical value below which the weight on future profits becomes too low to sustain any collusive behavior. Likewise, if $\gamma$ is sufficiently high, rational actors with full information will collude on the monopoly solution. In reality, there are various reasons why decision makers may end up charging prices between both extremes. For instance, they might not be fully aware of what exactly the benchmark prices are and might struggle to communicate and agree on a joint action (explicitly or tacitly). Similarly, in reinforcement learning, it is unlikely that there exists a strict dichotomy between fully collusive and perfectly competitive agents. Indeed, the results so far suggests that many intermediate levels are realistic. Nevertheless, with lower values of $\gamma$, less weight is put on the (expected) value of the future state in \autoref{td_error_expected} and the immediate reward $R_t$ gains relative importance. Accordingly, one would expect the agents to gradually approach the Nash benchmark as $\gamma$ decreases.

To gauge the actual effect of $ßgamma$ on outcomes, I conducted a series of experiments ranging from perfectly myopic ($\gamma =0$) to almost infinitely patient ($\gamma = 0.99$) agents.\footnote{While $\gamma = 1$ is usually easy to model in economics, it is highly problematic in continuing learning tasks due to its infinite sum property (this is the main reason why discounting is commonly utilized in reinforcement learning in the first place, see e.g.\ \textcite[p.3]{schwartz_reinforcement_1993}). Consider the following example. An agent with no time preference ($\gamma = 1$) in a continuous task explores  early that a particular action consistently yields positive rewards. When \emph{exploiting}, the agent keeps playing that action and the value estimate accumulates to infinity. This results in a significant bias towards actions that have been explored early and at some point becomes computationally infeasible. Through similar reasoning, values marginally below $1$ are known to be unstable \parencite{naik_discounted_2019}.} \autoref{gamma} summarizes the variation in average $\Delta$. Though the relationship is not as clear as anticipated, the curves of tabular learning and tile coding confirm the hypothesized pattern. With $\gamma = 0$, the average profits are much closer to the Nash benchmark.\footnote{The relationship is more pronounced with prices, see Appendix \ref{discounting_appendix}.} Another interesting revelation is that the \emph{average} profits for tabular learning are highest at $\gamma = 0.85$ (average $\Delta = 0.48$). This suggests the agents struggle with high variance due to large values of $\gamma$ \parencite[p.6]{naik_discounted_2019}.

\begin{figure}
	\includegraphics[width=\linewidth]{plots/gamma.png}
	\caption[Average $\Delta$ by \gls{fem} and $\gamma$]{Average $\Delta$ by \gls{fem} and $\gamma$. Includes converged and non-converged runs.}
	\label{gamma}
\end{figure}

With regard to the polynomial \gls{fem}s, the figure serves as further evidence of their ineptness for the considered learning task. Even without discounting ($\gamma = 0$), the outcomes remain high. In fact, they are even higher with separate polynomials. This clearly hints at a failure to learn how to compete when \emph{only} the immediate reward should matter.\footnote{I interpret this similar to the results in \textcite{waltman_q-learning_2008} where \emph{memoryless} agents without the ability to assert whether the opponent cheated still learn to charge supra-competitive prices.}

\subsection{Alternative algorithms}\label{vary_algorithm}

Of course, the specific algorithm described in \autoref{expected SARSA} is only one of many ways to use function approximation in learning tasks. I will consider two variations: \emph{Tree backup} and \emph{on-policy SARSA}.

\subsubsection{Tree backup}\label{tree_backup}

\textcite{precup_eligibility_2000} suggest the \emph{tree backup} algorithm as a successor to Q-Learning. Compared to the \emph{expected SARSA} algorithm, the update in \autoref{eligibility_trace_update} is replaced by:

\begin{gather}\label{eligibility_traces_tree_backup}
\boldsymbol{z}_t = \gamma \lambda \kappa(A_t | S_t) \boldsymbol{z}_{t-1} + \frac{\Delta \hat{q}}{\Delta \boldsymbol{w}_t}
\end{gather}

Recall that $\kappa(A_t | S_t)$ represents the probability of choosing $A_t$ if the agent were to follow a hypothetical target policy with $\epsilon= 0$. As with the eligibility trace in expected SARSA, the idea is that $\boldsymbol{z}$ resets to $\boldsymbol{0}$ as soon as a non-greedy action is played. Unsurprisingly, applying the tree backup algorithm with optimized values of $\alpha$ to the environment yields not very different results. \autoref{tb_violin} displays the distribution of $\Delta$ which is reminiscent of the violins for optimized values of $\alpha$ in \autoref{alpha_violin}. Appendix \ref{vary_algorithm_appendix} contains visualizations illustrating that the deviation experiments do not reveal new insights either.

\begin{figure}
	\includegraphics[width=\linewidth]{plots/tb_violin.png}
	\caption[Distribution of $\Delta$ by \gls{fem} with \emph{tree backup} algorithm]{Distribution of $\Delta$ by \gls{fem} with \emph{tree backup} algorithm. Includes converged and non-converged runs. Violin widths are scaled to maximize width of individual violins, comparisons of widths between violins are not meaningful. Violins are trimmed at smallest and largest observation respectively. Horizontal lines represent the median.}
	\label{tb_violin}
\end{figure}

\subsubsection{On-policy SARSA}
\emph{Q-Learning}, \emph{tree backup} and \emph{expected SARSA} all belong to the family of \emph{off-policy} learning algorithms. This stems from the simple fact that the (discounted) value estimation of the state-action combination at $t+1$ is not always based on the actually chosen action $A_{t+1}$ (see \autoref{td_error_expected}).\footnote{The only exception is $\epsilon = 0$.}  So, it is \emph{off-path} of the actually pursued policy. Off-policy methods tend to exhaust the entire range of state-action combination well, but convergence guarantees for them are generally weaker than for \emph{on-policy} algorithms \parencite[p.257-265]{sutton_reinforcement_2018}.\footnote{The main reason why I haven't put much consideration into this is that due to the \emph{moving target problem} described in \autoref{convergence_considerations}, convergence is not guaranteed anyway. Moreover, \textcite{hettich_algorithmic_2021} shows that off-policy methods can work well with function approximation.} As their name suggests, \emph{on-policy} algorithms wait until the state-action combination at $t+1$ is actually known and only then estimate the TD error $\delta_t$. A straightforward adaption is:\footnote{The full algorithm is documented in Appendix \ref{vary_algorithm_appendix}}

\begin{gather}\label{td_error_on_policy}
\delta_t^{SARSA} = r_t + \gamma \hat{q}(S_{t+1}, A_{t+1}, \boldsymbol{w}_t) - \hat{q}(S_t, A_t, \boldsymbol{w}_t) ~~ \text{,}
\end{gather}

Note that learning is delayed in the sense that $\delta_t^{SARSA}$ can only be calculated after the action in the next period has been taken. Using the optimized values of $\alpha$, I conducted one experiment per \gls{fem} with optimized $\alpha$. \autoref{op_violin} illustrates that the distribution of outcomes per experiment resembles the two \emph{off-policy} algorithms. Overall, the conclusions drawn in the previous section also apply to the \emph{on-policy} algorithm.

\begin{figure}
	\includegraphics[width=\linewidth]{plots/op_violin.png}
	\caption[Distribution of $\Delta$ by \gls{fem} with \emph{on-policy} algorithm ]{Distribution of $\Delta$ by \gls{fem} with \emph{on-policy} algorithm . Includes converged and non-converged runs. Violin widths are scaled to maximize width of individual violins, comparisons of widths between violins are not meaningful. Violins are trimmed at smallest and largest observation respectively. Horizontal lines represent the median.}
	\label{op_violin}
\end{figure}

		
\subsection{Differential reward setting}\label{differential}

In reinforcement learning, discounting is commonly used to avoid infinite value accumulation, but rarely has a practical interpretation \parencite{schwartz_reinforcement_1993}. Therefore, the blend with an economic task seems natural. However, despite wide usage, \textcite{naik_discounted_2019} argue that discounting in combination with function approximation is fundamentally incompatible in infinite sequences. They suggest an alternative \emph{differential reward} setting, where \autoref{td_error_expected} is replaced by:\footnote{See chapter 10 in \textcite[pp.249-252]{sutton_reinforcement_2018} for a rigorous treatment formulation. \textcite{hettich_algorithmic_2021} shows that the differential reward setting works well with agents in a Bertrand environment. He also compares both settings in a static environment and finds a tendency for oscillating behavior when discounting is used.}


\begin{gather}\label{differential_reward}
\delta_t^{differential} = R_t - \widetilde{R}_{t} + \hat{q}(S_{t+1}, A_{t+1}, \boldsymbol{w}_t) - \hat{q}(S_t, A_t, \boldsymbol{w}_t) ~~  \text{,}
\end{gather}

where $\widetilde{R}_{t}$ is a (weighted) average reward periodically updated according to

\begin{gather}
	\widetilde{R}_{t+1} = \widetilde{R}_t + \upsilon r_t ~~\text{,}
\end{gather}

where $\upsilon$ is a parameter controlling the speed of adjustment. The formulation ensures that recent rewards are weighted higher. The rest of \autoref{expected SARSA} remains untouched. Note that the differential reward setting does not involve any discounting. At first glance, this clashes with the economic understanding of time preferences.\footnote{This is the main reason why I have not utilized the differential setting in the main part of this study.} However, there are two arguments why the differential reward setting might still be well suited. First, \textcite[pp.253-254]{sutton_reinforcement_2018} proof that, due to the infinite nature of the Bertrand environment, the ordering of policies in the discounted value setting and the setting with average rewards are equivalent (irrespective of $\gamma$). Second, pricing algorithms tend to be used in markets with frequent price changes where it is less important whether a profit is realized immediately or in the next period.


\begin{figure}
	\includegraphics[width=\linewidth]{plots/converged_upsilon.png}
	\caption[Converged runs by \gls{fem} and $\upsilon$]{Number of runs per experiments that (i) achieved convergence, (ii) did not converge or (iii) failed to complete by \gls{fem} and $\upsilon$.}
	\label{converged_upsilon}
\end{figure}

I conducted a series of experiments varying over the following values of $\upsilon$: $0.001$, $0.005$, $0.01$, $0.025$, $0.05$ and $0.1$. As with the other variations, $\alpha$ is fixed at values deemed optimal. \autoref{converged_upsilon} shows the share of converged runs as a function of $\upsilon$ and the \gls{fem}. Disregarding two runs that failed to complete, convergence is consistently achieved for tabular learning, tile coding and separate polynomials. Contrary, only 74.2\% of polynomial tiles runs converged. This starkly contrasts the observation made in the experiments using the discounted reward setting. There, all runs with polynomial tiles converged for various values of $\alpha$.\footnote{The statements disregards runs that failed to complete. Refer back to the bottom right panel in \autoref{converged}. Appendix \ref{differential_appendix} shows that polynomial tiles in the differential reward setting also tend to converge later than the other \gls{fem}s.} Moreover, the plot suggests that low values of $\upsilon$ impede convergence for this \gls{fem}.


 \begin{figure}
	\includegraphics[width=\linewidth]{plots/upsilon.png}
	\caption[Average $\Delta$ by \gls{fem} and $\upsilon$]{Average $\Delta$ by \gls{fem} and $\upsilon$. Includes converged and non-converged runs. Beware the logarithmic x-scale.}
	\label{upsilon}
\end{figure}


\autoref{upsilon} displays how the average profits relative to $p_n$ and $p_m$ change with $\upsilon$. The overall impact is small. However, tabular learning and, to a lesser extent, tile coding seem to converge at higher profits when $\upsilon$ is very low.\footnote{Appendix \ref{differential_appendix} reveals that the variability of average $\Delta$ is higher than in the discounted setting.} With respect to punishment of price cuts, the results are similar to the discounted setting. Irrespective of $\upsilon$, the majority of deviations in experiments with separate polynomials and polynomial tiles is profitable and evokes no retaliation. With tabular learning, the share of profitable deviations is 24.7\% over all runs. There is slight evidence that the hint at some sort of punishment in tile coding is more pronounced in the differential reward setting. Only 48.8\% of deviations are strictly profitable.\footnote{The percentage drops to only 45.8\% if only runs with $\upsilon = 0.1$ or $\upsilon = 0.005$ are considered.} Appendix \ref{differential_appendix} shows that non deviating agents retaliate with a price cut at $\tau = 2$ in some runs.
	
	\section{Conclusions}\label{conclusions}

	\pagebreak
	\printbibliography
	
	\begin{appendices}
	\section{Results Appendix}\label{appendix}
	
		\subsection{Price range}\label{prices}

As established in \autoref{convergence}, many simulations converge in price cycles of various lengths. \autoref{price_range} plots the range between the lowest and highest price a single agent charges in a cycle upon convergence. Naturally, the price range is null if an agent does not vary its actions at all. Perhaps unsurprisingly, the price range then tends to increase with cycle length, at times becoming remarkably high. The range of prices due to tabular learning frequently exceeds the range between collusive and Nash prices. This is a clear indication that price setting is occasionally irrational. Irrespective of agents competing or colluding, prices outside this range are not economically optimal.

Recall from \autoref{convergence} that function approximation \gls{fem} tend to converge with lower cycle lengths. \autoref{price_range} extends that observation with the insight that those methods also produce lower price ranges. However, the inversion of the previous argument is dangerous. One should not deduct that behavior is \emph{closer to optimal} from the mere fact that prices appear more stable. In fact, the study shows that the strategies learned with function approximation are often far from optimal and easy to exploit.

\begin{figure}
	\includegraphics[width=\linewidth]{plots/price_range.png}
	\caption[Distribution of price range by \gls{fem} and cycle length]{Distribution of price range by \gls{fem} and cycle length. Every point represents a single run. Within groups, points are spaced out horizontally. Price range is defined as the difference between the highest and lowest price an agent charges within a cycle. Only converged runs are considered (as cycle length is unavailable for other runs). Horizontal line represents the difference between collusive and Nash outcome (i.e.\ $p_m - p_n$).}
	\label{price_range}
\end{figure}

\clearpage

\subsection{Price trajectory}\label{trajectory_prices_section}

\autoref{trajectory_Delta} displays the trajectory of $\Delta$ over time. Only runs of the \emph{optimized} experiments are printed, as explained in \autoref{justifications}. $\Delta$ is averaged over 50,000 periods apiece and over both players. Due to amassed exploration early in the simulation, average profits are low early on but increase over time. Furthermore, starting at $t = 250,000$ violin widths decrease because of some runs triggering the convergence criteria. Interestingly, the non-converging runs in the optimized separate polynomials experiments are characterized by profits \emph{below} the static Nash equilibrium. Arguably the most outstanding is the remarkable speed at which polynomial tiles increases profits. After mere $100,000$ periods, the median $\Delta$ hovers around $0.75$ already.




\begin{figure}
	\includegraphics[width=\linewidth]{plots/trajectory_Delta.png}
	\caption[Distribution of trajectory of $\Delta$ by \gls{fem}]{Distribution of trajectory of $\Delta$ by \gls{fem} with optimized $\alpha$ (see \autoref{justifications}). For individual runs, $\Delta$ is averaged over 50,000 periods apiece and both players. Plot includes converged and non-converged runs. Violin widths represent quantity of active runs at $t$ which enables comparisons between violins. As most runs converge after 200,000 to 300,000 periods, violin widths decrease thereafter. Violins are trimmed at smallest and largest observation respectively. Horizontal lines represent the median.}
	\label{trajectory_Delta}
\end{figure}

\begin{figure}
	\includegraphics[width=\linewidth]{plots/trajectory_price.png}
	\caption[Distribution of trajectory of average prices by \gls{fem}]{Distribution of trajectory average prices by \gls{fem} with optimized $\alpha$. For individual runs, $p$ is averaged over 50,000 periods apiece and both players. Plot includes converged and non-converged runs. Violin widths represent quantity of active runs at $t$ which enables comparisons between violins. As most runs converge after 200,000 to 300,000 periods, violin widths decrease thereafter. Violins are trimmed at smallest and largest observation respectively. Horizontal lines represent the median.}
	\label{trajectory_price}
\end{figure}

\autoref{trajectory_price} depicts the distribution of average \emph{prices} in optimized experiments over time and \autoref{all_runs} displays the price and profit trajectory of single runs over time. Both figures illustrate that, by and large, prices and profits remain within the benchmarks of Nash competition and the fully collusive case. Obviously, this does not apply to every single period, but holds true on average.
 
\begin{figure}
	\includegraphics[width=\linewidth]{plots/all_runs.png}
	\caption[Trajectories of average $\Delta$ and prices with optimized $\alpha$]{Trajectories of average $\Delta$ (top panel) and prices (bottom panel) with optimized $\alpha$. For individual runs, the respective metric is averaged over 50,000 periods apiece and both players. Plot includes converged and non-converged runs.}
	\label{all_runs}
\end{figure}

\clearpage

\subsection{Deviations}\label{deviations_appendix}

\autoref{intervention_profit_boxplot} displays the difference between profits agents receive after the forced deviation takes place and profits of the alternative path with no deviation. Naturally, the deviating agent makes larger profits at the deviation period $\tau = 1$. With tabular learning, her profits decrease thereafter due to retaliatory prices. The plot shows that this occasionally happens with tile coding and polynomial tiles. With all \gls{fem}s, profits tend to return to pre-deviation levels quickly. 

\begin{figure}
	\includegraphics[width=\linewidth]{plots/intervention_profit_boxplot.png}
	\caption[Distribution of profit differences around deviation by \gls{fem}]{Distribution of profit differences around deviation relative to counterfactual path \emph{without} forced deviation, i.e.\ the difference to the price had no deviation taken place, by \gls{fem}. Only includes converged runs because a clear counterfactual exists. Boxes demarcate 15th and 85th percentiles. They are extended by whiskers that mark the entire range of price differences. Horizontal lines represent the group median.}
	\label{intervention_profit_boxplot}
\end{figure}

\autoref{intervention_profitability_polygon} displays a frequency polygon to gauge how much more or less profitable the deviation is compared to the counterfactual of sticking to the learned strategy. With tabular learning, most deviations end up being unprofitable. Contrary, the line for polynomial tiles indicates that most deviations are profitable, some of them are impressively high. With regards to tile coding and separated polynomials, the deviations in many runs seem to yield profits similar to not deviating. Unsurprisingly, the bottom panel is skewed to the left suggesting that the deviation experiment is unprofitable for the non-deviating agent.

\begin{figure}
	\includegraphics[width=\linewidth]{plots/intervention_profitabiliy_polygon.png}
	\caption[Distribution of additional profitability due to deviation by agent and \gls{fem}]{Distribution of additional profitability due to deviation by agent and \gls{fem}. Width of bins: 0.02. 13 extreme data points ($<-0.5$ or $>0.5$) are excluded for better presentability. Deviations are deemed \emph{profitable} if the discounted profits until $\tau = 10$ due to the deviation exceed cash flows from a counterfactual without deviation. Only includes converged runs because a clear counterfactual exists. Discounting is equivalent to $\gamma$ in \autoref{td_error_expected}, i.e.\ 0.95. A significant number of 'deviations' are neither profitable nor unprofitable. In those runs, the learned strategy of the deviating agent is actually the best response at $\tau = 1$ and both agents keep following their respective price cycle.}
	\label{intervention_profitability_polygon}
\end{figure}


\autoref{intervention_poly_tiling} depicts the price trajectory of three individual runs belonging to the optimized polynomial tiles experiment. Run \emph{06} shows the deviating agent continuing to cheat and increasing her long-term profits. It also shows that price asymmetries \emph{between} players are not an uncommon phenomenon. Run \emph{08} shows the non deviating agent meeting the price cut. This culminates in a new equilibrium with lower prices. Since the pre-deviation prices were slightly above the collusive benchmark, the new equilibrium actually improves both agent's profits in this particular example. Run \emph{09} shows the common pattern of quick reversal to pre-deviation prices and profits.

\begin{figure}
	\includegraphics[width=\linewidth]{plots/intervention_poly_tiling.png}
	\caption[Prices and profits in polynomial tiles deviation experiment]{Prices and profits in 3 exemplary polynomial tiles deviation experiments with optimized $\alpha$. Top panels display prices, bottom panels profits. The exemplary runs are stacked horizontally	. Numbers on strip indicate assigned run id. Dashed horizontal lines represent the fully collusive and static Nash benchmarks. Dotted vertical line reflects time of convergence, i.e.\ the period immediately before the forced deviation.}
	\label{intervention_poly_tiling}
\end{figure}
	
	\pagebreak
	\section{Robustness and variations Appendix}\label{appendix_2}
	

	
		\subsection{Prolonged deviations}\label{prolonged_deviations_appendix}

	\textbf{some words on prolonged deviations}
	
\begin{figure}
		\includegraphics[width=\linewidth]{plots/prolonged_intervention_boxplot.png}
		\caption{distribution of prices at and after prolonged deviation relative to alternative path \emph{without} forced deviation, i.e.\ the difference to the price at the same $\tau$ had no deviation taken place. Only includes converged runs because a clear counterfactual exists. Boxes demarcate 15th and 85th percentiles and are extended by whiskers that mark the entire range of price differences. Horizontal lines represent the group median.}
		\label{prolonged_intervention_boxplot}
\end{figure}


\pagebreak
\subsection{$\beta$}\label{beta_appendix}

\textbf{some words on beta}

\begin{figure}
	\includegraphics[width=\linewidth]{plots/average_intervention_beta_tabular.png}
	\caption{Average price trajectory around deviation. Only tabular learning. Includes converged and non-converged runs}
	\label{average_intervention_beta_tabular}
\end{figure}


\pagebreak
\subsection{$\lambda$}\label{lambda_appendix}


Recall that high values of $\lambda$ increase the algorithm's hindsight but increases variance. This is reflected in both convergence rates and outcomes. \autoref{converged_lambda} clearly indicates that high values of $\lambda$ impedes convergence for tabular learning and the separated polynomial method. Similarly, \autoref{lambda_violin} exhibits greater variability in profits with increasing $\lambda$. This holds true for all feature extraction methods, but is particularly conspicuous for the separated polynomial method where a significant number of runs end in profits below the Nash equilibrium once $\lambda \ge 0.6$ .\footnote{The runs with $\Delta <0$ are largely the simulations without convergence.} Finally, the conclusions regarding deviation patterns are robust to variations in $\lambda$. Except for tabular learning, the non deviating agent fails to punish cheating which the deviating agent does not reliably exploit that weakness.

\begin{figure}
	\includegraphics[width=\linewidth]{plots/converged_m.png}
	\caption{Number of runs per experiment that achieved convergence as a function of $\lambda$.}
	\label{converged_lambda}
\end{figure}

\begin{figure}
	\includegraphics[width=\linewidth]{plots/lambda_violin.png}
	\caption{distribution of $\Delta$ for various experiments. Includes converged and non-converged runs. Violin widths are scaled to maximize width of single violins, comparisons of widths between violins are not meaningful. Violins are trimmed at smallest and largest observation respectively. Horizontal lines represent the median.}
	\label{lambda_violin}
\end{figure}


\pagebreak
\subsection{Price grid}\label{price_grid_appendix}

\textbf{some words on m}

\textbf{additional picutre showing stronger response for $m=10$ compared to other $m's$}

\begin{figure}
	\includegraphics[width=\linewidth]{plots/converged_m.png}
	\caption{Number of runs per experiment that achieved convergence as a function of $m$.}
	\label{converged_m}
\end{figure}


\begin{figure}
	\includegraphics[width=\linewidth]{plots/average_intervention_m_10.png}
	\caption{Average price trajectory around deviation. $m=10$. Includes converged and non-converged runs}
	\label{average_intervention_m_10}
\end{figure}

\textbf{some words on $zeta$}

Moreover, \autoref{zeta_violin_prices} confirms that average prices upon convergence largely remain within the Nash and collusive benchmarks. Polynomial tiles constitute the only exception which further discredits the method as appropriate for the learning task.

\begin{figure}
	\includegraphics[width=\linewidth]{plots/zeta_violin_prices.png}
	\caption[Distribution of average prices by \gls{fem} and $\zeta$]{Distribution of average prices (over cycle steps and both players) by \gls{fem} and $\zeta$. Includes converged and non-converged runs. Violin widths are scaled to maximize width of individual violins, comparisons of widths between violins are not meaningful. Violins are trimmed at smallest and largest observation respectively. Horizontal lines represent the median.}
	\label{zeta_violin_prices}
\end{figure}


\begin{figure}
	\includegraphics[width=\linewidth]{plots/average_intervention_zeta_tabular.png}
	\caption{Average price trajectory around deviation. Only tabular learning. Includes converged and non-converged runs}
	\label{average_intervention_zeta_tabular}
\end{figure}

\pagebreak
\subsection{Discount factor}\label{discounting_appendix}
\textbf{some words on $\gamma$}

\begin{figure}
	\includegraphics[width=\linewidth]{plots/gamma_violin_price.png}
	\caption{distribution of prices averaged over the last 100 time steps upon convergence and over both agents for various values of $\gamma$. Includes converged and non-converged runs. Violin widths are scaled to maximize width of single violins, comparisons of widths between violins are not meaningful. Isolated data points above $2$ are not displayed to improve presentability.}
	\label{gamma_violin_price}
\end{figure}



\pagebreak
\subsection{Algorithm Variations}\label{vary_algorithm_appendix}

\textbf{words on tree backup?}

Similarly,  \autoref{average_intervention_tb} reiterates that only tabular learning agents show a consistent punishment in response to a deviation and the cheated agents learning with feature approximation methods fail to respond. The bottom right panel, representing the polynomial tiles \gls{fem}, hints at a vague \emph{matching strategy} leading to new equilibria again, but here averaging turns out to be deceptive. In fact, only in 12.5\% of the runs does the non deviating agent respond with a price cut. Compared to the Expected SARSA algorithm, polynomial tiles exhibits a larger share of runs at the perfectly collusive solution.

\begin{figure}
	\includegraphics[width=\linewidth]{plots/average_intervention_tb.png}
	\caption[Average $\Delta$ for \emph{tree backup} algorithm with optimized $\alpha$ by \gls{fem}]{Average $\Delta$ for \emph{tree backup} algorithm with optimized $\alpha$ by \gls{fem}. Includes converged and non-converged runs.}
	\label{average_intervention_tb}
\end{figure}


The distribution of deviation prices and responses are displayed as boxplots in \autoref{intervention_boxplot_tb}.

\autoref{intervention_boxplot_tb} displays the distribution of prices around the forced deviation. With regard to polynomial tiles, \emph{some} runs show a sort of punishment or matching behavior in the wake of a price cut, but the vast majority (87.5\%) of runs show no response.

\begin{figure}
	\includegraphics[width=\linewidth]{plots/intervention_boxplot_tb.png}
	\caption{distribution of prices at and after deviation relative to alternative path \emph{without} forced deviation, i.e.\ the difference to the prices at the same $\tau$ had no deviation taken place. Boxes demarcate 15th and 85th percentiles and are extended by whiskers that mark the entire range of price differences. Horizontal lines represent the group median.}
	\label{intervention_boxplot_tb}
\end{figure}


\pagebreak

Box 2 describes the \emph{on-policy} algorithm used in \autoref{vary_algorithm}.

\begin{algorithm}
	\caption{Gradient Descend SARSA (on policy)}
	\begin{algorithmic}[]
		\label{SARSA}
		\small
		\STATE input feasible prices via $m \in \mathbb{N}$ and $\zeta > 0$
		\STATE configure static algorithm parameters $\alpha > 0$, $\beta > 0$, and $\lambda \in [0, 1]$
		\STATE initialize parameter vector and eligibility trace $\boldsymbol{w} = \boldsymbol{z} = \boldsymbol{0}$
		\STATE declare convergence rule (see \autoref{convergence})
		\STATE start tracking time: $t = 1$
		\STATE randomly initialize state $S_t$
		\STATE choose initial action $A_t$
		\WHILE{convergence is not achieved,}
		\STATE observe profit $\pi$, adjust to reward $r$
		\STATE move to next state: $t \leftarrow t+1$ and $S_{t+1} \leftarrow A_t$
		\STATE select action $A_{t+1}$ according to \autoref{action_selection}
		\STATE calculate TD-error: $\delta \leftarrow r + \gamma \hat{q}(S_{t+1}, A_{t+1}) - \hat{q}(S_t, A_t)$ (\autoref{td_error_on_policy})
		\STATE update eligibility trace: $\boldsymbol{z} \leftarrow \gamma \lambda \boldsymbol{z} + \boldsymbol{x}$
		\STATE update parameter vector: $\boldsymbol{w} \leftarrow \boldsymbol{w} + \alpha  \delta  \boldsymbol{z}$ (\autoref{update_rule})
		\STATE $S \leftarrow S_{t+1}$ and $A \leftarrow A_{t+1}$
		\ENDWHILE
	\end{algorithmic}
\end{algorithm}




\pagebreak
\subsection{Differential Reward}\label{differential_appendix}

\textbf{words on differential reward setting}

Curiously, the separated polynomial method struggles to achieve convergence in the differential reward setting. \autoref{convergence_at_upsilon} illustrates that a surprisingly large number of runs converge at a stage where exploration is incredibly rare. This suggests that agents frequently change their evaluation of what the optimal action is. Moreover, this may also help to explain the large share of non-converged runs (refer back to \autoref{converged_upsilon}) with an insufficient length of episodes per run.

\begin{figure}
	\includegraphics[width=\linewidth]{plots/convergence_at_upsilon.png}
	\caption{timing of convergence n the differential reward setting, runs that did not converge or failed to complete are excluded. Width of bins: 8,000}
	\label{convergence_at_upsilon}
\end{figure}


\autoref{upsilon_violin}

\begin{figure}
	\includegraphics[width=\linewidth]{plots/upsilon_violin.png}
	\caption{distribution of $\Delta$ in the differential reward setting for various values of $\upsilon$. Includes converged and non-converged runs. Violin widths are scaled to maximize width of single violins, comparisons of widths between violins are not meaningful. Violins are trimmed at smallest and largest observation respectively. Horizontal lines represent the median.}
	\label{upsilon_violin}
\end{figure}

\autoref{intervention_boxplot_tiling} illustrates the charged prices around the intervention relative to a counterfactual without a forced deviation for tile coding. Tabular learning shows a clear tendency to punish price cuts at $\tau = 2$. For tile coding and polynomial tiles, a price cut in response to the deviation occurs in \emph{some} runs.

\begin{figure}
	\includegraphics[width=\linewidth]{plots/intervention_boxplot_upsilon_005.png}
	\caption{distribution of prices in the differential reward setting with $\upsilon = 0.005$ at and after deviation relative to alternative path \emph{without} forced deviation, i.e.\ the difference to the prices at the same $\tau$ had no deviation taken place. Boxes demarcate 15th and 85th percentiles and are extended by whiskers that mark the entire range of price differences. Horizontal lines represent the group median.}
	\label{intervention_boxplot_tiling}
\end{figure}
	\end{appendices}
\end{document}
