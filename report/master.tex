\documentclass[a4paper]{scrartcl}
%alternatives to classic article are: book, report, scrartcl (recommended), scrreprt, scrbook
%more options Optionen: 11pt, 12pt, twoside, twocolumn

\usepackage[english]{babel}  %for German texts: ngerman
%\usepackage[utf8]{inputenc}   %encode in UTF-8 (I think TexStudio does this by default)

%\usepackage{natbib}  %used for bibliography (exact purpose unknown)
%\bibliographystyle{natdin} %bibliography style

\usepackage[backend = biber, citestyle=authoryear, bibstyle = authoryear]{biblatex} %more options, see e.g. https://www.overleaf.com/learn/latex/Biblatex_citation_styles
\addbibresource{zotero_refs.bib} %pass the name of the bib file

%Formatting and Layout
%\usepackage[left = 1.5cm, right = 1.5cm]{geometry} %This package allows for granular specified layouts. Optional arguments are, among others: left, right, top, bottom, width, height, textwidth, textheight, includeheadfoot
\usepackage[onehalfspacing]{setspace} %controls space between lines (line spacing)
\renewcommand{\baselinestretch}{1.5} %controls space between lines (line spacing)
\usepackage{blindtext} % to test layout via \blindtext

% Algorithm box
\usepackage{algorithm,algorithmic}

\usepackage{calligra}
\DeclareMathAlphabet{\mathcalligra}{T1}{calligra}{m}{n}


\pagestyle{headings}   %optional: section name is printed on the head of each page


\usepackage{xcolor}    %Colour package
\usepackage{graphicx}  %for graphics
\usepackage{amsmath,amsfonts,amssymb,amsthm,mathtools}  %math packages

%References
\usepackage[hidelinks]{hyperref} % should be called as last package
\usepackage{cleveref}


\begin{document}
	
	\def\sectionautorefname{section}
	\def\equationautorefname{equation}
	
	
	
	
	\title{\LaTeX Template}
	\author{Malte Jeschonneck}
	\date{\today}
	\maketitle
	
	\newpage
	
	\begin{titlepage}
		\centering
		\vspace{1cm}
		{\Large\bfseries \LaTeX Template \par}
		\vspace{4cm}
		{\large\itshape Heinrich-Heine-University Düsseldorf\par}
		\vspace{0.5cm}
		{\large\itshape Faculty of Business Administration and Economics\par}
		\vspace{0.5cm}
		{\large\itshape MW70 Competition Law and Policy\par}
		\vspace{0.5cm}
		{\large\itshape Summer Term 2018\par}
		{\large\itshape \par}
		\vfill
		by\par
		\vspace{0.5cm}
		Malte Jeschonneck\\
		Malte.Jechonneck@uni-duesseldorf.de\\
		Matriculation Number: 2307497\\
		Mörsenbroicher Weg 179, 40470 Düsseldorf\\
		Program: Economics, M. Sc.\\
		First Term
		
		
		\vfill
		
		% Bottom of the page
		{\large \today\par}
	\end{titlepage}

\newpage
	
	
	\tableofcontents
	\newpage
	\listoffigures
	\newpage
	\listoftables
	\newpage
	
	\section{Introduction}

There is little doubt that algorithms will play an increasingly important role in everyday life. Automated and dynamic pricing are frequently used in online retail \parencite{chen_empirical_2016}, but are also applied in the tourist industry \parencite[p.4]{den_boer_dynamic_2015} and at petrol stations \parencite[pp.7-9]{assad_algorithmic_2020}. 

\textbf{where is it prevalent}


As with many other technological advances, the economic advantages are conspicuous. Not only does automating pricing decisions cut costs and free up resources, algorithms may also be better at predicting demand and react faster to changing market conditions \parencite[p. 15]{oecd_algorithms_2017}. 
This non-exhaustive list of examples leaves little doubt that pricing algorithms may be used as tool by companies to gain competitive advantages and it is worth pointing out that consumers also benefit from intensified competition.\footnote{Moreover, there exist other types of algorithms that benefit consumers. Price comparison tools have been around for a while but the applications extend beyond that. \textcite{gal_algorithmic_2017} champion \emph{algorithmic consumers}, electronic assistants who make sophisticated product comparisons at low transaction costs enabling humans to completely outsource their purchase decisions. Moreover, algorithmic consumers may challenge market power of suppliers by bundling consumer interests.}

Nevertheless, concerns have been raised that ceding pricing authority to algorithms has the potential to create new forms of collusion that contemporaneous competition policy is not well equipped to deal with. The main issue is that the traditional dichotomy between \emph{explicit} and \emph{tacit} collusion is potentially unsuitable in the case of pricing software. Traditionally, competition authorities only prohibit and punish explicit pricing agreements. On the contrary, tacit collusion (e.g.\ \emph{intelligent market adaption}) is tolerated \textbf{citation needed?} despite the economic effect on consumers being equally detrimental \parencite[p. 141]{motta_competition_2004}.\footnote{A different issue is that pricing algorithms with information on consumer characteristics may be able to augment the scope of \emph{perfect price discrimination}, i.e.\ companies extracting rent by charging to every consumer the highest price he is willing to pay. Under which circumstances competition authorities should be concerned with this possibility is outlined in \textcite{oecd_price_2016}. \textcite{ezrachi_algorithmic_2017} develop a scenario where discriminatory pricing and tacit collusion occur simultaneously. Both issues remain outside the scope of this study.}


\textbf{The distinction was elusive before the advent of pricing algorithms. But it is even less clear now and might not capture the danger of algorithmic collusion}

\textbf{painpoint of of algorithms --> why do we care?} 

Unfortunately, there is a lack of empirical studies assessing the effects of autonomous pricing software in the real world. A notable study of the German retail gasoline market by \textcite{assad_algorithmic_2020} documents that margins in duopoly markets increased substantially after both gasoline stations switched from manual pricing to algorithmic-pricing software. Further field studies could prove instrumental to confirm and refine these findings. On the other hand, there is a growing number of simulation studies that show the capacity of \emph{reinforcement learning} algorithms to create and sustain collusive equilibria in repeated pricing competition games (see \autoref{simulation_studies}). However, the direct transferability of these findings to real markets is questionable. Most studies use tabular learning methods, mainly \emph{Q-Learning}, which requires discretizing prices and does not scale well if the complexity of the environment increases.

This study attempts to get rid of the problem by employing \emph{linear function approximation} to estimate the value of actions. More specifically, I develop three methods of function approximation and then run a series of experiments to assess how they compare to tabular learning. Moreover, I utilize \emph{eligibility traces} as an efficient way to increase the memory of agents interacting in the environment.\footnote{Neither linear function approximation nor eligibility traces are new concepts in reinforcement learning. However, to my knowledge, this is the first study to apply them to a repeated pricing game.}

To foreshadow the results, the simulations show that the developed function approximation methods, like tabular learning, result in supra-competitive prices upon convergence. However, \emph{unlike} tabular learning, the learned strategies are easy to exploit. By forcing one of the agents to diverge from the convergence equilibrium, I show that the cheated agent fails to punish that deviation. This indicates that the learned equilibrium strategies are unstable vis-à-vis rational agents with full information. This observation is robust to a number of variations and extensions. Also, the impact of eligibility traces in this study is small.

The remainder of this paper is organized as follows. The next section briefly reviews literature on algorithmic competition as well as contemporaneous regulation and presents results from previous simulations similar to this study. Section \ref{enironment} introduces the repeated pricing environment I let the competitors interact with. Section \ref{algorithm} present in detail the deployed learning algorithm with its parametrization and \autoref{feature_extraction} discusses the developed methods to estimate action values with function approximation. I present the results in \autoref{results} and consider variations and extensions in \autoref{robustness}. Finally, \autoref{conclusions} concludes.
	
	\section{Literature Review}
This study is related to three literature streams: (i) the scholarly debate on how competition law is supposed to manage autonomous pricing software, (ii)  repeated games in the realm of algorithms, and (iii) an increasing number of simulation studies that empirically examine the behavior of algorithms in simplified economic environments. I will provide a brief summary of the recent developments in each of these fields.

\subsection{Competition Policy and pricing algorithms}
As algorithms increasingly take over pricing authority from humans in a number of industries, some scholars have voiced concerns about the adequacy of current competition laws and practices. \textcite{mehra_antitrust_2015} points out that the traditional distinction between explicit and tacit collusion emerged with \emph{human sellers} in mind who differ from \emph{robo-sellers}. He argues that the latter are more likely to achieve cartel solutions in an oligopolistic setting due to superior speed, accuracy and even rationality when analyzing and adjusting prices. He concludes that the increasing prevalence of automated pricing software warrants a reassessment of current competition law and enforcement. 

* no human intent to achieve supracompetitive prices
	* but employed algorithms achieved that anyway. 
	* outcome eludes explicit agreement, still detrimental

\textcite{ezrachi_sustainable_2018} claim academic consensus that algorithms could at the very least be utilized to facilitate \emph{existing} collusive agreements. For instance, cartel members could automate the detection and even punishment of deviations from an agreement through an automated algorithm. Other conceivable schemes include facilitated market segmentation \parencite{oefgen_decision_2019} and price \emph{signaling} \parencite{oecd_price_2016}. While these scenarios may alter the operational scope of market investigations to account for the role of deployed algorithms, they are well covered by contemporary competition laws.  \footnote{see e.g.\ statements by \cite{bundeskartellamt_bundeskartellamt_nodate}. See \textcite{cma_case_2016} \textcite{oefgen_decision_2019}for two exemplary cases}. 




\textcite{noa}


\textcite{bundeskartellamt_bundeskartellamt_nodate} also note that the specifics of the algorithms are not highly important because the mere \emph{intention} to collude suffices to invoke competition laws.


\subsection{Algorithms in Game Theory}

something moer

\subsection{Simulation Studies}

While there are numerous studies on the behavior of algorithms in cooperative and competitive games, their application in industrial economics has been a little scarcer. A seminal study by \textcite{waltman_q-learning_2008} examines two \emph{Q-Learning} pricing agents in a \emph{Cournot} environment.\footnote{i.e.\ firms compete in quantities.} Their simulations result in supra-competitive equilibria, but even \emph{memoryless} agents without knowledge of past outcomes manage to attain quantities below the one-shot Nash equilibrium.  This casts doubt on the viability of the learned strategies vis-à-vis rational agents. Truly memoryless agents can't pursue \emph{trigger strategies} in the sense that they are unable to punish deviations as they fail to even detect them. Thus, constantly playing the one-shot solution \emph{should} be the only rational strategy. To that end, the agents seem to \emph{fail to learn how to compete} rather than to \emph{learn how to collude} \parencite{cooper_learning_2015}. 






More recently, \textcite{calvano_artificial_2019} simulate two Q-Learning agents that not only achieve collusive outcomes, but also sustain their tacit agreement through a \emph{reward-punishment} scheme. \textcite{klein_autonomous_2019} shows that Q-Learning algorithms converge faster in a sequential price setting environment.







Unfortunately, there is a lack of empirical studies assessing the workings and effects of autonomous pricing software in the real world. To my knowledge, a notable study of the German retail gasoline market by \textcite{assad_algorithmic_2020} remains the only exception. They document that margins in duopoly markets increased substantially if both actors switched from manual pricing to algorithmic-pricing software. Further field studies could prove instrumental to confirm and refine these results.
	
	\section{Environment}\label{enironment}

This section presents the simulated economic environment that the autonomous pricing agents interact with. I consider an infinitely repeated pricing game with a multinominal logit demand as in \textcite{calvano_artificial_2019}. Restricting the analysis to a symmetric oligopoly case with $n=2$ agents (where $i = 1,2$), the market comprises \emph{2} differentiated products and an outside option. In every period $t$, both agents simultaneously pick a price $p_i$. Demand for agent $i$ is then determined \parencite{anderson_logit_1992}:\footnote{Generalization to a model with \emph{n} agents is straightforward. In fact, the demand formula remains the same. The limitation to 2 agents is merely chosen for computational efficiency and the (intuitive) conjecture that the simulation results generalize to more players provided learning time is sufficiently high.}

\begin{gather}\label{quantity}
q_{i,t}=\frac{e^{\frac{a - p_{i,t}}{\mu}}}{\sum_{j=1}^{n}~ e^{\frac{a-p_{j,t}}{\mu}}+e^{\frac{a_0}{\mu}}}
\end{gather}

\textbf{Citations needed}
$\mu$ controls the degree of horizontal differentiation, where $\mu \rightarrow 0$ approximates perfect substitutability. While I forego to incorporate vertical differentiation throughout this study, it may be incorporated by choosing firm-specific quality parameters $a$. $a_0$ reflects the appeal of the outside good. It diminishes as $a_0 \rightarrow -\infty$. 

Profits of both agents $\pi_i$ are simply calculated as

\begin{gather}\label{profit}
\pi_{i,t} = (p_{i,t} - c) q_{i,t},
\end{gather}

where $c$ is the marginal cost.\footnote{Again, $c$ could be varied by adding suffixes accordingly.} \textcite{anderson_logit_1992} show that the multinominal logit demand model with symmetric firms entails a unqique one-shot equilibrium with best responses that solve:

\begin{gather}\label{best_response}
	p_n = p^* = c + \frac{\mu}{1 - (n + e^{\frac{a_0 - a + p^*}{\mu}})^{-1}}
\end{gather}

Naturally, the other extreme, a collusive (or monopoly) solution, is obtained by maximizing joint profits.\footnote{For this study, I approximate both cases using numerical optimization.} Both, the Nash outcomes characterized by $p_n$ and $\pi_n$ \emph{and} the fully collusive solution ($p_m$ and $\pi_m$) shall serve as benchmarks for the simulations.

Market entry and exit are not considered. The baseline parametrization is identical to \textcite{calvano_artificial_2019}:
$c = 1$,
$a = 2$,
$a_0 = 0$ and
$\mu = \frac{1}{4}$. These parameters give rise to a static Nash equilibrium with $p_n \approx 1.47$ and $\pi_n \approx 0.23$ per agent. The monopolist solution entails $p_m \approx 1.92$ with $\pi_m \approx 0.34$ for each product. Nevertheless, the following section covers the applied reinforcement learning methods for general parameters.

	
	\section{Reinforcement Learning with Function Approximation}
	
	

\pagebreak
\begin{algorithm}
	\caption{Gradient Descend Expected SARSA}
	\begin{algorithmic}[]
		\small
		\STATE Initialize state S
		\WHILE{convergence is not achieved,}
		\STATE Choose action A \~{} $\pi(.|S)$
		\STATE observe profit $\pi$, adjust to reward $R$
		\STATE observe next state: $S_{t+1} = A_t$
		\STATE calculate TD-error: $\delta \leftarrow R +  \gamma \bar{V}(S_{t+1}) - q(S_t)$
		\STATE update eligibility trace: $\boldsymbol{z} \leftarrow \gamma \lambda \rho \boldsymbol{z} + \boldsymbol{x} $
		\STATE update parameter vector: $\boldsymbol{w} \leftarrow \boldsymbol{w} + \alpha  \delta  \boldsymbol{z}$
		\STATE $S \leftarrow S_{t+1}$
		\STATE $t \leftarrow t+1$
		\ENDWHILE
	\end{algorithmic}
\end{algorithm}
















\section{Results}
Starting, with the baseline specification, this section reports on the simulation results. \textbf{TBD}. To foreshadow the results, profits mostly exceed Nash-predictions, but remain below monopoly profits. While agents learn to charge supra-competitive prices, they fail to incorporate \emph{reward-punishment} schemes consistently. Overall, the results crucially hinge on the combination of feature extraction method and selected parameters. Only tabular learning exhibits a clear tendency to punish deviations with lower prices in subsequent periods.

I report results for various specifications and will refer to every unique combination of feature extraction method and parameters as an \emph{experiment}. Every experiment consists of 48 \emph{runs}, i.e. repeated simulations with the exact same set of starting conditions. Lastly, within the scope of a particular \emph{run}, time steps are called \emph{periods}.

\subsection{Convergence}\label{convergence}

\textbf{TBD: As indicated,} convergence is not guaranteed in a non-stationary environment, much less so with function approximation. Notwithstanding the lack of a theoretical convergence guarantee, prior experiments have shown that simulation runs tend to approach a stable equilibrium in practice (\cite{calvano_artificial_2019} \textbf{and others}). Note that \emph{stability} simply refers to the observation that the same set of prices continuously recur over a longer time interval. The strategies upon convergence need not coincide with economic theory. In fact, at times the observed outcomes in this study contradict predictions from game theory. For instance, despite symmetric profit functions, the converged outcomes may display asymmetric prices. Moreover, price cycles, i.e.\ a recurring sequence of price combinations, occur frequently.\footnote{The model from \autoref{quantity} predicts symmetric outcomes without cycles. This is typical for simultaneous pricing games, but not universal across economic models. For instance, collusive outcomes in quantity competition (i.e.\ Cournot) may exhibit price asymmetries. The relevance of that prediction has been fortified in experimental settings, e.g.\ in \textcite{fischer_collusion_2019}. \textcite{maskine-tirole} pioneer a sequential pricing game that predicts \emph{Edgeworth price cycles} where agents successively undercut each other until one firm prefers to reset the cycle and increases its price. Based on their model, \textcite{klein_autonomous_2019} shows that \emph{Q-Learning} agents are indeed capable of learning those dynamic strategies.}


The following, arbitrary but practical, convergence rule was employed. If a price cycle recurred for 10,000 consecutive episodes, the algorithm is considered \emph{converged}. A price cycle requires both agents' adherence.\footnote{Of course it is possible that the cycle length differs between agents. For instance, one agent may continuously play the same price while the opponent keeps alternating between two prices. In this case, the cycle length is $1*2=2$.}

For efficiency reasons, price cycles up to a length of 10 are considered and a check for convergence is undertaken only every 2,000 episodes. If no convergence is achieved until 500,000 episodes, the simulation stops and the run is deemed \emph{not converged}. Furthermore, there are a number of runs that \emph{failed to complete} as a consequence of the program running into an error. Unfortunately, the program code does not allow to examine the exact cause of such occurrences in retrospect. However, by and large, the failed runs occurred with unsuitable specifications (see below for a detailed discussion).

\begin{figure}
	\includegraphics[width=\linewidth]{plots/converged.png}
	\caption{Number of runs that achieved convergence per experiment.}
	\label{converged}
\end{figure}

In accordance with the outlined convergence criteria above, \autoref{converged} displays the share of runs that, respectively, converged successfully, did not converge until the end of the simulation or failed to complete. Two main conclusions emerge. First, failed runs are mainly prevalent in specifications with a high value of $\alpha$ in conjunction with a polynomial feature method. Second, the tiling methods are more likely to converge. Both points deserve some further exploration.

Regarding the failed runs, \textbf{recall} from \autoref{feature_extraction} that features of polynomial extraction are not binary and warrant cautious adjustments of the coefficient vector. I suspect that with unreasonably large values of $\alpha$, the estimates of $\boldsymbol{\theta}$ overshoot early in the simulation, don't recover and at some point exceed the software's numerical limits.\footnote{Controlled runs where I could carefully monitor the development of the coefficient vector $\boldsymbol{w}$ seem to confirm the hypothesis. However, isolated errors \emph{with} reasonable parameter settings remain unexplained.} While imoportant to acknowledge, the failed runs are largely an artifact of unreasonable specifications and I will \textbf{disregard them for the remainder of this chapter}. For instance, the percentages in the subsequent paragraph don't account for the failed runs.

Out of thecompleted runs without program failure, 95.4\% did converge. Though there are subtle differences between feature extraction methods. With only one exception, both tiling methods converged consistently for various $\alpha$. With only 85.4\% of runs converging, separate polynomials constitute the other extreme. The figure also indicates that convergence becomes less likely for low values of $\alpha$. With tabular learning, 92.9\% of runs converged without clear relation to different values of $\alpha$.

\autoref{convergence_at} displays a frequency polygon of the runs that achieved convergence within 500,000 episodes. Clearly, the distribution is fairly uniform across feature extraction methods. Most runs converged between 200,000 and 300,000 runs. This is an artifact of the decay in exploration as dictated by $\beta$. Before the focal point of 200,000 is reached, agents probabilistically experiment too frequently to observe 10,000 consecutive episodes without any deviation from the learned strategies. Thereafter, it becomes increasingly likely that both agents keep \emph{exploiting} their current knowledge and continuously play the same strategy for a sufficiently long time to trigger the convergence criteria. Note that the low quantity of runs converging between 300,000 and 500,000 suggests that increasing the maximum of allowed episodes would not necessarily entail a significantly higher portion of converged runs.

\begin{figure}
	\includegraphics[width=\linewidth]{plots/convergence_at.png}
	\caption{timing of convergence, runs that did not converge or failed to complete are excluded. Width of bins: 8,000}
	\label{convergence_at}
\end{figure}

\autoref{cycle_length} visualizes the distribution of cycle length and offers some interesting insights. Unsurprisingly, a first glance suggests that the frequency of runs decreases with cycle length. Not accounting for differences between selection methods, the bars appear similar to a geometric distribution with the largest bar corresponding to a 'cycle length of 1' (i.e.\ no cycle at all). Moving towards the right, the frequency of observed runs decreases with cycle length, though at a decreasing pace. In fact, there are even 7 runs with the largest considered cycle length of 10. There are substantial differences between the different feature extraction methods. Polynomial tiling follows the described decaying pattern. Similarly, simple tile coding rarely converges in long cycles, though its spike of 194 runs corresponds to a cycle length of 2. Contrary, almost all runs of the separated polynomials converged without cycles.\footnote{Though barely visiblein \autoref{convergence_at}, there are 2 runs with a cycle length of 2.}. Lastly, the frequency of cycle length of converged tabular runs is distributed almost uniformly. This observation also suggests that the employed convergence rule may well have misclassified some of the runs in the top left panel of \autoref{converged} as \emph{not converged} where in reality the convergence cycle length simply exceeded the threshold arbitrarily set at 10. 

\begin{figure}
	\includegraphics[width=\linewidth]{plots/cycle_length.png}
	\caption{Number of converged runs with particular cycle length.}
	\label{cycle_length}
\end{figure}

\autoref{prices} unveils the ranges of prices within a cycle. For now, I proceed by examining profits upon convergence.

\subsection{Profits}

In order to benchmark the simulation profits, I normalize profits similar to \textcite{calvano_algorithmic_2018}:

\begin{gather}
\Delta = \frac{\bar{\pi} - p_n}{p_m - p_n}.
\end{gather}

$\bar{\pi}$ represents profits averaged over the last 100 time steps upon convergence and over both firms in a single run\footnote{Instead of looking just at the convergence profits, I average over the last 100 time steps to account for price cycles.}. The normalization implies that $\Delta = 0$ and $\Delta = 1$ respectively reference the Nash and monopoly solution. Note that it is possible to obtain a $\Delta$ below $0$ (e.g. if both agents charge prices equal to marginal costs), but not above $1$.\footnote{Strictly speaking, exactly 1 is not attainable either. Recall that $m$ was chosen to allow for prices very close, but not equal to both benchmark prices. With $m = 19$, the highest feasible $\Delta$ is 0.9997.} \autoref{alpha} displays the convergence profits as a function of the feature extraction method and $\alpha$. Every data point represents one experiment, more specifically the mean of $\Delta$ across all runs making up the experiment.

\begin{figure}
	\includegraphics[width=\linewidth]{plots/alpha.png}
	\caption{average $\Delta$ for various experiments. Includes converged and non-converged runs. One data point (poly tiling, $\alpha = 0.0004$) is excluded for better presentability. Beware the logarithmic x-scale.}
	\label{alpha}
\end{figure}

First of all, note that average profits consistently remain between both benchmarks $p_m$ and $p_n$ across specifications.\footnote{There is one exception. One data point is hidden in the plot to preserve reasonable y axis limits. More specifically, for the polynomial tiles and $\alpha = 0.0001$, the average $\Delta$ is -1.73. This extends the observation in \autoref{convergence}. It appears that this particular $\alpha$ constitutes a critical point. While the program does not crash, agents only learn strategies void of any reasonableness. As \autoref{alpha} displays, outcomes within the benchmarks are obtained by further decreasing $\alpha$.} As with prior results, the plot unveils salient differences between feature extraction methods.  On average, polynomial tiling runs yield the highest profits. The average $\Delta$ peaks at 0.85 for $\alpha = 10^{-8}$. Higher values of $\alpha$ tend to progressively decrease profits. Moving downwards on the y-axis, both the tabular method and tile coding yield similar average values of $\Delta$. Furthermore, the level of $\alpha$ does not seem to impact $\Delta$ much. For both methods $\alpha = 10^{-4}$ induces the highest average $\Delta$ at 0.487 and 0.478 respectively. Similarly for separated polynomials, $\Delta$ does not seem to respond to variations in $\alpha$. The maximum $\Delta$ is 0.35.

Naturally, averaging $\Delta$ over runs as in \autoref{alpha} potentially hides subtleties in the distribution of $\Delta$ per experiment. \autoref{alpha_violin} displays a violin plot that shows the distribution of $\Delta$ per experiment. The distribution largely confirms the conclusion that most runs converge between $\Delta_m$ and $\Delta_n$. The only method with a significant quantity of runs with profits below the Nash benchmark are the separated polynomials. Overall, 19.8\% of runs converged with profits below the Nash equilibrium, though most of them ended up reasonably close. The percentage is largest for the experiment with $\alpha = 10^{-8}$: 27.1\%. While the other methods tend to elicit runs within the set up benchmarks, the variability remains quite high. This indicates a degree of path dependence and suggests that the algorithms are prone to stick to early explored strategies that are \emph{above-average}, but \emph{sub-optimal}. Polynomial tiles exhibit the narrowest $\Delta$ range, in particular for low $\alpha$.

\begin{figure}
	\includegraphics[width=\linewidth]{plots/alpha_violin.png}
	\caption{distribution of $\Delta$ for various experiments. Includes converged and non-converged runs. Violin widths are scaled to maximize width of single violins, comparisons of widths between violins are not meaningful. Violins are trimmed at smallest and largest observation respectively. One violin (poly tiling, $\alpha = 0.0004$) is excluded for better presentability. Horizontal lines represent the median. Beware the logarithmic x-scale.}
	\label{alpha_violin}
\end{figure}


\autoref{convergence} and \autoref{alpha} established that, what constitutes a sensible value of $\alpha$ clearly depends on the feature extraction method. Hence, for the remainder of this chapter, I will select an \emph{optimal} $\alpha$ for every feature extraction method and present further results only for these combinations. In determining an `optimal` $\alpha$, I don't rely on a single hard criteria, rather I consider a number of factors including the percentage of converged runs, comparability with previous studies and prefer to select experiments with high average $\Delta$ as they are most central to the purpose of this study. \autoref{justifications} provides a justification for every experiment deemed \emph{optimal}. To get a sense of the variability of runs within the experiments and the price trajectory over time, \autoref{trajectory_Delta} displays the development of profits of all runs for the 'optimized' values of $\alpha$. \autoref{appendix} providisues further trajectory valizations of prices and profits.

\begin{table}

	\begin{tabular}{|l|c|l|}
		\hline
		\textbf{Feature Extraction Method}&$\boldsymbol{\alpha}$&\textbf{justification} \\
		\hline
		Tabular&0.1&- comparability with previous simulation studies \\
		&&- most pronounced response to price deviations \\
		&& \ \ (see \autoref{deviations}) \\
		\hline
		Tile Coding&0.001&- high $\Delta$ \\
		&&- most pronounced response to price deviations \\
		&&\ \ (see \autoref{deviations}) \\
		\hline
		Separated Polynomials&$10^{-6}$&- high percentage of converged runs \\
		\hline
		Polynomial Tiles&$10^{-8}$&- high $\Delta$ \\
		\hline
	\end{tabular}
	\caption{\emph{Optimized} values of $\alpha$ by feature extraction method}
	\label{justifications}
\end{table}

\begin{figure}
	\includegraphics[width=\linewidth]{plots/trajectory_Delta.png}
	\caption{distribution of $\Delta$ over time in 'optimized' experiments. For individual runs, $\Delta$ is averaged over 50,000 periods apiece and both players. Plot includes converged and non-converged runs. Violin widths represent quantity of active runs at $t$ which enables comparisons between violins. As most runs converge after 200,000 to 300,000 episodes, violin widths decrease thereafter. Violins are trimmed at smallest and largest observation respectively. Horizontal lines represent the median.}
	\label{trajectory_Delta}
\end{figure}



\subsection{Price Ranges}\label{prices}

As established in \autoref{convergence}, many simulations converge in price cycles of various lengths. Recall in particular that some feature extraction methods are more prone to long cycles than others. \autoref{price_range} plots the range between the lowest and highest price a single agent charges in a cycle upon convergence. Naturally, the price range is null if no cycle is present. Perhaps unsurprisingly, the price range then increases with the cycle length.

\begin{figure}
	\includegraphics[width=\linewidth]{plots/price_range.png}
	\caption{Distribution of price ranges as a function of feature extraction method and cycle length. Price range is defined as the range of prices \emph{one} agent charges upon convergence. Relationships to the opponent's prices are not examined. Only converged runs are considered (as cycle length is unavailable for other runs). Dashed line represents the difference between both collusive and Nash outcome (i.e.\ $p_m - p_n$). Box widths are scaled proportionally to the square-roots of number of observations within each group.}.
	\label{price_range}
\end{figure}


\subsection{Deviations}\label{deviations}

This section examines whether the learned strategies are stable in the face of deviations. As outlined before, collusion requires a \emph{reward punishment scheme} and it seems instructive to assess whether the agents learned to punish deviations. In order to scrutinize that, I had one agent deviate from the stable price cycle by playing the short-term best response to maximize own profits \emph{after} convergence was detected. Subsequently, both agents played the learned strategies again for 10 episodes. For the period of that intervention, learning and exploration was turned off.

\autoref{average_intervention} displays the average price trajectory around the manually imposed deviation. It exhibits clear differences between the considered feature extraction methods. 

\begin{figure}
	\includegraphics[width=\linewidth]{plots/average_intervention.png}
	\caption{average price trajectory around deviation}
	\label{average_intervention}
\end{figure}

As the average price trajectory might hide subtle differences between runs even within the same experiment, \autoref{intervention_violin} displays the distribution of at and after the deviation.


\begin{figure}
	\includegraphics[width=\linewidth]{plots/intervention_violin.png}
	\caption{distribution of prices at and after deviation relative to prior equilibrium. Every violin has the same maximum width, i.e.\ width between violins are not comparable. Tails are trimmed}
	\label{intervention_violin}
\end{figure}

\pagebreak
\subsection{responses off equilibrium}

\textbf{TBD}




	
	\section{Robustness}

\begin{enumerate}
	\item vary $\beta$
	\item vary discount factor
	\item vary $lambda$
	\item vary price range
	\item vary number of prices ($m$)
	\item other algorithms
	\item average reward setting
\end{enumerate}

\subsection{Algorithm Variations}\label{vary_algorithm}

\subsubsection{On Policy}

Note that $\delta_t$ can only be calculated after the action in the next period has been taken.\footnote{}

\begin{gather}
	\delta_t^{SARSA} = \pi_t + \gamma \hat{q}(S_{t+1}, A_{t+1}, \boldsymbol{w}) - \hat{q}(S_t, A_t, \boldsymbol{w})
\end{gather}
		

\subsection{Differential Reward Setting}

\begin{gather}
\delta_t = r_t - \widetilde{R}_{t-1} + \hat{q}(S_{t+1}, A_{t+1}, \boldsymbol{w}_t) - \hat{q}(S_t, A_t, \boldsymbol{w}_t) ~~   \text{,}
\end{gather}

$\widetilde{R}_{t-1}$ is a (weighted) average reward

\emph{ex post} \emph{differential} profit $\pi_t - \widetilde{R}_{t-1}$ in conjunction with the estimated value of the newly arising state-action combination in $t+1$

While they come with a meaningful economic interpretation, \textcite{sutton_reinforcement_2018} and \textcite{naik_discounted_2019} show that their use is inappropriate in infinite sequences with function approximation settings. Moreover, a policy maximizing average rewards is equivalent to a policy maximizing the average of discounted future values - irrespective of the particular discount factor.

Surprisingly, this system 
* discount factor can be = 1




\pagebreak
	
	\input{07_conclusions.tex}
	
	\section{To Do}
	List of useful \emph{quick \& dirty commands}:
	
	\begin{itemize}
		\item additional \space \space spaces \space \space \space with \texttt{\textbackslash space}
		\item \textbf{\textbackslash} \space via \texttt{\textbackslash textbackslash}
		\item \textbf{\textasciitilde} \space via \texttt{\textbackslash textasciitilde}
		\item \textbf{\ldots} \space via \texttt{\textbackslash ldots}
		
	\end{itemize}
	{\Huge Bibliography}
	also look at further hints in Microsoft/LinkedIn assignment file.
	
	\section{Hierarchy and References}\label{HaR}
	
	\subsection{Hierarchy}\label{Hie}
	
	The hierarchy is rather simple. There are \texttt{sections}, \texttt{subsections}, \texttt{subsubsections}, \texttt{paragraphs} \& \texttt{subparagraphs}. For instance, this text is located within a subsection. Luckily for us, \LaTeX \space takes care of incrementation.
	
	\subsubsection{Block Numbering}\label{sssbn}
	If we want don't want \LaTeX to assign numbers to a particular hierarchical element, one assign a \texttt{*} right after the command, e.g. \texttt{\textbackslash subsubsection*}
	
	\subsubsection*{Section without number}
	Hierarchically, this subsubsection is the successor of the previous subsubsection \ref{sssbn}.
	
	\subsubsection{Further hierarchy commands}
	The following hierarchical elements are rather uncommon and their functionality won't be described in detail. So, the following list serves merely as a collection:
	
	\begin{itemize}
		\item Naturally, \texttt{\textbackslash appendix} is used in the end of the text
		\item \texttt{\textbackslash part} \& \texttt{\textbackslash chapter} can only be used in the document class \emph{book}
		\item \texttt{\textbackslash frontmatter} is printed before the main text (if it is used, the main text should be started with \texttt{\textbackslash mainmatter})
		\item Similarly, \texttt{backmatter} is appended at the end
	\end{itemize}

\subsection{Calling other text files}
	In order to keep the size of a file manageable, a script can include references to other text files. The compiler will then include the text from that text file in the generated file. The required command is \texttt{\textbackslash include}. 

\subsection{References}
	\LaTeX \space can reference pretty much anything. The requirement is that the referenced object is labeled by \texttt{\textbackslash label}.
	
	\subsubsection{Default References}\label{dr}
	The default command to reference something is \texttt{\textbackslash ref}. For instance, we can refer to section \ref{HaR}, subsection \ref{Hie}, subsubsection \ref{sssbn}, table \ref{diamondsreg} \& figure \ref{tokyo}. On top of that one can refer to pages, e.g. the start of this paragraph is on page \pageref{dr}.
	
	\subsubsection{Hyperref}
	The package \emph{hyperref} steps up the default referencing. It introduces the command \texttt{\textbackslash autoref} which automatically returns the type of reference as well (e.g. you don't need to type `section' or `subsection' by yourself): As an example, look at the referencing towards \autoref{HaR}, \autoref{Hie}, \autoref{diamondsreg} \& \autoref{tokyo}.
	Hyperref does two more things:
	
	\begin{enumerate}
		\item it creates links between references and the Table of Contents. This allows a reader of the final \emph{pdf} to comfortably jump between sections.
		\item by default, it marks all references with a red frame. Personally, I like this while working in TexStudio but not really in the final pdf. It can be turned off by the option \texttt{hidelinks} when calling the package.
	\end{enumerate}



	\subsubsection{URLs and local file paths}
	
	Referencing URLs also works with \emph{hyperref}. Two options are available:
	\begin{itemize}
		\item using the \texttt{\textbackslash url} command: \url{https://r4ds.had.co.nz/index.html} 
		\item using the \texttt{\textbackslash href} command:
		\href{https://r4ds.had.co.nz/index.html}{R for Data Science}
	\end{itemize}
Both examples link the same page. The first displays the full link while the second uses a description of the page. Local file paths can be referenced in the same way. Here are links to the book
\href{run:C:/Users/psymo/OneDrive/Studium/Statistik/Dokumente/ISL/An Introduction to Statistical Learning.pdf}{\emph{Introduction to Statistical Learning}} and a picture of \href{run:C:/Users/psymo/Pictures/Korea/Korea/20190630_154735_HDR.jpg}{\emph{Ha Long Bay}}. Note that this linkage doesn't work in the previewer of TexStudio. You'll have to compile and open the real pdf for the links to work.

\subsection{Footnotes}
Footnotes are implemented with \texttt{\textbackslash footnote}. This is a sentence with an exemplary footnote\footnote{This is a footnote}.

	\section{Various}
	Due to the UTF-8 encoding, all German peculiarities are processed without problems: "ÄäÖöÜüß". Also note, that the language checking works as specified in the \emph{babel} package.
	
	That is great. At this point I want to check if \LaTeX knows about hyphenation and in order to do so I'll have to think of some sentences with some especially long word creations. The previous sentence convinced me that \LaTeX is capable of proper hyphenation.
	
	\subsection{Font Types}
	Single words can be printed in a different font. There are many types of fonts:
	\begin{itemize}
		\item \textmd{standard fonts via} \texttt{\textbackslash textmd}
		\item \textit{Italic} 
		\item \textbf{fat}
		\item \textsf{without serifs}
		\item \texttt{designed to print code where every letter has the same width}
		\item \textsc{capped names}
		\item \textsl{aslope}
		\item \underline{Underlined}
		\item \textcolor{blue}{Colored text} 
		\item \colorbox{green}{Marked text}
	\end{itemize}

\texttt{\textbackslash emph} ensures that an appropriate \emph{accentuation} for the current environment is chosen at all times.

	\subsection{Font Size}
One can change the font size, though in most circumstances this is not recommended, especially in a scientific text. The levels are:

\begin{itemize}
	\item {\tiny tiny}
	\item {\scriptsize scriptsize}
	\item {\footnotesize footnotesize}
	\item {\small small}
	\item {\normalsize normalsize (the default)}
	\item {\large large}
	\item {\Large Large}
	\item {\LARGE LARGE}
	\item {\huge huge}
	\item {\Huge Huge}
\end{itemize}


	\begin{scriptsize}
		Alternatively an environment can be called to change the text size for a longer text passage.
	\end{scriptsize}


\subsection{Quote and dash variations}
	There are many ways to quote. Which way is correct depends on the type of text and personal preferences. The most common ways are:
	
	\begin{itemize}
		\item `British single quotes' (I think this is my favorite)
		\item ``American double quotes'' (I think more common)
		\item \glqq German Quotation.\grqq.Test
		\item ``double quotes allows for `quotes inside quotes' ''
	\end{itemize}

The typical dashes that are available include:

	\begin{itemize}
		\item default - dash
		\item some -- dash
		\item large --- dash
		\item math $-$ dash
	\end{itemize}





	\subsection{Formatting}
	
\texttt{\textbackslash blindtext} is a nice way to test the current formatting options that are applied at a particular location of the text (in this example within the \texttt{quote} environment):
\begin{quote}
	\blindtext
\end{quote}


\subsubsection{line breaks}
Line breaks can be forced with \texttt{\textbackslash \textbackslash}. In order to increase the tweak the line spacing an optional argument can be used in square brackets.\\

{\setlength\parindent{0pt}
	I'm lost for words\\
	The truth hurts\\
	Behind walls of silence\\
	where I am caught \\[0.5cm]
	I'm lost for words\\
	The truth hurts\\
	The times I have the most to say\\
	are the times I can't talk
\\

Words or names that belong together can be glued together. A counterexample is \emph{Rio de Janeiro}. Spelling out the city name over two lines looks odd. Instead, use \textasciitilde \space to connect the single components as in \emph{Rio~de~Janeiro}.
}
\subsubsection{Indention}
By default, \LaTeX indents the first word of a a paragraph after another one has ended. Usually, this is useful and looks professional. However, sometimes this might be unwanted.

\begin{itemize}
	\item For instance, after an itemization, I think indention isn't particularly useful.
\end{itemize}

{\setlength\parindent{0pt}  \texttt{\textbackslash setlength\textbackslash parindent} in curly parentheses controls the amount of indention. Setting the mandatory argument to \emph{0pt} avoids indention altogether.}

Moreover, the indention of whole passage works with \texttt{\textbackslash quote}:
\begin{quote}
	``And I will strike down upon thee, those who attempt to poison and destroy my brothers, and you will know my name is the lord when I lay my vengeance upon thee.''
\end{quote}
		
	\subsubsection{Hyphenation}
	Given the right language package is used, \LaTeX \space usually hyphenates words automatically when appropriate. Occasionally, it may struggle with foreign words such as archae\-opterix. In those cases, use \textbackslash - within the word to mark the appropriate places to hyphenate.
	
	\subsection{Centering}
	Single text elements can be centered by creating a new \texttt{center} environment:
	
	\begin{center}
		This text is centered.\\
		$ a^2 + b^2 = c^2$
	\end{center}

A \emph{quick and dirty} alternative is \texttt{\textbackslash centering} but I don't like it that much.
	
	\subsection{Itemization \& Enumeration}
	Listings work via the \texttt{\textbackslash itemize} environment. Enumerations are built with \texttt{\textbackslash enumerate}. Partitioning over more than one layer works by using more than one environment:
	
	\begin{itemize}
		\item first bulletpoint
		\item second bulletpoint with subsections
		\begin{enumerate}
			\item first option
			\item second option
		\end{enumerate}
		\item third bulletpoint
			\begin{itemize}
				\item [a)]  The style of the bullet points can also be set manually
				\item [b)] However, \LaTeX \space doesn't really like that. It might be a good idea to stick to the defaults.
			\end{itemize}
	\end{itemize}

	
	\newpage
	
	\section{Tables and Figures}
	
	\subsection{Tables}
		Creating tables in \LaTeX is rather inefficient and involved. Below is an example of a manually created table:\\
		
	\begin{table}
		\caption{Excellent Songs}
	\begin{tabular}{|c|c|r|l|}
		\hline
		\multicolumn{4}{|c|}{\textbf{Excellent Songs}} \\
		\hline \hline
		\textbf{song}&\textbf{artist}&\textbf{price}&\textbf{comment} \\
		\hline
		Infinite&Eminem&0,99&Very Good \\
		\hline
		Lady (Hear me Tonight)&Modjo&0,69&Excellent track during summer \\
		\hline
		Still D.R.E.&Dr. D.R.E.&1,29&this is an excellent song \\
		&Snoop Dog&&of two gangsta exhibits who\\
		&&&swear that the game hasn't changed and\\
		&&& both of them remain rightful kings\\
		\hline
	\end{tabular}
	\end{table}
	
	There are many tools which ease the pain of \LaTeX tables (e.g. \emph{Stargazer} to print Regression tables from `R'):
	
	\begin{table}[!htbp] \centering 
		\caption{Diamonds Linear Regression} 
		\label{diamondsreg} 
		\begin{tabular}{@{\extracolsep{5pt}}lc} 
			\\[-1.8ex]\hline 
			\hline \\[-1.8ex] 
			& \multicolumn{1}{c}{\textit{Dependent variable:}} \\ 
			\cline{2-2} 
			\\[-1.8ex] & carat \\ 
			\hline \\[-1.8ex] 
			price & 0.00003$^{***}$ \\ 
			& (0.00000) \\ 
			& \\ 
			x & 0.312$^{***}$ \\ 
			& (0.001) \\ 
			& \\ 
			depth & 0.016$^{***}$ \\ 
			& (0.0002) \\ 
			& \\ 
			Constant & $-$2.133$^{***}$ \\ 
			& (0.016) \\ 
			& \\ 
			\hline \\[-1.8ex] 
			Observations & 53,940 \\ 
			R$^{2}$ & 0.969 \\ 
			Adjusted R$^{2}$ & 0.969 \\ 
			Residual Std. Error & 0.083 (df = 53936) \\ 
			F Statistic & 568,906.900$^{***}$ (df = 3; 53936) \\ 
			\hline 
			\hline \\[-1.8ex] 
			\textit{Note:}  & \multicolumn{1}{r}{$^{*}$p$<$0.1; $^{**}$p$<$0.05; $^{***}$p$<$0.01} \\ 
		\end{tabular} 
	\end{table}
	
	\subsection{Figures}
\LaTeX can import many pictures stored in various file formats and include them in the resulting pdf. The picture file must be located in the same folder as the \emph{.Tex} file.

		\newpage
	\section{Math} \label{form}
	This section is old but might help looking up some commands and templates.
	
	Formeln lassen sich ähnlich Microsofts Formel-Editor einfügen. Der Vorteil ist, dass man aufgrund der befehlbasierten Logik potenzielle Tippfehler besser korrigieren kann: $ x^2 =6 $. Zentrierte Formeln sind auch möglich:
	\begin{center}
		$ y_i =x_1*\beta_1 $
	\end{center}
	
	Hier sind die bekannten Regeln anzuwenden. Man kann tolle Sachen zaubern!
	\subsection{Matrizen}
	Matrizen lassen sich über '\&' sowie '\verb=\=' verarbeiten:
	
	\begin{center}
		$
		\begin{Bmatrix}
		y_{1,1} & y_{1,2} & \dots & y_{1,n} \\
		y_{2,1} & y_{2,2} & \dots & y_{2,n} \\
		\vdots & \vdots & \ddots & \vdots \\
		y_{m,1} & y_{m,2} & \vdots & y_{m,n}
		\end{Bmatrix}
		$
	\end{center}
	
	\subsection{Summen}
	\label{sums}
	Summen sind toll. Hier werden sie über die Alternative zur Einleitung von Formeln eingegeben. Dadurch wird immer ein neuer Absatz erstellt und zentriert.
	\[  \sum_{i=1}^n \]
	
	\subsection{Brüche und Wurzeln}
	Für Brüche wird der \textbf{frac}-Befehl genutzt.
	\[ \frac{a^2+b^2}{c^2+\frac{d}{n}} \]
	
	Wurzeln sind einfach ein Klassiker:
	\[ \sqrt[3]{Inhalt} \]
	
	
	\newpage
	
	\newpage
	\section{Crafting and automating the bibliography}
	Once set up there, \LaTeX in conjunction with JabRef and a bibliography tool takes care of citations. Take this as an example for an imaginary article in a journal citation\footnote{\cite{demo_art}}.
	Here comes a citation of an imaginary book.\footnote{\cite{demo_book}} Moreover, here is an American style reference to an article that was inserted into the pseudo database by Google Scholar's BibTeX function.(\cite{demo_source})
	
	\subsection{Workflow}
	
	Before passing different sources, a few global parameters must be set. These steps must be undertaken once:
	
	\begin{enumerate}
		\item Set up a .bib file and save it in the same directory as the .tex file. JabRef is a nice tool for that. The user interface suggests that a pseudo database is created which makes working with the files easier.
		\item In the file's header load the package \texttt{biblatex} with the appropriate options (see \autoref{config})
		\item Still in the file's header, specify the name of the .bib file with \texttt{\textbackslash addbibresource}
		\item Type \texttt{\textbackslash printbibliography} before the \emph{document} environment is closed (i.e. after the main text but before \texttt{\textbackslash end\{document\} }
	\end{enumerate}
	
	Once that is set up, the workflow for single quotes is like this.
	\begin{enumerate}
		\item include a source in the JabRef pseudo database. There are two options to do this.
			\begin{enumerate}
				\item manually enter all required information including a unique bibtexkey to refer to the source
				\item look online for the BibTex source code. There are some sites where those can be obtained (eg. Google Scholar). A bibtexkey is set by default but can be altered  to fit individual preferences
			\end{enumerate}
		\item use \texttt{\textbackslash cite} and refer to the Bibtexkey in curly brackets.
		\item optionally set up additional parameters in squared brackets to appear before or after the source\footnote{\cite[see][p. 12]{grapov2018rise}}.
	\end{enumerate}

\subsection{Configurations}\label{config}
	There are many options to control the citation style. For instance, the argument \texttt{citestyle} determines how citations are implemented in the text and \texttt{bibstyle} affects how the sources are implemented in the list of references at the end. Some information on style options can be obtained \href{https://www.overleaf.com/learn/latex/Biblatex_citation_styles}{\textbf{\color{blue}here}}.
	
	\subsection{Troubleshooting}
	
	If the workflow doesn't work, the first attempt to solve this should be to re-run the whole thing. If this doesn't work, check the following options:
	\begin{itemize}
		\item \textbf{TexStudio:} Bibliography $\rightarrow$ Type: \emph{BibLaTeX}
		\item \textbf{TexStudio:} Options $\rightarrow$ Configure TexStudio $\rightarrow$ Build\\ $\rightarrow$ Default Bibliography Tool = \emph{Biber}
		\item \textbf{JabRef:} Options $\rightarrow$ General $\rightarrow$ Default Bibliography mode = \emph{biblatex}
	\end{itemize}

The next step is to update or add one entry in JabRef and saving the changes in the .bib file. Following that, delete the existing .aux, .bbl, .bcf files in the directory of the .tex file and re-run three times. If this doesn't work use google for help. Those two sites are pretty extensive for troubleshooting:

\begin{itemize}
	\item \url{https://ipfs-sec.stackexchange.cloudflare-ipfs.com/tex/A/question/63852.html}
	\item 	\url{https://ipfs-sec.stackexchange.cloudflare-ipfs.com/tex/A/question/286706.html}
\end{itemize}

	\printbibliography
	
	\appendix
	\section{Appendix}\label{appendix}
	
		\autoref{trajectory_price} displays the distribution of average prices in various runs of the optimized experiments. \autoref{all_runs} displays the price and profit trajectory of single runs over time. Only runs of the \emph{optimal} runs are printed, as explained in \autoref{justifications}. Both metrics are averaged over 50,000 episodes apiece and over both players. Again, note that, by and large, prices and profits remain within the benchmarks of Nash competition and the cartel case.


\begin{figure}
	\includegraphics[width=\linewidth]{plots/trajectory_price.png}
	\caption{distribution of $p$ over time of 'optimized' experiments. For individual runs, $p$ is averaged over 50,000 periods apiece and both players. Plot includes converged and non-converged runs. Violin widths represent quantity of active runs at $t$ which enables comparisons between violins. As most runs converge after 200,000 to 300,000 episodes, violin widths decrease thereafter. Violins are trimmed at smallest and largest observation respectively. Horizontal lines represent the median.}
	\label{trajectory_price}
\end{figure}

\begin{figure}
	\includegraphics[width=\linewidth]{plots/all_runs.png}
	\caption{all runs for manually optimized $\alpha$}
	\label{all_runs}
\end{figure}

	
	\section{second appendix}
\end{document}
